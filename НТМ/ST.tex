\documentclass[12pt,a4paper]{article}
\usepackage{math-text}

\title{Основы наивной теории множеств.}
\author{\href{https://users.math-cs.spbu.ru/~speranski}{Станислав Олегович Сперанский}}
\date{}

\newcommand{\Pcal}{\ensuremath{\mathcal{P}}\xspace}
\DeclareMathOperator{\Ind}{Ind}
\DeclareMathOperator{\dom}{dom}
\DeclareMathOperator{\range}{range}

\begin{document}
    \maketitle

    Материалы лекций: \href{https://users.math-cs.spbu.ru/~speranski/courses/sets-2020-autumn/materials.html}{ссылка}
    
    Литература:
    \begin{itemize}
        \item K. Hrbacek and T. Jech. Introduction to Set Theory. 3rd ed., revised and expanded. Marcel Dekker, Inc., 1999.
        \item T. Jech. Set Theory. 3rd ed., revised and expanded. Springer, 2002.
    \end{itemize}

    Будем рассматривать как базовые выражения ``$x$ равен (совпадает с) $y$'' (``$x=y$'') ``$x$ лежит в $y$'' (``$x\in y$'').

    \begin{definition}[Наиваная схема аксиом выделения]
        Пусть $\Phi(x)$ --- произвольное условие на объекты. Тогда существует $X$, что $\forall u (\Phi(u) \leftrightarrow u \in X)$. В этом случае $X$ обозначается как $\{u \mid \Phi(u)\}$.
    \end{definition}

    \begin{statement}[парадокс Рассела]
        Пусть $R = \{u \mid u \notin u\}$. Тогда $R$ не может лежать в себе и не может не лежать в себе одновременно.
    \end{statement}

    Из-за данного парадокса будем рассматривать только условия, образованные переменными и $\in$, $=$, $\neg$, $\wedge$, $\vee$, $\leftarrow$, $\leftrightarrow$, $\forall$, $\exists$.

    \begin{definition}[аксиомы ZFC (= ZF (аксиомы Цермело-Френкеля) + C (аксиома выбора))]\ 
        \begin{itemize}
            \item[Ext)] ``Аксиома экстенциональности'':
                \[\forall X \forall Y (\forall u (u \in X \leftrightarrow u \in Y) \leftrightarrow X=Y)\]
            \item[Empty)] ``Аксиома пустого множества'':
                \[\exists \varnothing\; \forall u\, (u \notin \varnothing)\]
            \item[Pair)] ``Аксиома пары'':
                \[\forall X\, \forall Y\; \exists Z (\forall u\, (u \in Z \leftrightarrow (u = X \vee u = Y)))\]
                Обозначение: $Z = \{X, Y\}$.
            \item[Sep)] ``Схема аксиом выделения'':
                \[\forall \Phi(x)\quad \forall X\, \exists Y\; \forall u\, (u\in Y \leftrightarrow (u \in X \wedge \Phi(u)))\]
                Обозначение: $Y = \{u \in X \mid \Phi(u)\}$.

                \begin{corollary*}
                    Операторы
                    \begin{align*}
                        X \cap Y &:= \{u \mid u \in X \wedge u \in Y\}\\
                        X \setminus Y &:= \{u \in X \mid u \notin Y\}\\
                        \bigcap X &:= \{u \mid \forall v \in X \quad u \in v\}
                    \end{align*}
                    определены корректно.
                \end{corollary*}
            \item[Union)] ``Аксиома объединения'':
                \[\forall X\, \exists Y\; \forall u\, (u \in Y \leftrightarrow \exists v\, (v \in X \wedge u \in v))\]
                Обозначение: $Y=\bigcup X$.

                \begin{corollary*}
                    Оператор
                    \[
                        X \cup Y := \bigcup \{X, Y\} = \{u \mid u \in X \wedge u \in Y\}
                    \]
                    определён корректно.
                \end{corollary*}
            \item[Power)] Пусть $x \subseteq y := \forall v\, \{v \in x \rightarrow v \in y\}$. ``Аксиома степени'':
                \[
                    \forall X\, \exists Y\; \forall u\, (u \in Y \leftrightarrow u \subseteq X)
                \]
                Обозначение: $Y = \Pcal(X) := \{u \mid u \subseteq X\}$.
                $\Pcal(X)$ --- ``множество-степень X'' или ``булеан X''.

                \begin{definition}
                    Упорядоченная пара --- это объект от некоторых $X_1$ и $Y_1$, который равен другому такому объекту от $X_2$ и $Y_2$ тогда и только тогда, когда $X_1 = X_2 \wedge Y_1 = Y_2$.
                \end{definition}

                \begin{definition}
                    \emph{Декартово произведение} $X$ и $Y$ ($X \times Y$) --- $\{(x; y) \mid x \in X \wedge y \in Y\}$. 
                \end{definition}

                \begin{remark}
                    Можно нелсожно показать, что декартово произведение определено корректно.
                \end{remark}
            \item[Inf)] Пусть $\Ind(X) := \varnothing \in X \wedge \forall u\, (u \in X \wedge u \cup \{u\} \in X)$. Если $\Ind(X)$, то $X$ называется индуктивным. ``Аксиома бесконечности'': существует индуктивное множество.
            \item[Repl)] ``Схема аксиом подстановки'':
                \begin{align*}
                    \forall \Phi(x, y)\;&\\
                    &\forall x\, \forall y_1\, \forall y_2\, ((\Phi(x, y_1) \wedge \Phi(x, y_2)) \rightarrow y_1 = y_2) \rightarrow\\
                    &\forall X\, \exists Y\; \forall y\, (y \in Y \leftrightarrow \exists x (x \in X \wedge \Phi(x, y)))
                \end{align*}
            \item[Reg)] ``Аксиома регулярности'':
                \[
                    \forall X\, (X\neq \varnothing \rightarrow \exists u\, (u\in X \wedge X \cap u = \varnothing))
                \]
        \end{itemize}
    \end{definition}

    \section{Отношения.}

    \begin{definition}
        \emph{Бинарное (или двухместное) отношение} $R$ между $X$ и $Y$ --- подмножество $X \times Y$. Если $Y = X$, $R$ называется \emph{бинарным (или двухместным) отношением} на $X$.\\
        Обозначение: $(x, y) \in R \Leftrightarrow xRy$.
    \end{definition}

    \begin{definition}
        \begin{align*}
            \dom(R) &:= \{u \in X \mid \exists v\quad uRv\}&& \text{``область определения $R$''}\\
            \range(R) &:= \{v \in Y \mid \exists u\quad uRv\}&& \text{``область значений $R$''}\\
            R[U] &:= range(R \cap (U \times Y))\\
            R^{-1} &:= \{(y, x) \mid (x, y) \in R\}
        \end{align*}
    \end{definition}

    \begin{remark}
        \begin{align*}
            \range(R) = \dom&(R^{-1}) = R[X]\\
            \range(R^{-1}) = \dom&(R) = R^{-1}[Y]
        \end{align*}
    \end{remark}

    \begin{definition}
        Бинарные отношнения можно естественным образом комбинировать: для любых отношений $R$ и $Q$ между $X$ и $Y$, $Y$ и $Z$ соответственно отношение
        \[
            S = R \circ Q := \{(x, z) \in X \times Z \mid \exists y: xRy \wedge yQz\}
        \]
        называется композицией $R$ и $Q$.
    \end{definition}

    \begin{definition}
        \emph{Тождественное отображение} на $X$ --- $id_X := \{(x, x) \mid x \in X\}$.
    \end{definition}

    \begin{remark}
        Тождественное отображение при композиции (не важно, правой или левой) с другим отношением не меняет его. 
    \end{remark}

    \begin{definition}
        Отношение $R$ между $X$ и $Y$ называется функциональным, если
        \[
            \forall x\; \forall y_1\, \forall y_2\, ((xRy_1 \wedge xRy_2) \rightarrow y_1 = y_2).
        \]
    \end{definition}

    \begin{definition}
        \emph{Функция} из $X$ в $Y$ --- функциональное отношение $R$ между $X$ и $Y$, в котором $\dom(R)=X$.
    \end{definition}
\end{document}