\documentclass[12pt,a4paper]{article}
\usepackage{math-text}

\title{Математический анализ --- 1.}
\author{\href{https://vk.com/ybelov}{Юрий Сергеевич Белов}}
\date{}

\DeclareMathOperator{\Quot}{Quot}

\begin{document}
    \maketitle

    Литература:
    \begin{itemize}
        \item В. А. Зорич ``Математический анализ''
        \item О. Л. Виноградов ``Математический анализ''
        \item (подходит попозже) Г. М. Фихтенгельц ``Курс дифференциального и интегрального исчисления''
        \item У. Рудин ``Основы анализа''
        \item М. Спивак ``Математический анализ на многообразиях''
    \end{itemize}

    Мы начинаем с теории множеств.

    \begin{definition}\ 
        \begin{itemize}
            \item Множества и элемменты --- понятно.
            \item $a \in B$ --- понятно.
            \item $A \cup B := \{x \mid x\in A \vee x\in B\}$ --- объединение.
            \item $A \cap B := \{x \mid x\in A \wedge x\in B\}$ --- пересечение.
            \item $A \setminus B := \{x \mid x\in A \vee x\notin B\}$ --- разность.
            \item $A \bigtriangleup B := A \setminus B \cup B \setminus A$ --- симметрическая разница.
            \item $A^C := X\setminus A$ --- \emph{дополнение}, где $X$ --- некоторое фиксированное рассматриваемое множество.
            \item $A \subset B$ --- ``$A$ --- подмножество $B$'', т.е. $\forall x (X\in A \Rightarrow x\in B)$.
        \end{itemize}
    \end{definition}

    \begin{corollary*}\ 
        \begin{itemize}
            \item (первое правило Моргана) $(A\cup B)^C = A^C \cap B^C$.
                \begin{align*}
                    x\in (A\cup B)^C \Leftrightarrow
                    x \notin A \cup B \Leftrightarrow
                    \left\{ \begin{aligned}
                        &x \notin A\\
                        &x \notin B
                    \end{aligned}\right. \Leftrightarrow
                    \left\{ \begin{aligned}
                        &x \in A^c\\
                        &x \in B^C
                    \end{aligned} \right. \Leftrightarrow
                    x \in A^C \cap B^C
                \end{align*}
            \item (второе правило Моргана) $(A\cap B)^C = A^C \cup B^C$. Аналогично.
        \end{itemize}
    \end{corollary*}

    \begin{definition}
        (Аксиома индукции.) Пусть есть функция $A: \NN \to {true;false}$, что:
        \begin{enumerate}
            \item $A(1)=true$;
            \item $\forall n (A(n) \rightarrow A(n+1))$.
        \end{enumerate}
        Тогда $\forall n A(n)$.
    \end{definition}

    Определение натуральных чисел сложно, рассматривать его не будем. Важно также иметь в виду натуральные числа с операциями сложения и умножения.

    \begin{definition}
        Пусть есть кольцо без делителей нуля $R$. Рассмотрим отношение эквивалентности $\sim$ на $R \times (R\setminus \{0\})$, что $(a; b) \sim (c; d) \Leftrightarrow ad = bc$. Тогда $\Quot(R)$ --- фактор-множество по $\sim$ и поле.
    \end{definition}

    \begin{definition}
        Рациональные числа --- $\QQ := \Quot(\ZZ)$.
    \end{definition}

    \begin{theorem}
        $\nexists x\in \QQ, x^2 = 2$.
    \end{theorem}

    Теперь мы хотим понять, что есть вещественные числа. Тут есть несколько подходов.

    \begin{definition}[аксиоматический подход]
        Вещественные числа --- это полное упорядоченное поле $\RR$, состоящее не из одного элемента.
        
        Здесь ``поле'' значит, что на множестве (вместе с его операциями и выделенными элементами) верны акиомы $A_1$, $A_2$, $A_3$, $A_4$, $M_1$, $M_2$, $M_3$, $M_4$ и $D$.
        
        Упорядоченность значит, что есть рефлексивное транзитивное антисимметричное отношение $\preccurlyeq$, что все элементы сравнимы, согласованное с операциями, т.е.:
        \begin{itemize}
            \item[$A$)] $a \preccurlyeq b \Leftarrow a + x \preccurlyeq b + x$.
            \item[$M$)] $0 \preccurlyeq a \wedge 0 \preccurlyeq b \Rightarrow 0 \preccurlyeq ab$
        \end{itemize}

        Полнота поля значит любое из следующих утверждений (они равносильны):
        \begin{itemize}
            \item любое ограниченное сверху (снизу) подмножество поля имеет точную верхнюю (нижнюю) грань;
            \item (аксиома Кантора-Дедекинда) для любых двух множеств $A$ и $B$, что $A \preccurlyeq B$, есть разделяющий их элемент.
        \end{itemize}

        Итого мы имеем 9 аксиом поля, 2 аксиомы упорядоченности и 1 акиома полноты упорядоченности.
    \end{definition}

    \begin{statement*}
        Над $\QQ$ нет  элемента разделяющего $A := \{a > 0 \mid a^2 < 2\}$ и $B := \{b > 0 \mid b^2 > 2\}$.
    \end{statement*}

    \begin{proof}
        Предположим противное, т.е. есть $c > 0$, что $A < c < B$.

        Если $c^2 < 2$, то найдём $\varepsilon$, что $\varepsilon \in (0; 1)$ и $(c + \varepsilon)^2 < 2$. Заметим, что $(c + \varepsilon)^2 = c^2 + 2c\varepsilon + \varepsilon^2 < c^2 + (2c + 1)\varepsilon$. Пусть $\varepsilon < \frac{2 - c^2}{2c+ 1}$, тогда такое $\varepsilon$ точно подойдёт, ну а посокольку $\frac{2 - c^2}{2c + 1} > 0$, то такое $\varepsilon$ есть. Значит $c^2 \geqslant 2$.
        
        Аналогично имеем, что $\varepsilon \leqslant 2$. А значит $c^2 = 2$, что не бывает над $\QQ$.
    \end{proof}

    \begin{corollary*}
        $\QQ$ не полно.
    \end{corollary*}

    \begin{definition}\ 
        \begin{itemize}
            \item \emph{Закрытый интервал} или \emph{отрезок} $[a;b]:=\{x\in\RR \mid a \leqslant x \leqslant b\}$.
            \item \emph{Открытый интервал} или просто \emph{интервал} $(a;b):=\{x\in\RR \mid a < x < b\}$.
            \item \emph{Полуоткрытый интервал} или \emph{полуинтервал} $(a;b] := \{x\in\RR \mid a < x \leqslant b\}$, $[a;b):=\{x\in\RR \mid a \leqslant x < b\}$.
        \end{itemize}
    \end{definition}

    \begin{theorem}[Лемма о вложенных отрезках]\label{th_inter_segments}
        Пусть имеется $\{I_i\}_{i=1}^\infty$ --- множество вложенных (непустых) отрезков, т.е. $\forall n > 1 I_{n+1} \subset I_n$. Тогда $\bigcap_{i=1}^\infty I_i \neq \varnothing$.
    \end{theorem}

    \begin{proof}
        Заметим, что для любых натуральных $n < m$ верно, что $a_n \leqslant a_m \leqslant b_m \leqslant b_n$, где $I_n = [a_n;b_n]$. Тогда для $A:=\{a_i\}_{i=1}^\infty$ и $B:=\{b_i\}_{i=1}^\infty$ верно, что $A \leqslant B$. Значит есть разделяющий их элемент $t$, значит $A \leqslant t \leqslant B$, значит $t\in I_i$ для всех $i$, значит $t \in \bigcap_{i=1}^\infty I_i$.
    \end{proof}

    \begin{remark}
        Теорема \ref{th_inter_segments} не верна для не отрезков.
    \end{remark}

    \begin{remark}
        Если в теореме \ref{th_inter_segments} $b_i-a_i$ ``сходится к 0'', т.е. $\forall \varepsilon > 0\, \exists n\in\NN: \forall i > n\, b_i-a_i < \varepsilon$, то пересечение всех отрезков состоит из ровно одного элемента.
    \end{remark}
\end{document}