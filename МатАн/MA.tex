\documentclass[12pt,a4paper]{article}
\usepackage{math-text}

\title{Математический анализ --- 1.}
\author{\href{https://vk.com/ybelov}{Юрий Сергеевич Белов}}
\date{}

\DeclareMathOperator{\Quot}{Quot}

\begin{document}
    \maketitle

    Литература:
    \begin{itemize}
        \item В. А. Зорич ``Математический анализ''
        \item О. Л. Виноградов ``Математический анализ''
        \item (подходит попозже) Г. М. Фихтенгельц ``Курс дифференциального и интегрального исчисления''
        \item У. Рудин ``Основы анализа''
        \item М. Спивак ``Математический анализ на многообразиях''
    \end{itemize}

    \section{Множества, аксиоматика и вещественные числа.}

    Мы начинаем с теории множеств.

    \begin{definition}\ 
        \begin{itemize}
            \item Множества и элемменты --- понятно.
            \item $a \in B$ --- понятно.
            \item $A \cup B := \{x \mid x\in A \vee x\in B\}$ --- объединение.
            \item $A \cap B := \{x \mid x\in A \wedge x\in B\}$ --- пересечение.
            \item $A \setminus B := \{x \mid x\in A \vee x\notin B\}$ --- разность.
            \item $A \bigtriangleup B := A \setminus B \cup B \setminus A$ --- симметрическая разница.
            \item $A^C := X\setminus A$ --- \emph{дополнение}, где $X$ --- некоторое фиксированное рассматриваемое множество.
            \item $A \subset B$ --- ``$A$ --- подмножество $B$'', т.е. $\forall x (X\in A \Rightarrow x\in B)$.
        \end{itemize}
    \end{definition}

    \begin{corollary*}\ 
        \begin{itemize}
            \item (первое правило Моргана) $(A\cup B)^C = A^C \cap B^C$.
                \begin{align*}
                    x\in (A\cup B)^C \Leftrightarrow
                    x \notin A \cup B \Leftrightarrow
                    \left\{ \begin{aligned}
                        &x \notin A\\
                        &x \notin B
                    \end{aligned}\right. \Leftrightarrow
                    \left\{ \begin{aligned}
                        &x \in A^c\\
                        &x \in B^C
                    \end{aligned} \right. \Leftrightarrow
                    x \in A^C \cap B^C
                \end{align*}
            \item (второе правило Моргана) $(A\cap B)^C = A^C \cup B^C$. Аналогично.
        \end{itemize}
    \end{corollary*}

    \begin{definition}
        (Аксиома индукции.) Пусть есть функция $A: \NN \to {true;false}$, что:
        \begin{enumerate}
            \item $A(1)=true$;
            \item $\forall n (A(n) \rightarrow A(n+1))$.
        \end{enumerate}
        Тогда $\forall n A(n)$.
    \end{definition}

    Определение натуральных чисел сложно, рассматривать его не будем. Важно также иметь в виду натуральные числа с операциями сложения и умножения.

    \begin{definition}
        Пусть есть кольцо без делителей нуля $R$. Рассмотрим отношение эквивалентности $\sim$ на $R \times (R\setminus \{0\})$, что $(a; b) \sim (c; d) \Leftrightarrow ad = bc$. Тогда $\Quot(R)$ --- фактор-множество по $\sim$ и поле.
    \end{definition}

    \begin{definition}
        Рациональные числа --- $\QQ := \Quot(\ZZ)$.
    \end{definition}

    \begin{theorem}
        $\nexists x\in \QQ, x^2 = 2$.
    \end{theorem}

    Теперь мы хотим понять, что есть вещественные числа. Тут есть несколько подходов.

    \begin{definition}[аксиоматический подход]
        Вещественные числа --- это полное упорядоченное поле $\RR$, состоящее не из одного элемента.
        
        Здесь ``поле'' значит, что на множестве (вместе с его операциями и выделенными элементами) верны акиомы $A_1$, $A_2$, $A_3$, $A_4$, $M_1$, $M_2$, $M_3$, $M_4$ и $D$.
        
        Упорядоченность значит, что есть рефлексивное транзитивное антисимметричное отношение $\preccurlyeq$, что все элементы сравнимы, согласованное с операциями, т.е.:
        \begin{itemize}
            \item[$A$)] $a \preccurlyeq b \Leftarrow a + x \preccurlyeq b + x$.
            \item[$M$)] $0 \preccurlyeq a \wedge 0 \preccurlyeq b \Rightarrow 0 \preccurlyeq ab$
        \end{itemize}

        Полнота поля значит любое из следующих утверждений (они равносильны):
        \begin{itemize}
            \item любое ограниченное сверху (снизу) подмножество поля имеет точную верхнюю (нижнюю) грань;
            \item (аксиома Кантора-Дедекинда) для любых двух множеств $A$ и $B$, что $A \preccurlyeq B$, есть разделяющий их элемент.
        \end{itemize}

        Итого мы имеем 9 аксиом поля, 2 аксиомы упорядоченности и 1 акиома полноты упорядоченности.
    \end{definition}

    \begin{statement*}
        Над $\QQ$ нет  элемента разделяющего $A := \{a > 0 \mid a^2 < 2\}$ и $B := \{b > 0 \mid b^2 > 2\}$.
    \end{statement*}

    \begin{proof}
        Предположим противное, т.е. есть $c > 0$, что $A < c < B$.

        Если $c^2 < 2$, то найдём $\varepsilon$, что $\varepsilon \in (0; 1)$ и $(c + \varepsilon)^2 < 2$. Заметим, что $(c + \varepsilon)^2 = c^2 + 2c\varepsilon + \varepsilon^2 < c^2 + (2c + 1)\varepsilon$. Пусть $\varepsilon < \frac{2 - c^2}{2c+ 1}$, тогда такое $\varepsilon$ точно подойдёт, ну а посокольку $\frac{2 - c^2}{2c + 1} > 0$, то такое $\varepsilon$ есть. Значит $c^2 \geqslant 2$.
        
        Аналогично имеем, что $\varepsilon \leqslant 2$. А значит $c^2 = 2$, что не бывает над $\QQ$.
    \end{proof}

    \begin{corollary*}
        $\QQ$ не полно.
    \end{corollary*}

    \begin{definition}\ 
        \begin{itemize}
            \item \emph{Закрытый интервал} или \emph{отрезок} $[a;b]:=\{x\in\RR \mid a \leqslant x \leqslant b\}$.
            \item \emph{Открытый интервал} или просто \emph{интервал} $(a;b):=\{x\in\RR \mid a < x < b\}$.
            \item \emph{Полуоткрытый интервал} или \emph{полуинтервал} $(a;b] := \{x\in\RR \mid a < x \leqslant b\}$, $[a;b):=\{x\in\RR \mid a \leqslant x < b\}$.
        \end{itemize}
    \end{definition}

    \begin{theorem}[Лемма о вложенных отрезках]\label{th_inter_segments}
        Пусть имеется $\{I_i\}_{i=1}^\infty$ --- множество вложенных (непустых) отрезков, т.е. $\forall n > 1 I_{n+1} \subset I_n$. Тогда $\bigcap_{i=1}^\infty I_i \neq \varnothing$.
    \end{theorem}

    \begin{proof}
        Заметим, что для любых натуральных $n < m$ верно, что $a_n \leqslant a_m \leqslant b_m \leqslant b_n$, где $I_n = [a_n;b_n]$. Тогда для $A:=\{a_i\}_{i=1}^\infty$ и $B:=\{b_i\}_{i=1}^\infty$ верно, что $A \leqslant B$. Значит есть разделяющий их элемент $t$, значит $A \leqslant t \leqslant B$, значит $t\in I_i$ для всех $i$, значит $t \in \bigcap_{i=1}^\infty I_i$.
    \end{proof}

    \begin{remark}
        Теорема \ref{th_inter_segments} не верна для не отрезков.
    \end{remark}

    \begin{remark}
        Если в теореме \ref{th_inter_segments} $b_i-a_i$ ``сходится к 0'', т.е. $\forall \varepsilon > 0\, \exists n\in\NN: \forall i > n\, b_i-a_i < \varepsilon$, то пересечение всех отрезков состоит из ровно одного элемента.
    \end{remark}

    \begin{theorem}[индукция на вещественных числах]
        Пусть дано множество $X \subseteq [0;1]$, что
        \begin{enumerate}
            \item $0 \in X$;
            \item $\forall x \in X\; \exists \varepsilon > 0: U_\varepsilon(x) \cap [0;1] \subseteq X$;
            \item $\forall Y \subseteq X\; \sup(Y) \in X$.
        \end{enumerate}
        Тогда $X = [0;1]$.
    \end{theorem}

    \begin{proof}
        Предположим противное: $X \neq [0;1]$. Рассмотрим $Z := [0;1] \setminus X$ ($Z \neq \varnothing$!) и $Y := \{y \in [0;1] \mid y < Z\}$ ($Y \neq \varnothing$!). Заметим, что $Y \subseteq X$ и $\sup(Y) = \inf(Z) = t$. Тогда $t \in X$ по второму условию. Значит для некоторого $\varepsilon > 0$ верно, что $U_{\varepsilon}(t) \cap [0;1] \in X$, а т.е. $(U_\varepsilon(t) \cap [0;1]) \cap Z = \varnothing$, а тогда $t \neq \inf(Z)$ --- противоречие. Значит $X = [0;1]$.
    \end{proof}

    \section{Топология прямой, пределы и непрерывность.}

    \begin{definition}
        \emph{$\varepsilon$-окрестность} точки $x$ (для $\varepsilon > 0$) --- $(x - \varepsilon; x+ \varepsilon)$. Обозначение: $U_\varepsilon(x)$.

        \emph{Проколотая $\varepsilon$-окрестность} точки $x$ --- $(x - \varepsilon; x) \cup (x; x + \varepsilon)$. Обозначение: $V_\varepsilon(x)$.
    \end{definition}

    \begin{definition}
        Пусть дано некоторое множество $X \subseteq \RR$. Тогда точка $x \in X$ называется \emph{внутренней точкой множества} $X$, если она содержится в $X$ вместе со своей окрестностью.
        
        Само множество $X$ называется \emph{открытым}, если все его точки внутренние.
    \end{definition}

    \begin{example}
        Следующие множества открыты:
        \begin{itemize}
            \item $(a; b)$;
            \item $(a; +\infty)$;
            \item $\RR$;
            \item $\varnothing$;
            \item $\bigcup_{i=0}^\infty (a_i; b_i)$ (интервалы не обящательно не должны пересекаться).
        \end{itemize}
    \end{example}

    \begin{definition}
        Пусть дано множесство $X\subseteq \RR$. Точка $x \in \RR$ называется \emph{предельной точкой} множества, если в любой проколотой окрестности $x$ будет какая-либо точка $X$.
        
        Множество предельных точек $X$ называется \emph{производным множеством} множества $X$ и обозначается как $X'$.

        Множество $X$ называется замкнутым, если $X \supseteq X'$.
    \end{definition}

    \begin{definition}
        Пусть дано множество $X\subseteq \RR$. Если у любой последовательности его точек есть предельная точка из самого множества $X$.
    \end{definition}

    \begin{definition}
        \emph{Предел последовательности} $\{x_n\}_{n=0}^\infty$ --- такое число $x$, что для любой окрестности $x$ эта последовательность с некоторого момента будет лежать в этой окрестности:
        \[\forall \varepsilon > 0\; \exists N \in \NN: \forall n \geqslant N\quad x_n \in U_\varepsilon(x)\]
        Обозначение: $\lim \{x_n\}_{n=0}^\infty = x$.

        \emph{Предельная точка последовательности} $\{x_n\}_{n=0}^\infty$ --- такое число $x$, что в любой его окрестности после любого момента появится элемент данной последовательности:
        \[\forall \varepsilon > 0\, \forall N \in \NN\; \exists n > N: \quad x_n \in U_\varepsilon(x)\]
    \end{definition}

    \begin{definition}
        \emph{Предел} функции $f: X \to \RR$ при в точке $x$ --- такое значение $y$, что
        \[\forall \varepsilon > 0\, \exists \delta > 0: f(V_\delta(x) \cap X) = U_\varepsilon(y)\]
        Обозначение: $\lim\limits_{t \to x} f(t) = y$.
    \end{definition}

    \begin{definition}
        Функция $f: X \to \RR$ называется \emph{непрерывной в точке} $x$, если $\lim\limits_{t \to x} f(t) = f(x)$. 
    \end{definition}

    \begin{statement}
        Для последовательностей $\{x_n\}_{n=0}^\infty$ и $\{y_n\}_{n=0}^\infty$ верно, что
        \begin{enumerate}
            \item $\lim \{x_n\}_{n=0}^\infty + \lim \{y_n\}_{n=0}^\infty = \lim \{x_n + y_n\}_{n=0}^\infty$
            \item $\lim \{x_n\}_{n=0}^\infty \cdot \lim \{y_n\}_{n=0}^\infty = \lim \{x_n y_n\}_{n=0}^\infty$
            \item $\frac{1}{\lim \{x_n\}_{n=0}^\infty} = \lim \{\frac{1}{x_n}\}_{n=0}^\infty$ (если $\lim \{x_n\}_{n=0}^\infty \neq 0$)
        \end{enumerate}
        и одна из сторон равенства существует тогда и только тогда, когда существует другая (кроме случая во втором пункте, когда один из пределов равен $0$).
    \end{statement}

    \begin{statement}
        Для функций $f: \RR \to \RR$ и $g: \RR \to \RR$ верно, что
        \begin{enumerate}
            \item $\lim\limits_{x \to a} f(x) + \lim\limits_{x \to a} g(x) = \lim\limits_{x \to a} (f + g)(x)$
            \item $\lim\limits_{x \to a} f(x) \cdot \lim\limits_{x \to a} g(x) = \lim (fg)(x)$
            \item $\frac{1}{\lim\limits_{x \to a} f(x)} = \lim\limits_{x \to a} (\frac{1}{f})(x)$ (если $\lim\limits_{x \to a} f(x) \neq 0$)
            \item $\lim\limits_{y \to \lim\limits_{x \to a} g(x)} f(y) = \lim\limits_{x \to a} (f \circ g)(x)$
        \end{enumerate}
        и одна из сторон равенства существует тогда и только тогда, когда существует другая (кроме случая во втором пункте, когда один из пределов равен $0$).
    \end{statement}

    \begin{definition}
        Последовательность $\{x_n\}_{n=0}^\infty$ \emph{ассимптотически больше} последовательности $\{y_n\}_{n=0}^\infty$, если $x_n > y_n$ для всех натуральных $n$, начиная с некоторого. Обозначение: $\{x_n\}_{n=0}^\infty \succ \{y_n\}_{n=0}^\infty$.

        Аналогично определяются \emph{ассимптотически меньше} ($\{x_n\}_{n=0}^\infty \prec \{y_n\}_{n=0}^\infty$), \emph{ассимптотически не больше} ($\{x_n\}_{n=0}^\infty \preccurlyeq \{y_n\}_{n=0}^\infty$) и \emph{ассимптотически не меньше} ($\{x_n\}_{n=0}^\infty \succcurlyeq \{y_n\}_{n=0}^\infty$).
    \end{definition}

    \begin{statement}\label{stupid_seq_statement_1}
        Если $\{x_n\}_{n=0}^\infty \succcurlyeq \{y_n\}_{n=0}^\infty$, то $\lim \{x_n\}_{n=0}^\infty \geqslant \lim \{y_n\}_{n=0}^\infty$.
    \end{statement}

    \begin{statement}\label{stupid_seq_statement_2}
        Если $\lim \{x_n\}_{n=0}^\infty > \lim \{y_n\}_{n=0}^\infty$, то $\{x_n\}_{n=0}^\infty \succ \{y_n\}_{n=0}^\infty$.
    \end{statement}

    \begin{statement}[леммма о двух полицейских]\label{stupid_seq_statement_3}
        Если
        \[\{x_n\}_{n=0}^\infty \succcurlyeq \{y_n\}_{n=0}^\infty \succcurlyeq \{z_n\}_{n=0}^\infty\]
        и
        \[\lim \{x_n\}_{n=0}^\infty = \lim \{z_n\}_{n=0}^\infty = A,\]
        то предел $\{y_n\}_{n=0}^\infty$ определён и равен $A$.
    \end{statement}

    \begin{statement}
        Если $\{x_n\}_{n=0}^\infty \succcurlyeq \{y_n\}_{n=0}^\infty$, $\lim \{x_n\}_{n=0}^\infty = A$, а $\{y_n\}_{n=0}^\infty$, неубывает (с некоторого момента), то предел $\{y_n\}_{n=0}^\infty$ существует и не превосходит $A$.
    \end{statement}

    \begin{remark}
        Утверждения \ref{stupid_seq_statement_1}, \ref{stupid_seq_statement_2} и \ref{stupid_seq_statement_3} верны, если заменить последовательности на функции, пределы последовательностей на пределы функций в некоторой точке $x$, а асимптотические неравенства на неравенства на окрестности $x$.
    \end{remark}
\end{document}