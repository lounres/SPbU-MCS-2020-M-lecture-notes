\documentclass[12pt,a4paper]{article}
\usepackage{math-text}
\usepackage{todonotes}

\title{Математический анализ --- 1.}
\author{Лектор --- \href{https://vk.com/ybelov}{Юрий Сергеевич Белов}\\
        Создатель конспекта --- Глеб Минаев
        \footnote{Оригинал конспекта опубликован расположен \href{https://github.com/lounres/SPbU-MCS-2020-M-lecture-notes}{GitHub}}}
\date{}

\DeclareMathOperator{\Quot}{Quot}
\DeclareMathOperator{\osc}{osc}

\begin{document}
    \maketitle

    Литература:
    \begin{itemize}
        \item В. А. Зорич ``Математический анализ''
        \item О. Л. Виноградов ``Математический анализ''
        \item (подходит попозже) Г. М. Фихтенгельц ``Курс дифференциального и интегрального исчисления''
        \item У. Рудин ``Основы анализа''
        \item М. Спивак ``Математический анализ на многообразиях''
        \item В. М. Тихомиров ``Рассказы о максимумах и минимумах''
    \end{itemize}

    \section{Множества, аксиоматика и вещественные числа.}

    Мы начинаем с теории множеств.

    \begin{definition}\ 
        \begin{itemize}
            \item Множества и элементы --- понятно.
            \item $a \in B$ --- понятно.
            \item $A \cup B := \{x \mid x\in A \vee x\in B\}$ --- объединение.
            \item $A \cap B := \{x \mid x\in A \wedge x\in B\}$ --- пересечение.
            \item $A \setminus B := \{x \mid x\in A \vee x\notin B\}$ --- разность.
            \item $A \bigtriangleup B := A \setminus B \cup B \setminus A$ --- симметрическая разница.
            \item $A^C := X\setminus A$ --- \emph{дополнение}, где $X$ --- некоторое фиксированное рассматриваемое множество.
            \item $A \subset B$ --- ``$A$ --- подмножество $B$'', т.е. $\forall x (X\in A \Rightarrow x\in B)$.
        \end{itemize}
    \end{definition}

    \begin{corollary*}\ 
        \begin{itemize}
            \item (первое правило Моргана) $(A\cup B)^C = A^C \cap B^C$.
                \begin{align*}
                    x\in (A\cup B)^C \Leftrightarrow
                    x \notin A \cup B \Leftrightarrow
                    \left\{ \begin{aligned}
                        &x \notin A\\
                        &x \notin B
                    \end{aligned}\right. \Leftrightarrow
                    \left\{ \begin{aligned}
                        &x \in A^c\\
                        &x \in B^C
                    \end{aligned} \right. \Leftrightarrow
                    x \in A^C \cap B^C
                \end{align*}
            \item (второе правило Моргана) $(A\cap B)^C = A^C \cup B^C$. Аналогично.
        \end{itemize}
    \end{corollary*}

    \begin{definition}
        (Аксиома индукции.) Пусть есть функция $A: \NN \to {true;false}$, что:
        \begin{enumerate}
            \item $A(1)=true$;
            \item $\forall n (A(n) \rightarrow A(n+1))$.
        \end{enumerate}
        Тогда $\forall n A(n)$.
    \end{definition}

    Определение натуральных чисел сложно, рассматривать его не будем. Важно также иметь в виду натуральные числа с операциями сложения и умножения.

    \begin{definition}
        Пусть есть кольцо без делителей нуля $R$. Рассмотрим отношение эквивалентности $\sim$ на $R \times (R\setminus \{0\})$, что $(a; b) \sim (c; d) \Leftrightarrow ad = bc$. Тогда $\Quot(R)$ --- фактор-множество по $\sim$ и поле.
    \end{definition}

    \begin{definition}
        Рациональные числа --- $\QQ := \Quot(\ZZ)$.
    \end{definition}

    \begin{theorem}
        $\nexists x\in \QQ, x^2 = 2$.
    \end{theorem}

    \begin{proof}
        Предположим противное, т.е. существуют взаимно простые $m \in \ZZ$ и $n \in \NN\setminus\{0\}$, что $(\frac{m}{n})^2 = 2$. Тогда $m^2 = 2n^2$. Очевидно, что тогда $m^2 \divides 2$, значит $m\divides 2$, значит $m\divides 4$, значит $n^2 \divides 2$, значит $n \divides 2$, значит $n$ и $m$ не взаимно просты, так как делятся на $2$ --- противоречие.
    \end{proof}

    Теперь мы хотим понять, что есть вещественные числа. Тут есть несколько подходов.

    \begin{definition}[аксиоматический подход]
        Вещественные числа --- это полное упорядоченное поле $\RR$, состоящее не из одного элемента.
        
        Здесь ``поле'' значит, что на множестве (вместе с его операциями и выделенными элементами) верны аксиомы поля $A_1$, $A_2$, $A_3$, $A_4$, $M_1$, $M_2$, $M_3$, $M_4$ и $D$ (т.е. сложение и умножение ассоциативны, коммутативны имеют нейтральные элементы и удовлетворяют условию существованию обратных (по умножению --- для всех кроме нуля), а также дистрибутивности).
        
        Упорядоченность значит, что есть рефлексивное транзитивное антисимметричное отношение $\preccurlyeq$, что все элементы сравнимы, согласованное с операциями, т.е.:
        \begin{itemize}
            \item[$A$)] $a \preccurlyeq b \Rightarrow a + x \preccurlyeq b + x$.
            \item[$M$)] $0 \preccurlyeq a \wedge 0 \preccurlyeq b \Rightarrow 0 \preccurlyeq ab$.
        \end{itemize}

        Полнота поля значит любое из следующих утверждений (они равносильны):
        \begin{itemize}
            \item любое ограниченное сверху (снизу) подмножество поля имеет точную верхнюю (нижнюю) грань;
            \item (аксиома Кантора-Дедекинда) для любых двух множеств $A$ и $B$, что $A \preccurlyeq B$, есть разделяющий их элемент.
        \end{itemize}

        Итого мы имеем 9 аксиом поля, 2 аксиомы упорядоченности и 1 аксиома полноты упорядоченности.
    \end{definition}

    \begin{statement*}
        Над $\QQ$ нет  элемента разделяющего $A := \{a > 0 \mid a^2 < 2\}$ и $B := \{b > 0 \mid b^2 > 2\}$.
    \end{statement*}

    \begin{proof}
        Предположим противное, т.е. есть $c > 0$, что $A < c < B$.

        Если $c^2 < 2$, то найдём $\varepsilon$, что $\varepsilon \in (0; 1)$ и $(c + \varepsilon)^2 < 2$. Заметим, что $(c + \varepsilon)^2 = c^2 + 2c\varepsilon + \varepsilon^2 < c^2 + (2c + 1)\varepsilon$. Пусть $\varepsilon < \frac{2 - c^2}{2c+ 1}$, тогда такое $\varepsilon$ точно подойдёт, ну а поскольку $\frac{2 - c^2}{2c + 1} > 0$, то такое $\varepsilon$ есть. Значит $c^2 \geqslant 2$.
        
        Аналогично имеем, что $\varepsilon \leqslant 2$. А значит $c^2 = 2$, что не бывает над $\QQ$.
    \end{proof}

    \begin{corollary*}
        $\QQ$ не полно.
    \end{corollary*}
    
    \begin{definition}
        Значение $t$ является \emph{верхней (нижней) гранью} непустого множества $X \in \RR$ тогда и только тогда, когда $t \geqslant X$, т.е. любой элемент $x$ множества $X$ не более $t$.
        
        
        \emph{Точная верхняя (нижняя) грань} или \emph{супремум (инфимум)} непустого множества $X \subseteq \RR$ --- минимальная верхняя (нижняя) грань множества $X$. Он же является элементом разделяющим $X$ и множество всех его верхних (нижних) граней. Обозначение: $\sup(X)$ и $\inf(X)$ соответственно.

        \emph{Осцелляцией} множества $X$ называется значение $\osc X := \sup X - \inf X$.
    \end{definition}

    \begin{definition}\ 
        \begin{itemize}
            \item \emph{Закрытый интервал} или \emph{отрезок} $[a;b]:=\{x\in\RR \mid a \leqslant x \leqslant b\}$.
            \item \emph{Открытый интервал} или просто \emph{интервал} $(a;b):=\{x\in\RR \mid a < x < b\}$.
            \item \emph{Полуоткрытый интервал} или \emph{полуинтервал} $(a;b] := \{x\in\RR \mid a < x \leqslant b\}$, $[a;b):=\{x\in\RR \mid a \leqslant x < b\}$.
        \end{itemize}
    \end{definition}

    \begin{theorem}[Лемма о вложенных отрезках]\label{th_inter_segments}
        Пусть имеется $\{I_i\}_{i=1}^\infty$ --- множество вложенных (непустых) отрезков, т.е. $\forall n > 1\; I_{n+1} \subset I_n$. Тогда $\bigcap_{i=1}^\infty I_i \neq \varnothing$.
    \end{theorem}

    \begin{proof}
        Заметим, что для любых натуральных $n < m$ верно, что $a_n \leqslant a_m \leqslant b_m \leqslant b_n$, где $I_n = [a_n;b_n]$. Тогда для $A:=\{a_i\}_{i=1}^\infty$ и $B:=\{b_i\}_{i=1}^\infty$ верно, что $A \leqslant B$. Значит есть разделяющий их элемент $t$, значит $A \leqslant t \leqslant B$, значит $t\in I_i$ для всех $i$, значит $t \in \bigcap_{i=1}^\infty I_i$.
    \end{proof}

    \begin{remark}
        Теорема \ref{th_inter_segments} не верна для не отрезков.
    \end{remark}

    \begin{remark}
        Если в теореме \ref{th_inter_segments} $b_i-a_i$ ``сходится к 0'', т.е. $\forall \varepsilon > 0\, \exists n\in\NN: \forall i > n\, b_i-a_i < \varepsilon$, то пересечение всех отрезков состоит из ровно одного элемента.
    \end{remark}

    \begin{theorem}[индукция на вещественных числах]
        Пусть дано множество $X \subseteq [0;1]$, что
        \begin{enumerate}
            \item $0 \in X$;
            \item $\forall x \in X\; \exists \varepsilon > 0: U_\varepsilon(x) \cap [0;1] \subseteq X$;
            \item $\forall Y \subseteq X\; \sup(Y) \in X$.
        \end{enumerate}
        Тогда $X = [0;1]$.
    \end{theorem}

    \begin{proof}
        Предположим противное: $X \neq [0;1]$. Рассмотрим $Z := [0;1] \setminus X$ ($Z \neq \varnothing$!) и $Y := \{y \in [0;1] \mid y < Z\}$ ($Y \neq \varnothing$!). Заметим, что $Y \subseteq X$ и $\sup(Y) = \inf(Z) = t$. Тогда $t \in X$ по второму условию. Значит для некоторого $\varepsilon > 0$ верно, что $U_{\varepsilon}(t) \cap [0;1] \in X$, а т.е. $(U_\varepsilon(t) \cap [0;1]) \cap Z = \varnothing$, а тогда $t \neq \inf(Z)$ --- противоречие. Значит $X = [0;1]$.
    \end{proof}

    \section{Топология прямой, пределы и непрерывность.}

    \subsection{Последовательности, пределы и ряды}

    \begin{definition}
        \emph{Предел последовательности} $\{x_n\}_{n=0}^\infty$ --- такое число $x$, что для любой окрестности $x$ эта последовательность с некоторого момента будет лежать в этой окрестности:
        \[\forall \varepsilon > 0\; \exists N \in \NN: \forall n \geqslant N\quad x_n \in U_\varepsilon(x)\]
        Обозначение: $\lim \{x_n\}_{n=0}^\infty = x$.

        \emph{Предельная точка последовательности} $\{x_n\}_{n=0}^\infty$ --- такое число $x$, что в любой его окрестности после любого момента появится элемент данной последовательности:
        \[\forall \varepsilon > 0\, \forall N \in \NN\; \exists n > N: \quad x_n \in U_\varepsilon(x)\]
    \end{definition}

    \begin{definition}
        Последовательность $\{x_n\}_{n=0}^\infty$ называется \emph{фундаментальной}, если
        \[\forall \varepsilon > 0\; \exists N \in \NN:\; \forall n_1, n_2 > N\quad |x_{n_1} - x_{n_2}| < \varepsilon\]
    \end{definition}

    \begin{theorem}\label{fundamental_seq_theorem}
        Последовательность сходится тогда и только тогда, когда фундаментальна.
    \end{theorem}

    \begin{proof}
        \begin{enumerate}
            \item Пусть последовательность $\{x_n\}_{n=0}^\infty$ сходится к некоторому значению $X$, тогда
                \begin{multline*}
                    \forall \varepsilon > 0\; \exists N \in \NN:\; \forall n > N\quad |x_n - X| < \varepsilon/2 \Rightarrow\\
                    \forall n_1, n_2 > N\quad |x_{n_1} - x_{n_2}| = |x_{n_1} - X + X - x_{n_2}| \leqslant |x_{n_1} - X| + |X - x_{n_2}| < \varepsilon
                \end{multline*}
            \item Пусть последовательность $\{x_n\}_{n=0}^\infty$ фундаментальна. Мы знаем, что для каждого $\varepsilon > 0$ все члены, начиная с некоторого различаются менее чем на $\varepsilon$. Тогда возьмём какой-нибудь такой член $y_0$ для некоторого $\varepsilon$, затем какой-нибудь такой член $y_1$ для $\varepsilon/2$, который идёт после $y_0$ и так далее. Получим последовательность, что все члены, начиная с $n$-ого лежат в $\varepsilon/2^n$-окрестности $y_n$. Тогда рассмотрим последовательность $\{I_n\}_{n=0}^\infty$, где $I_n = [y_n - \varepsilon/2^{n-1}; y_n + \varepsilon/2^{n-1}]$. Несложно понять, что $I_n \supseteq I_{n+1}$, поэтому в пересечении $\{I_n\}_{n=0}^\infty$ лежит некоторый $X$. Несложно понять, что все члены начальной последовательности, начиная с $y_{n+2}$, лежат в $\varepsilon/2^{n+2}$-окрестности $y_{n+2}$. При этом $|y_{n+2} - X| \leqslant \varepsilon/2^{n+1}$, что значит, что все члены главной последовательности, начиная с $y_{n+2}$ лежат в $3\varepsilon/2^{n+2}$-окрестности $X$, а значит и в $\varepsilon/2^n$.
        \end{enumerate}
    \end{proof}

    \begin{statement}
        Для последовательностей $\{x_n\}_{n=0}^\infty$ и $\{y_n\}_{n=0}^\infty$ верно (если определено), что
        \begin{enumerate}
            \item $\lim \{x_n\}_{n=0}^\infty + \lim \{y_n\}_{n=0}^\infty = \lim \{x_n + y_n\}_{n=0}^\infty$
            \item $-\lim \{x_n\}_{n=0}^\infty = \lim \{-x_n\}_{n=0}^\infty$
            \item $\lim \{x_n\}_{n=0}^\infty \cdot \lim \{y_n\}_{n=0}^\infty = \lim \{x_n y_n\}_{n=0}^\infty$
            \item $\frac{1}{\lim \{x_n\}_{n=0}^\infty} = \lim \{\frac{1}{x_n}\}_{n=0}^\infty$ (если $\lim \{x_n\}_{n=0}^\infty \neq 0$)
        \end{enumerate}
        и всегда, когда определена левая сторона определена, правая тоже определена.
    \end{statement}

    \begin{proof}
        \begin{enumerate}
            \item Пусть $\lim \{x_n\}_{n=0}^\infty = X$, $\lim \{y_n\}_{n=0}^\infty = Y$. Тогда
                \[\forall \varepsilon > 0\; \exists N, M \in \NN:\quad \forall n > N\; |x_n - X| < \varepsilon/2\quad \wedge\quad \forall m > M\; |y_m - Y| < \varepsilon/2,\]
                тогда
                \[\forall n > \max(N, M)\quad |(x_n + y_n) - (X + Y)| \leqslant |x_n - X| + |y_n - Y| < \varepsilon,\]
                что означает, что $\{x_n + y_n\}_{n=0}^\infty$ сходится и сходится к $X + Y$.
            \item Пусть $\lim \{x_n\}_{n=0}^\infty = X$. Тогда
                \[\forall \varepsilon > 0\; \exists N \in \NN:\quad \forall n > N\; |x_n - X| < \varepsilon,\]
                тогда
                \[\forall n > N\quad |(-x_n) - (-X)| = |X - x_n| = |x_n - X| < \varepsilon,\]
                что означает, что $\{-x_n\}_{n=0}^\infty$ сходится и сходится к $-X$.
            \item Пусть $\lim \{x_n\}_{n=0}^\infty = X$, $\lim \{y_n\}_{n=0}^\infty = Y$. Определим также
                \[\delta: (0; +\infty) \to \RR, \varepsilon \mapsto \frac{\varepsilon}{\sqrt{\left(\frac{|x|+|y|}{2}\right)^2+\varepsilon} + \frac{|x|+|y|}{2}} = \sqrt{\left(\frac{|x|+|y|}{2}\right)^2+\varepsilon} - \frac{|x|+|y|}{2}\]
                Несложно видеть, что $\delta(\varepsilon)$ всегда определено и всегда положительно. Также несложно видеть, что $\delta(\varepsilon)$ есть корень уравнения $t^2 + t(|X| + |Y|) = \varepsilon$. Тогда
                \[\forall \varepsilon > 0\; \exists N, M \in \NN:\quad \forall n > N\; |x_n - X| < \delta(\varepsilon)\quad \wedge\quad \forall m > M\; |y_m - Y| < \delta(\varepsilon),\]
                тогда
                \begin{align*}
                    \forall n > \max(N, M)\quad |x_n \cdot y_n - X \cdot Y|
                    &= |x_n \cdot y_n - x_n \cdot Y + x_n \cdot Y - X \cdot Y|\\
                    &\leqslant |x_n \cdot (y_n-Y)| + |(x_n - X) \cdot Y|\\
                    &< |x_n|\cdot \delta(\varepsilon) + \delta(\varepsilon)\cdot |Y|\\
                    &< (|X|+\delta(\varepsilon)) \cdot \delta(\varepsilon) + |Y| \cdot \delta(\varepsilon)\\
                    &= \delta(\varepsilon)^2 + (|X| + |Y|)\delta(\varepsilon)\\
                    &= \varepsilon,
                \end{align*}
                что означает, что $\{x_n \cdot y_n\}_{n=0}^\infty$ сходится и сходится к $X \cdot Y$.
            \item Пусть $\lim \{x_n\}_{n=0}^\infty = X$. Определим также
                \[\delta: (0; +\infty) \to \RR, \varepsilon \mapsto \frac{\varepsilon |X|}{1 + \varepsilon |X|}\]
                Несложно видеть, что $\delta(\varepsilon)$ всегда определено и всегда меньше $|X|$. Также несложно видеть, что $\delta(\varepsilon)$ есть корень уравнения $\frac{t}{|X|(|X| - t)} = \varepsilon$. Тогда
                \[\forall \varepsilon > 0\; \exists N \in \NN:\quad \forall n > N\; |x_n - X| < \delta(\varepsilon),\]
                тогда
                \[\forall n > N\quad \left|\frac{1}{x_n} - \frac{1}{X}\right| = \left|\frac{X-x_n}{X\cdot x_n}\right| < \frac{\delta(\varepsilon)}{|X| \cdot |x_n|} < \frac{\delta(\varepsilon)}{|X|(|X|-\delta(\varepsilon))} = \varepsilon,\]
                что означает, что $\{\frac{1}{x_n}\}_{n=0}^\infty$ сходится и сходится к $1/X$.
        \end{enumerate}
    \end{proof}

    \begin{definition}
        Последовательность $\{x_n\}_{n=0}^\infty$ \emph{асимптотически больше} последовательности $\{y_n\}_{n=0}^\infty$, если $x_n > y_n$ для всех натуральных $n$, начиная с некоторого. Обозначение: $\{x_n\}_{n=0}^\infty \succ \{y_n\}_{n=0}^\infty$.

        Аналогично определяются \emph{асимптотически меньше} ($\{x_n\}_{n=0}^\infty \prec \{y_n\}_{n=0}^\infty$), \emph{асимптотически не больше} ($\{x_n\}_{n=0}^\infty \preccurlyeq \{y_n\}_{n=0}^\infty$) и \emph{асимптотически не меньше} ($\{x_n\}_{n=0}^\infty \succcurlyeq \{y_n\}_{n=0}^\infty$).
    \end{definition}

    \begin{statement}\label{stupid_seq_statement_1}
        Если $\{x_n\}_{n=0}^\infty \succcurlyeq \{y_n\}_{n=0}^\infty$, то $\lim \{x_n\}_{n=0}^\infty \geqslant \lim \{y_n\}_{n=0}^\infty$.
    \end{statement}

    \begin{proof}
        Предположим противное, т.е. $Y > X$, где $X := \lim \{x_n\}_{n=0}^\infty$, $Y := \lim \{y_n\}_{n=0}^\infty$. Тогда пусть $\varepsilon = \frac{|X - Y|}{2}$. С каких-то моментов $\{x_n\}_{n=0}^\infty$ и $\{y_n\}_{n=0}^\infty$ находятся в $\varepsilon$-окрестностях $X$ и $Y$ соответственно. Тогда начиная с позднего из этих моментов, $y_n > Y - \varepsilon = X + \varepsilon > x_n$, т.е. $\{x_n\}_{n=0}^\infty \prec \{y_n\}_{n=0}^\infty$ --- противоречие. Значит $X \geqslant Y$.
    \end{proof}

    \begin{statement}\label{stupid_seq_statement_2}
        Если $\lim \{x_n\}_{n=0}^\infty > \lim \{y_n\}_{n=0}^\infty$, то $\{x_n\}_{n=0}^\infty \succ \{y_n\}_{n=0}^\infty$.
    \end{statement}

    \begin{proof}
        Пусть $X := \lim \{x_n\}_{n=0}^\infty$, $Y := \lim \{y_n\}_{n=0}^\infty$. Тогда пусть $\varepsilon = \frac{|X - Y|}{2}$. С каких-то моментов $\{x_n\}_{n=0}^\infty$ и $\{y_n\}_{n=0}^\infty$ находятся в $\varepsilon$-окрестностях $X$ и $Y$ соответственно. Тогда начиная с позднего из этих моментов, $x_n > X - \varepsilon = Y + \varepsilon > y_n$, т.е. $\{x_n\}_{n=0}^\infty \succ \{y_n\}_{n=0}^\infty$.
    \end{proof}

    \begin{statement}[леммма о двух полицейских]\label{stupid_seq_statement_3}
        Если
        \[\{x_n\}_{n=0}^\infty \succcurlyeq \{y_n\}_{n=0}^\infty \succcurlyeq \{z_n\}_{n=0}^\infty\]
        и
        \[\lim \{x_n\}_{n=0}^\infty = \lim \{z_n\}_{n=0}^\infty = A,\]
        то предел $\{y_n\}_{n=0}^\infty$ определён и равен $A$.
    \end{statement}

    \begin{proof}
        Для каждого $\varepsilon > 0$ есть $N, M \in \NN$, что
        \[\forall n > N\; |x_n - A| < \varepsilon \quad \wedge \quad \forall m > M\; |z_n - A| < \varepsilon,\]
        значит
        \[\forall n > \max(N, M)\quad A + \varepsilon > x_n \geqslant y_n \geqslant z_n > A - \varepsilon \quad \text{т.е. } |y_n - A| < \varepsilon,\]
        что означает, что $\{y_n\}_{n=0}^\infty$ сходится и сходится к $A$.
    \end{proof}

    \begin{statement}
        Если $\{x_n\}_{n=0}^\infty \succcurlyeq \{y_n\}_{n=0}^\infty$, $\lim \{x_n\}_{n=0}^\infty = A$, а $\{y_n\}_{n=0}^\infty$, не убывает (с некоторого момента), то предел $\{y_n\}_{n=0}^\infty$ существует и не превосходит $A$.
    \end{statement}

    \begin{proof}
        Если последовательность $\{y_n\}_{n=0}^\infty$ возрастает не с самого начала, то отрежем её начало с до момента начала возрастания. Заметим, что она ограничена сверху (из-за последовательности $\{x_n\}_{n=0}^\infty$), тогда определим $B := \sup(\{y_n\}_{n=0}^\infty)$. Тогда $\forall \varepsilon > 0\; \exists N \in \NN:\quad |B-x_N| < \varepsilon$, тогда $\forall n > N\quad |B-x_n| < \varepsilon$, что означает, что $\{y_n\}_{n=0}^\infty$ сходится и сходится к $B$. По утверждению \ref{stupid_seq_statement_1} $A \geqslant B$.
    \end{proof}

    \begin{definition}
        Сумма ряда $\{a_k\}_{k=0}^\infty$ есть значение $\sum_{k=0}^\infty a_k := \lim \left\{\sum_{i=0}^k\right\}_{k=0}^\infty$. Частичной же суммой $s_k$ этого ряда называется просто $\sum_{i=0}^k a_i$.
    \end{definition}

    \begin{definition}
        Ряд $\sum_{i=0}^\infty a_i$ \emph{сильно сходится}, если $\sum_{i=0}^\infty |a_i|$ сходится.
    \end{definition}

    \begin{theorem}
        Если ряд сильно сходится сходится, то он сходится.
    \end{theorem}

    \begin{proof}
        \begin{thlemma}\label{lemma_sum_of_suffix}
            Пусть ряд $\sum_{i=0}^\infty a_i$ сходится, тогда сходится любой его ``хвост'' (суффикс), и для любого $\varepsilon > 0$ есть такой хвост, сумма которого меньше $\varepsilon$.
        \end{thlemma}

        \begin{proof}
            Пусть $A = \sum_{i=0}^\infty a_i$. Это значит, что для каждого $\varepsilon > 0$ существует $N \in \NN$, что для всех $n \geqslant N$ верно, что $\sum_{i=0}^n |a_i| \in U_\varepsilon(A)$. Тогда заметим, что
            \[\sum_{i=N+1}^\infty |a_i| = \lim_{n \to \infty} \sum_{i=N+1}^n |a_i| = \lim_{n \to \infty} \left(\sum_{i=0}^n |a_i| - \sum_{i=0}^N |a_i|\right) = \lim_{n \to \infty} \sum_{i=0}^n |a_i| - \sum_{i=0}^N |a_i| = A - \sum_{i=0}^N |a_i| \in U_\varepsilon(0)\]
            Это и означает, что любой хвост сходится. И так мы для каждого $\varepsilon$ нашли такой хвост, что его сумма меньше $\varepsilon$.
        \end{proof}

        Пусть дан сильно сходящийся ряд $\sum_{i=0}^\infty a_i$. Пусть $\varepsilon_n := \sum_{i=n}^\infty |a_i|$. Несложно видеть, что $\{\varepsilon_n\}_{n=0}^\infty$ монотонно уменьшается, сходясь к 0 (последнее следует из леммы \ref{lemma_sum_of_suffix}). Также несложно видеть по рассуждениям леммы \ref{lemma_sum_of_suffix}, что $\varepsilon_n - \varepsilon_{n+1} = |a_n|$. Тогда определим
        \[S_n := \overline{U}_{\varepsilon_{n+1}}(\sum_{i=0}^n a_i),\]
        где $\overline{U}_\varepsilon(x)$ --- закрытая $\varepsilon$-окрестность точки $x$. Тогда несложно видеть, что
        \[\left|\sum_{i=0}^{n+m} a_i - \sum_{i=0}^{n} a_i \right| = \left|\sum_{i=n+1}^{n+m} a_i \right| \leqslant \sum_{i=n+1}^{n+m} |a_i| \leqslant \varepsilon_{n+1}\]
        Тем самым сумма любого префикса длины хотя бы $n+1$ лежит в $\overline{U}_{\varepsilon_{n+1}}(\sum_{i=0}^{n} a_i) = S_n$. Также несложно видеть, что $S_{n+1} \subseteq S_n$. А также понятно, что $S_i$ замкнуто и ограничено (``компактно'').

        Пусть $A := \bigcap_{i=0}^\infty S_i$ (поскольку диаметры шаров сходятся к нулю, то в пересечении лежит не более одной точки). Тогда мы видим, что $|\sum_{i=0}^n a_i - A| \leqslant \varepsilon_{n+1} \to 0$, поэтому $\sum_{i=0}^n a_i$ сходится и сходится к $A$.
    \end{proof}

    \begin{corollary}
        Если $\{b_i\}_{i=0}^\infty \succcurlyeq \{|a_i|\}_{i=0}^n$ и $\sum_{i=0}^\infty |b_i|$ существует, то и $\sum_{i=0}^\infty a_i$ существует.
    \end{corollary}

    \begin{theorem}[признак Лейбница]
        Пусть дана последовательность $\{a_n\}$, монотонно сверху сходящаяся к $0$. Тогда ряд $\sum_{i=0}^\infty (-1)^i a_i$ сходится.
    \end{theorem}

    \begin{proof}
        Рассмотрим последовательности 
        \begin{align*}
            \{P_n\}_{n=0}^\infty &:= \{S_{2n}\}_{n=0}^\infty = \left\{\sum_{i=0}^{2n} (-1)^i a_i\right\}_{n=0}^\infty&
            \{Q_n\}_{n=0}^\infty &:= \{S_{2n + 1}\}_{n=0}^\infty = \left\{\sum_{i=0}^{2n + 1} (-1)^i a_i\right\}_{n=0}^\infty
        \end{align*}
        Несложно видеть, что 
        \begin{align*}
            P_{n+1} - P_n &= - a_{2n+1} + a_{2n+2} \leqslant 0&
            Q_{n+1} - Q_n &= a_{2n+2} - a_{2n-3} \geqslant 0\\
            Q_{n} - P_{n} &= - a_{2n+1} \leqslant 0&
            P_{n+1} - Q_{n} &= a_{2n+2} \geqslant 0
        \end{align*}
        Тогда имеем, что $\{P_n\}_{n=0}^\infty$ монотонно убывает, $\{Q_n\}_{n=0}^\infty$ монотонно возрастает, а также
        \[\{P_n\}_{n=0}^\infty \geqslant \{Q_n\}_{n=0}^\infty.\]
        Тогда последовательности $\{P_n\}_{n=0}^\infty$ и $\{Q_n\}_{n=0}^\infty$ сходятся и сходятся к $P$ и $Q$ соответственно. При этом последовательность
        \[\{P_n\}_{n=0}^\infty - \{Q_n\}_{n=0}^\infty = \{P_n - Q_n\}_{n=0}^\infty = a_{2n+1}\]
        тоже сходится по условию и сходится к $0$. Поэтому
        \[P - Q = \lim \{P_n\}_{n=0}^\infty - \lim \{Q_n\}_{n=0}^\infty = 0\]
        значит $P=Q$. Значит и последовательность префиксных сумм тоже сходится к $P=Q$.
    \end{proof}

    \begin{lemma}[преобразование Абеля]
        \[\sum_{k=0}^n a_k b_k = \sum_{k=0}^{n-1} (a_k - a_{k+1})B_k + a_n B_n\]
        где $B_n := \sum_{i=0}^n b_i$.
    \end{lemma}

    \begin{theorem}[признак Дирихле]
        Если даны $\{a_i\}_{i=0}^\infty$ и $\{b_i\}_{i=0}^\infty$, что $\{a_i\}_{i=0}^\infty \searrow 0$, а $\{B_n\}_{n=0}^\infty = \{\sum_{i=0}^n b_i\}_{i=0}^\infty$ ограничена, то ряд $\sum_{i=0}^\infty a_i b_i$ сходится.
    \end{theorem}

    \begin{proof}
        \[S_n = \sum_{i=0}^n a_k b_k = \sum_{i=0}^n (a_k - a_{k+1}) B_k + a_n B_n\]
        Пусть $|B_n| < C$ для всех $n$. Несложно видеть, что 
        \[\lim_{n \to \infty} |a_n B_n| \leqslant \lim a_n C = C \lim a_n = 0,\]
        поэтому $\lim a_n B_n = 0$. Также
        \[|(a_k-a_{k+1}) B_k| < C |a_k - a_{k+1}| = C(a_k - a_{k+1}),\]
        поэтому
        \[|S_n - a_n B_n| \leqslant \sum_{k=0}^{n-1} |(a_k - a_{k+1})B_k| < C \sum_{k=0}^{n-1} (a_k - a_{k+1}) = C(a_1 - a_{n+1}),\]
        что тоже сходится. Поэтому $\{S_n\}_{n=0}^\infty$ сходится, т.е. и ряд сходится.
    \end{proof}

    \subsection{Топология}

    \begin{definition}
        \emph{$\varepsilon$-окрестность} точки $x$ (для $\varepsilon > 0$) --- $(x - \varepsilon; x+ \varepsilon)$. Обозначение: $U_\varepsilon(x)$.

        \emph{Проколотая $\varepsilon$-окрестность} точки $x$ --- $(x - \varepsilon; x) \cup (x; x + \varepsilon)$. Обозначение: $V_\varepsilon(x)$.
    \end{definition}

    \begin{definition}
        Пусть дано некоторое множество $X \subseteq \RR$. Тогда точка $x \in X$ называется \emph{внутренней точкой множества} $X$, если она содержится в $X$ вместе со своей окрестностью.
        
        Само множество $X$ называется \emph{открытым}, если все его точки внутренние.
    \end{definition}

    \begin{example}
        Следующие множества открыты:
        \begin{itemize}
            \item $(a; b)$;
            \item $(a; +\infty)$;
            \item $\RR$;
            \item $\varnothing$;
            \item $\bigcup_{i=0}^\infty (a_i; b_i)$ (интервалы не обязательно не должны пересекаться).
        \end{itemize}
    \end{example}

    \begin{definition}
        Пусть дано множество $X\subseteq \RR$. Точка $x \in \RR$ называется \emph{предельной точкой} множества, если в любой проколотой окрестности $x$ будет какая-либо точка $X$.
        
        Множество предельных точек $X$ называется \emph{производным множеством} множества $X$ и обозначается как $X'$.

        Множество $X$ называется замкнутым, если $X \supseteq X'$.
    \end{definition}

    \begin{definition}
        Пусть дано множество $X\subseteq \RR$. Если у любой последовательности его точек есть предельная точка из самого множества $X$, то $X$ называется \emph{компактным}.
    \end{definition}

    \begin{theorem}
        Подмножество $\RR$ компактно тогда и только тогда, когда замкнуто и ограничено.
    \end{theorem}

    \begin{proof}\ 
        \begin{enumerate}
            \item Пусть $X \subseteq \RR$ компактно. Если $X$ неограниченно, то несложно построить последовательность элементов $X$, которая монотонно возрастает или убывает, а разность между членами не меньше любой фиксированной константы (например, не меньше $1$); такая последовательность не имеет предельных точек, что противоречит определению $X$, а значит $X$ ограничено. Если $X$ не замкнуто, то можно рассмотреть предельную точку $x$, не лежащую в $X$, и построить последовательность, сходящуюся к ней, а значит никаких других точек у последовательности быть не может, а значит опять получаем противоречие с определением $X$; значит $X$ ещё и замкнуто.
            \item Пусть $X$ замкнуто и ограничено. Пусть также дана некоторая последовательность $\{x_n\}_{n=0}^\infty$ элементов $X$. Поскольку $X$ ограничено, то значит лежит внутри некоторого отрезка $I_0$. Определим последовательность $\{I_n\}_{n=0}^\infty$ рекуррентно следующим образом. Пусть $I_n$ определено; разделим $I_n$ на две половины и определим $I_{n+1}$ как любую из половин, в которой находится бесконечное количество членов последовательности $\{x_n\}_{n=0}^\infty$. после этого определим последовательность $\{y_n\}_{n=0}^\infty$ как подпоследовательность $\{x_n\}_{n=0}^\infty$, что $y_n \in I_n$ для любого $n \in \NN$ (это можно сделать рекуррентно: если определён член $y_n$, то найдётся ещё бесконечное количество членов начальной последовательности в $I_{n+1}$, которые идут после $y_n$, так как отброшено конечное количество, а значит можно взять любой). Несложно видеть, что $\lim_{n \to \infty} y_n = \bigcap_{n \in \NN} I_n =: y$. Из-за замкнутости $y \in X$, а значит $y$ --- предельная точка $\{x_n\}_{n=0}^\infty$ --- лежит в $X$ и доказывает компактность $X$.
        \end{enumerate}
    \end{proof}

    \begin{lemma}\label{segment_edged_subcover_great_lemma}
        Пусть $\Sigma$ --- семейство интервалов длины больше некоторого $d > 0$, покрывающее отрезок $[a; b]$. Тогда у $\Sigma$ есть конечное подсемейство $\Sigma'$, покрывающее $[a; b]$.
    \end{lemma}
    
    \begin{proof}
        Давайте вести индукцию по $\lceil (b-a)/d \rceil$.

        \textbf{База.} $\lceil (b-a)/d \rceil = 0$. В таком случае $a = b$, а значит, можно взять любой интервал, покрывающий единственную точку и получить всё искомое семейство $\Sigma'$.

        \textbf{Шаг.} Рассмотрим $\Omega := \{I \in \Sigma \mid a \in I\}$. Заметим, что если у правых концов интервалов из $\Omega$ нет верхних граней (т.е. их множество не ограничено сверху), то значит найдётся интервал, покрывающий и $a$, и $b$, а значит его как единственный элемент семейства $\Sigma'$ будет достаточно. Иначе определим $a'$ как супремум правых концов интервалов из $\Omega$.
        
        Тогда мы имеем, что есть интервалы из $\Omega$, подбирающиеся сколь угодно близко к $a'$, а также что все интервалы из $\Sigma$, покрывающие $a'$ не покрывают $a$. Если $a' > b$, то можно опять же взять интервал, который покроет весь $[a; b]$, и остановится. Иначе рассмотрим любой интервал $I$, покрывающий $a'$ и любой интервал $J$ из $\Omega$, перекрывающийся с $I$. Пусть $a''$ --- правый конец $J$.

        Заметим, что $I$ и $J$ покрывают $[a; a'')$. При этом $a < J < a''$, значит $a'' - a \geqslant \osc(J) > d$. Если $a'' > b$, то $\Sigma = \{I, J\}$ будет достаточно. Иначе заметим, что
        \[
            \left\lceil \frac{b-a''}{d} \right\rceil =
            \left\lceil \frac{b-a}{d} - \frac{a''-a}{d} \right\rceil \leqslant
            \left\lceil \frac{b-a}{d} - 1 \right\rceil =
            \left\lceil \frac{b-a}{d} \right\rceil - 1 <
            \left\lceil \frac{b-a}{d} \right\rceil
        \]
        Тогда по предположению индукции есть конечное подпокрытие $\Sigma''$ покрытия $\Sigma$ отрезка $[a''; b]$. Значит $\Sigma' := \Sigma'' \cup \{I, J\}$ является конечным подпокрытием покрытия $\Sigma$ множества $[a; b]$.
    \end{proof}

    \begin{lemma}\label{edged_subcover_great_lemma}
        Пусть $\Sigma$ --- семейство интервалов длины больше некоторого $d > 0$. Тогда найдётся не более чем счётное подсемейство $\Sigma'$, имеющее такое же объединение, т.е. $|\Sigma'| \leqslant |\NN|$, а $\bigcup \Sigma = \bigcup \Sigma'$.
    \end{lemma}

    \begin{proof}
        Несложно видеть, что $A := \bigcup \Sigma$ представляется в виде дизъюнктного объединения интервалов. Каждый из них можно представить как объединение не более чем счётного отрезков. Итого мы получим не более чем счётное семейство $\Omega$ отрезков, что $\bigcup \Omega = A$. Для каждого отрезка из $\Omega$ построим по лемме \ref{segment_edged_subcover_great_lemma} конечное подпокрытие покрытия $\Sigma$, а затем объединив их, получим не более чем счётное семейство $\Sigma'$, покрывающее любой из них, а значит и $\bigcup \Omega = A = \bigcup \Sigma$. С другой стороны $\Sigma'$ --- подмножество $\Sigma$, значит и $\bigcup \Sigma'$ --- подмножество $\bigcup \Sigma$.

        В итоге $\bigcup \Sigma' = \bigcup \Sigma$, и при этом $\Sigma'$ --- не более чем счётное подмножество $\Sigma$.
    \end{proof}

    \begin{lemma}
        Пусть дано семейство $\Sigma$ интервалов. Тогда из него можно выделить не более чем счётное подсемейство $\Sigma'$ с тем же объединением, т.е. $|\Sigma'| \leqslant |\NN|$, а $\bigcup \Sigma = \bigcup \Sigma'$.
    \end{lemma}

    \begin{proof}
        Рассмотрим для каждого $n \in \ZZ$ семейство
        \[\Sigma_n = \{I \in \Sigma \mid \osc(I) \in [2^n; 2^{n+1})\}\]
        Применим лемму к $\Sigma_n$ и получим $\Sigma'_n$. Тогда $\Sigma' := \bigcup_{n \in \ZZ} \Sigma'_n$ является подмножеством $\Sigma$, даёт в объединении то же, что и $\Sigma$, и при этом имеет мощность не более $|\NN \times \NN| = |\NN|$.
    \end{proof}

    \begin{theorem}
        Подмножество $\RR$ компактно тогда и только тогда, когда из любого его покрытия интервалами можно выделить конечное подпокрытие.
    \end{theorem}

    \begin{proof}
        \begin{enumerate}
            \item Пусть $X$ компактно, а $\Sigma$ --- некоторое его покрытие интервалами. Определим для каждого $d > 0$
                \[\Sigma_d := \{I \in \Sigma\mid \osc(I) > d\}\]
                Если никакое из $\Sigma_d$ не является подпокрытием множества $X$, то рассмотрим последовательность $\{x_n\}_{n=0}^\infty$, где $x_n$ --- любой элемент $X \setminus \Sigma_{1/2^n}$. У $\{x_n\}_{n=0}^\infty$ есть предельная точка $x \in X$. Значит должен быть интервал, покрывающий $x$, но тогда он же покрывает весь некоторый хвост нашей последовательности, а сам лежит в некотором $\Sigma_{1/2^n}$ --- противоречие. Значит некоторое $\Sigma_d$ является подпокрытие, а значит далее можно рассматривать его в качестве $\Sigma$.

                $\bigcup \Sigma$ --- открытое множество, поэтому является дизъюнктным объединением семейства $\Omega$ интервалов. Поскольку в $\Sigma$ длины всех интервалов больше $d$, то в $\Omega$ тоже. Но также $X$ ограничено, поэтому $\Omega$ конечно, да и все интервалы в нём ограничены. Заметим, что $X \cap I$, где $I$ --- любой интервал из $\Omega$, является замкнутым множеством, поэтому его можно накрыть некоторым отрезком $S \subseteq I$ (для этого можно взять отрезок $[\inf(X \cap I); \sup(X \cap I)]$). Значит из накрытия $\Sigma$ выделить $|\Omega|$ конечных подпокрытий для каждого отрезка (по лемме \ref{edged_subcover_great_lemma}), а их объединение даст конечное покрытие $X$.

            \item Пусть $X$ таково, что из любого покрытия можно выбрать конечное подпокрытие.
            
                Если $X$ неограничено, то тогда несложно будет видеть, что покрытие $\{(n; n+2) \mid n \in \ZZ\}$ нельзя уменьшить до конечного. Значит $X$ конечно.

                Если $X$ не замкнуто, то значит есть точка $x \notin X$, что в любой окрестности $x$ будет точка. Тогда рассмотрим покрытие $\{(x + 2^n; x^{n+2}) \mid n \in \ZZ\} \cup \{(x - 2^{n+2}; x^n) \mid n \in \ZZ\}$. Несложно видеть, что если взять любое конечное подсемейство интервалов, то оно не накроет некоторую окрестность $x$, а значит и $X$. Значит $X$ замкнуто.

                Итого получаем, что $X$ компактно.
        \end{enumerate}
    \end{proof}

    \subsection{Пределы функций, непрерывность}

    \begin{definition}[по Коши]
        \emph{Предел} функции $f: X \to \RR$ в точке $x$ --- такое значение $y$, что
        \[\forall \varepsilon > 0\, \exists \delta > 0: f(V_\delta(x) \cap X) = U_\varepsilon(y)\]
        Обозначение: $\lim\limits_{t \to x} f(t) = y$.
    \end{definition}

    \begin{definition}[по Гейне]
        \emph{Предел} функции $f: X \to \RR$ в точке $x$ --- такое значение $y$, что для любой последовательность $\{x_n\}_{n=0}^\infty$ элементов $X \setminus \{x\}$ последовательность $\{f(x_n)\}_{n=0}^\infty$ сходится к $y$. Обозначение: $\lim\limits_{t \to x} f(t) = y$.
    \end{definition}

    \begin{theorem}
        Определения пределов по Коши и по Гейне равносильны.
    \end{theorem}

    \begin{proof}
        Будем доказывать равносильность отрицаний утверждений, ставимых в определениях.
        \begin{enumerate}
            \item Пусть функция $f: X \to \RR$ не сходится по Коши в $x$ к значению $y$. Значит есть такое $\varepsilon > 0$, что в любой проколотой окрестности $x$ (в множестве $X$) есть точка, значение $f$ в которой не лежит в $\varepsilon$-окрестности. Рассмотрев любую такую проколотую окрестность $I_0 = V_{\delta_0}(x)$, берём в ней любую такую точку $x_0$. Далее рассмотрев $I_1 = V_{\delta_1}(x)$, где $\delta_1 = \min(\delta_0/2, |x-x_0|)$, берём там любую точку $x_1$, где значение $f$ вылетает вне $\varepsilon$-окрестности $y$. Так далее строим последовательность $\{x_n\}_{n=0}^\infty$, сходящуюся к $x$, значения $f$ в которой не лежат в $\varepsilon$-окрестности $y$, что означает, что $\{f(x_n)\}_{n=0}^\infty$ не сходится к $y$, что означает, что $f$ не сходится по Гейне в $x$ к значению $y$.
            \item Пусть функция $f: X \to \RR$ не сходится по Гейне в $x$ к значению $y$. Значит есть последовательность $\{x_n\}_{n=0}^\infty$, сходящаяся к $x$, что последовательность её значений не сходится к $y$. Значит есть $\varepsilon > 0$, что после любого момента в последовательности будет член, значение в котором вылезает вне $\varepsilon$-окрестности $y$. Поскольку для любой проколотой окрестности $x$ есть момент, начиная с которого вся последовательность лежит в этой окрестности, то в любой проколотой окрестности $x$ есть член, значение которого вылезает вне $\varepsilon$-окрестности $y$, что означает, что $f$ не сходится по Коши в $x$ к $y$.
        \end{enumerate}
    \end{proof}

    \begin{statement}
        Функция $f: X \to \RR$ имеет в $x$ предел тогда и только тогда, когда
        \[\forall \varepsilon > 0\; \exists \delta > 0:\; \forall x_1, x_2 \in V_\delta(x)\quad |f(x_1) - f(x_2)| < \varepsilon\]
    \end{statement}

    \begin{proof}
        Такое же как для последовательностей: см. теорему \ref{fundamental_seq_theorem}.
    \end{proof}

    \begin{statement}
        Для функций $f: \RR \to \RR$ и $g: \RR \to \RR$ верно, что
        \begin{enumerate}
            \item $\lim\limits_{x \to a} f(x) + \lim\limits_{x \to a} g(x) = \lim\limits_{x \to a} (f + g)(x)$
            \item $\lim\limits_{x \to a} (-f)(x) = -\lim\limits_{x \to a} f(x)$
            \item $\lim\limits_{x \to a} f(x) \cdot \lim\limits_{x \to a} g(x) = \lim (f \cdot g)(x)$
            \item $\frac{1}{\lim\limits_{x \to a} f(x)} = \lim\limits_{x \to a} (\frac{1}{f})(x)$ (если $\lim\limits_{x \to a} f(x) \neq 0$)
            \item $\lim\limits_{y \to \lim\limits_{x \to a} g(x)} f(y) = \lim\limits_{x \to a} (f \circ g)(x)$
        \end{enumerate}
        и всегда, когда определена левая сторона определена, правая тоже определена.
    \end{statement}

    \begin{remark}
        Утверждения \ref{stupid_seq_statement_1}, \ref{stupid_seq_statement_2} и \ref{stupid_seq_statement_3} верны, если заменить последовательности на функции, пределы последовательностей на пределы функций в некоторой точке $x$, а асимптотические неравенства на неравенства на окрестности $x$.
    \end{remark}

    \begin{definition}
        \emph{Верхним пределом} функции $f$ в точке $x_0$ называется
        \[\varlimsup\limits_{x \to x_0} f(x) = \inf_{\delta > 0} (\sup_{V_\delta(x_0)} f)\]
        \emph{Нижним пределом} функции $f$ в точке $x_0$ называется
        \[\varliminf\limits_{x \to x_0} f(x) = \sup_{\delta > 0} (\inf_{V_\delta(x_0)} f)\]
    \end{definition}

    \begin{statement}
        Функция $f: X \to \RR$ имеет в $x$ предел тогда и только тогда, когда $\varlimsup\limits_{t \to x} f(t) = \varliminf\limits_{t \to x} f(t)$.
    \end{statement}

    \begin{definition}
        Функция $f: X \to \RR$ называется \emph{непрерывной в точке} $x$, если $\lim\limits_{t \to x} f(t) = f(x)$. В изолированных точках $f$ всегда непрерывна.
    \end{definition}

    \begin{definition}
        Функция $f: X \to \RR$ называется \emph{непрерывной на множестве} $Y \subseteq X$, если она непрерывна во всех точках $Y$.
    \end{definition}

    \begin{statement}
        Для непрерывных на $X$ функций $f$ и $g$ верно, что
        \begin{itemize}
            \item $f+g$ непрерывна на $X$;
            \item $fg$ непрерывна на $X$;
            \item $\frac{1}{f}$ непрерывна на $X$ (если $f \neq 0$).
        \end{itemize}
    \end{statement}

    \begin{statement}
        Для $f$, непрерывной в $x_0$, и $g$, непрерывной в $f(x_0)$, $g\circ f$ непрерывна в $x_0$.
    \end{statement}

    \begin{theorem}[Вейерштрасса]
        Непрерывная функция на компакте ограничена на нём и принимает на нём свои минимум и максимум.
    \end{theorem}

    \begin{proof}
        Докажем утверждение для ограниченности сверху и максимума; для ограниченности снизу и минимума рассуждения аналогичны.

        Пусть множество неограниченно сверху. Тогда есть $\{x_n\}_{n=0}^\infty$, что $\{f(x_n)\}_{n=0}^\infty \to +\infty$. Тогда рассмотрим подпоследовательность $\{y_n\}_{n=0}^\infty$ последовательности $\{x_n\}_{n=0}^\infty$, сходящуюся к $y$. Тогда
        \[f(y) = \lim_{n \to \infty}\limits f(y_n) = +\infty\]
        --- противоречие.

        Тогда существует последовательность $\{x_n\}_{n=0}^\infty$, что $\{f(x_n)\}_{n=0}^\infty$ сходится к супремуму $S$ функции. Рассмотрим подпоследовательность $\{y_n\}_{n=0}^\infty$ последовательности $\{x_n\}_{n=0}^\infty$, сходящуюся к $y$. Тогда
        \[f(y) = \lim_{n \to \infty} f(y_n) = S\]
    \end{proof}

    \begin{corollary}
        Так как отрезок компактен, то любая непрерывная на нём функция ограничена и принимает на нём свои максимум и минимум.
    \end{corollary}

    \begin{theorem}[о промежуточном значении]
        Пусть $f$ непрерывна на $[a; b]$, а $f(a) < f(b)$. Тогда $\forall y \in [f(a); f(b)]$ найдётся $c \in [a; b]$, что $f(c) = y$.
    \end{theorem}

    \begin{proof}
        Рассмотрим последовательность $\{(a_n; b_n)\}_{n=0}^\infty$, что $(a; b) = (a_0; b_0)$, а следующие пары определяются так: если $f(\frac{a_n+b_n}{2}) < y$, то $(a_{n+1}; b_{n+1}) = (\frac{a_n+b_n}{2}; b_n)$, иначе $(a_{n+1}; b_{n+1}) = (a_n; \frac{a_n+b_n}{2})$. Тогда $c = \lim \{a_n\}_{n=0}^\infty = \lim \{b_n\}_{n=0}^\infty$. Тогда
        \[f(c) = \lim \{f(a_n)\}_{n=0}^\infty = \lim \{f(b_n)\}_{n=0}^\infty,\]
        откуда получаем, что $f(c) \geqslant y$ и $f(c) \leqslant y$, т.е. $f(c) = y$.
    \end{proof}

    \begin{definition}
        Функция $f$ \emph{равномерно непрерывна} на $X$, если
        \[\forall \varepsilon > 0\; \exists \delta > 0:\; \forall x \in X\quad f(U_\delta(x)) \subseteq U_\varepsilon(f(x))\]
    \end{definition}

    \begin{theorem}[Кантор]
        Непрерывная на компакте функция равномерно непрерывна.
    \end{theorem}

    \begin{proof}
        Предположим противное. Тогда
        \[\exists \varepsilon > 0: \forall \delta > 0\; \exists x, y:\quad |x - y| < \delta \wedge |f(x) - f(y)| > \varepsilon\]
        Тогда рассмотрим последовательность пар $x$ и $y$ построенных так для $\delta$, сходящихся к $0$. Из неё выделим подпоследовательность, что $x$ сходится к некоторому $a$. Тогда $y$ сойдутся к нему же. Тогда в любой окрестности $a$ будет пара точек $(x'; y')$, что $|f(x') - f(y')| > \varepsilon$, значит будет в любой окрестности $x$ будет точка, выбивающаяся из $\varepsilon/2$-окрестности --- противоречие с непрерывностью.
    \end{proof}

    \begin{definition}
        Пусть есть функции $f$ и $g$, что $|f| \leqslant C|g|$ в окрестности $x$ для некоторого $C \in \RR$, тогда пишут, что $f = O(g)$ (при $t \to x$).
        
        Если же $\forall \varepsilon > 0$ будет такая окрестность $x_0$, что $|f| \leqslant \varepsilon |g|$ в этой окрестности, тогда пишут, что $f = o(g)$ (при $t \to x$).
    \end{definition}

    \subsection{Гладкость (дифференцируемость)}

    \begin{definition}\label{def_derivative_1}
        Функция $f$ называется \emph{гладкой (дифференцируемой)} в $x$, если $f(x + \delta) = f(x) + A \delta + o(\delta)$ для некоторого $A \in \RR$. В таком случае $A$ называется \emph{дифференциалом (производной)} $f$ в точке $x$.

        Обозначение: $f'(x) = A$.
    \end{definition}

    \begin{definition}\label{def_derivative_2}
        Функция $f$ называется \emph{гладкой (дифференцируемой)} в $x$, если предел
        \[\lim\limits_{\delta \to 0} \frac{f(x+\delta) - f(x)}{\delta}\]
        определён. В таком случае его значение называется \emph{дифференциалом (производной)} $f$ в точке $x$.
    \end{definition}

    \begin{statement}
        Определения \ref{def_derivative_1} и \ref{def_derivative_2} равносильны.
    \end{statement}

    \begin{statement}
        Непрерывная в некоторой точке функция там же непрерывна.
    \end{statement}

    \begin{definition}  
        Функция, значения которой равны производным функции $f$ в тех же точках называется \emph{производной функцией} (или просто \emph{производной}) функции $f$. Обозначение: $f'$.
    \end{definition}

    \begin{lemma}
        Для дифференцируемых в $x$ функций $f$ и $g$
        \begin{enumerate}
            \item $(f \pm g)'(x) = f'(x) \pm g'(x)$;
            \item $(f \cdot g)'(x) = f'(x) g(x) + f(x) g'(x)$ (правило Лейбница);
            \item $(\frac{1}{f})'(x) = \frac{-f'(x)}{f(x)^2}$;
            \item $(f \circ g)'(x) = f'(g(x))\cdot g'(x)$.
        \end{enumerate}
    \end{lemma}

    \begin{lemma}
        Пусть дана $f: [a; b] \to \RR$ --- непрерывная монотонно возрастающая (убывающая) функция. Тогда существует $g: [f(a); f(b)] \to \RR$ --- непрерывная монотонно возрастающая (убывающая) функция, что $g \circ f = Id$.
    \end{lemma}

    \begin{proof}
        Заметим, что $f$ --- монотонно возрастающая (убывающая) биекция из $[a; b]$ в $[f(a); f(b)]$. Тогда существует монотонно возрастающая (убывающая) биекция $g: [f(a); f(b)] \to [a; b]$, что $g \circ f = id$. Осталось показать, что $g$ непрерывна.

        Предположим противное, тогда в любой окрестности некоторой точки $f(x)$ из $[f(a); f(b)]$ есть точки вылетающие вне $\varepsilon$-окрестности. Значит все точки из либо $(x - \varepsilon; x)$, либо $(x; x + \varepsilon)$ не принимаются, значит $g$ не биекция --- противоречие. Значит $g$ непрерывна.
    \end{proof}

    \begin{lemma}
        \[(f^{-1})' = \frac{1}{f' \circ f^{-1}}\]
    \end{lemma}

    \begin{proof}
        Пусть $g := f^{-1}$. Тогда
        \[1 = Id' = (f \circ g)' = f' \circ g \cdot g'\]
        Откуда следует, что
        \[(f^{-1})' = g' = \frac{1}{f' \circ g} = \frac{1}{f' \circ f^{-1}}\]
    \end{proof}

    \begin{definition}
        Функция $f$ \emph{возрастает в точке $y$}, если есть $\varepsilon > 0$, что $f(x) \leqslant f(y)$ для любого $x \in (y-\varepsilon; y)$ и $f(x) \geqslant f(y)$ для любого $x \in (y; y+\varepsilon)$.

        Аналогично определяется убываемость функции в точке.
    \end{definition}

    \begin{lemma}
        Если $f$ возрастает в любой точке на $[a;b]$, то $f(a) \leqslant f(b)$.
    \end{lemma}

    \begin{proof}
        \begin{enumerate}\ 
            \item Можно рассмотреть для каждой точки $[a; b]$ окрестность, для которой верна её возрастаемость, и из покрытия, ими образуемого, выделить конечное. А тогда перебираясь между общими точками окрестностей, получим искомое.
            \item Также можно предположить противное, рассмотреть последовательность вложенных отрезков, у которых левый конец выше правого, и тогда для точки пересечения отрезков будет противоречие.
        \end{enumerate}
    \end{proof}

    \begin{corollary}
        $f$ возрастает на всём отрезке.
    \end{corollary}

    \begin{theorem}
        Если $f$ гладка, а $f'$ положительна на $[a; b]$, то $f$ строго возрастает на $[a; b]$.
    \end{theorem}

    \begin{proof}
        Несложно видеть, что в любой точке на $[a; b]$ у функции есть окрестность, где она строго возрастает, так как если $t \in [a; b]$, а $f'(t) = \lambda > 0$, то в некоторой окрестности
        \begin{align*}
            \frac{f(x) - f(t)}{x - t} &\in (0; 2\lambda)&
            &\Longrightarrow&
            f(x) \in (f(t); f(t) + 2\lambda(x-t))
        \end{align*}
        что значит, что эта окрестность --- подтверждение для возрастания $f$ в $t$. Тогда по предыдущему следствию $f$ возрастает на $[a; b]$. Если вдруг функция возрастает нестрого, то тогда найдётся подотрезок на $[a;b]$, на котором функция константа, а значит на интервале с теми же концами производная тождественна равна нулю.
    \end{proof}

    \begin{theorem}
        Если $f$ возрастает, то $f'$ в своей области определения неотрицательно.
    \end{theorem}

    \begin{proof}
        Если функция в точке $t$ равна $\lambda < 0$, то в некоторой окрестности $t$
        \begin{align*}
            \frac{f(x)-f(t)}{x-t} &\in \left(\frac{3}{2}\lambda; \frac{1}{2}\lambda\right)&
            &\Longrightarrow&
            f(x) &\in \left(f(t) + \frac{3}{2}\lambda(x-t); f(t) + \frac{1}{2}\lambda(x-t)\right)
        \end{align*}
        что значит, что $f$ в точке $t$ "строго" убывает --- противоречие. Значит $f'(t) \geqslant 0$.
    \end{proof}

    \begin{definition}
        \emph{$f$ имеет локальный максимум в $x$}, если для некоторого $\varepsilon > 0$ верно, что $f(x) \geqslant f(y)$ для любого $y \in (x - \varepsilon; x + \varepsilon)$.

        Аналогично определяется точка локального минимума.
    \end{definition}

    \begin{theorem}
        В точках локальных максимумов и минимумов функции $f$ функция $f'$ принимает нули (если определена).
    \end{theorem}

    \begin{proof}
        Слева от точки максимума функция возрастает в данной точке, значит производная в данной точке $\geqslant 0$, а справа --- убывает, значит производная $\leqslant 0$, значит производная равна $0$. Аналогично для точки минимума.
    \end{proof}

    \begin{theorem}[Ролль]
        Если $f$ --- гладкая функция на $[a; b]$, и $f(a) = f(b)$, то существует $c \in (a; b)$, что $f'(c) = 0$.
    \end{theorem}

    \begin{proof}
        В точке максимума или минимума $f$ на $[a;b]$ достигается ноль производной. Если они обе совпадают с концами отрезка, то значит функция константа, а тогда в любой точке отрезка производная равна нулю.
    \end{proof}

    \begin{theorem}
        Если $f$ и $g$ непрерывные на $[a; b]$ и гладкие на $(a; b)$ функции, а $g' \neq 0$, то существует $c \in (a; b)$, что
        \[\frac{f(a) - f(b)}{g(a) - g(b)} = \frac{f'(c)}{g'(c)}\]
    \end{theorem}

    \begin{proof}
        Пусть
        \[\lambda := \frac{f(a) - f(b)}{g(a) - g(b)}\]
        а $\tau(x) := f(x) - \lambda g(x)$. В таком случае
        \[\frac{\tau(a) - \tau(b)}{g(a) - g(b)} = \frac{(f(a) - f(b) - \lambda (g(a) - g(b)))}{g(a) - g(b)} = \lambda - \lambda = 0\]
        значит $\tau(a) = \tau(b)$, значит есть $c \in [a; b]$, что $\tau(c) = 0$. Тогда
        \[\frac{f'(c)}{g'(c)} = \frac{(\tau + \lambda g)'(c)}{g(c)} = \frac{\tau'(c)}{g(c)} + \lambda = \lambda = \frac{f(a) - f(b)}{g(a) - g(b)}\]
    \end{proof}

    \begin{theorem}[Лагранж]
        Если $f$ непрерывна на $[a; b]$ и гладка на $(a; b)$, то существует $c \in (a; b)$, что
        \[\frac{f(a) - f(b)}{a - b} = f'(c)\]
    \end{theorem}

    \begin{proof}
        Очевидно следует из предыдущей теоремы с помощью подстановки $g(x) = x$.
    \end{proof}

    \begin{theorem}
        Пусть $f$ --- гладкая на $(a; b)$ функция.
        \begin{enumerate}
            \item Если $f' \geqslant 0$, то $f$ возрастающая функция.
            \item Если $f' > 0$, то $f$ строго возрастающая функция.
            \item Если $f$ возрастающая функция, то $f' > 0$.
        \end{enumerate}
    \end{theorem}

    \begin{theorem}
        Пусть $f$ --- гладкая на $[a; b]$ функция. Если $f'(x) = 0$ для всех $x \in [a; b]$, то $f \equiv const$ на том же отрезке.
    \end{theorem}

    \begin{remark}
        Функция $f(x) := x^2 \sin(1/x)$ (доопределённая в нуле) имеет производную $f'(x) = 2x\sin(1/x) - \cos(1/x)$ в случае ненулевых $x$ и производную $f'(0) = 0$. При этом легко видно, что $f'$ не является непрерывной функцией (она имеет разрыв в том же нуле).
    \end{remark}

    \begin{theorem}
        Если $f$ гладка на $(a; b)$, а $f'$ не равна нулю, то $f'$ либо положительна, либо отрицательна.
    \end{theorem}

    \begin{proof}
        $f$ не принимает никакое значение на $(a; b)$ дважды (т.к. иначе у производной был бы корень), значит она либо строго возрастает, либо строго убывает, а значит $f'$ либо неотрицательна, либо неположительна соответственно. Но ноль принимать не может, поэтому последнее утверждение равносильно тому, что $f$ либо строго положительна, либо строго отрицательна.
    \end{proof}

    \begin{theorem}
        Пусть $f$ гладка на $(a; b)$ и для некоторых $u, v \in (a; b)$ верно, что $f'(u) < \alpha < f'(v)$. Тогда существует $c \in (u; v)$, что $f'(c) = \alpha$.
    \end{theorem}

    \begin{proof}
        Пусть $g(x) := f(x) - \alpha x$. Тогда $g'(u) < 0 < g'(v)$, значит $g$ не может строго возрастать или убывать на $(u; v)$, значит $\exists c \in (u; v)$, что $g'(c) = 0$, а значит $f'(c)=\alpha$.
    \end{proof}

    \begin{remark}
        Данная теорема по сути является теоремой о промежуточном значении для производной.
    \end{remark}

    \begin{theorem}
        Пусть $f$ непрерывна на $[a; b)$ и гладка на $(a; b)$. Пусть также $\lim_{x \to a^+} f'(x)$ существует и равен $d$. Тогда $f'(a)$ тоже существует и равна $d$.
    \end{theorem}

    \begin{proof}
        Есть несколько способов:
        \begin{enumerate}
            \item Несложно видеть, что для любого $\varepsilon > 0$ есть некоторая правая окрестность $a$, в которой функция $f'$ лежит в $\varepsilon$-окрестности $d$. Тогда $f(x) - (d-\varepsilon)x$ убывает в данной окрестности, а $f(x) - (d+\varepsilon)x$ возрастает, значит $f(x) - f(a) \in ((d + \varepsilon)(x-a); (d-\varepsilon)(x-a))$. В таком случае $f'(a)$ определена и равна $d$.
            \item По теореме Лагранжа для любого $x \in [a; b)$ найдётся $\xi \in (a; x)$, что
                \[\frac{f(x) - f(a)}{x - a} = f'(\xi)\]
                Значит
                \[\lim_{x \to a^+} \frac{f(x) - f(a)}{x - a} = \lim_{x \to a^+} f'(\xi) = d\]
                что буквально значит, что $f'(a) = d$.
        \end{enumerate}
    \end{proof}

    \begin{theorem}[правило Лопиталя]
        Пусть $\lim_{x \to a^+} f(x) = \lim_{x \to a^+} g(x) = 0$. Пусть также $f$ и $g$ гладки и $g' \neq 0$ на $(a; b)$. Тогда
        \[\lim_{x \to a^+} \frac{f(x)}{g(x)} = \lim_{x \to a^+} \frac{f'(x)}{g'(x)}\]
        если второй предел определён.
    \end{theorem}

    \begin{proof}
        Пусть дано $\varepsilon > 0$, а
        \[d := \lim_{x \to a^+} \frac{f'(x)}{g'(x)}\]. Тогда есть $\delta > 0$, что для любого $t \in (a; a + \delta)$ значение $f'(t)/g'(t)$ лежит в $U_\varepsilon(d)$. Легко видеть, что для любых $x, y \in (a; a + \delta)$ существует $\xi \in (x; y) \subseteq (a; a + \delta)$, что
        \[\frac{f(x) - f(y)}{g(x) - g(y)} = f'(\xi) \in U_\varepsilon(d)\]
        Устремляя $x$ к $a$, получаем, что $f(y)/g(y)$ тоже лежит в $U_\varepsilon(d)$. Тогда по определению предела
        \[\lim_{x \to a^+} \frac{f(x)}{g(x)} = d = \lim_{x \to a^+} \frac{f'(x)}{g'(x)}\]
    \end{proof}

    \begin{definition}
        $f''$ --- вторая производная $f$, т.е. $(f')'$, а $f^{(n)}$ --- $n$-ая производная $f$, т.е. $f^{(n)} := (f^{(n-1)})'$, $f^{(0)} := f$.
    \end{definition}

    \begin{definition}
        $P(x)$ --- полином Тейлора степени $n$ функции $f$, если $\deg(P) \leqslant n$, а
        \[f(x) - P(x) = o((x-a)^n),\quad x \to a\]
    \end{definition}

    \begin{theorem}
        Если $P_1$ и $P_2$ --- полиномы Тейлора степени $n$ функции $f$, то $P_1 = P_2$.
    \end{theorem}

    \begin{theorem}
        Пусть $f: (a; b) \to \RR$, $f^{(1)}$, \dots, $f^{(n-1)}$ определены на $(t-\delta; t+\delta)$ для некоторого $\delta > 0$ и определена $f^{(n)}(t)$. Тогда
        \[f(x) = f(t) + \frac{f^{(1)}(t)}{1!}(x-t) + \dots + \frac{f^{(n)}(t)}{n!}(x-t)^n + o((x-t)^n)\]
    \end{theorem}

    \begin{proof}
        Рассмотрим $g(x) := f(x) - f(t)/0! \cdot (x-t)^0 - \dots - f^{(n)}(t)/n! \cdot (x-t)^n$. Тогда задача сведена к следующей лемме.

        \begin{thlemma}
            Если $g^{(1)}$, \dots, $g^{(n-1)}$ определены на $(t-\delta; t+\delta)$ для некоторого $\delta > 0$ и
            \[g(t) = g^{(1)}(t) = \dots = g^{(n)}(t) = 0.\]
            Тогда $g(x) = o((x-t)^n)$.
        \end{thlemma}

        \begin{proof}
            Докажем по индукции по $n$.

            \textbf{База.} Пусть $n = 1$. Тогда очевидно, что $f(x) = f(t) + f'(t)(x-t) + o(x-t) = o(x-t)$.

            \textbf{Шаг.} По предположению индукции $f'(x) = o((x-t)^n)$. Тогда мы имеем, что
            \[f(x) = f(x) - f(t) = f'(\xi) (x - t)\]
            для некоторого $\xi \in (x, t)$. Тогда
            \[\frac{f(x) - f(t)}{(x-t)^n} = \frac{f'(\xi)}{(x-t)^{n-1}} = \frac{o((\xi-t)^{n-1})}{(x-t)^{n-1}} = o(1) \frac{(\xi - t)^{n-1}}{(x-t)^{n-1}} = o(1)\]
        \end{proof}
    \end{proof}


    \begin{theorem}
        Пусть $f(t) = f^{(1)}(t) = \dots = f^{(n)}(t) = 0$, а $f^{(n+1)} \neq 0$. Если $n$ чётно, то $t$ --- не экстремальные точка функции $f$, иначе $t$ --- экстремальная точка функции $f$.
    \end{theorem}

    \begin{theorem}
        Пусть $f: (a; b) \to \RR$, $f^{(1)}$, \dots, $f^{(n+1)}$ определены на $(t-\delta; t+\delta)$ для некоторого $\delta > 0$. Тогда существует $\xi \in (x; t)$, что
        \[f(x) = f(t) + \frac{f^{(1)}(t)}{1!}(x-t) + \dots + \frac{f^{(n)}(t)}{n!}(x-t)^n + \frac{f^{(n+1)}(\xi)}{(n+1)!}(x-t)^{n+1}\]
    \end{theorem}

    \begin{proof}
        Точно так же сведём $f$ к $g$, что $g(t) = \dots g^{(n)}(t) = 0$. Тогда требуется показать, что $g(x) = g^{(n+1)}(\xi)/(n+1)! \cdot (x-t)^{n+1}$ для некоторого $\xi \in (x, t)$. Докажем это по индукции.

        \textbf{База.} $n=0$. Теорема Лагранжа.

        \textbf{Шаг.}
        \[\frac{f(x)}{(x-t)^{n+1}} = \frac{f(x) - f(t)}{(x-t)^{n+1} - (t-t)^{n+1}} = \frac{f'(\xi)}{(n+1)(\xi - t)^{n}} = \frac{f^{(n+1)}(\eta)}{(n+1)!}\]
        где $\xi \in (x, t)$ (существует по теореме Лагранжа), а $\eta \in (\xi, t) \subseteq (x, t)$ (существует по предположению индукции для $f'$ и $\xi$). Отсюда следует искомое утверждение.
    \end{proof}

    \subsection{Стандартные функции, ряды Тейлора и их сходимость}

    \todo[inline]{Тут нужно рассказать про функции $\exp$, $\sin$, $\cos$ и $(1+x)^\alpha$ и их ряды}

    \begin{definition}
        $f$ является \emph{(поточечным) пределом} $\{f_n\}_{n=0}^\infty$ на $E$, если $\lim \{f_n(x)\}_{n=0}^\infty = f(x)$ для любого $x \in E$.
    \end{definition}

    \begin{definition}
        $f$ является \emph{равномерным пределом} $\{f_n\}_{n=0}^\infty$ на $E$, если для любого $\varepsilon > 0$ найдётся $N \in \NN$, что $|f_n(x) - f(x)| < \varepsilon$ для всех $n > N$ и $x \in E$.
    \end{definition}

    \begin{theorem}[Стокс, Зейдель]
        Пусть $\{f_n\}_{n=0}^\infty$ --- последовательность непрерывных функций, и $f_n \to f$ равномерно на $E$. Тогда $f$ непрерывна.
    \end{theorem}

    \begin{proof}
        Для любого $\varepsilon > 0$ есть такое $n \in \NN$, что $|f_n(x) - f(x)| < \varepsilon/3$ для всех $x \in E$. Тогда существует $\delta > 0$, что $f_n(U_\delta(t)) \subseteq U_{\varepsilon/3}(f_n(t))$ для данного $t$. Тогда
        \[f(U_\delta(t)) \subseteq U_{\varepsilon/3}(f_n(U_\delta(t))) \subseteq U_{2\varepsilon/3}(f_n(t)) \subseteq U_\varepsilon(f(t)).\]
    \end{proof}

    \begin{theorem}[Коши]
        TFAE (the following are equivalent):
        \begin{enumerate}
            \item $f_n \to f$ равномерно сходится на $E$.
            \item Для любого $\varepsilon > 0$ существует $N \in \NN$, что $|f_k(x) - f_l(x)| < \varepsilon$ для любых $k, l > N$ и $x \in E$.
        \end{enumerate}
    \end{theorem}

    \begin{theorem}[Вейерштрасс]
        Пусть $\{u_n\}_{n=0}^\infty$ --- последовательность непрерывных функций, что есть последовательность чисел $\{d_n\}_{n=0}^\infty$, для которой верно, что $|u_n| < d_n$ для всех $n \in \NN$, и $\sum_{n=0}^\infty d_n$ сходится. Тогда $\sum_{n=0}^\infty u_n$ равномерно сходится.
    \end{theorem}

    \begin{theorem}
        Пусть $f_n \to f$ на $E$ и $\{f_n\}_{n=0}^\infty$ гладкие. Если $f_n' \to g$ равномерно, то $f$ тогда тоже гладка и $f' = g$.
    \end{theorem}

    \begin{proof}
        Для любого $\varepsilon > 0$ существует $N \in \NN$, что $|f'_k - f'_l| < \varepsilon/3$ для всех $k, l > N$. Тогда имеем, что
        \[\left|\frac{f_k(x) - f_k(y)}{x - y} - \frac{f_l(x) - f_l(y)}{x - y}\right| = \left| \frac{(f_k - f_l)(x) - (f_k - f_l)(y)}{x - y} \right| = |(f_k - f_l)'(\xi)| < \varepsilon/3 \]
        Устремляя $l$ к бесконечности получаем, что
        \[\left|\frac{f_k(x) - f_k(y)}{x - y} - \frac{f(x) - f(y)}{x - y}\right| \leqslant \varepsilon/3\]
        Также имеем, что есть такое $\delta > 0$, что для всех $y \in U_\delta(x)$
        \[\left|\frac{f_k(x) - f_k(y)}{x - y} - f'_k(x)\right| < \varepsilon/3\]
        Также есть $M \in \NN$, что $|f'_k - g| < \varepsilon/3$ для любого $k > M$. Складывая всё вместе, получаем, что для всех $k > \max(N, M)$ и $y \in U_\delta(x)$
        \[\left|\frac{f(x) - f(y)}{x - y} - g(x)\right| < \varepsilon\]
        Значит $f$ гладка и $f' = g$.
    \end{proof}

    \begin{corollary}
        Если $\{f^{(0)}\}$, \dots, $\{f^{(n-1)}\}$ сходятся, а $f^{(n)}$ равномерно сходится. Тогда то же верно и про первые $n$ производных.
    \end{corollary}

    \begin{corollary}
        Если ряд Тейлора сходится, то функция бесконечно гладкая.
    \end{corollary}
\end{document}