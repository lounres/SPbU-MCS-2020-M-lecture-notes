\documentclass[12pt,a4paper]{article}
\usepackage{math-text}
\usepackage{todonotes}
\usepackage{multicol}

\title{Математический анализ --- 1.}
\author{Лектор --- \href{https://vk.com/ybelov}{Юрий Сергеевич Белов} \\
        Создатель конспекта --- Глеб Минаев
        \footnote{Оригинал конспекта расположен на \href{https://github.com/lounres/SPbU-MCS-2020-M-lecture-notes/blob/master/\%D0\%9C\%D0\%B0\%D1\%82\%D0\%90\%D0\%BD/MA.pdf}{GitHub}. Также на GitHub доступен \href{https://github.com/lounres/SPbU-MCS-2020-M-lecture-notes}{репозиторий} с другими конспектами.}}
\date{}

\DeclareMathOperator{\Quot}{Quot}
\DeclareMathOperator*{\osc}{osc}
\DeclareMathOperator{\sign}{sign}
\DeclareMathOperator{\const}{const}

\begin{document}
    \maketitle

    \listoftodos[TODOs]

    \tableofcontents

    \vspace{2em}
    Литература:
    \begin{itemize}
        \item В. А. Зорич ``Математический анализ''
        \item О. Л. Виноградов ``Математический анализ''
        \item (подходит попозже) Г. М. Фихтенгельц ``Курс дифференциального и интегрального исчисления''
        \item У. Рудин ``Основы анализа''
        \item М. Спивак ``Математический анализ на многообразиях''
        \item В. М. Тихомиров ``Рассказы о максимумах и минимумах''
    \end{itemize}

    \section{Множества, аксиоматика и вещественные числа.}

    Мы начинаем с теории множеств.

    \begin{definition}\ 
        \begin{itemize}
            \item Множества и элементы --- понятно.
            \item $a \in B$ --- понятно.
            \item $A \cup B := \{x \mid x\in A \vee x\in B\}$ --- объединение.
            \item $A \cap B := \{x \mid x\in A \wedge x\in B\}$ --- пересечение.
            \item $A \setminus B := \{x \mid x\in A \vee x\notin B\}$ --- разность.
            \item $A \bigtriangleup B := A \setminus B \cup B \setminus A$ --- симметрическая разница.
            \item $A^C := X\setminus A$ --- \emph{дополнение}, где $X$ --- некоторое фиксированное рассматриваемое множество.
            \item $A \subset B$ --- ``$A$ --- подмножество $B$'', т.е. $\forall x (X\in A \Rightarrow x\in B)$.
        \end{itemize}
    \end{definition}

    \begin{corollary*}\ 
        \begin{itemize}
            \item (первое правило Моргана) $(A\cup B)^C = A^C \cap B^C$.
                \begin{align*}
                    x\in (A\cup B)^C \Leftrightarrow
                    x \notin A \cup B \Leftrightarrow
                    \left\{ \begin{aligned}
                        &x \notin A\\
                        &x \notin B
                    \end{aligned}\right. \Leftrightarrow
                    \left\{ \begin{aligned}
                        &x \in A^c\\
                        &x \in B^C
                    \end{aligned} \right. \Leftrightarrow
                    x \in A^C \cap B^C
                \end{align*}
            \item (второе правило Моргана) $(A\cap B)^C = A^C \cup B^C$. Аналогично.
        \end{itemize}
    \end{corollary*}

    \begin{definition}
        (Аксиома индукции.) Пусть есть функция $A: \NN \to {true;false}$, что:
        \begin{enumerate}
            \item $A(1)=true$;
            \item $\forall n (A(n) \rightarrow A(n+1))$.
        \end{enumerate}
        Тогда $\forall n A(n)$.
    \end{definition}

    Определение натуральных чисел сложно, рассматривать его не будем. Важно также иметь в виду натуральные числа с операциями сложения и умножения.

    \begin{definition}
        Пусть есть кольцо без делителей нуля $R$. Рассмотрим отношение эквивалентности $\sim$ на $R \times (R\setminus \{0\})$, что $(a; b) \sim (c; d) \Leftrightarrow ad = bc$. Тогда $\Quot(R)$ --- фактор-множество по $\sim$ и поле.
    \end{definition}

    \begin{definition}
        Рациональные числа --- $\QQ := \Quot(\ZZ)$.
    \end{definition}

    \begin{theorem}
        $\nexists x\in \QQ, x^2 = 2$.
    \end{theorem}

    \begin{proof}
        Предположим противное, т.е. существуют взаимно простые $m \in \ZZ$ и $n \in \NN\setminus\{0\}$, что $(\frac{m}{n})^2 = 2$. Тогда $m^2 = 2n^2$. Очевидно, что тогда $m^2 \divided 2$, значит $m\divided 2$, значит $m\divided 4$, значит $n^2 \divided 2$, значит $n \divided 2$, значит $n$ и $m$ не взаимно просты, так как делятся на $2$ --- противоречие.
    \end{proof}

    Теперь мы хотим понять, что есть вещественные числа. Тут есть несколько подходов.

    \begin{definition}[аксиоматический подход]
        Вещественные числа --- это полное упорядоченное поле $\RR$, состоящее не из одного элемента.
        
        Здесь ``поле'' значит, что на множестве (вместе с его операциями и выделенными элементами) верны аксиомы поля $A_1$, $A_2$, $A_3$, $A_4$, $M_1$, $M_2$, $M_3$, $M_4$ и $D$ (т.е. сложение и умножение ассоциативны, коммутативны имеют нейтральные элементы и удовлетворяют условию существованию обратных (по умножению --- для всех кроме нуля), а также дистрибутивности).
        
        Упорядоченность значит, что есть рефлексивное транзитивное антисимметричное отношение $\preccurlyeq$, что все элементы сравнимы, согласованное с операциями, т.е.:
        \begin{itemize}
            \item[$A$)] $a \preccurlyeq b \Rightarrow a + x \preccurlyeq b + x$.
            \item[$M$)] $0 \preccurlyeq a \wedge 0 \preccurlyeq b \Rightarrow 0 \preccurlyeq ab$.
        \end{itemize}

        Полнота поля значит любое из следующих утверждений (они равносильны):
        \begin{itemize}
            \item любое ограниченное сверху (снизу) подмножество поля имеет точную верхнюю (нижнюю) грань;
            \item (аксиома Кантора-Дедекинда) для любых двух множеств $A$ и $B$, что $A \preccurlyeq B$, есть разделяющий их элемент.
        \end{itemize}

        Итого мы имеем 9 аксиом поля, 2 аксиомы упорядоченности и 1 аксиома полноты упорядоченности.
    \end{definition}

    \begin{statement*}
        Над $\QQ$ нет  элемента разделяющего $A := \{a > 0 \mid a^2 < 2\}$ и $B := \{b > 0 \mid b^2 > 2\}$.
    \end{statement*}

    \begin{proof}
        Предположим противное, т.е. есть $c > 0$, что $A < c < B$.

        Если $c^2 < 2$, то найдём $\varepsilon$, что $\varepsilon \in (0; 1)$ и $(c + \varepsilon)^2 < 2$. Заметим, что $(c + \varepsilon)^2 = c^2 + 2c\varepsilon + \varepsilon^2 < c^2 + (2c + 1)\varepsilon$. Пусть $\varepsilon < \frac{2 - c^2}{2c+ 1}$, тогда такое $\varepsilon$ точно подойдёт, ну а поскольку $\frac{2 - c^2}{2c + 1} > 0$, то такое $\varepsilon$ есть. Значит $c^2 \geqslant 2$.
        
        Аналогично имеем, что $\varepsilon \leqslant 2$. А значит $c^2 = 2$, что не бывает над $\QQ$.
    \end{proof}

    \begin{corollary*}
        $\QQ$ не полно.
    \end{corollary*}
    
    \begin{definition}
        Значение $t$ является \emph{верхней (нижней) гранью} непустого множества $X \in \RR$ тогда и только тогда, когда $t \geqslant X$, т.е. любой элемент $x$ множества $X$ не более $t$.
        
        
        \emph{Точная верхняя (нижняя) грань} или \emph{супремум (инфимум)} непустого множества $X \subseteq \RR$ --- минимальная верхняя (нижняя) грань множества $X$. Он же является элементом разделяющим $X$ и множество всех его верхних (нижних) граней. Обозначение: $\sup(X)$ и $\inf(X)$ соответственно.

        \emph{Осцелляцией} множества $X$ называется значение $\osc X := \sup X - \inf X$.
    \end{definition}

    \begin{definition}\ 
        \begin{itemize}
            \item \emph{Закрытый интервал} или \emph{отрезок} $[a;b]:=\{x\in\RR \mid a \leqslant x \leqslant b\}$.
            \item \emph{Открытый интервал} или просто \emph{интервал} $(a;b):=\{x\in\RR \mid a < x < b\}$.
            \item \emph{Полуоткрытый интервал} или \emph{полуинтервал} $(a;b] := \{x\in\RR \mid a < x \leqslant b\}$, $[a;b):=\{x\in\RR \mid a \leqslant x < b\}$.
        \end{itemize}
    \end{definition}

    \begin{theorem}[Лемма о вложенных отрезках]\label{th_inter_segments}
        Пусть имеется $\{I_i\}_{i=1}^\infty$ --- множество вложенных (непустых) отрезков, т.е. $\forall n > 1\; I_{n+1} \subset I_n$. Тогда $\bigcap_{i=1}^\infty I_i \neq \varnothing$.
    \end{theorem}

    \begin{proof}
        Заметим, что для любых натуральных $n < m$ верно, что $a_n \leqslant a_m \leqslant b_m \leqslant b_n$, где $I_n = [a_n;b_n]$. Тогда для $A:=\{a_i\}_{i=1}^\infty$ и $B:=\{b_i\}_{i=1}^\infty$ верно, что $A \leqslant B$. Значит есть разделяющий их элемент $t$, значит $A \leqslant t \leqslant B$, значит $t\in I_i$ для всех $i$, значит $t \in \bigcap_{i=1}^\infty I_i$.
    \end{proof}

    \begin{remark}
        Теорема \ref{th_inter_segments} не верна для не отрезков.
    \end{remark}

    \begin{remark}
        Если в теореме \ref{th_inter_segments} $b_i-a_i$ ``сходится к 0'', т.е. $\forall \varepsilon > 0\, \exists n\in\NN: \forall i > n\, b_i-a_i < \varepsilon$, то пересечение всех отрезков состоит из ровно одного элемента.
    \end{remark}

    \begin{theorem}[индукция на вещественных числах]
        Пусть дано множество $X \subseteq [0;1]$, что
        \begin{enumerate}
            \item $0 \in X$;
            \item $\forall x \in X\; \exists \varepsilon > 0: U_\varepsilon(x) \cap [0;1] \subseteq X$;
            \item $\forall Y \subseteq X\; \sup(Y) \in X$.
        \end{enumerate}
        Тогда $X = [0;1]$.
    \end{theorem}

    \begin{proof}
        Предположим противное: $X \neq [0;1]$. Рассмотрим $Z := [0;1] \setminus X$ ($Z \neq \varnothing$!) и $Y := \{y \in [0;1] \mid y < Z\}$ ($Y \neq \varnothing$!). Заметим, что $Y \subseteq X$ и $\sup(Y) = \inf(Z) = t$. Тогда $t \in X$ по второму условию. Значит для некоторого $\varepsilon > 0$ верно, что $U_{\varepsilon}(t) \cap [0;1] \in X$, а т.е. $(U_\varepsilon(t) \cap [0;1]) \cap Z = \varnothing$, а тогда $t \neq \inf(Z)$ --- противоречие. Значит $X = [0;1]$.
    \end{proof}

    \section{Топология прямой, пределы и непрерывность.}

    \subsection{Последовательности, пределы и ряды}

    \begin{definition}
        \emph{Предел последовательности} $\{x_n\}_{n=0}^\infty$ --- такое число $x$, что для любой окрестности $x$ эта последовательность с некоторого момента будет лежать в этой окрестности:
        \[\forall \varepsilon > 0\; \exists N \in \NN: \forall n \geqslant N\quad x_n \in U_\varepsilon(x)\]
        Обозначение: $\lim \{x_n\}_{n=0}^\infty = x$.

        \emph{Предельная точка последовательности} $\{x_n\}_{n=0}^\infty$ --- такое число $x$, что в любой его окрестности после любого момента появится элемент данной последовательности:
        \[\forall \varepsilon > 0\, \forall N \in \NN\; \exists n > N: \quad x_n \in U_\varepsilon(x)\]
    \end{definition}

    \begin{definition}
        Последовательность $\{x_n\}_{n=0}^\infty$ называется \emph{фундаментальной}, если
        \[\forall \varepsilon > 0\; \exists N \in \NN:\; \forall n_1, n_2 > N\quad |x_{n_1} - x_{n_2}| < \varepsilon\]
    \end{definition}

    \begin{theorem}\label{fundamental_seq_theorem}
        Последовательность сходится тогда и только тогда, когда фундаментальна.
    \end{theorem}

    \begin{proof}
        \begin{enumerate}
            \item Пусть последовательность $\{x_n\}_{n=0}^\infty$ сходится к некоторому значению $X$, тогда
                \begin{multline*}
                    \forall \varepsilon > 0\; \exists N \in \NN:\; \forall n > N\quad |x_n - X| < \varepsilon/2 \Rightarrow\\
                    \forall n_1, n_2 > N\quad |x_{n_1} - x_{n_2}| = |x_{n_1} - X + X - x_{n_2}| \leqslant |x_{n_1} - X| + |X - x_{n_2}| < \varepsilon
                \end{multline*}
            \item Пусть последовательность $\{x_n\}_{n=0}^\infty$ фундаментальна. Мы знаем, что для каждого $\varepsilon > 0$ все члены, начиная с некоторого различаются менее чем на $\varepsilon$. Тогда возьмём какой-нибудь такой член $y_0$ для некоторого $\varepsilon$, затем какой-нибудь такой член $y_1$ для $\varepsilon/2$, который идёт после $y_0$ и так далее. Получим последовательность, что все члены, начиная с $n$-ого лежат в $\varepsilon/2^n$-окрестности $y_n$. Тогда рассмотрим последовательность $\{I_n\}_{n=0}^\infty$, где $I_n = [y_n - \varepsilon/2^{n-1}; y_n + \varepsilon/2^{n-1}]$. Несложно понять, что $I_n \supseteq I_{n+1}$, поэтому в пересечении $\{I_n\}_{n=0}^\infty$ лежит некоторый $X$. Несложно понять, что все члены начальной последовательности, начиная с $y_{n+2}$, лежат в $\varepsilon/2^{n+2}$-окрестности $y_{n+2}$. При этом $|y_{n+2} - X| \leqslant \varepsilon/2^{n+1}$, что значит, что все члены главной последовательности, начиная с $y_{n+2}$ лежат в $3\varepsilon/2^{n+2}$-окрестности $X$, а значит и в $\varepsilon/2^n$.
        \end{enumerate}
    \end{proof}

    \begin{statement}
        Для последовательностей $\{x_n\}_{n=0}^\infty$ и $\{y_n\}_{n=0}^\infty$ верно (если определено), что
        \begin{enumerate}
            \item $\lim \{x_n\}_{n=0}^\infty + \lim \{y_n\}_{n=0}^\infty = \lim \{x_n + y_n\}_{n=0}^\infty$
            \item $-\lim \{x_n\}_{n=0}^\infty = \lim \{-x_n\}_{n=0}^\infty$
            \item $\lim \{x_n\}_{n=0}^\infty \cdot \lim \{y_n\}_{n=0}^\infty = \lim \{x_n y_n\}_{n=0}^\infty$
            \item $\frac{1}{\lim \{x_n\}_{n=0}^\infty} = \lim \{\frac{1}{x_n}\}_{n=0}^\infty$ (если $\lim \{x_n\}_{n=0}^\infty \neq 0$)
        \end{enumerate}
        и всегда, когда определена левая сторона определена, правая тоже определена.
    \end{statement}

    \begin{proof}
        \begin{enumerate}
            \item Пусть $\lim \{x_n\}_{n=0}^\infty = X$, $\lim \{y_n\}_{n=0}^\infty = Y$. Тогда
                \[\forall \varepsilon > 0\; \exists N, M \in \NN:\quad \forall n > N\; |x_n - X| < \varepsilon/2\quad \wedge\quad \forall m > M\; |y_m - Y| < \varepsilon/2,\]
                тогда
                \[\forall n > \max(N, M)\quad |(x_n + y_n) - (X + Y)| \leqslant |x_n - X| + |y_n - Y| < \varepsilon,\]
                что означает, что $\{x_n + y_n\}_{n=0}^\infty$ сходится и сходится к $X + Y$.
            \item Пусть $\lim \{x_n\}_{n=0}^\infty = X$. Тогда
                \[\forall \varepsilon > 0\; \exists N \in \NN:\quad \forall n > N\; |x_n - X| < \varepsilon,\]
                тогда
                \[\forall n > N\quad |(-x_n) - (-X)| = |X - x_n| = |x_n - X| < \varepsilon,\]
                что означает, что $\{-x_n\}_{n=0}^\infty$ сходится и сходится к $-X$.
            \item Пусть $\lim \{x_n\}_{n=0}^\infty = X$, $\lim \{y_n\}_{n=0}^\infty = Y$. Определим также
                \[\delta: (0; +\infty) \to \RR, \varepsilon \mapsto \frac{\varepsilon}{\sqrt{\left(\frac{|x|+|y|}{2}\right)^2+\varepsilon} + \frac{|x|+|y|}{2}} = \sqrt{\left(\frac{|x|+|y|}{2}\right)^2+\varepsilon} - \frac{|x|+|y|}{2}\]
                Несложно видеть, что $\delta(\varepsilon)$ всегда определено и всегда положительно. Также несложно видеть, что $\delta(\varepsilon)$ есть корень уравнения $t^2 + t(|X| + |Y|) = \varepsilon$. Тогда
                \[\forall \varepsilon > 0\; \exists N, M \in \NN:\quad \forall n > N\; |x_n - X| < \delta(\varepsilon)\quad \wedge\quad \forall m > M\; |y_m - Y| < \delta(\varepsilon),\]
                тогда
                \begin{align*}
                    \forall n > \max(N, M)\quad |x_n \cdot y_n - X \cdot Y|
                    &= |x_n \cdot y_n - x_n \cdot Y + x_n \cdot Y - X \cdot Y|\\
                    &\leqslant |x_n \cdot (y_n-Y)| + |(x_n - X) \cdot Y|\\
                    &< |x_n|\cdot \delta(\varepsilon) + \delta(\varepsilon)\cdot |Y|\\
                    &< (|X|+\delta(\varepsilon)) \cdot \delta(\varepsilon) + |Y| \cdot \delta(\varepsilon)\\
                    &= \delta(\varepsilon)^2 + (|X| + |Y|)\delta(\varepsilon)\\
                    &= \varepsilon,
                \end{align*}
                что означает, что $\{x_n \cdot y_n\}_{n=0}^\infty$ сходится и сходится к $X \cdot Y$.
            \item Пусть $\lim \{x_n\}_{n=0}^\infty = X$. Определим также
                \[\delta: (0; +\infty) \to \RR, \varepsilon \mapsto \frac{\varepsilon |X|^2}{1 + \varepsilon |X|}\]
                Несложно видеть, что $\delta(\varepsilon)$ всегда определено и всегда меньше $|X|$. Также несложно видеть, что $\delta(\varepsilon)$ есть корень уравнения $\frac{t}{|X|(|X| - t)} = \varepsilon$. Тогда
                \[\forall \varepsilon > 0\; \exists N \in \NN:\quad \forall n > N\; |x_n - X| < \delta(\varepsilon),\]
                тогда
                \[\forall n > N\quad \left|\frac{1}{x_n} - \frac{1}{X}\right| = \left|\frac{X-x_n}{X\cdot x_n}\right| < \frac{\delta(\varepsilon)}{|X| \cdot |x_n|} < \frac{\delta(\varepsilon)}{|X|(|X|-\delta(\varepsilon))} = \varepsilon,\]
                что означает, что $\{\frac{1}{x_n}\}_{n=0}^\infty$ сходится и сходится к $1/X$.
        \end{enumerate}
    \end{proof}

    \begin{definition}
        Последовательность $\{x_n\}_{n=0}^\infty$ \emph{асимптотически больше} последовательности $\{y_n\}_{n=0}^\infty$, если $x_n > y_n$ для всех натуральных $n$, начиная с некоторого. Обозначение: $\{x_n\}_{n=0}^\infty \succ \{y_n\}_{n=0}^\infty$.

        Аналогично определяются \emph{асимптотически меньше} ($\{x_n\}_{n=0}^\infty \prec \{y_n\}_{n=0}^\infty$), \emph{асимптотически не больше} ($\{x_n\}_{n=0}^\infty \preccurlyeq \{y_n\}_{n=0}^\infty$) и \emph{асимптотически не меньше} ($\{x_n\}_{n=0}^\infty \succcurlyeq \{y_n\}_{n=0}^\infty$).
    \end{definition}

    \begin{statement}\label{stupid_seq_statement_1}
        Если $\{x_n\}_{n=0}^\infty \succcurlyeq \{y_n\}_{n=0}^\infty$, то $\lim \{x_n\}_{n=0}^\infty \geqslant \lim \{y_n\}_{n=0}^\infty$.
    \end{statement}

    \begin{proof}
        Предположим противное, т.е. $Y > X$, где $X := \lim \{x_n\}_{n=0}^\infty$, $Y := \lim \{y_n\}_{n=0}^\infty$. Тогда пусть $\varepsilon = \frac{|X - Y|}{2}$. С каких-то моментов $\{x_n\}_{n=0}^\infty$ и $\{y_n\}_{n=0}^\infty$ находятся в $\varepsilon$-окрестностях $X$ и $Y$ соответственно. Тогда начиная с позднего из этих моментов, $y_n > Y - \varepsilon = X + \varepsilon > x_n$, т.е. $\{x_n\}_{n=0}^\infty \prec \{y_n\}_{n=0}^\infty$ --- противоречие. Значит $X \geqslant Y$.
    \end{proof}

    \begin{statement}\label{stupid_seq_statement_2}
        Если $\lim \{x_n\}_{n=0}^\infty > \lim \{y_n\}_{n=0}^\infty$, то $\{x_n\}_{n=0}^\infty \succ \{y_n\}_{n=0}^\infty$.
    \end{statement}

    \begin{proof}
        Пусть $X := \lim \{x_n\}_{n=0}^\infty$, $Y := \lim \{y_n\}_{n=0}^\infty$. Тогда пусть $\varepsilon = \frac{|X - Y|}{2}$. С каких-то моментов $\{x_n\}_{n=0}^\infty$ и $\{y_n\}_{n=0}^\infty$ находятся в $\varepsilon$-окрестностях $X$ и $Y$ соответственно. Тогда начиная с позднего из этих моментов, $x_n > X - \varepsilon = Y + \varepsilon > y_n$, т.е. $\{x_n\}_{n=0}^\infty \succ \{y_n\}_{n=0}^\infty$.
    \end{proof}

    \begin{statement}[леммма о двух полицейских]\label{stupid_seq_statement_3}
        Если
        \[\{x_n\}_{n=0}^\infty \succcurlyeq \{y_n\}_{n=0}^\infty \succcurlyeq \{z_n\}_{n=0}^\infty\]
        и
        \[\lim \{x_n\}_{n=0}^\infty = \lim \{z_n\}_{n=0}^\infty = A,\]
        то предел $\{y_n\}_{n=0}^\infty$ определён и равен $A$.
    \end{statement}

    \begin{proof}
        Для каждого $\varepsilon > 0$ есть $N, M \in \NN$, что
        \[\forall n > N\; |x_n - A| < \varepsilon \quad \wedge \quad \forall m > M\; |z_n - A| < \varepsilon,\]
        значит
        \[\forall n > \max(N, M)\quad A + \varepsilon > x_n \geqslant y_n \geqslant z_n > A - \varepsilon \quad \text{т.е. } |y_n - A| < \varepsilon,\]
        что означает, что $\{y_n\}_{n=0}^\infty$ сходится и сходится к $A$.
    \end{proof}

    \begin{statement}
        Если $\{x_n\}_{n=0}^\infty \succcurlyeq \{y_n\}_{n=0}^\infty$, $\lim \{x_n\}_{n=0}^\infty = A$, а $\{y_n\}_{n=0}^\infty$, не убывает (с некоторого момента), то предел $\{y_n\}_{n=0}^\infty$ существует и не превосходит $A$.
    \end{statement}

    \begin{proof}
        Если последовательность $\{y_n\}_{n=0}^\infty$ возрастает не с самого начала, то отрежем её начало с до момента начала возрастания. Заметим, что она ограничена сверху (из-за последовательности $\{x_n\}_{n=0}^\infty$), тогда определим $B := \sup(\{y_n\}_{n=0}^\infty)$. Тогда $\forall \varepsilon > 0\; \exists N \in \NN:\quad |B-x_N| < \varepsilon$, тогда $\forall n > N\quad |B-x_n| < \varepsilon$, что означает, что $\{y_n\}_{n=0}^\infty$ сходится и сходится к $B$. По утверждению \ref{stupid_seq_statement_1} $A \geqslant B$.
    \end{proof}

    \begin{definition}
        Сумма ряда $\{a_k\}_{k=0}^\infty$ есть значение $\sum_{k=0}^\infty a_k := \lim \left\{\sum_{i=0}^k\right\}_{k=0}^\infty$. Частичной же суммой $s_k$ этого ряда называется просто $\sum_{i=0}^k a_i$.
    \end{definition}

    \begin{definition}
        Ряд $\sum_{i=0}^\infty a_i$ \emph{сильно сходится}, если $\sum_{i=0}^\infty |a_i|$ сходится.
    \end{definition}

    \begin{theorem}
        Если ряд сильно сходится сходится, то он сходится.
    \end{theorem}

    \begin{proof}
        \begin{thlemma}\label{lemma_sum_of_suffix}
            Пусть ряд $\sum_{i=0}^\infty a_i$ сходится, тогда сходится любой его ``хвост'' (суффикс), и для любого $\varepsilon > 0$ есть такой хвост, сумма которого меньше $\varepsilon$.
        \end{thlemma}

        \begin{proof}
            Пусть $A = \sum_{i=0}^\infty a_i$. Это значит, что для каждого $\varepsilon > 0$ существует $N \in \NN$, что для всех $n \geqslant N$ верно, что $\sum_{i=0}^n |a_i| \in U_\varepsilon(A)$. Тогда заметим, что
            \[\sum_{i=N+1}^\infty |a_i| = \lim_{n \to \infty} \sum_{i=N+1}^n |a_i| = \lim_{n \to \infty} \left(\sum_{i=0}^n |a_i| - \sum_{i=0}^N |a_i|\right) = \lim_{n \to \infty} \sum_{i=0}^n |a_i| - \sum_{i=0}^N |a_i| = A - \sum_{i=0}^N |a_i| \in U_\varepsilon(0)\]
            Это и означает, что любой хвост сходится. И так мы для каждого $\varepsilon$ нашли такой хвост, что его сумма меньше $\varepsilon$.
        \end{proof}

        Пусть дан сильно сходящийся ряд $\sum_{i=0}^\infty a_i$. Пусть $\varepsilon_n := \sum_{i=n}^\infty |a_i|$. Несложно видеть, что $\{\varepsilon_n\}_{n=0}^\infty$ монотонно уменьшается, сходясь к 0 (последнее следует из леммы \ref{lemma_sum_of_suffix}). Также несложно видеть по рассуждениям леммы \ref{lemma_sum_of_suffix}, что $\varepsilon_n - \varepsilon_{n+1} = |a_n|$. Тогда определим
        \[S_n := \overline{U}_{\varepsilon_{n+1}}(\sum_{i=0}^n a_i),\]
        где $\overline{U}_\varepsilon(x)$ --- закрытая $\varepsilon$-окрестность точки $x$. Тогда несложно видеть, что
        \[\left|\sum_{i=0}^{n+m} a_i - \sum_{i=0}^{n} a_i \right| = \left|\sum_{i=n+1}^{n+m} a_i \right| \leqslant \sum_{i=n+1}^{n+m} |a_i| \leqslant \varepsilon_{n+1}\]
        Тем самым сумма любого префикса длины хотя бы $n+1$ лежит в $\overline{U}_{\varepsilon_{n+1}}(\sum_{i=0}^{n} a_i) = S_n$. Также несложно видеть, что $S_{n+1} \subseteq S_n$. А также понятно, что $S_i$ замкнуто и ограничено (``компактно'').

        Пусть $A := \bigcap_{i=0}^\infty S_i$ (поскольку диаметры шаров сходятся к нулю, то в пересечении лежит не более одной точки). Тогда мы видим, что $|\sum_{i=0}^n a_i - A| \leqslant \varepsilon_{n+1} \to 0$, поэтому $\sum_{i=0}^n a_i$ сходится и сходится к $A$.
    \end{proof}

    \begin{corollary}
        Если $\{b_i\}_{i=0}^\infty \succcurlyeq \{|a_i|\}_{i=0}^n$ и $\sum_{i=0}^\infty |b_i|$ существует, то и $\sum_{i=0}^\infty a_i$ существует.
    \end{corollary}

    \begin{theorem}[признак Лейбница]
        Пусть дана последовательность $\{a_n\}$, монотонно сверху сходящаяся к $0$. Тогда ряд $\sum_{i=0}^\infty (-1)^i a_i$ сходится.
    \end{theorem}

    \begin{proof}
        Рассмотрим последовательности 
        \begin{align*}
            \{P_n\}_{n=0}^\infty &:= \{S_{2n}\}_{n=0}^\infty = \left\{\sum_{i=0}^{2n} (-1)^i a_i\right\}_{n=0}^\infty&
            \{Q_n\}_{n=0}^\infty &:= \{S_{2n + 1}\}_{n=0}^\infty = \left\{\sum_{i=0}^{2n + 1} (-1)^i a_i\right\}_{n=0}^\infty
        \end{align*}
        Несложно видеть, что 
        \begin{align*}
            P_{n+1} - P_n &= - a_{2n+1} + a_{2n+2} \leqslant 0&
            Q_{n+1} - Q_n &= a_{2n+2} - a_{2n-3} \geqslant 0\\
            Q_{n} - P_{n} &= - a_{2n+1} \leqslant 0&
            P_{n+1} - Q_{n} &= a_{2n+2} \geqslant 0
        \end{align*}
        Тогда имеем, что $\{P_n\}_{n=0}^\infty$ монотонно убывает, $\{Q_n\}_{n=0}^\infty$ монотонно возрастает, а также
        \[\{P_n\}_{n=0}^\infty \geqslant \{Q_n\}_{n=0}^\infty.\]
        Тогда последовательности $\{P_n\}_{n=0}^\infty$ и $\{Q_n\}_{n=0}^\infty$ сходятся и сходятся к $P$ и $Q$ соответственно. При этом последовательность
        \[\{P_n\}_{n=0}^\infty - \{Q_n\}_{n=0}^\infty = \{P_n - Q_n\}_{n=0}^\infty = a_{2n+1}\]
        тоже сходится по условию и сходится к $0$. Поэтому
        \[P - Q = \lim \{P_n\}_{n=0}^\infty - \lim \{Q_n\}_{n=0}^\infty = 0\]
        значит $P=Q$. Значит и последовательность префиксных сумм тоже сходится к $P=Q$.
    \end{proof}

    \begin{lemma}[преобразование Абеля]
        \[\sum_{k=0}^n a_k b_k = \sum_{k=0}^{n-1} (a_k - a_{k+1})B_k + a_n B_n\]
        где $B_n := \sum_{i=0}^n b_i$.
    \end{lemma}

    \begin{theorem}[признак Дирихле]
        Если даны $\{a_i\}_{i=0}^\infty$ и $\{b_i\}_{i=0}^\infty$, что $\{a_i\}_{i=0}^\infty \searrow 0$, а $\{B_n\}_{n=0}^\infty = \{\sum_{i=0}^n b_i\}_{i=0}^\infty$ ограничена, то ряд $\sum_{i=0}^\infty a_i b_i$ сходится.
    \end{theorem}

    \begin{proof}
        \[S_n = \sum_{i=0}^n a_k b_k = \sum_{i=0}^n (a_k - a_{k+1}) B_k + a_n B_n\]
        Пусть $|B_n| < C$ для всех $n$. Несложно видеть, что 
        \[\lim_{n \to \infty} |a_n B_n| \leqslant \lim a_n C = C \lim a_n = 0,\]
        поэтому $\lim a_n B_n = 0$. Также
        \[|(a_k-a_{k+1}) B_k| < C |a_k - a_{k+1}| = C(a_k - a_{k+1}),\]
        поэтому
        \[|S_n - a_n B_n| \leqslant \sum_{k=0}^{n-1} |(a_k - a_{k+1})B_k| < C \sum_{k=0}^{n-1} (a_k - a_{k+1}) = C(a_1 - a_{n+1}),\]
        что тоже сходится. Поэтому $\{S_n\}_{n=0}^\infty$ сходится, т.е. и ряд сходится.
    \end{proof}

    \subsection{Топология}

    \begin{definition}
        \emph{$\varepsilon$-окрестность} точки $x$ (для $\varepsilon > 0$) --- $(x - \varepsilon; x+ \varepsilon)$. Обозначение: $U_\varepsilon(x)$.

        \emph{Проколотая $\varepsilon$-окрестность} точки $x$ --- $(x - \varepsilon; x) \cup (x; x + \varepsilon)$. Обозначение: $V_\varepsilon(x)$.
    \end{definition}

    \begin{definition}
        Пусть дано некоторое множество $X \subseteq \RR$. Тогда точка $x \in X$ называется \emph{внутренней точкой множества} $X$, если она содержится в $X$ вместе со своей окрестностью.
        
        Само множество $X$ называется \emph{открытым}, если все его точки внутренние.
    \end{definition}

    \begin{example}
        Следующие множества открыты:
        \begin{itemize}
            \item $(a; b)$;
            \item $(a; +\infty)$;
            \item $\RR$;
            \item $\varnothing$;
            \item $\bigcup_{i=0}^\infty (a_i; b_i)$ (интервалы не обязательно не должны пересекаться).
        \end{itemize}
    \end{example}

    \begin{definition}
        Пусть дано множество $X\subseteq \RR$. Точка $x \in \RR$ называется \emph{предельной точкой} множества, если в любой проколотой окрестности $x$ будет какая-либо точка $X$.
        
        Множество предельных точек $X$ называется \emph{производным множеством} множества $X$ и обозначается как $X'$.

        Множество $X$ называется замкнутым, если $X \supseteq X'$.
    \end{definition}

    \begin{definition}
        Пусть дано множество $X\subseteq \RR$. Если у любой последовательности его точек есть предельная точка из самого множества $X$, то $X$ называется \emph{компактным}.
    \end{definition}

    \begin{theorem}
        Подмножество $\RR$ компактно тогда и только тогда, когда замкнуто и ограничено.
    \end{theorem}

    \begin{proof}\ 
        \begin{enumerate}
            \item Пусть $X \subseteq \RR$ компактно. Если $X$ неограниченно, то несложно построить последовательность элементов $X$, которая монотонно возрастает или убывает, а разность между членами не меньше любой фиксированной константы (например, не меньше $1$); такая последовательность не имеет предельных точек, что противоречит определению $X$, а значит $X$ ограничено. Если $X$ не замкнуто, то можно рассмотреть предельную точку $x$, не лежащую в $X$, и построить последовательность, сходящуюся к ней, а значит никаких других точек у последовательности быть не может, а значит опять получаем противоречие с определением $X$; значит $X$ ещё и замкнуто.
            \item Пусть $X$ замкнуто и ограничено. Пусть также дана некоторая последовательность $\{x_n\}_{n=0}^\infty$ элементов $X$. Поскольку $X$ ограничено, то значит лежит внутри некоторого отрезка $I_0$. Определим последовательность $\{I_n\}_{n=0}^\infty$ рекуррентно следующим образом. Пусть $I_n$ определено; разделим $I_n$ на две половины и определим $I_{n+1}$ как любую из половин, в которой находится бесконечное количество членов последовательности $\{x_n\}_{n=0}^\infty$. после этого определим последовательность $\{y_n\}_{n=0}^\infty$ как подпоследовательность $\{x_n\}_{n=0}^\infty$, что $y_n \in I_n$ для любого $n \in \NN$ (это можно сделать рекуррентно: если определён член $y_n$, то найдётся ещё бесконечное количество членов начальной последовательности в $I_{n+1}$, которые идут после $y_n$, так как отброшено конечное количество, а значит можно взять любой). Несложно видеть, что $\lim_{n \to \infty} y_n = \bigcap_{n \in \NN} I_n =: y$. Из-за замкнутости $y \in X$, а значит $y$ --- предельная точка $\{x_n\}_{n=0}^\infty$ --- лежит в $X$ и доказывает компактность $X$.
        \end{enumerate}
    \end{proof}

    \begin{lemma}\label{segment_edged_subcover_great_lemma}
        Пусть $\Sigma$ --- семейство интервалов длины больше некоторого $d > 0$, покрывающее отрезок $[a; b]$. Тогда у $\Sigma$ есть конечное подсемейство $\Sigma'$, покрывающее $[a; b]$.
    \end{lemma}
    
    \begin{proof}
        Давайте вести индукцию по $\lceil (b-a)/d \rceil$.

        \textbf{База.} $\lceil (b-a)/d \rceil = 0$. В таком случае $a = b$, а значит, можно взять любой интервал, покрывающий единственную точку и получить всё искомое семейство $\Sigma'$.

        \textbf{Шаг.} Рассмотрим $\Omega := \{I \in \Sigma \mid a \in I\}$. Заметим, что если у правых концов интервалов из $\Omega$ нет верхних граней (т.е. их множество не ограничено сверху), то значит найдётся интервал, покрывающий и $a$, и $b$, а значит его как единственный элемент семейства $\Sigma'$ будет достаточно. Иначе определим $a'$ как супремум правых концов интервалов из $\Omega$.
        
        Тогда мы имеем, что есть интервалы из $\Omega$, подбирающиеся сколь угодно близко к $a'$, а также что все интервалы из $\Sigma$, покрывающие $a'$ не покрывают $a$. Если $a' > b$, то можно опять же взять интервал, который покроет весь $[a; b]$, и остановится. Иначе рассмотрим любой интервал $I$, покрывающий $a'$ и любой интервал $J$ из $\Omega$, перекрывающийся с $I$. Пусть $a''$ --- правый конец $J$.

        Заметим, что $I$ и $J$ покрывают $[a; a'')$. При этом $a < J < a''$, значит $a'' - a \geqslant \osc(J) > d$. Если $a'' > b$, то $\Sigma = \{I, J\}$ будет достаточно. Иначе заметим, что
        \[
            \left\lceil \frac{b-a''}{d} \right\rceil =
            \left\lceil \frac{b-a}{d} - \frac{a''-a}{d} \right\rceil \leqslant
            \left\lceil \frac{b-a}{d} - 1 \right\rceil =
            \left\lceil \frac{b-a}{d} \right\rceil - 1 <
            \left\lceil \frac{b-a}{d} \right\rceil
        \]
        Тогда по предположению индукции есть конечное подпокрытие $\Sigma''$ покрытия $\Sigma$ отрезка $[a''; b]$. Значит $\Sigma' := \Sigma'' \cup \{I, J\}$ является конечным подпокрытием покрытия $\Sigma$ множества $[a; b]$.
    \end{proof}

    \begin{lemma}\label{edged_subcover_great_lemma}
        Пусть $\Sigma$ --- семейство интервалов длины больше некоторого $d > 0$. Тогда найдётся не более чем счётное подсемейство $\Sigma'$, имеющее такое же объединение, т.е. $|\Sigma'| \leqslant |\NN|$, а $\bigcup \Sigma = \bigcup \Sigma'$.
    \end{lemma}

    \begin{proof}
        Несложно видеть, что $A := \bigcup \Sigma$ представляется в виде дизъюнктного объединения интервалов. Каждый из них можно представить как объединение не более чем счётного отрезков. Итого мы получим не более чем счётное семейство $\Omega$ отрезков, что $\bigcup \Omega = A$. Для каждого отрезка из $\Omega$ построим по лемме \ref{segment_edged_subcover_great_lemma} конечное подпокрытие покрытия $\Sigma$, а затем объединив их, получим не более чем счётное семейство $\Sigma'$, покрывающее любой из них, а значит и $\bigcup \Omega = A = \bigcup \Sigma$. С другой стороны $\Sigma'$ --- подмножество $\Sigma$, значит и $\bigcup \Sigma'$ --- подмножество $\bigcup \Sigma$.

        В итоге $\bigcup \Sigma' = \bigcup \Sigma$, и при этом $\Sigma'$ --- не более чем счётное подмножество $\Sigma$.
    \end{proof}

    \begin{lemma}
        Пусть дано семейство $\Sigma$ интервалов. Тогда из него можно выделить не более чем счётное подсемейство $\Sigma'$ с тем же объединением, т.е. $|\Sigma'| \leqslant |\NN|$, а $\bigcup \Sigma = \bigcup \Sigma'$.
    \end{lemma}

    \begin{proof}
        Рассмотрим для каждого $n \in \ZZ$ семейство
        \[\Sigma_n = \{I \in \Sigma \mid \osc(I) \in [2^n; 2^{n+1})\}\]
        Применим лемму к $\Sigma_n$ и получим $\Sigma'_n$. Тогда $\Sigma' := \bigcup_{n \in \ZZ} \Sigma'_n$ является подмножеством $\Sigma$, даёт в объединении то же, что и $\Sigma$, и при этом имеет мощность не более $|\NN \times \NN| = |\NN|$.
    \end{proof}

    \begin{theorem}
        Подмножество $\RR$ компактно тогда и только тогда, когда из любого его покрытия интервалами можно выделить конечное подпокрытие.
    \end{theorem}

    \begin{proof}
        \begin{enumerate}
            \item Пусть $X$ компактно, а $\Sigma$ --- некоторое его покрытие интервалами. Определим для каждого $d > 0$
                \[\Sigma_d := \{I \in \Sigma\mid \osc(I) > d\}\]
                Если никакое из $\Sigma_d$ не является подпокрытием множества $X$, то рассмотрим последовательность $\{x_n\}_{n=0}^\infty$, где $x_n$ --- любой элемент $X \setminus \Sigma_{1/2^n}$. У $\{x_n\}_{n=0}^\infty$ есть предельная точка $x \in X$. Значит должен быть интервал, покрывающий $x$, но тогда он же покрывает весь некоторый хвост нашей последовательности, а сам лежит в некотором $\Sigma_{1/2^n}$ --- противоречие. Значит некоторое $\Sigma_d$ является подпокрытие, а значит далее можно рассматривать его в качестве $\Sigma$.

                $\bigcup \Sigma$ --- открытое множество, поэтому является дизъюнктным объединением семейства $\Omega$ интервалов. Поскольку в $\Sigma$ длины всех интервалов больше $d$, то в $\Omega$ тоже. Но также $X$ ограничено, поэтому $\Omega$ конечно, да и все интервалы в нём ограничены. Заметим, что $X \cap I$, где $I$ --- любой интервал из $\Omega$, является замкнутым множеством, поэтому его можно накрыть некоторым отрезком $S \subseteq I$ (для этого можно взять отрезок $[\inf(X \cap I); \sup(X \cap I)]$). Значит из накрытия $\Sigma$ выделить $|\Omega|$ конечных подпокрытий для каждого отрезка (по лемме \ref{edged_subcover_great_lemma}), а их объединение даст конечное покрытие $X$.

            \item Пусть $X$ таково, что из любого покрытия можно выбрать конечное подпокрытие.
            
                Если $X$ неограничено, то тогда несложно будет видеть, что покрытие $\{(n; n+2) \mid n \in \ZZ\}$ нельзя уменьшить до конечного. Значит $X$ конечно.

                Если $X$ не замкнуто, то значит есть точка $x \notin X$, что в любой окрестности $x$ будет точка. Тогда рассмотрим покрытие $\{(x + 2^n; x^{n+2}) \mid n \in \ZZ\} \cup \{(x - 2^{n+2}; x^n) \mid n \in \ZZ\}$. Несложно видеть, что если взять любое конечное подсемейство интервалов, то оно не накроет некоторую окрестность $x$, а значит и $X$. Значит $X$ замкнуто.

                Итого получаем, что $X$ компактно.
        \end{enumerate}
    \end{proof}

    \subsection{Пределы функций, непрерывность}

    \begin{definition}[по Коши]
        \emph{Предел} функции $f: X \to \RR$ в точке $x$ --- такое значение $y$, что
        \[\forall \varepsilon > 0\, \exists \delta > 0: f(V_\delta(x) \cap X) = U_\varepsilon(y)\]
        Обозначение: $\lim\limits_{t \to x} f(t) = y$.
    \end{definition}

    \begin{definition}[по Гейне]
        \emph{Предел} функции $f: X \to \RR$ в точке $x$ --- такое значение $y$, что для любой последовательность $\{x_n\}_{n=0}^\infty$ элементов $X \setminus \{x\}$ последовательность $\{f(x_n)\}_{n=0}^\infty$ сходится к $y$. Обозначение: $\lim\limits_{t \to x} f(t) = y$.
    \end{definition}

    \begin{theorem}
        Определения пределов по Коши и по Гейне равносильны.
    \end{theorem}

    \begin{proof}
        Будем доказывать равносильность отрицаний утверждений, ставимых в определениях.
        \begin{enumerate}
            \item Пусть функция $f: X \to \RR$ не сходится по Коши в $x$ к значению $y$. Значит есть такое $\varepsilon > 0$, что в любой проколотой окрестности $x$ (в множестве $X$) есть точка, значение $f$ в которой не лежит в $\varepsilon$-окрестности. Рассмотрев любую такую проколотую окрестность $I_0 = V_{\delta_0}(x)$, берём в ней любую такую точку $x_0$. Далее рассмотрев $I_1 = V_{\delta_1}(x)$, где $\delta_1 = \min(\delta_0/2, |x-x_0|)$, берём там любую точку $x_1$, где значение $f$ вылетает вне $\varepsilon$-окрестности $y$. Так далее строим последовательность $\{x_n\}_{n=0}^\infty$, сходящуюся к $x$, значения $f$ в которой не лежат в $\varepsilon$-окрестности $y$, что означает, что $\{f(x_n)\}_{n=0}^\infty$ не сходится к $y$, что означает, что $f$ не сходится по Гейне в $x$ к значению $y$.
            \item Пусть функция $f: X \to \RR$ не сходится по Гейне в $x$ к значению $y$. Значит есть последовательность $\{x_n\}_{n=0}^\infty$, сходящаяся к $x$, что последовательность её значений не сходится к $y$. Значит есть $\varepsilon > 0$, что после любого момента в последовательности будет член, значение в котором вылезает вне $\varepsilon$-окрестности $y$. Поскольку для любой проколотой окрестности $x$ есть момент, начиная с которого вся последовательность лежит в этой окрестности, то в любой проколотой окрестности $x$ есть член, значение которого вылезает вне $\varepsilon$-окрестности $y$, что означает, что $f$ не сходится по Коши в $x$ к $y$.
        \end{enumerate}
    \end{proof}

    \begin{statement}
        Функция $f: X \to \RR$ имеет в $x$ предел тогда и только тогда, когда
        \[\forall \varepsilon > 0\; \exists \delta > 0:\; \forall x_1, x_2 \in V_\delta(x)\quad |f(x_1) - f(x_2)| < \varepsilon\]
    \end{statement}

    \begin{proof}
        Такое же как для последовательностей: см. теорему \ref{fundamental_seq_theorem}.
    \end{proof}

    \begin{statement}
        Для функций $f: \RR \to \RR$ и $g: \RR \to \RR$ верно, что
        \begin{enumerate}
            \item $\lim\limits_{x \to a} f(x) + \lim\limits_{x \to a} g(x) = \lim\limits_{x \to a} (f + g)(x)$
            \item $\lim\limits_{x \to a} (-f)(x) = -\lim\limits_{x \to a} f(x)$
            \item $\lim\limits_{x \to a} f(x) \cdot \lim\limits_{x \to a} g(x) = \lim (f \cdot g)(x)$
            \item $\frac{1}{\lim\limits_{x \to a} f(x)} = \lim\limits_{x \to a} (\frac{1}{f})(x)$ (если $\lim\limits_{x \to a} f(x) \neq 0$)
            \item $\lim\limits_{y \to \lim\limits_{x \to a} g(x)} f(y) = \lim\limits_{x \to a} (f \circ g)(x)$
        \end{enumerate}
        и всегда, когда определена левая сторона определена, правая тоже определена.
    \end{statement}

    \begin{remark}
        Утверждения \ref{stupid_seq_statement_1}, \ref{stupid_seq_statement_2} и \ref{stupid_seq_statement_3} верны, если заменить последовательности на функции, пределы последовательностей на пределы функций в некоторой точке $x$, а асимптотические неравенства на неравенства на окрестности $x$.
    \end{remark}

    \begin{definition}
        \emph{Верхним пределом} функции $f$ в точке $x_0$ называется
        \[\varlimsup\limits_{x \to x_0} f(x) = \inf_{\delta > 0} (\sup_{V_\delta(x_0)} f)\]
        \emph{Нижним пределом} функции $f$ в точке $x_0$ называется
        \[\varliminf\limits_{x \to x_0} f(x) = \sup_{\delta > 0} (\inf_{V_\delta(x_0)} f)\]
    \end{definition}

    \begin{statement}
        Функция $f: X \to \RR$ имеет в $x$ предел тогда и только тогда, когда $\varlimsup\limits_{t \to x} f(t) = \varliminf\limits_{t \to x} f(t)$.
    \end{statement}

    \begin{definition}
        Функция $f: X \to \RR$ называется \emph{непрерывной в точке} $x$, если $\lim\limits_{t \to x} f(t) = f(x)$. В изолированных точках $f$ всегда непрерывна.
    \end{definition}

    \begin{definition}
        Функция $f: X \to \RR$ называется \emph{непрерывной на множестве} $Y \subseteq X$, если она непрерывна во всех точках $Y$.
    \end{definition}

    \begin{statement}
        Для непрерывных на $X$ функций $f$ и $g$ верно, что
        \begin{itemize}
            \item $f+g$ непрерывна на $X$;
            \item $fg$ непрерывна на $X$;
            \item $\frac{1}{f}$ непрерывна на $X$ (если $f \neq 0$).
        \end{itemize}
    \end{statement}

    \begin{statement}
        Для $f$, непрерывной в $x_0$, и $g$, непрерывной в $f(x_0)$, $g\circ f$ непрерывна в $x_0$.
    \end{statement}

    \begin{theorem}[Вейерштрасса]
        Непрерывная функция на компакте ограничена на нём и принимает на нём свои минимум и максимум.
    \end{theorem}

    \begin{proof}
        Докажем утверждение для ограниченности сверху и максимума; для ограниченности снизу и минимума рассуждения аналогичны.

        Пусть множество неограниченно сверху. Тогда есть $\{x_n\}_{n=0}^\infty$, что $\{f(x_n)\}_{n=0}^\infty \to +\infty$. Тогда рассмотрим подпоследовательность $\{y_n\}_{n=0}^\infty$ последовательности $\{x_n\}_{n=0}^\infty$, сходящуюся к $y$. Тогда
        \[f(y) = \lim_{n \to \infty}\limits f(y_n) = +\infty\]
        --- противоречие.

        Тогда существует последовательность $\{x_n\}_{n=0}^\infty$, что $\{f(x_n)\}_{n=0}^\infty$ сходится к супремуму $S$ функции. Рассмотрим подпоследовательность $\{y_n\}_{n=0}^\infty$ последовательности $\{x_n\}_{n=0}^\infty$, сходящуюся к $y$. Тогда
        \[f(y) = \lim_{n \to \infty} f(y_n) = S\]
    \end{proof}

    \begin{corollary}
        Так как отрезок компактен, то любая непрерывная на нём функция ограничена и принимает на нём свои максимум и минимум.
    \end{corollary}

    \begin{theorem}[о промежуточном значении]
        Пусть $f$ непрерывна на $[a; b]$, а $f(a) < f(b)$. Тогда $\forall y \in [f(a); f(b)]$ найдётся $c \in [a; b]$, что $f(c) = y$.
    \end{theorem}

    \begin{proof}
        Рассмотрим последовательность $\{(a_n; b_n)\}_{n=0}^\infty$, что $(a; b) = (a_0; b_0)$, а следующие пары определяются так: если $f(\frac{a_n+b_n}{2}) < y$, то $(a_{n+1}; b_{n+1}) = (\frac{a_n+b_n}{2}; b_n)$, иначе $(a_{n+1}; b_{n+1}) = (a_n; \frac{a_n+b_n}{2})$. Тогда $c = \lim \{a_n\}_{n=0}^\infty = \lim \{b_n\}_{n=0}^\infty$. Тогда
        \[f(c) = \lim \{f(a_n)\}_{n=0}^\infty = \lim \{f(b_n)\}_{n=0}^\infty,\]
        откуда получаем, что $f(c) \geqslant y$ и $f(c) \leqslant y$, т.е. $f(c) = y$.
    \end{proof}

    \begin{definition}
        Функция $f$ \emph{равномерно непрерывна} на $X$, если
        \[\forall \varepsilon > 0\; \exists \delta > 0:\; \forall x \in X\quad f(U_\delta(x)) \subseteq U_\varepsilon(f(x))\]
    \end{definition}

    \begin{theorem}[Кантор]
        Непрерывная на компакте функция равномерно непрерывна.
    \end{theorem}

    \begin{proof}
        Предположим противное. Тогда
        \[\exists \varepsilon > 0: \forall \delta > 0\; \exists x, y:\quad |x - y| < \delta \wedge |f(x) - f(y)| > \varepsilon\]
        Тогда рассмотрим последовательность пар $x$ и $y$ построенных так для $\delta$, сходящихся к $0$. Из неё выделим подпоследовательность, что $x$ сходится к некоторому $a$. Тогда $y$ сойдутся к нему же. Тогда в любой окрестности $a$ будет пара точек $(x'; y')$, что $|f(x') - f(y')| > \varepsilon$, значит будет в любой окрестности $x$ будет точка, выбивающаяся из $\varepsilon/2$-окрестности --- противоречие с непрерывностью.
    \end{proof}

    \begin{definition}
        Пусть есть функции $f$ и $g$, что $|f| \leqslant C|g|$ в окрестности $x$ для некоторого $C \in \RR$, тогда пишут, что $f = O(g)$ (при $t \to x$).
        
        Если же $\forall \varepsilon > 0$ будет такая окрестность $x_0$, что $|f| \leqslant \varepsilon |g|$ в этой окрестности, тогда пишут, что $f = o(g)$ (при $t \to x$).
    \end{definition}

    \subsection{Гладкость (дифференцируемость)}

    \begin{definition}\label{def_derivative_1}
        Функция $f$ называется \emph{гладкой (дифференцируемой)} в $x$, если $f(x + \delta) = f(x) + A \delta + o(\delta)$ для некоторого $A \in \RR$. В таком случае $A$ называется \emph{дифференциалом (производной)} $f$ в точке $x$.

        Обозначение: $f'(x) = A$.
    \end{definition}

    \begin{definition}\label{def_derivative_2}
        Функция $f$ называется \emph{гладкой (дифференцируемой)} в $x$, если предел
        \[\lim\limits_{\delta \to 0} \frac{f(x+\delta) - f(x)}{\delta}\]
        определён. В таком случае его значение называется \emph{дифференциалом (производной)} $f$ в точке $x$.
    \end{definition}

    \begin{statement}
        Определения \ref{def_derivative_1} и \ref{def_derivative_2} равносильны.
    \end{statement}

    \begin{statement}
        Непрерывная в некоторой точке функция там же непрерывна.
    \end{statement}

    \begin{definition}  
        Функция, значения которой равны производным функции $f$ в тех же точках называется \emph{производной функцией} (или просто \emph{производной}) функции $f$. Обозначение: $f'$.
    \end{definition}

    \begin{lemma}
        Для дифференцируемых в $x$ функций $f$ и $g$
        \begin{enumerate}
            \item $(f \pm g)'(x) = f'(x) \pm g'(x)$;
            \item $(f \cdot g)'(x) = f'(x) g(x) + f(x) g'(x)$ (правило Лейбница);
            \item $(\frac{1}{f})'(x) = \frac{-f'(x)}{f(x)^2}$;
            \item $(f \circ g)'(x) = f'(g(x))\cdot g'(x)$.
        \end{enumerate}
    \end{lemma}

    \begin{lemma}
        Пусть дана $f: [a; b] \to \RR$ --- непрерывная монотонно возрастающая (убывающая) функция. Тогда существует $g: [f(a); f(b)] \to \RR$ --- непрерывная монотонно возрастающая (убывающая) функция, что $g \circ f = Id$.
    \end{lemma}

    \begin{proof}
        Заметим, что $f$ --- монотонно возрастающая (убывающая) биекция из $[a; b]$ в $[f(a); f(b)]$. Тогда существует монотонно возрастающая (убывающая) биекция $g: [f(a); f(b)] \to [a; b]$, что $g \circ f = id$. Осталось показать, что $g$ непрерывна.

        Предположим противное, тогда в любой окрестности некоторой точки $f(x)$ из $[f(a); f(b)]$ есть точки вылетающие вне $\varepsilon$-окрестности. Значит все точки из либо $(x - \varepsilon; x)$, либо $(x; x + \varepsilon)$ не принимаются, значит $g$ не биекция --- противоречие. Значит $g$ непрерывна.
    \end{proof}

    \begin{lemma}
        \[(f^{-1})' = \frac{1}{f' \circ f^{-1}}\]
    \end{lemma}

    \begin{proof}
        Пусть $g := f^{-1}$. Тогда
        \[1 = Id' = (f \circ g)' = f' \circ g \cdot g'\]
        Откуда следует, что
        \[(f^{-1})' = g' = \frac{1}{f' \circ g} = \frac{1}{f' \circ f^{-1}}\]
    \end{proof}

    \begin{definition}
        Функция $f$ \emph{возрастает в точке $y$}, если есть $\varepsilon > 0$, что $f(x) \leqslant f(y)$ для любого $x \in (y-\varepsilon; y)$ и $f(x) \geqslant f(y)$ для любого $x \in (y; y+\varepsilon)$.

        Аналогично определяется убываемость функции в точке.
    \end{definition}

    \begin{lemma}
        Если $f$ возрастает в любой точке на $[a;b]$, то $f(a) \leqslant f(b)$.
    \end{lemma}

    \begin{proof}
        \begin{enumerate}\ 
            \item Можно рассмотреть для каждой точки $[a; b]$ окрестность, для которой верна её возрастаемость, и из покрытия, ими образуемого, выделить конечное. А тогда перебираясь между общими точками окрестностей, получим искомое.
            \item Также можно предположить противное, рассмотреть последовательность вложенных отрезков, у которых левый конец выше правого, и тогда для точки пересечения отрезков будет противоречие.
        \end{enumerate}
    \end{proof}

    \begin{corollary}
        $f$ возрастает на всём отрезке.
    \end{corollary}

    \begin{theorem}
        Если $f$ гладка, а $f'$ положительна на $[a; b]$, то $f$ строго возрастает на $[a; b]$.
    \end{theorem}

    \begin{proof}
        Несложно видеть, что в любой точке на $[a; b]$ у функции есть окрестность, где она строго возрастает, так как если $t \in [a; b]$, а $f'(t) = \lambda > 0$, то в некоторой окрестности
        \begin{align*}
            \frac{f(x) - f(t)}{x - t} &\in (0; 2\lambda)&
            &\Longrightarrow&
            f(x) \in (f(t); f(t) + 2\lambda(x-t))
        \end{align*}
        что значит, что эта окрестность --- подтверждение для возрастания $f$ в $t$. Тогда по предыдущему следствию $f$ возрастает на $[a; b]$. Если вдруг функция возрастает нестрого, то тогда найдётся подотрезок на $[a;b]$, на котором функция константа, а значит на интервале с теми же концами производная тождественна равна нулю.
    \end{proof}

    \begin{theorem}
        Если $f$ возрастает, то $f'$ в своей области определения неотрицательно.
    \end{theorem}

    \begin{proof}
        Если функция в точке $t$ равна $\lambda < 0$, то в некоторой окрестности $t$
        \begin{align*}
            \frac{f(x)-f(t)}{x-t} &\in \left(\frac{3}{2}\lambda; \frac{1}{2}\lambda\right)&
            &\Longrightarrow&
            f(x) &\in \left(f(t) + \frac{3}{2}\lambda(x-t); f(t) + \frac{1}{2}\lambda(x-t)\right)
        \end{align*}
        что значит, что $f$ в точке $t$ "строго" убывает --- противоречие. Значит $f'(t) \geqslant 0$.
    \end{proof}

    \begin{definition}
        \emph{$f$ имеет локальный максимум в $x$}, если для некоторого $\varepsilon > 0$ верно, что $f(x) \geqslant f(y)$ для любого $y \in (x - \varepsilon; x + \varepsilon)$.

        Аналогично определяется точка локального минимума.
    \end{definition}

    \begin{theorem}
        В точках локальных максимумов и минимумов функции $f$ функция $f'$ принимает нули (если определена).
    \end{theorem}

    \begin{proof}
        Слева от точки максимума функция возрастает в данной точке, значит производная в данной точке $\geqslant 0$, а справа --- убывает, значит производная $\leqslant 0$, значит производная равна $0$. Аналогично для точки минимума.
    \end{proof}

    \begin{theorem}[Ролль]
        Если $f$ --- гладкая функция на $[a; b]$, и $f(a) = f(b)$, то существует $c \in (a; b)$, что $f'(c) = 0$.
    \end{theorem}

    \begin{proof}
        В точке максимума или минимума $f$ на $[a;b]$ достигается ноль производной. Если они обе совпадают с концами отрезка, то значит функция константа, а тогда в любой точке отрезка производная равна нулю.
    \end{proof}

    \begin{theorem}
        Если $f$ и $g$ непрерывные на $[a; b]$ и гладкие на $(a; b)$ функции, а $g' \neq 0$, то существует $c \in (a; b)$, что
        \[\frac{f(a) - f(b)}{g(a) - g(b)} = \frac{f'(c)}{g'(c)}\]
    \end{theorem}

    \begin{proof}
        Пусть
        \[\lambda := \frac{f(a) - f(b)}{g(a) - g(b)}\]
        а $\tau(x) := f(x) - \lambda g(x)$. В таком случае
        \[\frac{\tau(a) - \tau(b)}{g(a) - g(b)} = \frac{(f(a) - f(b) - \lambda (g(a) - g(b)))}{g(a) - g(b)} = \lambda - \lambda = 0\]
        значит $\tau(a) = \tau(b)$, значит есть $c \in [a; b]$, что $\tau(c) = 0$. Тогда
        \[\frac{f'(c)}{g'(c)} = \frac{(\tau + \lambda g)'(c)}{g(c)} = \frac{\tau'(c)}{g(c)} + \lambda = \lambda = \frac{f(a) - f(b)}{g(a) - g(b)}\]
    \end{proof}

    \begin{theorem}[Лагранж]
        Если $f$ непрерывна на $[a; b]$ и гладка на $(a; b)$, то существует $c \in (a; b)$, что
        \[\frac{f(a) - f(b)}{a - b} = f'(c)\]
    \end{theorem}

    \begin{proof}
        Очевидно следует из предыдущей теоремы с помощью подстановки $g(x) = x$.
    \end{proof}

    \begin{theorem}
        Пусть $f$ --- гладкая на $(a; b)$ функция.
        \begin{enumerate}
            \item Если $f' \geqslant 0$, то $f$ возрастающая функция.
            \item Если $f' > 0$, то $f$ строго возрастающая функция.
            \item Если $f$ возрастающая функция, то $f' > 0$.
        \end{enumerate}
    \end{theorem}

    \begin{theorem}
        Пусть $f$ --- гладкая на $[a; b]$ функция. Если $f'(x) = 0$ для всех $x \in [a; b]$, то $f \equiv const$ на том же отрезке.
    \end{theorem}

    \begin{remark}
        Функция $f(x) := x^2 \sin(1/x)$ (доопределённая в нуле) имеет производную $f'(x) = 2x\sin(1/x) - \cos(1/x)$ в случае ненулевых $x$ и производную $f'(0) = 0$. При этом легко видно, что $f'$ не является непрерывной функцией (она имеет разрыв в том же нуле).
    \end{remark}

    \begin{theorem}
        Если $f$ гладка на $(a; b)$, а $f'$ не равна нулю, то $f'$ либо положительна, либо отрицательна.
    \end{theorem}

    \begin{proof}
        $f$ не принимает никакое значение на $(a; b)$ дважды (т.к. иначе у производной был бы корень), значит она либо строго возрастает, либо строго убывает, а значит $f'$ либо неотрицательна, либо неположительна соответственно. Но ноль принимать не может, поэтому последнее утверждение равносильно тому, что $f$ либо строго положительна, либо строго отрицательна.
    \end{proof}

    \begin{theorem}
        Пусть $f$ гладка на $(a; b)$ и для некоторых $u, v \in (a; b)$ верно, что $f'(u) < \alpha < f'(v)$. Тогда существует $c \in (u; v)$, что $f'(c) = \alpha$.
    \end{theorem}

    \begin{proof}
        Пусть $g(x) := f(x) - \alpha x$. Тогда $g'(u) < 0 < g'(v)$, значит $g$ не может строго возрастать или убывать на $(u; v)$, значит $\exists c \in (u; v)$, что $g'(c) = 0$, а значит $f'(c)=\alpha$.
    \end{proof}

    \begin{remark}
        Данная теорема по сути является теоремой о промежуточном значении для производной.
    \end{remark}

    \begin{theorem}
        Пусть $f$ непрерывна на $[a; b)$ и гладка на $(a; b)$. Пусть также $\lim_{x \to a^+} f'(x)$ существует и равен $d$. Тогда $f'(a)$ тоже существует и равна $d$.
    \end{theorem}

    \begin{proof}
        Есть несколько способов:
        \begin{enumerate}
            \item Несложно видеть, что для любого $\varepsilon > 0$ есть некоторая правая окрестность $a$, в которой функция $f'$ лежит в $\varepsilon$-окрестности $d$. Тогда $f(x) - (d-\varepsilon)x$ убывает в данной окрестности, а $f(x) - (d+\varepsilon)x$ возрастает, значит $f(x) - f(a) \in ((d + \varepsilon)(x-a); (d-\varepsilon)(x-a))$. В таком случае $f'(a)$ определена и равна $d$.
            \item По теореме Лагранжа для любого $x \in [a; b)$ найдётся $\xi \in (a; x)$, что
                \[\frac{f(x) - f(a)}{x - a} = f'(\xi)\]
                Значит
                \[\lim_{x \to a^+} \frac{f(x) - f(a)}{x - a} = \lim_{x \to a^+} f'(\xi) = d\]
                что буквально значит, что $f'(a) = d$.
        \end{enumerate}
    \end{proof}

    \begin{theorem}[правило Лопиталя]
        Пусть $\lim_{x \to a^+} f(x) = \lim_{x \to a^+} g(x) = 0$. Пусть также $f$ и $g$ гладки и $g' \neq 0$ на $(a; b)$. Тогда
        \[\lim_{x \to a^+} \frac{f(x)}{g(x)} = \lim_{x \to a^+} \frac{f'(x)}{g'(x)}\]
        если второй предел определён.
    \end{theorem}

    \begin{proof}
        Пусть дано $\varepsilon > 0$, а
        \[d := \lim_{x \to a^+} \frac{f'(x)}{g'(x)}\]. Тогда есть $\delta > 0$, что для любого $t \in (a; a + \delta)$ значение $f'(t)/g'(t)$ лежит в $U_\varepsilon(d)$. Легко видеть, что для любых $x, y \in (a; a + \delta)$ существует $\xi \in (x; y) \subseteq (a; a + \delta)$, что
        \[\frac{f(x) - f(y)}{g(x) - g(y)} = f'(\xi) \in U_\varepsilon(d)\]
        Устремляя $x$ к $a$, получаем, что $f(y)/g(y)$ тоже лежит в $U_\varepsilon(d)$. Тогда по определению предела
        \[\lim_{x \to a^+} \frac{f(x)}{g(x)} = d = \lim_{x \to a^+} \frac{f'(x)}{g'(x)}\]
    \end{proof}

    \begin{definition}
        $f''$ --- вторая производная $f$, т.е. $(f')'$, а $f^{(n)}$ --- $n$-ая производная $f$, т.е. $f^{(n)} := (f^{(n-1)})'$, $f^{(0)} := f$.
    \end{definition}

    \begin{definition}
        $P(x)$ --- полином Тейлора степени $n$ функции $f$, если $\deg(P) \leqslant n$, а
        \[f(x) - P(x) = o((x-a)^n),\quad x \to a\]
    \end{definition}

    \begin{theorem}
        Если $P_1$ и $P_2$ --- полиномы Тейлора степени $n$ функции $f$, то $P_1 = P_2$.
    \end{theorem}

    \begin{theorem}
        Пусть $f: (a; b) \to \RR$, $f^{(1)}$, \dots, $f^{(n-1)}$ определены на $(t-\delta; t+\delta)$ для некоторого $\delta > 0$ и определена $f^{(n)}(t)$. Тогда для всякого $x \in U_\delta(t)$
        \[f(x) = f(t) + \frac{f^{(1)}(t)}{1!}(x-t) + \dots + \frac{f^{(n)}(t)}{n!}(x-t)^n + o((x-t)^n)\]
    \end{theorem}

    \begin{proof}
        Рассмотрим $g(x) := f(x) - f(t)/0! \cdot (x-t)^0 - \dots - f^{(n)}(t)/n! \cdot (x-t)^n$. Тогда задача сведена к следующей лемме.

        \begin{thlemma}
            Если $g^{(1)}$, \dots, $g^{(n-1)}$ определены на $(t-\delta; t+\delta)$ для некоторого $\delta > 0$ и
            \[g(t) = g^{(1)}(t) = \dots = g^{(n)}(t) = 0.\]
            Тогда $g(x) = o((x-t)^n)$.
        \end{thlemma}

        \begin{proof}
            Докажем по индукции по $n$.

            \textbf{База.} Пусть $n = 1$. Тогда очевидно, что $f(x) = f(t) + f'(t)(x-t) + o(x-t) = o(x-t)$.

            \textbf{Шаг.} По предположению индукции $f'(x) = o((x-t)^n)$. Тогда мы имеем, что
            \[f(x) = f(x) - f(t) = f'(\xi) (x - t)\]
            для некоторого $\xi \in (x, t)$. Тогда
            \[\frac{f(x) - f(t)}{(x-t)^n} = \frac{f'(\xi)}{(x-t)^{n-1}} = \frac{o((\xi-t)^{n-1})}{(x-t)^{n-1}} = o(1) \frac{(\xi - t)^{n-1}}{(x-t)^{n-1}} = o(1)\]
        \end{proof}
    \end{proof}


    \begin{theorem}
        Пусть $f(t) = f^{(1)}(t) = \dots = f^{(n)}(t) = 0$, а $f^{(n+1)} \neq 0$. Если $n$ чётно, то $t$ --- не экстремальные точка функции $f$, иначе $t$ --- экстремальная точка функции $f$.
    \end{theorem}

    \begin{theorem}\label{finite_Teylor_series_theorem_engineers_variation}
        Пусть $f: (a; b) \to \RR$, $f^{(1)}$, \dots, $f^{(n+1)}$ определены на $(t-\delta; t+\delta)$ для некоторого $\delta > 0$. Тогда для всякого $x \in U_\delta(t)$ существует $\xi \in (x; t)$, что
        \[f(x) = f(t) + \frac{f^{(1)}(t)}{1!}(x-t) + \dots + \frac{f^{(n)}(t)}{n!}(x-t)^n + \frac{f^{(n+1)}(\xi)}{(n+1)!}(x-t)^{n+1}\]
    \end{theorem}

    \begin{proof}
        Точно так же сведём $f$ к $g$, что $g(t) = \dots g^{(n)}(t) = 0$. Тогда требуется показать, что $g(x) = g^{(n+1)}(\xi)/(n+1)! \cdot (x-t)^{n+1}$ для некоторого $\xi \in (x, t)$. Докажем это по индукции.

        \textbf{База.} $n=0$. Теорема Лагранжа.

        \textbf{Шаг.}
        \[\frac{f(x)}{(x-t)^{n+1}} = \frac{f(x) - f(t)}{(x-t)^{n+1} - (t-t)^{n+1}} = \frac{f'(\xi)}{(n+1)(\xi - t)^{n}} = \frac{f^{(n+1)}(\eta)}{(n+1)!}\]
        где $\xi \in (x, t)$ (существует по теореме Лагранжа), а $\eta \in (\xi, t) \subseteq (x, t)$ (существует по предположению индукции для $f'$ и $\xi$). Отсюда следует искомое утверждение.
    \end{proof}

    \subsection{Стандартные функции, ряды Тейлора и их сходимость}

    \todo[inline]{Тут нужно рассказать про функции $\exp$, $\sin$, $\cos$ и $(1+x)^\alpha$ и их ряды}

    \begin{definition}
        $f$ является \emph{(поточечным) пределом} $\{f_n\}_{n=0}^\infty$ на $E$, если $\lim \{f_n(x)\}_{n=0}^\infty = f(x)$ для любого $x \in E$.
    \end{definition}

    \begin{definition}
        $f$ является \emph{равномерным пределом} $\{f_n\}_{n=0}^\infty$ на $E$, если для любого $\varepsilon > 0$ найдётся $N \in \NN$, что $|f_n(x) - f(x)| < \varepsilon$ для всех $n > N$ и $x \in E$.
    \end{definition}

    \begin{theorem}[Стокс, Зейдель]\label{Stokes_Zaidel_theorem}
        Пусть $\{f_n\}_{n=0}^\infty$ --- последовательность непрерывных функций, и $f_n \to f$ равномерно на $E$. Тогда $f$ непрерывна.
    \end{theorem}

    \begin{proof}
        Для любого $\varepsilon > 0$ есть такое $n \in \NN$, что $|f_n(x) - f(x)| < \varepsilon/3$ для всех $x \in E$. Тогда существует $\delta > 0$, что $f_n(U_\delta(t)) \subseteq U_{\varepsilon/3}(f_n(t))$ для данного $t$. Тогда
        \[f(U_\delta(t)) \subseteq U_{\varepsilon/3}(f_n(U_\delta(t))) \subseteq U_{2\varepsilon/3}(f_n(t)) \subseteq U_\varepsilon(f(t)).\]
    \end{proof}

    \begin{theorem}[Коши]
        TFAE (the following are equivalent):
        \begin{enumerate}
            \item $f_n \to f$ равномерно сходится на $E$.
            \item Для любого $\varepsilon > 0$ существует $N \in \NN$, что $|f_k(x) - f_l(x)| < \varepsilon$ для любых $k, l > N$ и $x \in E$.
        \end{enumerate}
    \end{theorem}

    \begin{theorem}[Вейерштрасс]\label{Weierstrass_uniform_convergence_theorem}
        Пусть $\{u_n\}_{n=0}^\infty$ --- последовательность непрерывных функций, что есть последовательность чисел $\{d_n\}_{n=0}^\infty$, для которой верно, что $|u_n| < d_n$ для всех $n \in \NN$, и $\sum_{n=0}^\infty d_n$ сходится. Тогда $\sum_{n=0}^\infty u_n$ равномерно сходится.
    \end{theorem}

    \begin{theorem}
        Пусть $f_n \to f$ на $E$ и $\{f_n\}_{n=0}^\infty$ гладкие. Если $f_n' \to g$ равномерно, то $f$ тогда тоже гладка и $f' = g$.
    \end{theorem}

    \begin{proof}
        Для любого $\varepsilon > 0$ существует $N \in \NN$, что $|f'_k - f'_l| < \varepsilon/3$ для всех $k, l > N$. Тогда имеем, что
        \[\left|\frac{f_k(x) - f_k(y)}{x - y} - \frac{f_l(x) - f_l(y)}{x - y}\right| = \left| \frac{(f_k - f_l)(x) - (f_k - f_l)(y)}{x - y} \right| = |(f_k - f_l)'(\xi)| < \varepsilon/3 \]
        Устремляя $l$ к бесконечности получаем, что
        \[\left|\frac{f_k(x) - f_k(y)}{x - y} - \frac{f(x) - f(y)}{x - y}\right| \leqslant \varepsilon/3\]
        Также имеем, что есть такое $\delta > 0$, что для всех $y \in U_\delta(x)$
        \[\left|\frac{f_k(x) - f_k(y)}{x - y} - f'_k(x)\right| < \varepsilon/3\]
        Также есть $M \in \NN$, что $|f'_k - g| < \varepsilon/3$ для любого $k > M$. Складывая всё вместе, получаем, что для всех $k > \max(N, M)$ и $y \in U_\delta(x)$
        \[\left|\frac{f(x) - f(y)}{x - y} - g(x)\right| < \varepsilon\]
        Значит $f$ гладка и $f' = g$.
    \end{proof}

    \begin{corollary}
        Если $\{f^{(0)}\}$, \dots, $\{f^{(n-1)}\}$ сходятся, а $f^{(n)}$ равномерно сходится. Тогда то же верно и про первые $n$ производных.
    \end{corollary}

    \begin{corollary}
        Если ряд Тейлора сходится, то функция бесконечно гладкая.
    \end{corollary}

    \section{Примеры и контрпримеры}
    \todo[inline]{Название раздела?}

    \begin{theorem}
        Существует непрерывная функция $f$ на отрезке $[a; b]$, которая не имеет производной ни в какой точке на отрезке $[a; b]$
    \end{theorem}

    \begin{proof}
        Можно привести примеры данной функции $f$.
        \begin{enumerate}
            \item (функция Вейерштрасса) Определим
                \begin{align*}
                    f_0(x) &:= \frac{1}{2} - \left|x - \lfloor x \rfloor - \frac{1}{2}\right|&
                    f_n(x) &:= \frac{f_0(4^n x)}{4^n}&
                    f(x) &:= \sum_{i = 0}^\infty f_i(x)\\
                \end{align*}
                \todo{Картиночки!}
                Поскольку $|f_n| = 1/4^n$, а $\sum_{i = 0}^\infty 1/4^i$ сходится, то по \hyperref[Weierstrass_uniform_convergence_theorem]{теореме Вейерштрасса} ряд равномерно сходится к $f$, а поскольку каждая $f_n$ непрерывна, то по \hyperref[Stokes_Zaidel_theorem]{теореме Стокса-Зейделя} функция $f$ непрерывна. Теперь осталось показать, что у $f$ нет производных.

                Пусть $a$ --- произвольная точка из $\RR$. Заметим, что для всяких $m$ и $n$, что $m \geqslant n$, период $f_m$ равен $1/4^m$, значит $1/4^m \mid 1/4^n$, а тогда $f_m(a \pm 1/4^n) = f_m(a)$. Значит для всякого $n \in \NN \cup \{0\}$
                \[f(a \pm 1/4^n) - f(a) = \sum_{i=0}^{n-1} f_i(a \pm 1/4^n) - f_i(a)\]
                Заметим, что $a$ находится на отрезке монотонности функции $f_{n-1}$ длины $1/(2 \cdot 4^{n-1})=2/4^n$, который также является отрезком монотонности каждой функции из $f_0$, \dots, $f_{n-2}$. Поскольку $1/4^n$ в два раза меньше, то либо $a + 1/4^n$, либо $a - 1/4^n$ лежит на том же отрезке монотонности; пусть это будет точка $b_n$. Тогда имеем, что
                \[
                    \left|\frac{f_0(b_n) - f_0(a)}{b_n - a}\right|
                    = \left|\frac{f_1(b_n) - f_1(a)}{b_n - a}\right|
                    = \dots
                    = \left|\frac{f_{n-1}(b_n) - f_{n-1}(a)}{b_n - a}\right|
                    = 1\]
                Следовательно
                \[
                    \frac{f(b_n) - f(a)}{b_n - a}
                    = \sum_{i=0}^\infty \frac{f_i(b_n) - f_i(a)}{b_n - a}
                    = \sum_{i=0}^{n-1} \frac{f_i(b_n) - f_i(a)}{b_n - a}
                \]
                --- целое число, совпадающее по чётности с $n$. Если $f'(a)$ определено, то $\lim_{x \to a} \frac{f(x) - f(a)}{x-a}$ сходится, а значит должен сойтись и
                \[\lim_{n \to \infty} \frac{f(b_n) - f(a)}{b_n - a};\]
                но это последовательность целых значений, значит с какого-то момента она должна быть тождественно равна $0$, но это не так, так как нечётных членов бесконечно много в этой последовательности.
            \item (пример Глеба Минаева) Рассмотрим $f_0(x) := x$. Представим её как бесконечную ломанную $\dots \leftrightarrow (-2, -2) \leftrightarrow (-1, -1) \leftrightarrow (0, 0) \leftrightarrow (1, 1) \leftrightarrow (2, 2) \leftrightarrow ...$. Далее будем получать $f_{n+1}$ из $f_n$ следующим образом.
                \todo{Тем более картиночки!!!}

                $f_n$ будет некоторой бесконечной в обе стороны ломанной, при этом всегда $f_n(x) = f_n(x + 1)$. Следующая функция будет получаться заменой ребра $(a_1, b_1) \leftrightarrow (a_2, b_2)$ на три ребра:
                \[
                    (a_1, b_1)
                    \quad \longleftrightarrow \quad
                    \left(\frac{a_1 + 2a_2}{3}, \frac{2b_1 + b_2}{3}\right)
                    \quad \longleftrightarrow \quad
                    \left(\frac{2a_1 + a_2}{3}, \frac{b_1 + 2b_2}{3}\right)
                    \quad \longleftrightarrow \quad
                    (a_2, b_2)
                \]
                Так мы получим $f_{n+1}$. Рассматриваемой же функцией будет $f := \lim_{n \to \infty} f_n$.

                Несложно видеть, что звено высоты $h$ каждый раз заменяется на три ребра: два высоты $2h/3$ и одно высоты $h/3$. При этом описанный прямоугольник любого ребра содержит описанные прямоугольники рёбер, на которые он был заменён, а значит, окажись точка на ребре, из его описанного прямоугольника больше не вылезет. Таким образом после функции $f_n$ разброс положений $f_i(x)$ не более $1/3^n$, поэтому поточечный предел определён.
                
                При этом значения в точках $k/3^m$ с некоторого момента неподвижны: после функции $f_n$ значения во всех точках $k/3^n$ не меняются. Таким образом мы имеем, что во всякой окрестности будут точки вида $k/3^n$, $(3k+1)/3^{n+1}$, $(3k+2)/3^{n+1}$ и $(k+1)/3^n$, а они ломают монотонность функции на данном интервале. Таким образом $f$ нигде не монотонна.

                Также предположим в точке $a$ есть производная. Рассмотрим для каждого $n \in \NN \cup \{0\}$ пару $(p_n, q_n)$, что $p_n$ и $q_n$ --- абсциссы концов звена на котором лежит $(a, f_n(a))$ в ломаной функции $f_n$ ($q_n > p_n$). Тогда заметим, что $q_n-p_n = 1/3^n$, $f_n(p_n) = f(p_n)$, $f_n(q_n) = f(q_n)$, а тогда
                \[
                    \frac{f_n(q_n) - f_n(p_n)}{q_n - p_n}
                    = \frac{f(q_n) - f(p_n)}{q_n - p_n}
                    = \frac{f(q_n) - f(a)}{q_n - a} \cdot \frac{q_n - a}{q_n - p_n} + \frac{f(a) - f(p_n)}{a - p_n} \cdot \frac{a - p_n}{q_n - p_n}
                \]
                Следовательно значение $\frac{f_n(q_n) - f_n(p_n)}{q_n - p_n}$ лежит на отрезке между $\frac{f(q_n) - f(a)}{q_n - a}$ и $\frac{f(p_n) - f(a)}{p_n - a}$; при этом оно является коэффициентом наклона звена $(p_n, f(p_n)) \leftrightarrow (q_n, f(q_n))$.

                Заметим, что звено $(p_n, f(p_n)) \leftrightarrow (q_n, f(q_n))$ будет заменено на три, среди которых будет и $(p_{n+1}, f(p_{n+1})) \leftrightarrow (q_{n+1}, f(q_{n+1}))$. Значит коэффициент наклона $(p_{n+1}, f(p_{n+1})) \leftrightarrow (q_{n+1}, f(q_{n+1}))$ можно получить из коэффициента наклона $(p_n, f(p_n)) \leftrightarrow (q_n, f(q_n))$ домножением либо на $2$, либо на $-1$.

                Таким образом мы имеем, что последовательность
                \[\left(\frac{f_n(q_n) - f_n(p_n)}{q_n - p_n}\right)_{n=0}^\infty\]
                либо расходится по модулю, либо с некоторого момента не меняет модуль, но знакочередуется. При этом если $f'(a)$ определена, то в некоторой окрестности $a$ значение
                \[\frac{f(x) - f(a)}{x-a}\]
                несильно отличается от $f'(a)$ (чем меньше окрестность, тем меньше отличается). Но если мы будем рассматривать точки $p_n$ и $q_n$, то для одной из них (обозначим её за $x_n$) верно, что
                \begin{align*}
                    \left|\frac{f(x_n) - f(a)}{x_n-a}\right| &\geqslant \left|\frac{f_n(q_n) - f_n(p_n)}{q_n - p_n}\right|&
                    \sign\left(\frac{f(x_n) - f(a)}{x_n-a}\right) &= \sign\left(\frac{f_n(q_n) - f_n(p_n)}{q_n - p_n}\right)
                \end{align*}
                Тогда про последовательность
                \[\left(\frac{f_n(x_n) - f_n(a)}{x_n - a}\right)_{n=0}^\infty\]
                с одной стороны можно сказать, что она сходится к $f'(a)$ (т.к. $|p_n - a|$ и $|q_n - a|$ не более $1/3^n$, а следовательно и $|x_n - a|$); с другой же стороны эта последовательность либо неограниченно растёт по модулю, либо с некоторого момента знакочередуется, а значит навряд ли сходится --- противоречие. Значит ни в какой точке $f'$ не определена.
        \end{enumerate}
    \end{proof}

    \section{Интегрирование}

    \subsection{Первообразная}

    \begin{definition}
        $g$ --- \emph{первообразная} функции $f$, если на области определения $f$ верно, что $g' = f$.
    \end{definition}

    \begin{theorem}
        Если $g_1$ и $g_2$ --- первообразные $f$ на отрезке $[a; b]$, то $g_1 - g_2 = \const$ на том же отрезке.
    \end{theorem}

    \begin{proof}
        Очевидно, что $(g_1 - g_2)' = f' - f' = 0$ на отрезке $[a; b]$. Если $g_1 - g_2$ не константна, то есть две точки на отрезке $[a; b]$, в которых принимаются разные значения, а тогда по теореме Лагранжа будет точка строго между ними (а значит и на отрезке), где производная не равна нулю --- противоречие. Следовательно $g_1 - g_2$ является константой.
    \end{proof}

    \begin{remark}
        Для несвязного множества утверждение неверно. Например, если областью определения $f$ будут два отрезка, то $g_1 - g_2$ будет константной на каждом отрезке, но константы могут быть различны.
    \end{remark}

    \begin{definition}
        Семейство первообразных функции $f$ обозначается как
        \[\int f\]
    \end{definition}

    \begin{definition}
        \emph{Линейная форма} --- линейная однородная функция ($f(x) = \alpha x$).
    \end{definition}

    \begin{theorem}\ 
        \begin{multicols}{2}
            \begin{enumerate}
                \item \[\int \alpha f = \alpha\int f + C\]
                \item \[\int f + g = \int f + \int g\]
                \item \[\int f\, dg = fg - \int g\, df\]
                \item \[\int f(\varphi(x))\varphi'(x)dx = \left(\int f\right) \circ \varphi\]
            \end{enumerate}
        \end{multicols}
    \end{theorem}

    \begin{proof}
        \begin{enumerate}
            \item Продифференцируем обе части:
                \[\left(\int \alpha f\right)' = \alpha f = \alpha \left(\int f\right)' = \left(\alpha \int f\right)'\]
                Таким образом обе стороны отличаются на константу: её корректность гарантирует $+ C$.
            
            \item Продифференцируем обе части:
                \[\left(\int f + g\right)' = f + g = \left(\int f\right)' + \left(\int g'\right) = \left(\int f + \int g\right)\]
                Таким образом обе стороны отличаются на константу; эта константа поглощается первообразными слева и справа (так как это семейства функций).
            
            \item Продифференцируем обе части:
                \[\left(\int f\, dg\right)' = f\cdot g' = (fg)' - g\cdot f' = \left(fg - \int g\, df\right)'\]
                Таким образом обе стороны отличаются на константу; эта константа поглощается первообразными слева и справа (так как это семейства функций).
            
            \item Продифференцируем обе части:
                \[\left(\int f(\varphi(x))\varphi'(x)dx\right)' = (f \circ \varphi) \cdot \varphi' = \left(\left(\int f\right) \circ \varphi\right)'\]
                Таким образом обе стороны отличаются на константу; эта константа поглощается первообразными слева и справа (так как это семейства функций).
        \end{enumerate}
    \end{proof}

    \subsection{Суммы Дарбу и интеграл Римана}

    \begin{definition}
        \emph{Разбиение} отрезка $[a; b]$ --- такое семейство $\Sigma := \{I_k\}_{k=1}^n$ отрезков (ненулевой длины), что $[a; b] = \bigcup_{k=1}^n I_k$, и все отрезки из $\Sigma$ попарно пересекаются не более, чем по одной точке.

        Пусть дана функция $f: E \to \RR$, где $E \supseteq [a; b]$, и некоторое разбиение $\Sigma$ отрезка $[a; b]$. Тогда \emph{верхняя и нижняя суммы Дарбу функции $f$ при разбиении $\Sigma$} есть выражения
        \begin{align*}
            S^+(f, \Sigma) &:= \sum_{I \in \Sigma} |I| \cdot \sup_{x \in I} f(x)&
            S^-(f, \Sigma) &:= \sum_{I \in \Sigma} |I| \cdot \inf_{x \in I} f(x)
        \end{align*}
        соответственно. (При этом $\sup$ и $\inf$ могут принимать значения $+\infty$ и $-\infty$ соответственно; и в таких случаях соответствующие суммы Дарбу тоже будут принимать значения $\pm \infty$.)
    \end{definition}

    \begin{example}\ 
        \begin{itemize}
            \item Пусть $f(x) := x^\alpha$, $\alpha > 0$, $[a; b] := [0; 1]$, а $\Sigma := \{[\frac{k-1}{n}; \frac{k}{n}]\}_{k=1}^n$. Тогда
                \begin{gather*}
                    S^+(f, \Sigma) = \sum_{I \in \Sigma} |I| \cdot \sup_{x \in I} f(x) = \sum_{k=1}^n \frac{1}{n} \cdot f\left(\frac{k}{n}\right) = \frac{\sum_{k=1}^n k^\alpha}{n^{\alpha + 1}}\\
                    S^-(f, \Sigma) = \sum_{I \in \Sigma} |I| \cdot \inf_{x \in I} f(x) = \sum_{k=0}^{n-1} \frac{1}{n} \cdot f\left(\frac{k}{n}\right) = \frac{\sum_{k=1}^{n-1} k^\alpha}{n^{\alpha + 1}}
                \end{gather*}

            \item Пусть $f$ --- функция Дирихле, отрезок $[a; b]$ --- любой, и его разбиение $\Sigma$ --- любое. Тогда
                \begin{gather*}
                    S^+(f, \Sigma) = \sum_{I \in \Sigma} |I| \cdot \sup_{x \in I} f(x) = \sum_{I \in \Sigma} |I| \cdot 1 = b - a\\
                    S^-(f, \Sigma) = \sum_{I \in \Sigma} |I| \cdot \inf_{x \in I} f(x) = \sum_{I \in \Sigma} |I| \cdot 0 = 0
                \end{gather*}
        \end{itemize}
    \end{example}

    \begin{lemma}
        Пусть даны функция $f$, отрезка $[a; b]$ и его разбиение $\Sigma$. Назовём его \emph{подразбиением} семейство отрезков $\Sigma'$, которое является объединением разбиений отрезков из $\Sigma$ (иначе говоря, множество концов отрезков $\Sigma$ является подмножеством концов отрезков $\Sigma'$). Тогда верны неравенства
        \begin{align*}
            S^+(f, \Sigma) &\geqslant S^+(f, \Sigma')&
            S^-(f, \Sigma) &\leqslant S^-(f, \Sigma')
        \end{align*}
    \end{lemma}

    \begin{proof}
        Покажем это для верхних сумм Дарбу; для нижних доказательство аналогично.

        Пусть $\{\Lambda_I\}_{I \in \Sigma}$ --- набор разбиений каждого отрезка $I$ из $\Sigma$, что $\Sigma' = \bigcup_{I \in \Sigma} \Lambda_I$. Тогда мы имеем, что для всяких $I \in \Sigma$ и $J \in \Lambda_I$ верно, что
        \[\sup_{x \in I} f(x) \geqslant \sup_{x \in J} f(x)\]
        Следовательно
        \[
            \sum_{J \in \Lambda_I} |J| \cdot \sup_{x \in J} f(x)
            \leqslant \sum_{J \in \Lambda_I} |J| \cdot \sup_{x \in I} f(x)
            = \left(\sum_{J \in \Lambda_I} |J|\right) \cdot \sup_{x \in I} f(x)
            = |I| \cdot \sup_{x \in I} f(x)
        \]
        Значит, суммируя обе части по $\Sigma$, получаем, что
        \[
            S^+(f, \Sigma')
            = \sum_{J \in \Sigma'} |J| \cdot \sup_{x \in J} f(x)
            = \sum_{I \in \Sigma} \sum_{J \in \Lambda_I} |J| \cdot \sup_{x \in J} f(x)
            \leqslant \sum_{I \in \Sigma} |I| \cdot \sup_{x \in I} f(x)
            = S^+(f, \Sigma)
        \]
    \end{proof}

    \begin{lemma}
        Пусть даны функция $f$, отрезок $[a; b]$, его разбиения $\Sigma_1$ и $\Sigma_2$. Тогда
        \[S^+(f, \Sigma_1) \geqslant S^-(f, \Sigma_2)\]
    \end{lemma}

    \begin{proof}
        Рассмотрим
        \[\Sigma := \{I \cap J \mid I \in \Sigma_1 \wedge J \in \Sigma_2 \wedge |I \cap J| > 1\}\]
        --- (минимальное) подразбиение $\Sigma_1$ и $\Sigma_2$. Тогда верно, что
        \[S^+(f, \Sigma_1) \geqslant S^+(f, \Sigma) \geqslant S^-(f, \Sigma) \geqslant S^-(f, \Sigma_2)\]
    \end{proof}

    \begin{corollary}
        Пусть фиксированы функция $f$ и отрезок $[a; b]$. Рассмотрим множества
        \begin{align*}
            D^+ &:= \{S^+(f, \Sigma) \mid \text{$\Sigma$ --- разбиение $[a; b]$}\}&
            D^- &:= \{S^-(f, \Sigma) \mid \text{$\Sigma$ --- разбиение $[a; b]$}\}
        \end{align*}
        Тогда $D^+ \geqslant D^-$.
    \end{corollary}

    \begin{definition}
        Пусть фиксированы функция $f$ и отрезок $[a; b]$, разбиения которого рассматриваются. Если
        \[\sup_{\Sigma} S^-(f, \Sigma) = \inf_{\Sigma} S^+(f, \Sigma) = S,\]
        то тогда $f$ называется \emph{интегрируемой по Риману}, а $S$ называют \emph{интегралом Римана функции $f$ на отрезке $[a;b]$}. Обозначение:
        \[\int_a^b f(x) dx := S\]
    \end{definition}

    \begin{lemma}
        Пусть даны функция $f$ и отрезок $[a; b]$. Тогда если для всякого $\varepsilon > 0$ есть разбиение $\Sigma$ отрезка $[a; b]$, что
        \[\forall I \in \Sigma\quad \osc_I f < \varepsilon\]
        то $f$ интегрируема по Риману на $[a; b]$.
    \end{lemma}

    \begin{proof}
        Обозначим для каждого такого $\varepsilon$ разбиение из условия за $\Sigma_\varepsilon$. Тогда мы имеем, что
        \[
            S^+(f, \Sigma_\varepsilon) - S^-(f, \Sigma_\varepsilon)
            = \sum_{I \in \Sigma} |I| \cdot (\sup_I f - \inf_I f)
            = \sum_{I \in \Sigma} |I| \cdot \osc_I f
            < \varepsilon \cdot \sum_{I \in \Sigma} |I|
            = \varepsilon \cdot (b - a)
        \]
        Т.е. для всякого $\varepsilon > 0$ верно, что
        \[
            \inf_{\Sigma} S^+(f, \Sigma) - \sup_{\Sigma} S^-(f, \Sigma)
            \leqslant S^+(f, \Sigma_{\varepsilon / (b - a)}) - S^-(f, \Sigma_{\varepsilon / (b - a)})
            < \frac{\varepsilon}{b - a} \cdot (b - a)
            = \varepsilon
        \]
        Следовательно
        \[\inf_{\Sigma} S^+(f, \Sigma) = \sup_{\Sigma} S^-(f, \Sigma),\]
        что значит, что $f$ интегрируема по Риману.
    \end{proof}

    \begin{lemma}
        Пусть даны функция $f$ и отрезок $[a; b]$. Тогда $f$ интегрируема по Риману на $[a; b]$ тогда и только тогда, когда для всякого $\varepsilon > 0$ существует $\delta > 0$, что для всякого разбиение $\Sigma$ отрезка $[a; b]$, где $\forall I \in \Sigma\; |I| < \delta$, верно, что
        \[\sum_{I \in \Sigma} |I| \cdot \osc_I f < \varepsilon\]
    \end{lemma}

    \begin{proof}\ 
        \begin{itemize}
            \item[($\Rightarrow$)] Пусть $f$ интегрируема по Риману на $[a; b]$. Тогда для всякого $\varepsilon > 0$ есть разбиения $\Sigma_1$ и $\Sigma_2$ отрезка $[a; b]$, что
                \[\{S^+(f, \Sigma_1); S^-(f, \Sigma_2)\} \subseteq U_{\varepsilon/4}\left(\int_a^b f(x) dx\right)\]
                Пусть $\Sigma$ --- общее подразбиение $\Sigma_1$ и $\Sigma_2$ (например, минимальное). Тогда
                \[S^+(f, \Sigma_1) \geqslant S^+(f, \Sigma) \geqslant S^-(f, \Sigma) \geqslant S^-(f, \Sigma_2)\]
                Следовательно,
                \[\{S^+(f, \Sigma); S^-(f, \Sigma)\} \subseteq U_{\varepsilon/4}\left(\int_a^b f(x) dx\right)\]

                Заметим, что в таком случае $\sup_{[a; b]} f$ и $\inf_{[a; b]} f$ ограничены (равны вещественным значениям, а не $\pm \infty$). Поэтому $A := \osc_{[a; b]} f$ является вещественной величиной. Определим также $L := \min_\Sigma |I|$.

                Пусть $\Lambda$ --- некоторое разбиение $[a; b]$, что длина всякого отрезка не больше $L \cdot \alpha$, где
                \[\alpha := \min\left(1, \frac{\varepsilon \cdot |\Sigma|}{2 \cdot A \cdot L}\right) \in (0; 1].\]
                Тогда мы имеем, что всякий отрезок $I$ из $\Lambda$ либо является подотрезком некоторого отрезка $J_I$ из $\Sigma$ (обозначим множество таких $I$ за $\Gamma$), либо является подотрезком объединения двух соседних отрезков $K_1$ и $K_2$ из $\Sigma$ и содержит их общую границу (обозначим множество таких $I$ за $\Theta$). В случае $I$ и $J$ мы имеем, что $\osc_I f \leqslant \osc_J f$; в случае $I$, $K_1$ и $K_2$ мы имеем, что $\osc_I f \leqslant \osc_{K_1 \cup K_2} f \leqslant A$. Следовательно, используя только что оговоренные оценки,
                \begin{multline*}
                    \sum_{I \in \Lambda} |I| \cdot \osc_I f\\
                    = \sum_{I \in \Gamma} |I| \cdot \osc_I f
                        + \sum_{I \in \Theta} |I| \cdot \osc_I f\\
                    \leqslant \sum_{I \in \Gamma} |I| \cdot \osc_{J_I} f
                        + \sum_{I \in \Theta} |I| \cdot \osc_{[a; b]} f\\
                    \leqslant \sum_{I \in \Sigma} |I| \cdot \osc_{I} f
                        + A \cdot \sum_{I \in \Theta} |I|\\
                    \leqslant \sum_{I \in \Sigma} |I| \cdot \osc_{I} f
                        + A \cdot L \cdot \alpha \cdot |\Theta|\\
                    \leqslant S^+(f, \Sigma) - S^-(f, \Sigma) + A \cdot L \cdot \alpha \cdot |\Sigma|\\
                    < \frac{\varepsilon}{2} + \frac{\varepsilon}{2}\\
                    = \varepsilon
                \end{multline*}
                Таким образом $\delta := L \cdot \alpha$.
            
            \item[($\Leftarrow$)] Пусть для всякого $\varepsilon > 0$ существует $\delta > 0$, что для всякого разбиение $\Sigma$ отрезка $[a; b]$, где $\forall I \in \Sigma\; |I| < \delta$, верно, что
                \[\sum_{I \in \Sigma} |I| \cdot \osc_I f < \varepsilon\]
                Тогда
                \[
                    S^+(f, \Sigma) - S^-(f, \Sigma)
                    = \sum_{I \in \Sigma} |I| \cdot (\sup_I f - \inf_I f)
                    = \sum_{I \in \Sigma} |I| \cdot \osc_I f
                    < \varepsilon
                \]
                Т.е. для всякого $\varepsilon > 0$
                \[\inf_{\Sigma} S^+(f, \Sigma) - \sup_{\Sigma} S^-(f, \Sigma) < \varepsilon\]
                Следовательно
                \[\inf_{\Sigma} S^+(f, \Sigma) = \sup_{\Sigma} S^-(f, \Sigma)\]
                т.е. $f$ интегрируема на $[a; b]$ по Риману.
        \end{itemize}
    \end{proof}

    \begin{theorem}
        Пусть $f$ --- непрерывная на $[a; b]$ функция. Тогда она интегрируема по Риману на $[a; b]$.
    \end{theorem}

    \begin{proof}
        Поскольку $f$ непрерывна на компакте $[a; b]$, то она равномерно непрерывна, т.е.
        \[\forall \varepsilon > 0\, \exists \delta > 0:\; \forall x \in [a; b]\qquad f(U_\delta(x)) \subseteq U_\varepsilon(f(x))\]
        Для каждого такого $\varepsilon$ получаемое $\delta$ обозначим за $\delta_\varepsilon$. Тогда для всякого подотрезка $I$ отрезка $[a; b]$ длины менее $\delta_{\varepsilon/ 2}$ верно, что $\osc_I f < \varepsilon$. Следовательно для любого разбиения $\Sigma$ с шагом не более $\delta_{\varepsilon/2}$ (т.е. $\forall I \in \Sigma\; |I| < \delta_{\varepsilon/2}$) мы имеем, что
        \[\sum_{I \in \Sigma} |I| \cdot \osc_I f < [a; b] \cdot \varepsilon\]
        Поэтому $f$ интегрируема по Риману на $[a; b]$.
    \end{proof}

    \begin{theorem}\ 
        \begin{multicols}{2}
            \begin{enumerate}
                \item \[\int_a^b \lambda dx = \lambda(b-a)\]
                \item Если $f$ интегрируема по Риману на $[a; b]$, то
                    \[f \geqslant 0 \Longrightarrow \int_a^b f(x)dx \geqslant 0\]
                \item Если $f$ и $g$ интегрируемы по Риману на $[a; b]$, то
                    \[\int_a^b (f+g)(x)dx = \int_a^b f(x)dx + \int_a^b g(x)dx\]
                \item Если $f$ интегрируема по Риману на $[a; b]$, то
                    \[\int_a^b \lambda f(x)dx = \lambda \int_a^b f(x)dx\]
                \item $f$ интегрируема по Риману на $[a; b]$ и $[b; c]$ тогда и только тогда, когда на $[c; a]$, и во всех случаях
                    \[\int_a^b f(x)dx + \int_b^c f(x)dx = \int_a^c f(x)dx\]
            \end{enumerate}
        \end{multicols}
    \end{theorem}

    \begin{proof}
        \begin{enumerate}
            \item Очевидно, что для всякого разбиения $\Sigma$ верно, что $S^+(f, \Sigma) = S^-(f, \Sigma) = \lambda (b-a)$, следовательно и интеграл Римана равен $\lambda (b-a)$.
            
            \item Очевидно, что для всякого разбиения $\Sigma$ верно, что $S^-(f, \Sigma) \geqslant 0$, следовательно, если интеграл Римана определён, то он неотрицателен.
            
            \item Очевидно, что для всякого $\varepsilon > 0$ есть разбиения $\Sigma_1$ и $\Sigma_2$, что
                \begin{align*}
                    \sum_{I \in \Sigma_1} |I| \cdot \osc_I f &< \frac{\varepsilon}{2}&
                    \sum_{I \in \Sigma_2} |I| \cdot \osc_I g &< \frac{\varepsilon}{2}
                \end{align*}
                Рассмотрим любое подразбиение $\Sigma$ разбиений $\Sigma_1$ и $\Sigma_2$. Тогда
                \begin{multline*}
                    S^+(f+g, \Sigma)\\
                    = \sum_{I \in \Sigma} |I| \cdot \sup_I (f + g)\\
                    \leqslant \sum_{I \in \Sigma} |I| \cdot \sup_I f + \sum_{I \in \Sigma} |I| \cdot \sup_I g\\
                    \leqslant \sum_{I \in \Sigma_1} |I| \cdot \sup_I f + \sum_{I \in \Sigma_2} |I| \cdot \sup_I g\\
                    = S^+(f, \Sigma_1) + S^+(g, \Sigma_2) 
                \end{multline*}
                Аналогично мы имеем, что $S^-(f+g, \Sigma) \geqslant S^-(f, \Sigma_1) + S^-(g, \Sigma_2)$. Таким образом отметим две важные строки неравенств.
                \begin{gather*}
                    S^+(f, \Sigma_1) + S^+(g, \Sigma_2) \geqslant S^+(f+g, \Sigma) \geqslant S^-(f+g, \Sigma) \geqslant S^-(f, \Sigma_1) + S^-(g, \Sigma_2)\\
                    S^+(f, \Sigma_1) + S^+(g, \Sigma_2) \geqslant \int_a^b f(x)dx + \int_a^b g(x)dx \geqslant S^-(f, \Sigma_1) + S^-(g, \Sigma_2)\\
                \end{gather*}
                Так мы получаем, что $S^+(f+g, \Sigma)$, $S^-(f+g, \Sigma)$ и $\int_a^b f(x)dx + \int_a^b g(x)dx$ --- три числа с отрезка
                \[\left[S^-(f, \Sigma_1) + S^-(g, \Sigma_2); S^+(f, \Sigma_1) + S^+(g, \Sigma_2)\right],\]
                длина которого меньше $\varepsilon$. Следовательно $f + g$ интегрируема по Риману на $[a; b]$, и интеграл равен
                \[\int_a^b f(x)dx + \int_a^b g(x)dx\]

            \item Докажем сначала для $\lambda \geqslant 0$. Для всякого $\varepsilon > 0$ есть разбиение $\Sigma$ отрезка $[a; b]$, что
                \[\sum_{I \in \Sigma} |I| \cdot \osc_I f < \varepsilon\]
                Также имеем, что
                \begin{gather*}
                    S^+(\lambda f, \Sigma) = \sum_{I \in \Sigma} |I| \cdot \sup_I \lambda f = \sum_{I \in \Sigma} |I| \cdot \lambda \cdot \sup_I f = \lambda \sum_{I \in \Sigma} |I| \cdot \sup_I f = \lambda S^+(f, \Sigma)\\
                    S^-(\lambda f, \Sigma) = \sum_{I \in \Sigma} |I| \cdot \inf_I \lambda f = \sum_{I \in \Sigma} |I| \cdot \lambda \cdot \inf_I f = \lambda \sum_{I \in \Sigma} |I| \cdot \inf_I f = \lambda S^-(f, \Sigma)\\
                \end{gather*}
                Следовательно
                \begin{gather*}
                    S^+(\lambda f, \Sigma) = \lambda S^+(f, \Sigma) \geqslant \lambda \int_a^b f(x)dx \geqslant \lambda S^-(f, \Sigma) = S^-(\lambda f, \Sigma)\\
                    S^+(\lambda f, \Sigma) - S^-(\lambda f, \Sigma) < \lambda \varepsilon
                \end{gather*}
                Таким образом интеграл $\lambda f$ по Риману на $[a; b]$ определён и равен $\lambda \int_a^b f(x)dx$.

                Теперь покажем для $\lambda = -1$. Заметим, что
                \begin{gather*}
                    S^+(-f, \Sigma) = \sum_{I \in \Sigma} |I| \cdot \sup_I -f = \sum_{I \in \Sigma} |I| \cdot -\inf_I f = -\sum_{I \in \Sigma} |I| \cdot \inf_I f = -S^-(f, \Sigma)\\
                    S^-(-f, \Sigma) = \sum_{I \in \Sigma} |I| \cdot \inf_I -f = \sum_{I \in \Sigma} |I| \cdot -\sup_I f = -\sum_{I \in \Sigma} |I| \cdot \sup_I f = -S^+(f, \Sigma)\\
                \end{gather*}
                Следовательно
                \begin{gather*}
                    S^+(-f, \Sigma) = -S^-(f, \Sigma) \geqslant -\int_a^b f(x)dx \geqslant -S^+(f, \Sigma) = S^-(-f, \Sigma)\\
                    S^+(-f, \Sigma) - S^-(-f, \Sigma) < \varepsilon
                \end{gather*}
                Таким образом интеграл $-f$ по Риману на $[a; b]$ определён и равен $-\int_a^b f(x)dx$.

                Используя доказанные утверждения получаем, что для всякого $\lambda$ верно, что
                \begin{multline*}
                    \int_a^b \lambda f(x)dx\\
                    = \int_a^b \sign(\lambda)\cdot |\lambda| f(x)dx\\
                    = \sign(\lambda) \int_a^b |\lambda| f(x)dx\\
                    = \sign(\lambda)|\lambda| \int_a^b f(x)dx\\
                    = \lambda \int_a^b f(x)dx
                \end{multline*}

            \item Если $f$ интегрируема по Риману на некотором отрезке $I$, то $\sup_I f$ и $\inf_I f$ равны некоторым вещественным значениям (не $\pm\infty$).

                Таким образом пусть $f$ интегрируема по Риману на $[a; b]$ и $[b; c]$. Тогда для всякого $\varepsilon > 0$ есть разбиения $\Sigma_L$ отрезка $[a; b]$ и $\Sigma_R$ отрезка $[b; c]$, что
                \begin{align*}
                    \sum_{I \in \Sigma_L} |I| \osc_I f &< \frac{\varepsilon}{2}&
                    \sum_{I \in \Sigma_R} |I| \osc_I f &< \frac{\varepsilon}{2}
                \end{align*}
                Следовательно, если определить $\Sigma := \Sigma_L \cup \Sigma_R$,
                \[
                    S^+(f, \Sigma)
                    = \sum_{I \in \Sigma} |I| \sup_I f
                    = \sum_{I \in \Sigma_L} |I| \sup_I f + \sum_{I \in \Sigma_R} |I| \sup_I f
                    \geqslant \int_a^b f(x)dx + \int_b^c f(x)dx
                \]
                По аналогии получаем, что
                \[S^+(f, \Sigma) \geqslant \int_a^b f(x)dx + \int_b^c f(x)dx \geqslant S^-(f, \Sigma)\]
                При этом
                \[
                    S^+(f, \Sigma) - S^-(f, \Sigma)
                    = \sum_{I \in \Sigma} |I| \osc_I f
                    = \sum_{I \in \Sigma_L} |I| \osc_I f + \sum_{I \in \Sigma_R} |I| \osc_I f
                    < \varepsilon
                \]
                Таким образом $f$ интегрируема по Риману на $[a; c]$, а интеграл равен $\int_a^b f(x)dx + \int_b^c f(x)dx$.

                Пусть теперь $f$ интегрируема на $[a; c]$. Тогда для всякого разбиения $\Sigma$ отрезка $[a; c]$ мы можем рассмотреть
                \[\Sigma_L := \{I \cap [a; b] \mid I \in \Sigma\}\]
                и тогда
                \[\sum_{I \in \Sigma_L} |I| \cdot \osc_I f < \sum_{I \in \Sigma} |I| \cdot \osc_I f\]
                Следовательно есть разбиения со сколь угодно маленькой осцелляцией $f$ на них, а значит $f$ интегрируема на $[a; b]$; аналогично и на $[b; c]$. А по предыдущим рассуждениям достигается равенство в тождестве интегралов.
        \end{enumerate}
    \end{proof}

    \begin{theorem}
        Пусть дана функция $f$, а $M = \sup_{[a; b]} |f|$. Тогда
        \[\left|\int_a^b f(x)dx\right| \leqslant M(b-a)\]
    \end{theorem}

    \begin{proof}
        Очевидно, что при разбиении $\Sigma := \{[a; b]\}$
        \begin{align*}
            S^+(f, \Sigma) = \sum_{I \in \Sigma} |I| \cdot \sup_I f = (b - a) \cdot \sup_{[a; b]} f\\
            S^-(f, \Sigma) = \sum_{I \in \Sigma} |I| \cdot \inf_I f = (b - a) \cdot \inf_{[a; b]} f
        \end{align*}
        Следовательно
        \[
            M(b-a)
            \geqslant (b-a)\sup_{[a; b]} f
            = S^+(f, \Sigma)
            \geqslant \int_a^b f(x)dx
            \geqslant S^-(f, \Sigma)
            = (b-a)\sup_{[a; b]} f
            \geqslant -M(b-a)
        \]
        откуда следует требуемое.
    \end{proof}

    \begin{theorem}
        Пусть $f: [a; b] \to \RR$ непрерывна. Тогда
        \[F: [a, b] \to \RR, x \mapsto \int_a^x f(t)dt\]
        является первообразной $f$.
    \end{theorem}

    \begin{proof}
        Рассмотрим какие-то $x$ и $y$, что $a \leqslant x < y \leqslant b$. Тогда
        \[F(y) - F(x) = \int_a^y f(t)dt - \int_a^x f(t)dt = \int_x^y f(t)dt\]
        Следовательно, рассматривая $\Sigma := \{[x; y]\}$,
        \begin{gather*}
            \sum_{I \in \Sigma} |I| \cdot \inf_I f \leqslant F(y) - F(x) \leqslant \sum_{I \in \Sigma} |I| \cdot \sup_I f\\
            (y-x) \cdot \inf_{[x; y]} f \leqslant F(y) - F(x) \leqslant (y-x) \cdot \sup_{[x; y]} f\\
            \inf_{[x; y]} f \leqslant \frac{F(y) - F(x)}{y - x} \leqslant \sup_{[x; y]} f
        \end{gather*}
        Немного меняя обозначения, получаем, что для всякого $\varepsilon \in [a - x; b - x] \setminus \{0\}$
        \[\inf_{U_{|\varepsilon|}(x)} f \leqslant \frac{F(x + \varepsilon) - F(x)}{\varepsilon} \leqslant \sup_{U_{|\varepsilon|}(x)} f\]

        Заметим, что по непрерывности $f$
        \[\lim_{\varepsilon \to 0^+} \inf_{U_\varepsilon(x)} f = \lim_{\varepsilon \to 0^+} \sup_{U_\varepsilon(x)} f = f(x)\]
        Следовательно
        \[\lim_{\varepsilon \to 0} \frac{F(x + \varepsilon) - F(x)}{\varepsilon} = f(x)\]
        Иначе говоря $F'(x) = f(x)$. Таким образом $F' = f$.
    \end{proof}

    \begin{corollary}[формула Ньютона-Лейбница]
        Пусть $F$ --- первообразная непрерывной на $[a; b]$ функции $f$. Тогда
        \[\int_a^b f(x)dx = F(b) - F(a)\]
    \end{corollary}

    \begin{proof}
        Заметим, что $G(x) := \int_a^x f(t)dt$ --- первообразная $f$. Следовательно $G(x) - F(x) = C$ на $[a; b]$. Значит
        \[\int_a^b f(x)dx = G(b) = G(b) - G(a) = (F(b) + C) - (F(a) + C) = F(b) - F(a)\]
    \end{proof}

    \begin{theorem}
        Пусть $(f_n)_{n=0}^\infty$ --- последовательность интегрируемых по Риману на $[a; b]$ функций --- равномерно сходится к $f$. Тогда $f$ интегрируема по Риману на $[a; b]$, и
        \[\int_a^b f(x)dx = \lim_{n \to \infty} \int_a^b f_n(x)dx\]
    \end{theorem}

    \begin{proof}
        Заметим, что для всякого $\varepsilon > 0$ есть $N \in \NN \cup \{0\}$, что для всяких $n, m > N$ верно, что
        \[|f_n - f_m| \leqslant \frac{\varepsilon}{3(b-a)};\]
        следовательно для всякого $n > N$ верно, что
        \[|f - f_n| \leqslant \frac{\varepsilon}{3(b-a)}.\]
        При этом существует такое разбиение $\Sigma$ отрезка $[a; b]$, что
        \[S^+(f_{N+1}, \Sigma) - S^-(f_{N+1}, \Sigma) = \sum_{I \in \Sigma} |I| \cdot \osc_I f < \frac{\varepsilon}{3}.\]
        Таким образом для всякого $n > N$ верно, что
        \begin{multline*}
            S^+(f_n, \Sigma)\\
            = \sum_{I \in \Sigma} |I| \cdot \sup_I f_n
            \leqslant \sum_{I \in \Sigma} |I| \cdot \left(\frac{\varepsilon}{3(b-a)} + \sup_I f_{N+1}\right)
            = \frac{\varepsilon}{3} + \sum_{I \in \Sigma} |I| \cdot \sup_I f_{N+1}\\
            = S^+(f_{N+1}, \Sigma) + \frac{\varepsilon}{3};
        \end{multline*}
        аналогично $S^-(f_n, \Sigma) \geqslant S^-(f_{N+1}, \Sigma) - \frac{\varepsilon}{3}$. Аналогично данные утверждения верны и для $f$ (вместо $f_n$).

        Заметим, что
        \begin{gather*}
            S^+(f_{N+1}, \Sigma) + \frac{\varepsilon}{3} \geqslant S^+(f_n, \Sigma) \geqslant \int_a^b f_n(x)dx \geqslant S^-(f_n, \Sigma) \geqslant S^-(f_{N+1}, \Sigma) - \frac{\varepsilon}{3}\\
            S^+(f_{N+1}, \Sigma) + \frac{\varepsilon}{3} \geqslant S^+(f, \Sigma) \geqslant S^-(f, \Sigma) \geqslant S^-(f_{N+1}, \Sigma) - \frac{\varepsilon}{3}\\
        \end{gather*}
        Таким образом
        \[S^+(f, \Sigma) - S^-(f, \Sigma) < \varepsilon,\]
        следовательно $f$ интегрируема по Риману на $[a; b]$. Таким образом мы имеем, что
        \begin{gather*}
            S^+(f_{N+1}, \Sigma) + \frac{\varepsilon}{3} \geqslant \int_a^b f_n(x)dx \geqslant S^-(f_{N+1}, \Sigma) - \frac{\varepsilon}{3}\\
            S^+(f_{N+1}, \Sigma) + \frac{\varepsilon}{3} \geqslant \int_a^b f(x)dx \geqslant S^-(f_{N+1}, \Sigma) - \frac{\varepsilon}{3}\\
        \end{gather*}
        т.е. для всех $n > N$
        \[\left|\int_a^b f_n(x)dx - \int_a^b f(x)dx\right| < \varepsilon\]
        Это значит, что
        \[\int_a^b f(x)dx = \lim_{n \to \infty} \int_a^b f_n(x)dx\]
    \end{proof}

    \begin{lemma}
        Если $f$ и $g$ интегрируемы по Риману на $[a; b]$, то и $f \cdot g$.
    \end{lemma}

    \begin{proof}
        Заметим, что, поскольку $f$ и $g$ интегрируемы по Риману, есть такие константы $C_f > 0$ и $C_g > 0$, что $|f| \leqslant C_f$ и $|g| \geqslant C_g$. Следовательно для всяких $x, y \in [a; b]$
        \begin{multline*}
            |f(x)g(x) - f(y)g(y)|\\
            \leqslant |f(x)g(x) - f(y)g(x)| + |f(y)g(x) - f(y)g(y)|\\
            \leqslant C_g|f(x) - f(y)| + C_f|g(x) - g(y)|
        \end{multline*}
        Значит для всякого отрезка $I$ верно, что
        \[\osc_I f\cdot g \leqslant C_g \osc_I f + C_f \osc_I g\]
        Следовательно для всякого разбиения $\Sigma$ отрезка $[a; b]$
        \[\sum_{I \in \Sigma} |I| \cdot \osc_I f \cdot g \leqslant C_g \sum_{I \in \Sigma} |I| \cdot \osc_I f + C_f \sum_{I \in \Sigma} |I| \cdot \osc_I g\]
        Вспомним, что для всякого $\varepsilon > 0$ есть разбиения $\Sigma_f$ и $\Sigma_g$ отрезка $[a; b]$, что
        \begin{align*}
            \sum_{I \in \Sigma_f} |I| \cdot \osc_I f &< \frac{\varepsilon}{2C_g}&
            \sum_{I \in \Sigma_f} |I| \cdot \osc_I g &< \frac{\varepsilon}{2C_f}
        \end{align*}
        Рассмотрим общее подразбиение $\Sigma$ разбиений $\Sigma_f$ и $\Sigma_g$. Для него верны предыдущие предыдущие неравенства. Следовательно
        \[\sum_{I \in \Sigma} |I| \cdot \osc_I f \cdot g < C_g \frac{\varepsilon}{2C_g} + C_f \frac{\varepsilon}{2C_f} = \varepsilon\]
        Поэтому $f \cdot g$ интегрируема по Риману на $[a; b]$.
    \end{proof}

    \subsection{У чего есть выражаемая первообразная?}

    \begin{lemma}\ 
        \begin{multicols}{2}
            \begin{enumerate}
                \item \[\int 0 = C\]
                \item \[\int a_0 + \dots + a_n x^n = C + a_0 x + \dots + a_n x^{n+1} \]
                \item $\forall \alpha \in \RR \setminus \{0\}, x > 0$
                    \[\int x^\alpha = \frac{x^{\alpha + 1}}{\alpha + 1} + C\]
                \item $\forall x > 0$
                    \[\int \frac{1}{x} = \log(x) + C\]
                \item \[\int e^x = e^x + C\]
                \item \[\int \sin(x) = -\cos(x) + C\] \[\int \cos(x) = \sin(x) + C\]
                \item \[\int \frac{1}{1+x^2} = \arctan(x) + C\]
                \item \[\int \frac{1}{\sqrt{1-x^2}} = \arcsin(x) + C\]
            \end{enumerate}
        \end{multicols}
    \end{lemma}

    \begin{theorem}
        Следующие виды функций имеют выражаемую первообразную:
        \begin{multicols}{2}
            \begin{enumerate}
                \item рациональные функции;
                \item рациональные функции от $\sin$ и $\cos$;
                \item рациональные функции от $\sinh$ и $\cosh$;
                \item рациональные функции от $x$ и $\sqrt{ax^2 + bx + 1}$, где $a \neq 0$.
            \end{enumerate}
        \end{multicols}
    \end{theorem}

    \begin{proof}
        \begin{enumerate}
            \item Каждая рациональная функция представляется в виде суммы полиномов, членов вида $\frac{k}{(x+a)^n}$ и членов вида $\frac{px+q}{(ax^2+bx+c)^n}$. Покажем, что каждый из них имеет выражаемую первообразную.
                \begin{itemize}
                    \item Первообразная многочлена очевидна.
                    \item \[\int \frac{dx}{x+a} = \log(x+a) + C\]
                    \item Для всякого $n > 1$
                        \[\int \frac{dx}{(x+a)^n} = \frac{1}{(1-n)(x-a)^{n-1}} + C\]
                    \item \[\int \frac{dx}{x^2+1} = \tg^{-1}(x)\]
                    \item \[\int \frac{xdx}{x^2+1} = \frac{1}{2} \log(x^2+1)\]
                    \item Для всякого $n > 1$
                        \[\int \frac{xdx}{(x^2 + 1)^n} = \frac{1}{2(1-n)(x^2+1)^{n-1}}\]
                    \item Заметим, что
                        \[
                            \left(\frac{x}{(x^2+1)^n}\right)'
                            = \frac{1}{(x^2+1)^n} - 2n \frac{x^2}{(x^2+1)^{n+1}}
                            = \frac{2n}{(x^2+1)^{n+1}} - \frac{2n-1}{(x^2+1)^n}
                        \]
                        Следовательно
                        \[\int \frac{dx}{(x^2+1)^{n+1}} = \frac{x}{2n(x^2+1)^n} + \frac{2n-1}{2n} \int \frac{dx}{(x^2+1)^n}\]
                        Таким образом несложно понять по индукции, что $\int \frac{dx}{(x^2+1)^n}$ для $n > 1$ есть некоторая сумма рациональных функций и $\tg^{-1}$.
                    \item Линейными заменами задача нахождения первообразных у $\frac{1}{(x+a)^n}$ и $\frac{px+q}{(ax^2+bx+c)^n}$ сводится к нахождению первообразных $\frac{1}{x^n}$, $\frac{1}{(x^2+1)^n}$ и $\frac{x}{(x^2+1)^n}$.
                \end{itemize}

            \item Заметим, что
                \begin{align*}
                    \sin(x) &= \frac{2 \tg(x/2)}{1 + \tg(x/2)^2}&
                    \cos(x) &= \frac{1 - \tg(x/2)^2}{1 + \tg(x/2)^2}&
                    dx &= \frac{2d(\tg(x/2))}{1+\tg(x/2)^2}
                \end{align*}
                Следовательно задача сводится к нахождению первообразной рациональной функции при помощи подстановки $t := \tg(x)$.

            \item С одной стороны это можно свести к предыдущей задаче заменой $t := ix$. С другой стороны можно заметить, что
                \begin{align*}
                    \sinh(x) &= \frac{e^x - e^{-x}}{2}&
                    \cosh(x) &= \frac{e^x + e^{-x}}{2}&
                    dx &= \frac{d(e^x)}{e^x}
                \end{align*}
                Следовательно задача сводится к нахождению первообразной рациональной функции при помощи подстановки $t := e^x$.
            
            \item Линейными подстановками можно свести задачу к нахождению первообразной рациональной функции от $y$ и $\sqrt{\pm y^2 \pm 1}$ или только от $y$.
                \begin{itemize}
                    \item Случай рациональной функции только от $y$ уже был разобран.
                    \item Если нам дана рациональная функция от $y$ и $\sqrt{1 - y^2}$, то заменой $t := \sin^{-1}(y)$ она сводится к рациональной функции от $\sin$ и $\cos$.
                    \item Если нам дана рациональная функция от $y$ и $\sqrt{y^2 - 1}$, то заменой $t := \cosh^{-1}(y)$ она сводится к рациональной функции от $\sinh$ и $\cosh$.
                    \item Если нам дана рациональная функция от $y$ и $\sqrt{1 + y^2}$, то заменой $t := \sinh^{-1}(y)$ она сводится к рациональной функции от $\sinh$ и $\cosh$.
                \end{itemize}
        \end{enumerate}
    \end{proof}

    \section{Логарифм? Полезности?? Что???}

    \begin{theorem}
        Пусть дана функция $\varphi: (0; +\infty) \to \RR$, что
        \begin{itemize}
            \item $\forall a, b \in (0; +\infty)\quad \varphi(ab) = \varphi(a) + \varphi(b)$
            \item $\varphi$ монотонна.
        \end{itemize}
        Тогда верны следующие утверждения:
        \begin{enumerate}
            \item $\varphi(1) = 0$;
            \item $\forall a \in (0; +\infty), n \in \NN\quad \varphi(a^n) = n\varphi(a)$;
            \item $\forall a \in (0; +\infty)\quad \varphi(a^{-1}) = -\varphi(a)$;
            \item $\forall a \in (0; +\infty), q \in \QQ\quad \varphi(a^q) = q\varphi(a)$;
            \item $\varphi$ непрерывна;
            \item $\varphi$ бесконечно дифференцируема;
            \item $\varphi'(x) = \frac{C}{x}$ для некоторого $C \in \RR$;
            \item все такие $\phi$ имеют вид $\int_1^x \frac{Cdt}{t}$ для некоторого $C > 0$ и наоборот: любая такая функция удовлетворяет условиям на $\phi$.
        \end{enumerate}
    \end{theorem}

    \begin{proof}\ 
        \begin{enumerate}
            \item \[\varphi(1) = \varphi(1 \cdot 1) - \varphi(1) = \varphi(1) - \varphi(1) = 0.\]

            \item Докажем по индукции по $n$. База при $n=0$ и $n=1$ очевидна. Весь шаг:
                \[\varphi(a^{n+1}) = \varphi(a^n) + \varphi(a) = n\varphi(a) + \varphi(a) = (n+1)\varphi(a).\]

            \item \[\varphi(a^{-1}) = \varphi(a \cdot a^{-1}) - \varphi(a) = \varphi(1) - \varphi(a) = -\varphi(a)\]

            \item Пусть $q = \frac{kn}{m}$, где $n, m \in \NN \setminus \{0\}$, а $k = \pm 1$. Тогда
                \[m\varphi(a^q) = \varphi(a^{mq}) = \varphi(a^{kn}) = k\varphi(a^n) = kn\varphi(a) = qm\varphi(a)\]
                Следовательно
                \[\varphi(a^q) = q\varphi(a)\]

            \item Заметим, что $\lim_{x \to 1^+} \varphi(x)$ в связи с монотонностью и ограниченностью (например, значением $\varphi(1/2)$) функции $\varphi$ определён, значит равен некоторому $b$. Тогда
                \[2b = \lim_{x \to 1^+} 2\varphi(x) = \lim_{x \to 1^+} \varphi(x^2) = \lim_{y \to 1^+} \varphi(y) = b\]
                значит $b = 0$. Следовательно
                \[\lim_{x \to 1^-} \varphi(x) = - \lim_{x \to 1^-} \varphi(x^{-1}) = - \lim_{y \to 1^+} \varphi(y) = 0\]
                Таким образом $\varphi$ непрерывна в $1$. Следовательно
                \[\lim_{x \to y} \varphi(x) = \varphi(y) + \lim_{x \to y} \varphi\left(\frac{x}{y}\right) = \varphi(y) + \lim_{\alpha \to 1} \varphi(\alpha)\]
                т.е. $\varphi$ непрерывна во всех точках $(0; +\infty)$.

            \item Рассмотрим
                \[\Phi := (0; +\infty) \to \RR, x \mapsto \int_1^x \varphi(t)dt\]
                Тогда мы имеем, что $\Phi$ --- первообразная, поскольку $\varphi$ непрерывна. А тогда если $\varphi$ имеет $n \in \NN \cup \{0\}$ производных, то $\Phi$ имеет $n+1$ производную. Заметим, что
                \[
                    \Phi(2x) - \Phi(x)
                    = \int_x^{2x}\varphi(t)dt
                    = x\int_x^{2x}\varphi\left(\frac{t}{x}\right)d\frac{t}{x} + x\int_x^{2x}\varphi(x)d\frac{t}{x}
                    = Cx + \varphi(x)x
                \]
                где
                \[C := \int_1^2 \varphi(t)dt\]
                Следовательно
                \[\varphi(x) = \frac{\Phi(2x) - \Phi(x)}{x} - C\]
                Таким образом, если $\Phi$ имеет $n \in \NN \cup \{0\}$ производных, то $\varphi$ тоже. Значит $\Phi$ и $\phi$ бесконечно дифференцируемы.

            \item Пусть фиксировано некоторое $y > 0$. Следовательно
                \[y\varphi'(xy) = (\varphi(xy))' = (\varphi(x) + \varphi(y))' = \varphi'(x)\]
                а значит, если подставить $y = x^{-1}$
                \[\varphi'(x) = \frac{\varphi'(1)}{x}\]
                Таким образом определяя $C := \varphi'(1)$ имеем, что
                \[\varphi'(x) = \frac{C}{x}\]
            
            \item Действительно, если есть некоторое $\phi$, то $\phi$ и $\int_1^x \frac{\phi'(1)dt}{t}$ являются первообразными $\frac{\phi'(1)}{x}$, значит отличаются на константу. При этом в $1$ они обе равны $0$, значит функции совпадают.

                Теперь же покажем, что $\psi(x) := \int_1^x \frac{Cdt}{t}$ является корнем функционального уравнения.
                \begin{itemize}
                    \item Поскольку $\frac{C}{x}$ --- функция одного знака, то $\psi$ монотонна.
                    \item
                        \begin{align*}
                            \psi(xy)
                            &= \int_1^{xy} \frac{Cdt}{t}&
                            &= \int_1^x \frac{Cdt}{t} + \int_x^{xy} \frac{Cdt}{t}\\
                            &= \int_1^x \frac{Cdt}{t} + \int_x^{xy} \frac{Cd(t/x)}{(t/x)}&
                            &= \int_1^x \frac{Cdt}{t} + \int_1^{y} \frac{Cds}{s}&
                            &= \psi(x) + \psi(y)
                        \end{align*}
                \end{itemize}
        \end{enumerate}
    \end{proof}

    \begin{definition}
        \emph{Натуральный логарифм} --- функция
        \[\log: (0; +\infty) \to \RR, x \mapsto \int_1^x \frac{dt}{t}\]

        \emph{Экспонента} --- функция
        \[\exp: \RR \to (0; +\infty), x \mapsto \log^{-1}(x)\]
    \end{definition}

    \begin{theorem}\ 
        \begin{enumerate}
            \item $\exp$ корректно определена;
            \item $\exp$ непрерывна;
            \item $\exp$ бесконечно дифференцируема, и каждая производная $\exp$ равна $\exp$;
            \item $\exp(0) = 1$;
            \item $\exp(a + b) = \exp(a) \cdot \exp(b)$;
            \item \[\exp(x) = \sum_{n=0}^\infty \frac{x^n}{n!};\]
        \end{enumerate}
    \end{theorem}

    \begin{proof}\ 
        \begin{enumerate}
            \item Поскольку $\log$ монотонна, то всякое значение из области значений $\log$ --- всё $\RR$ --- принимается единожды. Следовательно $\exp$ корректно определена.
            \item Поскольку всякая монотонная биекция из интервала в интервал является непрерывной функцией, то и $\exp$ непрерывна на всяком интервале.
            \item По свойству дифференцирования
                \[\exp'(x) = \frac{1}{\log'(\exp(x))} = \frac{1}{1/\exp(x)} = \exp(x)\]
                Таким образом $\exp$ дифференцируема раз, и при дифференцировании не меняется. Следовательно $\exp$ бесконечно дифференцируема.
            \item Следует из того, что $\log(1) = 0$.
            \item Следует из того, что $\log(xy) = \log(x) + \log(y)$, подстановкой $x := \log^{-1}(a)$ и $y := \log^{-1}(b)$.
            \item Вспомним, что по теореме \ref{finite_Teylor_series_theorem_engineers_variation} для всякого $x \in \RR$ и $n \in \NN \cup \{0\}$ есть $\xi_n \in (0; x)$, что
                \[
                    \exp(x)
                    = \sum_{k=0}^n \frac{\exp^{(k)}(0)}{k!}(x-0)^k + \frac{\exp^{(n+1)}(\xi_n)}{(n+1)!}(x-0)^{n+1}
                    = \sum_{k=0}^n \frac{x^k}{k!} + \frac{\exp(\xi_n)x^{n+1}}{(n+1)!}
                \]
                Вспомним также, что
                \[
                    \sum_{k=0}^\infty \frac{x^k}{k!} \stackrel{\mathrm{def}}{=} \lim \left(\sum_{k=0}^n \frac{x^k}{k!}\right)_{n=0}^\infty
                \]
                Чтобы показать, что предел этой последовательности реально совпадает с $\exp(x)$, покажем, что
                \[\lim \left(\exp(x) - \sum_{k=0}^n \frac{x^k}{k!}\right)_{n=0}^\infty = 0\]
                Действительно,
                \[
                    \left(\left|\exp(x) - \sum_{k=0}^n \frac{x^k}{k!}\right|\right)_{n=0}^\infty
                    = \left(\left|\frac{\exp(\xi_n)x^{n+1}}{(n+1)!}\right|\right)_{n=0}^\infty
                    \leqslant \exp(|x|) \left(\frac{|x|^{n+1}}{(n+1)!}\right)_{n=0}^\infty
                \]
                Пусть $N = \lceil 2|x| \rceil$. Тогда для всякого $n \geqslant N$ имеем, что
                \[
                    \frac{|x|^{n+1}}{(n+1)!}
                    = \frac{|x|^N}{N!} \prod_{k=N+1}^{n+1} \frac{|x|}{k}
                    < \frac{|x|^N}{N!} \prod_{k=N+1}^{n+1} \frac{1}{2}
                    = \frac{|x|^N\cdot 2^{N-1}}{N!} \cdot \frac{1}{2^n}
                \]
                Следовательно, с момента $N$ последовательность
                \[\left(\frac{|x|^{n+1}}{(n+1)!}\right)_{n=0}^\infty\]
                сходится к $0$ немедленнее, чем геометрическая прогрессия, а тогда и
                \[\lim \left(\exp(x) - \sum_{k=0}^n \frac{x^k}{k!}\right)_{n=0}^\infty = 0\]
        \end{enumerate}
    \end{proof}

    \begin{theorem}
        \begin{multicols}{2}
            \begin{enumerate}
                \item $(a^x)' = \log(a)a^x$
                \item $(x^a)' = a x^{a-1}$
            \end{enumerate}
        \end{multicols}
    \end{theorem}

    \begin{proof}\ 
        \begin{enumerate}
            \item \[(a^x)' = \exp(x \log(a))' = (x \log(a))' a^x = \log(a) a^x\]
            \item \[(x^a)' = \exp(a \log(x))' = (a \log(x))' x^a = \frac{a}{x} x^a = a x^{a-1}\]
        \end{enumerate}
    \end{proof}

    \begin{theorem}
        Пусть $f: (a; b) \to \RR$, $f^{(1)}$, \dots, $f^{(n+1)}$ определены на $(t-\delta; t+\delta)$ для некоторого $\delta > 0$. Тогда для всякого $x \in U_\delta(t)$
        \[f(x) = \sum_{k=0}^n \frac{f^{(k)}(t)}{k!}(x-t)^k + \frac{1}{n!} \int_t^x f^{(n+1)}(s)(x-s)^n ds\]
    \end{theorem}

    \begin{proof}
        Рассмотрим функцию
        \[g(x) := f(x) - \sum_{k=0}^n \frac{f^{(k)}(t)}{k!}(x-t)^k\]
        Тогда мы имеем, что $g(t) = g'(t) = g^{(2)}(t) = \dots = g^{(n)}(t) = 0$, а $g^{(n+1)}(t) = f^{(n+1)}(t)$. Тогда нужно показать, что
        \[g(x) = \frac{1}{n!} \int_t^x f^{(n+1)}(s)(x-s)^n ds\]
        Докажем это по индукции по $n$.

        \textbf{База.} $n=0$. Тогда
        \[g(x) = g(t) + \int_t^x g'(s)ds = \frac{1}{n!} \int_t^x g^{(n+1)}(s) (x-s)^n ds\]

        \textbf{Шаг.} Пусть утверждение верно для $n$ докажем для $n+1$.
        \begin{align*}
            g(x)
            &= \frac{1}{n!} \int_t^x g^{(n+1)}(s) (x-s)^n ds\\
            &= \left. \frac{-g^{(n+1)}(s) (x-s)^{(n+1)}}{(n+1)!} \right|_x^t
                + \frac{1}{(n+1)!}\int_t^x g^{(n+1)}(s)(x-s)^{(n+1)} ds\\
            &= \frac{1}{(n+1)!}\int_t^x g^{(n+1)}(s)(x-s)^{(n+1)} ds
        \end{align*}
    \end{proof}

    \begin{theorem}[формула Валлиса]
        \[
            \frac{\pi}{2}
            = \lim_{n \to \infty} \left(\frac{(2n)!!}{(2n-1)!!}\right)^2 \cdot \frac{1}{2n+1}
            = \frac{2}{1} \cdot \frac{2}{3} \cdot \frac{4}{3} \cdot \frac{4}{5} \cdot \frac{6}{5} \cdot \cdots
        \]
    \end{theorem}

    \begin{proof}
        Для всякого $n \in \NN \cup \{0\}$ определим
        \[I_n := \int_0^{\pi/2} \sin(x)^n dx\]
        Заметим, что $\sin(x)^{n+1} < \sin(x)^n$ на $(0; \frac{\pi}{2})$, следовательно $I_n > I_{n+1}$. Также
        \begin{align*}
            I_{n+2}
            &= \int_0^{\pi/2} \sin(x)^{n+2} dx\\
            &= \left. -\sin(x)^{n+1}\cos(x) \right|_0^{\pi/2} + (n+1)\int_0^{\pi/2} \sin(x)^n \cos(x)^2 dx\\
            &= (n+1)\int_0^{\pi/2} \sin(x)^n (1 - \sin(x)^2) dx\\
            &= (n+1)\int_0^{\pi/2} \sin(x)^n dx - (n+1)\int_0^{\pi/2} \sin(x)^{n+2} dx\\
            &= (n+1) I_n - (n+1) I_{n+2}
        \end{align*}
        Следовательно $I_{n+2} = \frac{n+1}{n+2} I_n$. При этом понятно, что $I_0 = \frac{\pi}{2}$, а $I_1 = 1$. Таким образом для всякого $n \in \NN \cup \{0\}$
        \begin{align*}
            &I_{2n}&
            &= \frac{2n-1}{2n} \cdot \frac{2n-3}{2n-2} \cdot \cdots \cdot \frac{1}{2} \cdot I_0&
            &= \frac{(2n-1)!!}{(2n)!!} \frac{\pi}{2}\\
            &I_{2n+1}&
            &= \frac{2n}{2n+1} \cdot \frac{2n-2}{2n-1} \cdot \cdots \cdot \frac{2}{3} \cdot I_1&
            &= \frac{(2n)!!}{(2n+1)!!}
        \end{align*}
        Вспомним, что $I_{2n+1} < I_{2n} < I_{2n-1}$. Следовательно
        \begin{gather*}
            \frac{(2n)!!}{(2n+1)!!} < \frac{\pi}{2} \cdot \frac{(2n-1)!!}{(2n)!!} < \frac{(2n-2)!!}{(2n-1)!!}\\
            \frac{(2n)!! \cdot (2n)!!}{(2n-1)!! \cdot (2n+1)!!} < \frac{\pi}{2} < \frac{(2n)!! \cdot (2n-2)!!}{(2n-1)!! \cdot (2n-1)!!}\\
            \left(\frac{(2n)!!}{(2n-1)!!}\right)^2 \frac{1}{2n+1} < \frac{\pi}{2} < \left(\frac{(2n)!!}{(2n-1)!!}\right)^2 \frac{1}{2n}\\
        \end{gather*}
        Заметим, что в последнем неравенстве отношение значений слева и справа сходится к $1$, значит
        \[\lim_{n \to \infty} \left(\frac{(2n)!!}{(2n-1)!!}\right)^2 \frac{1}{2n+1} = \frac{\pi}{2}\]
    \end{proof}

    \newpage
    \section{Мои придумки вне лекций}

    \subsection{Сильная сходимость}

    \begin{definition}
        Ряд $\sum_{n=0}^\infty a_n$ \emph{сильно сходится}, если сходится ряд $\sum_{n=0}^\infty |a_n|$.
    \end{definition}

    \begin{remark*}
        Это понятие распространяемо не только на $\RR$, а также на $\CC$. Также можно пытаться рассматривать любые векторные пространства с метрикой и полнотой.
    \end{remark*}

    \begin{lemma}
        Если ряд сильно сходится, то он сходится.
    \end{lemma}

    \begin{proof}
        Пусть дан сильно сходящийся ряд $\sum_{n=0}^\infty a_n$.

        \begin{thlemma}
            Для всякого $\varepsilon > 0$ есть такое натуральное $N$, что $\sum_{n=N}^\infty |a_n| < \varepsilon$.
        \end{thlemma}

        \begin{proof}
            Давайте рассматривать префиксные суммы $S_n$ последовательности $(a_n)_{n=0}^\infty$. Заметим, что $(S_n)_{n=0}^\infty$ является неубывающей сходящейся последовательностью. Пусть предел равен $A$. Следовательно какой-то член $S_N$ будет в $\varepsilon$-окрестности $A$ (а на деле в $(A - \varepsilon; A]$). Следовательно для всякого $M > N$
            \[\sum_{n=N+1}^M |a_n| = S_M - S_N\]
            При этом $\{S_n - S_N\}_{n=N+1}^\infty$ сходится, поскольку равна $\{S_n\}_{n=N+1}^\infty - S_N$, а последняя последовательность сходится, поскольку является подпоследовательностью сходящейся $\{S_n\}_{n=0}^\infty$. При этом $S_M \in [S_N; A]$. Следовательно $S_M - S_N \in [0; A - S_N]$. Значит и предел $\{S_n - S_N\}_{n=N+1}^\infty$ лежит в $[0; A - S_N] \subseteq [0; \varepsilon)$.
        \end{proof}

        Заметим, что $|x_1 + \dots + x_n| \leqslant |x_1| + \dots + |x_n|$. Пусть $\{S_n\}_{n=0}^\infty$ --- префиксные суммы последовательности $(a_n)_{n=0}^\infty$, т.е. $S_n := \sum_{k=0}^n a_k$. Мы видим, что
        \[|S_M - S_N| = \left|\sum_{n=N+1}^M a_k\right| \leqslant \sum_{n=N+1}^M |a_k| \leqslant \sum_{n=N+1}^\infty |a_k|\]
        Пусть $r_N := \sum_{n=N+1}^\infty |a_k|$. Тогда имеем, что все $S_M$ для $M \geqslant N$ лежат в $\overline{U}_{r_N}(S_N)$. Несложно видеть, что
        \[\left(\overline{U}_{r_N}(S_N)\right)_{N=0}^\infty\]
        --- последовательность замкнутых убывающих по включение множеств. Также по доказанной лемме последовательность $(r_N)_{N=0}^\infty$ сходится монотонно сверху к нулю. Таким образом последовательность замкнутых окрестностей имеет предел и ровно один. Поскольку размеры окрестностей сходятся, а для каждой из них верно, что все члены последовательности $(S_N)_{N=0}^\infty$ начиная с некоторого лежат в этой окрестности, то сама последовательность $(S_N)_{N=0}^\infty$ сходится к пересечению окрестностей.
    \end{proof}

    \begin{theorem}
        Пусть дан ряд $f(x) := \sum_{n=0}^\infty a_n x^n$.
        \begin{enumerate}
            \item Если $f$ сходится в некоторой точке $t$, то сильно сходится и во всех точках $s$, где $|s| < |t|$.
            \item При этом если $f$ сильно сходится в точке $t$, то сильно сходится и во всех точках $s$, где $|s| \leqslant |t|$.
        \end{enumerate}
    \end{theorem}

    \begin{proof}\ 
        \begin{enumerate}
            \item Мы имеем, что ряд $\sum_{n=0}^\infty a_n t^n$ сходится. Это значит, что есть такая константа $C$, что для всякого $n \in \NN \cup \{0\}$ верно, что $|a_n t^n| \leqslant C$, а тогда $|a_n| \cdot |t|^n \leqslant C$.

                Пусть дана точка $s$, что $|s| < |t|$. Тогда
                \[|a_n s^n| = |a_n| \cdot |s|^n = |a_n| \cdot |t|^n \cdot \left(\frac{|s|}{|t|}\right)^n \leqslant C \cdot \left(\frac{|s|}{|t|}\right)^n\]

                Обозначим $|s|/|t|$ за $\alpha$. Понятно, что $\alpha \in [0; 1)$. Следовательно
                \[\sum_{n=0}^N |a_n s^n| \leqslant \sum_{n=0}^N C \alpha^n = C \frac{1 - \alpha^{N+1}}{1 - \alpha}\]
                Таким образом $C \frac{1 - \alpha^{N+1}}{1 - \alpha}$ сходится, а значит сходится и последовательность префиксных сумм $(|a_n s^n|)_{n=0}^\infty$ (так как не убывает и ограничена). Следовательно $f$ в точке $s$ сильно сходится.
            
            \item Мы имеем, что ряд $\sum_{n=0}^\infty a_n t^n$ сильно сходится. Пусть дана точка $s$, что $|s| \leqslant |t|$. Тогда
            \[|a_n s^n| = |a_n| \cdot |s|^n \leqslant |a_n| \cdot |t|^n = |a_n t^n|\]
            Следовательно
            \[\sum_{n=0}^N |a_n s^n| \leqslant \sum_{n=0}^N |a_n t^n|\]
            Поскольку последовательность правых сумм сходится, то сходится и последовательность левых сумм (так как не убывает и ограничена). Следовательно $f$ в точке $s$ сильно сходится.
        \end{enumerate}
    \end{proof}

    \begin{corollary}
        Область определения $f$ как функция есть открытый шар с центром в нуле (возможно, нулевого радиуса) вместе с некоторыми (возможно, со всеми, возможно, с ни одной) точками на его границе.
    \end{corollary}

    \begin{theorem}
        Пусть дана функция $f(t) := \sum_{n=0}^\infty a_n t^n$. Пусть также известно, что для некоторой точки $x$ и некоторого $\varepsilon > 0$ верно, что $f$ сильно сходится в $\varepsilon$-окрестности $x$.
        \begin{enumerate}
            \item Значит $f$ бесконечно дифференцируема в этой окрестности.
            \item Для всякого $k \in \NN \cup \{0\}$ функция $g_k(t) := \sum_{n=0}^\infty a_{n+k}\frac{(n+k)!}{n!}x^n$ сильно сходится в этой окрестности.
            \item Для всякого $k \in \NN \cup \{0\}$ верно, что $f^{(k)} = g_k$ в этой окрестности.
        \end{enumerate}
    \end{theorem}

    \begin{proof}
        Докажем все утверждения теоремы только для первой производной $f$. Тогда для всех последующих производных утверждение будет выводится по индукции. Также докажем сильную сходимость не для всей окрестности, а только для точки $x$. Тогда, поскольку все другие точки окрестности так же имеют окрестность, где функция сильно сходится, то для них утверждение будет верно по аналогии.

        Возьмём некоторую точку $y$ из $\varepsilon$-окрестности $x$, что $|y| > |x|$. Заметим, что
        \[\sum_{n=0}^\infty |a_{n+1} (n+1) x^n| = \sum_{n=0}^\infty |a_{n+1} y^n| \cdot (n+1) \left(\frac{|x|}{|y|}\right)^n\]
        Понятно, что $|x|/|y| < 1$; обозначим $\alpha := |x|/|y|$. Поэтому последовательность $\{(n+1)\alpha^n\}_{n=0}^\infty$ сходится к нулю, а значит ограничена сверху константой $C$. Следовательно
        \[\sum_{n=0}^\infty |a_{n+1} (n+1) x^n| \leqslant \sum_{n=0}^\infty C|a_{n+1} y^n| = C \sum_{n=0}^\infty |a_{n+1} y^n|,\]
        причём последний ряд сходится по условию, значит и первый тоже, что означает сильную сходимость $g_1$ в $x$. Таким образом $g_1$ сильно сходится на всей $\varepsilon$-окрестности $x$.

        Теперь покажем, что $f'(x) = g_1(x)$. Пусть $t$ --- любая точка из проколотой $\varepsilon$-окрестности $x$. Тогда
        \[
            \frac{f(t)-f(x)}{t-x} - g_1(x)
            = \sum_{n=0}^\infty a_{n+1}\left(\frac{t^{n+1} - x^{n+1}}{t - x} - (n+1) x^n\right)
        \]
        Заметим, что
        \[
            \frac{t^{n+1} - x^{n+1}}{t - x} - (n+1) x^n
            = t^n + t^{n-1}x + \cdots + tx^{n-1} - n x^n
            = (t-x)(t^{n-1} + 2t^{n-2}x + \cdots + n x^{n-1})
        \]
        Следовательно
        \[\frac{f(t)-f(x)}{t-x} - g_1(x) = (t-x) \sum_{n=0}^\infty a_{n+2}(t^n + 2t^{n-1}x + \cdots + (n+1) x^n)\]
        Пусть $y$ некоторая точка из $\varepsilon$-окрестности $x$, что $\delta := |y| - |x| > 0$. Тогда для всякой точки $t$ из проколотой $\delta/2$-окрестности $x$
        \begin{align*}
            |a_{n+2}(t^n + 2t^{n-1}x + \cdots + (n+1) x^n)|
            &= |a_{n+2}| (|t|^n + 2|t|^{n-1}|x| + \cdots + (n+1) |x|^n)\\
            &< |a_{n+2}| (|y| - \delta/2)^n \frac{(n+1)(n+2)}{2}\\
            &= |a_{n+2} y^{n+2}| \cdot |y|^{-2} \frac{(n+1)(n+2)}{2} \left(\frac{|y|-\delta/2}{|y|}\right)^n
        \end{align*}
        Тогда мы имеем, что $\alpha := \frac{|y|-\delta/2}{|y|} \in [0; 1)$, а тогда последовательность $\left(|y|^{-2} \frac{(n+1)(n+2)}{2} \alpha^n\right)_{n=0}^\infty$ сходится к $0$, значит ограничена сверху константой $C$. Следовательно
        \[
            \sum_{n=0}^\infty |a_{n+2}(t^n + 2t^{n-1}x + \cdots + (n+1) x^n)|
            < \sum_{n=0}^\infty |a_{n+2} y^{n+2}| C
            = C \sum_{n=0}^\infty |a_{n+2} y^{n+2}|
        \]
        причём последняя сумма сходится, так как $f$ сильно сходится в $y$. Тогда пусть
        \[A := C \sum_{n=0}^\infty |a_{n+2} y^{n+2}|.\]
        Значит
        \begin{align*}
            \left|\frac{f(t)-f(x)}{t-x} - g_1(x)\right|
            &= |t-x| \left|\sum_{n=0}^\infty a_{n+2}(t^n + 2t^{n-1}x + \cdots + (n+1) x^n)\right|\\
            &\leqslant |t-x| \sum_{n=0}^\infty |a_{n+2}(t^n + 2t^{n-1}x + \cdots + (n+1) x^n)|\\
            &< |t-x| A
        \end{align*}
        Таким образом
        \[\lim_{t \to x} \frac{f(t) - f(x)}{t-x} = g_1(x),\]
        что значит, что $f'(x) = g_1(x)$.
    \end{proof}

    \begin{lemma}
        Пусть $\sum_{n=0}^\infty a_n$ --- сильно сходящийся ряд, а $(\sum_{n=0}^\infty b_{k,n})_{k=0}^\infty$ --- последовательность сильно сходящихся рядов, что есть некоторая константа $C$, что для всякого $n \in \NN \cup \{0\}$ верно, что
        \begin{itemize}
            \item $(b_{k, n})_{k=0}^\infty \to a_n$,
            \item $|b_{k, n}| \leqslant C |a_n|$ для всякого $k \in \NN \cup \{0\}$,
            \item для всякого $k \in \NN \cup \{0\}$ ряд $\sum_{n=0}^\infty b_{k,n}$ сходится и равен $B_k$.
        \end{itemize}
        Тогда $\sum_{n=0}^\infty a_n = \lim_{k \to \infty} B_k$.
    \end{lemma}

    \begin{proof}
        Вспомним, что для всякого $\varepsilon > 0$ есть $N \in \NN \cup \{0\}$, что $\sum_{n=N}^\infty |a_n| < \frac{\varepsilon}{2(C+1)}$. Следовательно для всякого $k \in \NN \cup \{0\}$
        \begin{align*}
            \left|\sum_{n=0}^\infty a_n - \sum_{n=0}^\infty b_{k, n}\right|
            &\leqslant \sum_{n=0}^{N-1} |a_n - b_{k,n}| + \sum_{n=N}^\infty |a_n| + |b_{k, n}|&
            &\leqslant \sum_{n=0}^{N-1} |a_n - b_{k,n}| + (C+1) \sum_{n=N}^\infty |a_n|\\
            &< \sum_{n=0}^{N-1} |a_n - b_{k,n}| + (C+1) \frac{\varepsilon}{2(C+1)}&
            &= \sum_{n=0}^{N-1} |a_n - b_{k,n}| + \frac{\varepsilon}{2}\\
        \end{align*}
        Тогда для всякого $n \in [0; N-1]$ есть такое $K_n$, что для всех $k \geqslant K_n$ верно, что $|a_n - b_{k, n}| < \frac{\varepsilon}{2N}$. Пусть $K := \max\{K_n\}_{n=0}^{N-1}$. Следовательно для всякого $k \geqslant K$
        \[
            \left|\sum_{n=0}^\infty a_n - \sum_{n=0}^\infty b_{k, n}\right|
            < \sum_{n=0}^{N-1} |a_n - b_{k,n}| + \frac{\varepsilon}{2}
            < \sum_{n=0}^{N-1} \frac{\varepsilon}{2N} + \frac{\varepsilon}{2}
            = \varepsilon
        \]
        Таким образом для всякого $k \geqslant K$
        \[\left|\sum_{n=0}^\infty a_n - B_k\right| < \varepsilon,\]
        что значит, что
        \[\lim_{k \to \infty} B_k = \sum_{n=0}^\infty a_n\]
    \end{proof}

    \begin{theorem}
        Пусть рассматривается ряд вещественных чисел.
        \begin{enumerate}
            \item Если ряд сильно сходится, то любая его перестановка сильно сходится к тому же значению.
            \item Если ряд сходится, но не сильно, то для любого значения $S$ можно так переставить его члены, что итоговый ряд сойдётся к $S$.
        \end{enumerate}
    \end{theorem}

    \begin{proof}\ 
        \begin{enumerate}
            \item Пусть даны сильно сходящийся ряд $\sum_{n=0}^\infty a_n$ и любая его перестановка $\sum_{n=0}^\infty a_{\sigma(n)}$. Заметим, что для всякого $\varepsilon > 0$ есть $N \in \NN \cup \{0\}$, что $\sum_{n=N}^\infty |a_n| < \varepsilon$. Значит есть $M := \sigma^{-1}(N)$. Следовательно для всякого $L \geqslant M$
            \[
                \left|\sum_{n=0}^L a_{\sigma(n)} - \sum_{n=0}^\infty a_n\right|
                \leqslant \sum_{n=N+1}^\infty |a_n| < \varepsilon
            \]
            Таким образом $\sum_{n=0}^\infty a_{\sigma(n)} = \sum_{n=0}^\infty a_n$; чтобы показать сильную сходимость достаточно проделать те же рассуждения для ряда $\sum_{n=0}^\infty |a_{\sigma(n)}|$.

            \item Заметим, что сходимость ряда значит, что $(|a_n|)_{n=0}^\infty \to 0$. При этом несильная сходимость значит, что сумма положительных членов ряда и сумма отрицательных расходятся (уходят в $+\infty$ и $-\infty$ соответственно).
            
            Выделим из последовательности $(a_n)_{n=0}^\infty$ подпоследовательности положительных членов $(a^+_n)_{n=0}^\infty$ и отрицательных $(a^-_n)_{n=0}^\infty$ (в основной последовательности могут быть нули, но их можно раскидывать по последовательности как угодно --- это не изменит результат). Заметим, что для всякого $\varepsilon > 0$ есть $N(\varepsilon) \in \NN \cup \{0\}$, что для всякого $n \geqslant N$ верно, что $|a^+_n| < \varepsilon$ и $|a^-_n| < \varepsilon$.

            Теперь составим нашу последовательность-перестановку следующим образом. В начальный момент у нас есть число $t = 0$ и пустая последовательность $T$. Также у нас есть две бесконечных последовательности: одна с положительными --- $A^+$, другая с отрицательными членами --- $A^-$. Суммы обеих бесконечны. Каждый ход мы можем только взять из одной из $A^+$ и $A^-$ первое невзятое число, записать его в конец последовательности $T$ и прибавить это число к $t$.
            
            Какие же конкретно ходы мы хотим делать? Пусть $C_0$ --- максимальный размер членов в $A^+$ и $A^-$, а $C := \max(C_0, S+1)$. Тогда на итерации № $n$ ($n \in \NN \cup \{0\}$) будем удерживаться в интервале $(S-C/2^n; S+C/2^n)$. Чтобы достичь этой цели будем использовать следующую стратегию:
            \begin{itemize}
                \item если $t \in (S-C/2^n; S]$, то возьмём число из $A^+$;
                \item если $t \in [S; S-C/2^n)$, то возьмём число из $A^-$.
            \end{itemize}
            Заканчивать итерацию № $n$ будем, когда из обеих последовательностей будут вытащены первые $N(C/2^{n+1})$ членов. Поскольку к началу каждой итерации № $n$ оставшиеся члены в $A^+$ и $A^-$ по модулю меньше $C/2^n$, то описанными выше шагами мы не выйдем из интервала $(S-C/2^n; S+C/2^n)$. При этом после каждого момента мы не можем вытаскивать только члены одной из $A^+$ и $A^-$, так как их суммы стремятся к $\pm \infty$, значит каждый член каждой последовательности будет взят, а следовательно мы попадём на каждую итерацию. Значит префиксные суммы будут сходится к $S$ (после конца каждой итерации префиксные суммы будут находиться во всё меньшей окрестности $S$). Таким образом сумма полученного ряда-перестановки будет равна $S$.
        \end{enumerate}
    \end{proof}

    \subsection*{Практика}

    Рассмотрим функцию $\exp(x) := \sum_{n=0}^\infty \frac{x^n}{n!}$.

    \begin{statement}
        $\exp(x)$ сильно сходится при любых $x \in \CC$.
    \end{statement}

    \begin{proof}
        Действительно, есть натуральное $N > 2|x|$, значит
        \begin{align*}
            \sum_{n=0}^\infty \left|\frac{x^n}{n!}\right|
            &= \sum_{n=0}^{N-1} \left|\frac{x^n}{n!}\right| + \frac{|x|^N}{N!}\sum_{n=0}^\infty \frac{|x|^n}{(N+n)!/N!}&
            &< \sum_{n=0}^{N-1} \left|\frac{x^n}{n!}\right| + \frac{|x|^N}{N!}\sum_{n=0}^\infty \frac{|x|^n}{N^n}\\
            &< \sum_{n=0}^{N-1} \left|\frac{x^n}{n!}\right| + \frac{|x|^N}{N!}\sum_{n=0}^\infty \frac{1}{2^n}&
            &= \sum_{n=0}^{N-1} \left|\frac{x^n}{n!}\right| + 2\frac{|x|^N}{N!}\\
        \end{align*}
        т.е. $\exp$ сильно сходится во всякой точке $x \in \CC$.
    \end{proof}

    \begin{corollary}
        $\exp$ бесконечно дифференцируема. Причём $\exp'(x) = \sum_{n=0}^\infty \frac{(n+1)x^n}{(n+1)!} = \exp(x)$.
    \end{corollary}

    \begin{statement}
        Для абсолютных любых $a$ и $b$ верно, что $\exp(a+b) = \exp(a) \cdot \exp(b)$.
    \end{statement}

    \begin{proof}
        Давайте рассмотрим следующую таблицу.
        \begin{table}[h]
            \centering
            \begin{tabular}{c||c|c|c|c|c|c}
                \raisebox{-14pt}{\rule{0pt}{36pt}}$\times$ & $\dfrac{a^0}{0!}$ & $\dfrac{a^1}{1!}$ & $\dfrac{a^2}{2!}$ & $\dfrac{a^3}{3!}$ & $\dfrac{a^4}{4!}$ & $\cdots$\\
                \hline
                \hline
                \raisebox{-14pt}{\rule{0pt}{36pt}}$\dfrac{b^0}{0!}$ & $\dfrac{a^0 \cdot b^0}{0! \cdot 0!}$ & $\dfrac{a^1 \cdot b^0}{1! \cdot 0!}$ & $\dfrac{a^2 \cdot b^0}{2! \cdot 0!}$ & $\dfrac{a^3 \cdot b^0}{3! \cdot 0!}$ & $\dfrac{a^4 \cdot b^0}{4! \cdot 0!}$ & $\cdots$\\
                \hline
                \raisebox{-14pt}{\rule{0pt}{36pt}}$\dfrac{b^1}{1!}$ & $\dfrac{a^0 \cdot b^1}{0! \cdot 1!}$ & $\dfrac{a^1 \cdot b^1}{1! \cdot 1!}$ & $\dfrac{a^2 \cdot b^1}{2! \cdot 1!}$ & $\dfrac{a^3 \cdot b^1}{3! \cdot 1!}$ & $\dfrac{a^4 \cdot b^1}{4! \cdot 1!}$ & $\cdots$\\
                \hline
                \raisebox{-14pt}{\rule{0pt}{36pt}}$\dfrac{b^2}{2!}$ & $\dfrac{a^0 \cdot b^2}{0! \cdot 2!}$ & $\dfrac{a^1 \cdot b^2}{1! \cdot 2!}$ & $\dfrac{a^2 \cdot b^2}{2! \cdot 2!}$ & $\dfrac{a^3 \cdot b^2}{3! \cdot 2!}$ & $\dfrac{a^4 \cdot b^2}{4! \cdot 2!}$ & $\cdots$\\
                \hline
                \raisebox{-14pt}{\rule{0pt}{36pt}}$\dfrac{b^3}{3!}$ & $\dfrac{a^0 \cdot b^3}{0! \cdot 3!}$ & $\dfrac{a^1 \cdot b^3}{1! \cdot 3!}$ & $\dfrac{a^2 \cdot b^3}{2! \cdot 3!}$ & $\dfrac{a^3 \cdot b^3}{3! \cdot 3!}$ & $\dfrac{a^4 \cdot b^3}{4! \cdot 3!}$ & $\cdots$\\
                \hline
                \raisebox{-14pt}{\rule{0pt}{36pt}}$\dfrac{b^4}{4!}$ & $\dfrac{a^0 \cdot b^4}{0! \cdot 4!}$ & $\dfrac{a^1 \cdot b^4}{1! \cdot 4!}$ & $\dfrac{a^2 \cdot b^4}{2! \cdot 4!}$ & $\dfrac{a^3 \cdot b^4}{3! \cdot 4!}$ & $\dfrac{a^4 \cdot b^4}{4! \cdot 4!}$ & $\cdots$\\
                \hline
                $\vdots$ & $\vdots$ & $\vdots$ & $\vdots$ & $\vdots$ & $\vdots$ & $\ddots$\\
            \end{tabular}
        \end{table}
        В ней в верхней строчке написаны подряд члены ряда $\exp(a)$, в левом столбце --- $\exp(b)$, а в пересечении строки и столбца --- произведение соответствующих членов. Тогда легко видеть, что
        \[\exp(a)\exp(b) = \lim_{n \to \infty} \left(\sum_{k=0}^\infty \frac{a^k}{k!}\right)\left(\sum_{k=0}^\infty \frac{b^k}{k!}\right).\]
        В таком случае при раскрытии скобок мы получаем члены, находящиеся в угловом квадрате $n+1 \times n+1$ таблицы. Если же раскрыть скобки у члена $(a+b)^n/n!$ ряда $\exp(a+b)$, то получим члены на диагонали № $n$ (без учёта особой строки и особого столбца); значит при раскрытии скобок у префиксных суммы в $\exp(a+b)$ мы получаем члены в равнобедренных прямоугольных треугольниках размера $n$, вписанных прямым углом в угол таблицы. Тогда получаем, что $\exp(a)\exp(b)$ и $\exp(a+b)$ --- просто суммы рядов, членами которых являются значения из таблицы, но в разных порядках.

        Заметим, что ряд членов из таблицы сильно сходится. Действительно, давайте в последовательность подряд выпишем члены диагоналей. Получается, что эо всё та же последовательность $\exp(a+b)$, но только в ней каждый член разбили сразу на несколько подряд идущих. Заметим, что для всякого $n \in \NN \cup \{0\}$
        \[\frac{(|a|+|b|)^n}{n!} = \frac{|a|^n \cdot |b|^0}{n! \cdot 0!} + \cdots + \frac{|a|^0 \cdot |b|^n}{0! \cdot n!}.\]
        Соответственно выписанный ряд сильно сходится, поскольку сильно сходится ряд $\exp(|a| + |b|)$. Значит при любых перестановках мы получим одно и то же, значит $\exp(a + b) = \exp(a) \exp(b)$.
    \end{proof}

    \begin{statement}
        \[\lim_{n \to \infty} \left(1 + \frac{x}{n}\right)^n = \exp(x)\]
    \end{statement}

    \begin{proof}
        Давайте раскроем скобки в подпредельном выражении:
        \[
            \left(1 + \frac{x}{n}\right)^n
            = 1 + \frac{\binom{n}{1}}{n}x + \frac{\binom{n}{2}}{n^2}x^2 + \cdots + \frac{\binom{n}{n}}{n^n}x^n
        \]
        Заметим, что при всяком $n \in \NN \cup \{0\}$ член степени $k$ равен
        \[\frac{\binom{n}{k}}{n^k}x^k = \frac{x^k}{k!} \cdot \frac{n}{n} \cdot \frac{n-1}{n} \cdot \cdots \cdot \frac{n-k+1}{n}\]
        Тогда видно, что для всякого $k$
        \[\lim_{n \to \infty} \frac{\binom{n}{k}}{n^k}x^k = \frac{x^k}{k!}\]
        и заодно
        \[\left|\frac{\binom{n}{k}}{n^k}x^k\right| < \left|\frac{x^k}{k!}\right|\]
        Следовательно по теореме мы получаем, что последовательность рядов, почленно сходящихся к $\exp(x)$ и почленно ограниченных по модулю константой $C=1$. Значит предел определён и равен $\exp(x)$.
    \end{proof}

    \begin{exercise}
        Рассмотрим функцию \[f(x) := x \prod_{n=1}^\infty \left(1 - \frac{x^2}{n^2}\right)\]
        \begin{enumerate}
            \item Покажите, что $f$ сходится при любом $x$.
            \item Покажите, что $f(x+1) = -f(x)$.
            \item Превратите $f$ в ряд и покажите, что $f$ сильно сходится при всяком $x$ (осторожно, аккуратная возня с суммами и не очень аккуратная с оценками).
        \end{enumerate}
    \end{exercise}

    \begin{remark*}
        До синуса недалеко, но я пока не знаю, что нужно сделать, чтобы окончательно доказать, что это он\dots
    \end{remark*}

    \begin{exercise}
        Рассмотрим функции
        \begin{align*}
            &s: \RR \to \RR, x \mapsto \frac{\exp(xi) - \exp(-xi)}{2i}&
            &\text{ и }&
            &c: \RR \to \RR, x \mapsto \frac{\exp(xi) + \exp(-xi)}{2}
        \end{align*}
        \begin{enumerate}
            \item Покажите, что $s$ и $c$ определены корректно, т.е. они действительно возвращают вещественные значения.
            \item Докажите, что $s(x+y) = s(x)c(y) + s(y)c(x)$ и $c(x+y) = c(x)c(y) - s(y)s(x)$ для всевозможных $x, y \in \RR$.
            \item Докажите, что $c(x)^2 + s(x)^2 = 1$ для всевозможных $x \in \RR$.
            \item Докажите, что $s$ и $c$ --- бесконечно дифференцируемые функции.
            \item Докажите, что $s'(x) = c(x)$ и $c'(x) = -s(x)$ для всевозможных $x \in \RR$.
            \item Докажите, что $\lim_{x \to 0} \frac{s(x)}{x} = 1$.
            \item Докажите, что $s(x) = \sin(x)$, а $c(x) = \cos(x)$.
        \end{enumerate}
    \end{exercise}
\end{document}