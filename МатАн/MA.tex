\documentclass[12pt,a4paper]{article}
\usepackage{math-text}

\title{Математический анализ --- 1.}
\author{\href{https://vk.com/ybelov}{Юрий Сергеевич Белов}}
\date{}

\DeclareMathOperator{\Quot}{Quot}
\DeclareMathOperator{\osc}{osc}

\begin{document}
    \maketitle

    Литература:
    \begin{itemize}
        \item В. А. Зорич ``Математический анализ''
        \item О. Л. Виноградов ``Математический анализ''
        \item (подходит попозже) Г. М. Фихтенгельц ``Курс дифференциального и интегрального исчисления''
        \item У. Рудин ``Основы анализа''
        \item М. Спивак ``Математический анализ на многообразиях''
    \end{itemize}

    \section{Множества, аксиоматика и вещественные числа.}

    Мы начинаем с теории множеств.

    \begin{definition}\ 
        \begin{itemize}
            \item Множества и элемменты --- понятно.
            \item $a \in B$ --- понятно.
            \item $A \cup B := \{x \mid x\in A \vee x\in B\}$ --- объединение.
            \item $A \cap B := \{x \mid x\in A \wedge x\in B\}$ --- пересечение.
            \item $A \setminus B := \{x \mid x\in A \vee x\notin B\}$ --- разность.
            \item $A \bigtriangleup B := A \setminus B \cup B \setminus A$ --- симметрическая разница.
            \item $A^C := X\setminus A$ --- \emph{дополнение}, где $X$ --- некоторое фиксированное рассматриваемое множество.
            \item $A \subset B$ --- ``$A$ --- подмножество $B$'', т.е. $\forall x (X\in A \Rightarrow x\in B)$.
        \end{itemize}
    \end{definition}

    \begin{corollary*}\ 
        \begin{itemize}
            \item (первое правило Моргана) $(A\cup B)^C = A^C \cap B^C$.
                \begin{align*}
                    x\in (A\cup B)^C \Leftrightarrow
                    x \notin A \cup B \Leftrightarrow
                    \left\{ \begin{aligned}
                        &x \notin A\\
                        &x \notin B
                    \end{aligned}\right. \Leftrightarrow
                    \left\{ \begin{aligned}
                        &x \in A^c\\
                        &x \in B^C
                    \end{aligned} \right. \Leftrightarrow
                    x \in A^C \cap B^C
                \end{align*}
            \item (второе правило Моргана) $(A\cap B)^C = A^C \cup B^C$. Аналогично.
        \end{itemize}
    \end{corollary*}

    \begin{definition}
        (Аксиома индукции.) Пусть есть функция $A: \NN \to {true;false}$, что:
        \begin{enumerate}
            \item $A(1)=true$;
            \item $\forall n (A(n) \rightarrow A(n+1))$.
        \end{enumerate}
        Тогда $\forall n A(n)$.
    \end{definition}

    Определение натуральных чисел сложно, рассматривать его не будем. Важно также иметь в виду натуральные числа с операциями сложения и умножения.

    \begin{definition}
        Пусть есть кольцо без делителей нуля $R$. Рассмотрим отношение эквивалентности $\sim$ на $R \times (R\setminus \{0\})$, что $(a; b) \sim (c; d) \Leftrightarrow ad = bc$. Тогда $\Quot(R)$ --- фактор-множество по $\sim$ и поле.
    \end{definition}

    \begin{definition}
        Рациональные числа --- $\QQ := \Quot(\ZZ)$.
    \end{definition}

    \begin{theorem}
        $\nexists x\in \QQ, x^2 = 2$.
    \end{theorem}

    \begin{proof}
        Предположим противное, т.е. существуют взаимно простиые $m \in \ZZ$ и $n \in \NN\setminus\{0\}$, что $(\frac{m}{n})^2 = 2$. Тогда $m^2 = n^2$. Очевидно, что тогда $m^2 \vdots 2$, значит $m\vdots 2$, значит $m\vdots 4$, значит $n^2 \vdots 2$, значит $n \vdots 2$, значит $n$ и $m$ не взаимно просты, так как делятся на $2$ --- противоречие.
    \end{proof}

    Теперь мы хотим понять, что есть вещественные числа. Тут есть несколько подходов.

    \begin{definition}[аксиоматический подход]
        Вещественные числа --- это полное упорядоченное поле $\RR$, состоящее не из одного элемента.
        
        Здесь ``поле'' значит, что на множестве (вместе с его операциями и выделенными элементами) верны аксиомы поля $A_1$, $A_2$, $A_3$, $A_4$, $M_1$, $M_2$, $M_3$, $M_4$ и $D$ (т.е. сложение и умножение ассоциативны, коммутативны имеют нейтральные элементы и удовлетворяют условию существованию обратных (по умножению --- для всех кроме нуля), а также дистрибутивности).
        
        Упорядоченность значит, что есть рефлексивное транзитивное антисимметричное отношение $\preccurlyeq$, что все элементы сравнимы, согласованное с операциями, т.е.:
        \begin{itemize}
            \item[$A$)] $a \preccurlyeq b \Rightarrow a + x \preccurlyeq b + x$.
            \item[$M$)] $0 \preccurlyeq a \wedge 0 \preccurlyeq b \Rightarrow 0 \preccurlyeq ab$.
        \end{itemize}

        Полнота поля значит любое из следующих утверждений (они равносильны):
        \begin{itemize}
            \item любое ограниченное сверху (снизу) подмножество поля имеет точную верхнюю (нижнюю) грань;
            \item (аксиома Кантора-Дедекинда) для любых двух множеств $A$ и $B$, что $A \preccurlyeq B$, есть разделяющий их элемент.
        \end{itemize}

        Итого мы имеем 9 аксиом поля, 2 аксиомы упорядоченности и 1 акиома полноты упорядоченности.
    \end{definition}

    \begin{statement*}
        Над $\QQ$ нет  элемента разделяющего $A := \{a > 0 \mid a^2 < 2\}$ и $B := \{b > 0 \mid b^2 > 2\}$.
    \end{statement*}

    \begin{proof}
        Предположим противное, т.е. есть $c > 0$, что $A < c < B$.

        Если $c^2 < 2$, то найдём $\varepsilon$, что $\varepsilon \in (0; 1)$ и $(c + \varepsilon)^2 < 2$. Заметим, что $(c + \varepsilon)^2 = c^2 + 2c\varepsilon + \varepsilon^2 < c^2 + (2c + 1)\varepsilon$. Пусть $\varepsilon < \frac{2 - c^2}{2c+ 1}$, тогда такое $\varepsilon$ точно подойдёт, ну а посокольку $\frac{2 - c^2}{2c + 1} > 0$, то такое $\varepsilon$ есть. Значит $c^2 \geqslant 2$.
        
        Аналогично имеем, что $\varepsilon \leqslant 2$. А значит $c^2 = 2$, что не бывает над $\QQ$.
    \end{proof}

    \begin{corollary*}
        $\QQ$ не полно.
    \end{corollary*}

    \begin{definition}\ 
        \begin{itemize}
            \item \emph{Закрытый интервал} или \emph{отрезок} $[a;b]:=\{x\in\RR \mid a \leqslant x \leqslant b\}$.
            \item \emph{Открытый интервал} или просто \emph{интервал} $(a;b):=\{x\in\RR \mid a < x < b\}$.
            \item \emph{Полуоткрытый интервал} или \emph{полуинтервал} $(a;b] := \{x\in\RR \mid a < x \leqslant b\}$, $[a;b):=\{x\in\RR \mid a \leqslant x < b\}$.
        \end{itemize}
    \end{definition}

    \begin{theorem}[Лемма о вложенных отрезках]\label{th_inter_segments}
        Пусть имеется $\{I_i\}_{i=1}^\infty$ --- множество вложенных (непустых) отрезков, т.е. $\forall n > 1\; I_{n+1} \subset I_n$. Тогда $\bigcap_{i=1}^\infty I_i \neq \varnothing$.
    \end{theorem}

    \begin{proof}
        Заметим, что для любых натуральных $n < m$ верно, что $a_n \leqslant a_m \leqslant b_m \leqslant b_n$, где $I_n = [a_n;b_n]$. Тогда для $A:=\{a_i\}_{i=1}^\infty$ и $B:=\{b_i\}_{i=1}^\infty$ верно, что $A \leqslant B$. Значит есть разделяющий их элемент $t$, значит $A \leqslant t \leqslant B$, значит $t\in I_i$ для всех $i$, значит $t \in \bigcap_{i=1}^\infty I_i$.
    \end{proof}

    \begin{remark}
        Теорема \ref{th_inter_segments} не верна для не отрезков.
    \end{remark}

    \begin{remark}
        Если в теореме \ref{th_inter_segments} $b_i-a_i$ ``сходится к 0'', т.е. $\forall \varepsilon > 0\, \exists n\in\NN: \forall i > n\, b_i-a_i < \varepsilon$, то пересечение всех отрезков состоит из ровно одного элемента.
    \end{remark}

    \begin{theorem}[индукция на вещественных числах]
        Пусть дано множество $X \subseteq [0;1]$, что
        \begin{enumerate}
            \item $0 \in X$;
            \item $\forall x \in X\; \exists \varepsilon > 0: U_\varepsilon(x) \cap [0;1] \subseteq X$;
            \item $\forall Y \subseteq X\; \sup(Y) \in X$.
        \end{enumerate}
        Тогда $X = [0;1]$.
    \end{theorem}

    \begin{proof}
        Предположим противное: $X \neq [0;1]$. Рассмотрим $Z := [0;1] \setminus X$ ($Z \neq \varnothing$!) и $Y := \{y \in [0;1] \mid y < Z\}$ ($Y \neq \varnothing$!). Заметим, что $Y \subseteq X$ и $\sup(Y) = \inf(Z) = t$. Тогда $t \in X$ по второму условию. Значит для некоторого $\varepsilon > 0$ верно, что $U_{\varepsilon}(t) \cap [0;1] \in X$, а т.е. $(U_\varepsilon(t) \cap [0;1]) \cap Z = \varnothing$, а тогда $t \neq \inf(Z)$ --- противоречие. Значит $X = [0;1]$.
    \end{proof}

    \section{Топология прямой, пределы и непрерывность.}

    \subsection{Топология}

    \begin{definition}
        \emph{$\varepsilon$-окрестность} точки $x$ (для $\varepsilon > 0$) --- $(x - \varepsilon; x+ \varepsilon)$. Обозначение: $U_\varepsilon(x)$.

        \emph{Проколотая $\varepsilon$-окрестность} точки $x$ --- $(x - \varepsilon; x) \cup (x; x + \varepsilon)$. Обозначение: $V_\varepsilon(x)$.
    \end{definition}

    \begin{definition}
        Пусть дано некоторое множество $X \subseteq \RR$. Тогда точка $x \in X$ называется \emph{внутренней точкой множества} $X$, если она содержится в $X$ вместе со своей окрестностью.
        
        Само множество $X$ называется \emph{открытым}, если все его точки внутренние.
    \end{definition}

    \begin{example}
        Следующие множества открыты:
        \begin{itemize}
            \item $(a; b)$;
            \item $(a; +\infty)$;
            \item $\RR$;
            \item $\varnothing$;
            \item $\bigcup_{i=0}^\infty (a_i; b_i)$ (интервалы не обязательно не должны пересекаться).
        \end{itemize}
    \end{example}

    \begin{definition}
        Пусть дано множесство $X\subseteq \RR$. Точка $x \in \RR$ называется \emph{предельной точкой} множества, если в любой проколотой окрестности $x$ будет какая-либо точка $X$.
        
        Множество предельных точек $X$ называется \emph{производным множеством} множества $X$ и обозначается как $X'$.

        Множество $X$ называется замкнутым, если $X \supseteq X'$.
    \end{definition}

    \begin{definition}
        Пусть дано множество $X\subseteq \RR$. Если у любой последовательности его точек есть предельная точка из самого множества $X$, то $X$ называется \emph{компактным}.
    \end{definition}

    \begin{statement}
        Подмножество $\RR$ компактно тогда и только тогда, когда замкнуто и ограничено.
    \end{statement}

    \subsection{Последовательности, пределы и ряды}

    \begin{definition}
        \emph{Предел последовательности} $\{x_n\}_{n=0}^\infty$ --- такое число $x$, что для любой окрестности $x$ эта последовательность с некоторого момента будет лежать в этой окрестности:
        \[\forall \varepsilon > 0\; \exists N \in \NN: \forall n \geqslant N\quad x_n \in U_\varepsilon(x)\]
        Обозначение: $\lim \{x_n\}_{n=0}^\infty = x$.

        \emph{Предельная точка последовательности} $\{x_n\}_{n=0}^\infty$ --- такое число $x$, что в любой его окрестности после любого момента появится элемент данной последовательности:
        \[\forall \varepsilon > 0\, \forall N \in \NN\; \exists n > N: \quad x_n \in U_\varepsilon(x)\]
    \end{definition}

    \begin{definition}
        Последовательность $\{x_n\}_{n=0}^\infty$ называется \emph{фундаментальной}, если
        \[\forall \varepsilon > 0\; \exists N \in \NN:\; \forall n_1, n_2 > N\quad |x_{n_1} - x_{n_2}| < \varepsilon\]
    \end{definition}

    \begin{theorem}\label{fundamental_seq_theorem}
        Последовательность сходится тогда и только тогда, когда фундаментальна.
    \end{theorem}

    \begin{proof}
        \begin{enumerate}
            \item Пусть последовательность $\{x_n\}_{n=0}^\infty$ сходится к некоторому значению $X$, тогда
                \begin{multline*}
                    \forall \varepsilon > 0\; \exists N \in \NN:\; \forall n > N\quad |x_n - X| < \varepsilon/2 \Rightarrow\\
                    \forall n_1, n_2 > N\quad |x_{n_1} - x_{n_2}| = |x_{n_1} - X + X - x_{n_2}| \leqslant |x_{n_1} - X| + |X - x_{n_2}| < \varepsilon
                \end{multline*}
            \item Пусть последовательность $\{x_n\}_{n=0}^\infty$ фундаментальна. Мы знаем, что для каждого $\varepsilon > 0$ все члены, начиная с некоторого различаются менее чем на $\varepsilon$. Тогда возьмём какой-нибудь такой член $y_0$ для некоторого $\varepsilon$, затем какой-нибудь такой член $y_1$ для $\varepsilon/2$, который идёт после $y_0$ и так далее. Получим последовательность, что все члены, начиная с $n$-ого лежат в $\varepsilon/2^n$-окрестности $y_n$. Тогда рассмотрим последовательность $\{I_n\}_{n=0}^\infty$, где $I_n = [y_n - \varepsilon/2^{n-1}; y_n + \varepsilon/2^{n-1}]$. Несложно понять, что $I_n \supseteq I_{n+1}$, поэтому в пересечении $\{I_n\}_{n=0}^\infty$ лежит некоторый $X$. Несложно понять, что все члены начальной последовательности, начиная с $y_{n+2}$, лежат в $\varepsilon/2^{n+2}$-окрестности $y_{n+2}$. При этом $|y_{n+2} - X| \leqslant \varepsilon/2^{n+1}$, что значит, что все члены главной последовательности, начиная с $y_{n+2}$ лежат в $3\varepsilon/2^{n+2}$-окрестности $X$, а значит и в $\varepsilon/2^n$.
        \end{enumerate}
    \end{proof}

    \begin{statement}
        Для последовательностей $\{x_n\}_{n=0}^\infty$ и $\{y_n\}_{n=0}^\infty$ верно (если определено), что
        \begin{enumerate}
            \item $\lim \{x_n\}_{n=0}^\infty + \lim \{y_n\}_{n=0}^\infty = \lim \{x_n + y_n\}_{n=0}^\infty$
            \item $-\lim \{x_n\}_{n=0}^\infty = \lim \{-x_n\}_{n=0}^\infty$
            \item $\lim \{x_n\}_{n=0}^\infty \cdot \lim \{y_n\}_{n=0}^\infty = \lim \{x_n y_n\}_{n=0}^\infty$
            \item $\frac{1}{\lim \{x_n\}_{n=0}^\infty} = \lim \{\frac{1}{x_n}\}_{n=0}^\infty$ (если $\lim \{x_n\}_{n=0}^\infty \neq 0$)
        \end{enumerate}
        и всегда, когда определена левая сторона определена, правая тоже определена.
    \end{statement}

    \begin{proof}
        \begin{enumerate}
            \item Пусть $\lim \{x_n\}_{n=0}^\infty = X$, $\lim \{y_n\}_{n=0}^\infty = Y$. Тогда
                \[\forall \varepsilon > 0\; \exists N, M \in \NN:\quad \forall n > N\; |x_n - X| < \varepsilon/2\quad \wedge\quad \forall m > M\; |y_m - Y| < \varepsilon/2,\]
                тогда
                \[\forall n > \max(N, M)\quad |(x_n + y_n) - (X + Y)| \leqslant |x_n - X| + |y_n - Y| < \varepsilon,\]
                что означает, что $\{x_n + y_n\}_{n=0}^\infty$ сходится и сходится к $X + Y$.
            \item Пусть $\lim \{x_n\}_{n=0}^\infty = X$. Тогда
                \[\forall \varepsilon > 0\; \exists N \in \NN:\quad \forall n > N\; |x_n - X| < \varepsilon,\]
                тогда
                \[\forall n > N\quad |(-x_n) - (-X)| = |X - x_n| = |x_n - X| < \varepsilon,\]
                что означает, что $\{-x_n\}_{n=0}^\infty$ сходится и сходится к $-X$.
            \item Пусть $\lim \{x_n\}_{n=0}^\infty = X$, $\lim \{y_n\}_{n=0}^\infty = Y$. Определим также
                \[\delta: (0; +\infty) \to \RR, \varepsilon \mapsto \frac{\varepsilon}{\sqrt{\left(\frac{|x|+|y|}{2}\right)^2+\varepsilon} + \frac{|x|+|y|}{2}} = \sqrt{\left(\frac{|x|+|y|}{2}\right)^2+\varepsilon} - \frac{|x|+|y|}{2}\]
                Несложно видеть, что $\delta(\varepsilon)$ всегда определено и всегда положительно. Также несложно видеть, что $\delta(\varepsilon)$ есть корень уравнения $t^2 + t(|X| + |Y|) = \varepsilon$. Тогда
                \[\forall \varepsilon > 0\; \exists N, M \in \NN:\quad \forall n > N\; |x_n - X| < \delta(\varepsilon)\quad \wedge\quad \forall m > M\; |y_m - Y| < \delta(\varepsilon),\]
                тогда
                \begin{align*}
                    \forall n > \max(N, M)\quad |x_n \cdot y_n - X \cdot Y|
                    &= |x_n \cdot y_n - x_n \cdot Y + x_n \cdot Y - X \cdot Y|\\
                    &\leqslant |x_n \cdot (y_n-Y)| + |(x_n - X) \cdot Y|\\
                    &< |x_n|\cdot \delta(\varepsilon) + \delta(\varepsilon)\cdot |Y|\\
                    &< (|X|+\delta(\varepsilon)) \cdot \delta(\varepsilon) + |Y| \cdot \delta(\varepsilon)\\
                    &= \delta(\varepsilon)^2 + (|X| + |Y|)\delta(\varepsilon)\\
                    &= \varepsilon,
                \end{align*}
                что означает, что $\{x_n \cdot y_n\}_{n=0}^\infty$ сходится и сходится к $X \cdot Y$.
            \item Пусть $\lim \{x_n\}_{n=0}^\infty = X$. Определим также
                \[\delta: (0; +\infty) \to \RR, \varepsilon \mapsto \frac{\varepsilon |X|}{1 + \varepsilon |X|}\]
                Несложно видеть, что $\delta(\varepsilon)$ всегда определено и всегда меньше $|X|$. Также несложно видеть, что $\delta(\varepsilon)$ есть корень уравнения $\frac{t}{|X|(|X| - t)} = \varepsilon$. Тогда
                \[\forall \varepsilon > 0\; \exists N \in \NN:\quad \forall n > N\; |x_n - X| < \delta(\varepsilon),\]
                тогда
                \[\forall n > N\quad \left|\frac{1}{x_n} - \frac{1}{X}\right| = \left|\frac{X-x_n}{X\cdot x_n}\right| < \frac{\delta(\varepsilon)}{|X| \cdot |x_n|} < \frac{\delta(\varepsilon)}{|X|(|X|-\delta(\varepsilon))} = \varepsilon,\]
                что означает, что $\{\frac{1}{x_n}\}_{n=0}^\infty$ сходится и сходится к $1/X$.
        \end{enumerate}
    \end{proof}

    \begin{definition}
        Последовательность $\{x_n\}_{n=0}^\infty$ \emph{ассимптотически больше} последовательности $\{y_n\}_{n=0}^\infty$, если $x_n > y_n$ для всех натуральных $n$, начиная с некоторого. Обозначение: $\{x_n\}_{n=0}^\infty \succ \{y_n\}_{n=0}^\infty$.

        Аналогично определяются \emph{ассимптотически меньше} ($\{x_n\}_{n=0}^\infty \prec \{y_n\}_{n=0}^\infty$), \emph{ассимптотически не больше} ($\{x_n\}_{n=0}^\infty \preccurlyeq \{y_n\}_{n=0}^\infty$) и \emph{ассимптотически не меньше} ($\{x_n\}_{n=0}^\infty \succcurlyeq \{y_n\}_{n=0}^\infty$).
    \end{definition}

    \begin{statement}\label{stupid_seq_statement_1}
        Если $\{x_n\}_{n=0}^\infty \succcurlyeq \{y_n\}_{n=0}^\infty$, то $\lim \{x_n\}_{n=0}^\infty \geqslant \lim \{y_n\}_{n=0}^\infty$.
    \end{statement}

    \begin{proof}
        Предположим противное, т.е. $Y > X$, где $X := \lim \{x_n\}_{n=0}^\infty$, $Y := \lim \{y_n\}_{n=0}^\infty$. Тогда пусть $\varepsilon = \frac{|X - Y|}{2}$. С каких-то моментов $\{x_n\}_{n=0}^\infty$ и $\{y_n\}_{n=0}^\infty$ находятся в $\varepsilon$-окрестностях $X$ и $Y$ соответственно. Тогда начиная с позднего из этих моментов, $y_n > Y - \varepsilon = X + \varepsilon > x_n$, т.е. $\{x_n\}_{n=0}^\infty \prec \{y_n\}_{n=0}^\infty$ --- противоречие. Значит $X \geqslant Y$.
    \end{proof}

    \begin{statement}\label{stupid_seq_statement_2}
        Если $\lim \{x_n\}_{n=0}^\infty > \lim \{y_n\}_{n=0}^\infty$, то $\{x_n\}_{n=0}^\infty \succ \{y_n\}_{n=0}^\infty$.
    \end{statement}

    \begin{proof}
        Пусть $X := \lim \{x_n\}_{n=0}^\infty$, $Y := \lim \{y_n\}_{n=0}^\infty$. Тогда пусть $\varepsilon = \frac{|X - Y|}{2}$. С каких-то моментов $\{x_n\}_{n=0}^\infty$ и $\{y_n\}_{n=0}^\infty$ находятся в $\varepsilon$-окрестностях $X$ и $Y$ соответственно. Тогда начиная с позднего из этих моментов, $x_n > X - \varepsilon = Y + \varepsilon > y_n$, т.е. $\{x_n\}_{n=0}^\infty \succ \{y_n\}_{n=0}^\infty$.
    \end{proof}

    \begin{statement}[леммма о двух полицейских]\label{stupid_seq_statement_3}
        Если
        \[\{x_n\}_{n=0}^\infty \succcurlyeq \{y_n\}_{n=0}^\infty \succcurlyeq \{z_n\}_{n=0}^\infty\]
        и
        \[\lim \{x_n\}_{n=0}^\infty = \lim \{z_n\}_{n=0}^\infty = A,\]
        то предел $\{y_n\}_{n=0}^\infty$ определён и равен $A$.
    \end{statement}

    \begin{proof}
        Для каждого $\varepsilon > 0$ есть $N, M \in \NN$, что
        \[\forall n > N\; |x_n - A| < \varepsilon \quad \wedge \quad \forall m > M\; |z_n - A| < \varepsilon,\]
        значит
        \[\forall n > \max(N, M)\quad A + \varepsilon > x_n \geqslant y_n \geqslant z_n > A - \varepsilon \quad \text{т.е. } |y_n - A| < \varepsilon,\]
        что означает, что $\{y_n\}_{n=0}^\infty$ сходится и сходится к $A$.
    \end{proof}

    \begin{statement}
        Если $\{x_n\}_{n=0}^\infty \succcurlyeq \{y_n\}_{n=0}^\infty$, $\lim \{x_n\}_{n=0}^\infty = A$, а $\{y_n\}_{n=0}^\infty$, неубывает (с некоторого момента), то предел $\{y_n\}_{n=0}^\infty$ существует и не превосходит $A$.
    \end{statement}

    \begin{proof}
        Если последовательность $\{y_n\}_{n=0}^\infty$ возрастает не с самого начала, то отрежем её начало с до момента начала возрастания. Заметим, что она ограничена сверху (из-за последовательности $\{x_n\}_{n=0}^\infty$), тогда определим $B := \sup(\{y_n\}_{n=0}^\infty)$. Тогда $\forall \varepsilon > 0\; \exists N \in \NN:\quad |B-x_N| < \varepsilon$, тогда $\forall n > N\quad |B-x_n| < \varepsilon$, что означает, что $\{y_n\}_{n=0}^\infty$ сходится и сходится к $B$. По утверждению \ref{stupid_seq_statement_1} $A \geqslant B$.
    \end{proof}

    \begin{definition}
        Сумма ряда $\{a_k\}_{k=0}^\infty$ есть значение $\sum_{k=0}^\infty a_k := \lim \left\{\sum_{i=0}^k\right\}_{k=0}^\infty$. Частичной же суммой $s_k$ этого ряда называется просто $\sum_{i=0}^k a_i$.
    \end{definition}

    \begin{definition}
        Ряд $\sum_{i=0}^\infty a_i$ \emph{сильно сходится}, если $\sum_{i=0}^\infty |a_i|$ сходится.
    \end{definition}

    \begin{theorem}
        Еcли ряд сильно сходится сходится, то он сходится.
    \end{theorem}

    \begin{proof}
        \begin{thlemma}\label{lemma_sum_of_suffix}
            Пусть ряд $\sum_{i=0}^\infty a_i$ сходится, тогда сходится любой его ``хвост'' (суффикс), и для любого $\varepsilon > 0$ есть такой хвост, сумма которого меньше $\varepsilon$.
        \end{thlemma}

        \begin{proof}
            Пусть $A = \sum_{i=0}^\infty a_i$. Это значит, что для каждого $\varepsilon > 0$ существует $N \in \NN$, что для всех $n \geqslant N$ верно, что $\sum_{i=0}^n |a_i| \in U_\varepsilon(A)$. Тогда заметим, что
            \[\sum_{i=N+1}^\infty |a_i| = \lim_{n \to \infty} \sum_{i=N+1}^n |a_i| = \lim_{n \to \infty} \left(\sum_{i=0}^n |a_i| - \sum_{i=0}^N |a_i|\right) = \lim_{n \to \infty} \sum_{i=0}^n |a_i| - \sum_{i=0}^N |a_i| = A - \sum_{i=0}^N |a_i| \in U_\varepsilon(0)\]
            Это и означает, что любой хвост сходится. И так мы для каждого $\varepsilon$ нашли такой хвост, что его сумма меньше $\varepsilon$.
        \end{proof}

        Пусть дан сильно сходящийся ряд $\sum_{i=0}^\infty a_i$. Пусть $\varepsilon_n := \sum_{i=n}^\infty |a_i|$. Несложно видеть, что $\{\varepsilon_n\}_{n=0}^\infty$ монотонно уменьшается, сходясь к 0 (последнее следует из леммы \ref{lemma_sum_of_suffix}). Также несложно видеть по рассуждениям леммы \ref{lemma_sum_of_suffix}, что $\varepsilon_n - \varepsilon_{n+1} = |a_n|$. Тогда определим
        \[S_n := \overline{U}_{\varepsilon_{n+1}}(\sum_{i=0}^n a_i),\]
        где $\overline{U}_\varepsilon(x)$ --- закрытая $\varepsilon$-окрестность точки $x$. Тогда несложно видеть, что
        \[\left|\sum_{i=0}^{n+m} a_i - \sum_{i=0}^{n} a_i \right| = \left|\sum_{i=n+1}^{n+m} a_i \right| \leqslant \sum_{i=n+1}^{n+m} |a_i| \leqslant \varepsilon_{n+1}\]
        Тем самым сумма любого префикса длины хотя бы $n+1$ лежит в $\overline{U}_{\varepsilon_{n+1}}(\sum_{i=0}^{n} a_i) = S_n$. Также несложно видеть, что $S_{n+1} \subseteq S_n$. А также понятно, что $S_i$ замкнуто и ограничено (``компактно'').

        Пусть $A := \bigcap_{i=0}^\infty S_i$ (поскольку диаметры шаров сходятся к нулю, то в пересечении лежит не более одной точки). Тогда мы видим, что $|\sum_{i=0}^n a_i - A| \leqslant \varepsilon_{n+1} \to 0$, поэтому $\sum_{i=0}^n a_i$ сходится и сходится к $A$.
    \end{proof}

    \begin{corollary}
        Еcли $\{b_i\}_{i=0}^\infty \succcurlyeq \{|a_i|\}_{i=0}^n$ и $\sum_{i=0}^\infty |b_i|$ существует, то и $\sum_{i=0}^\infty a_i$ существует.
    \end{corollary}

    \begin{theorem}[признак Лейбница]
        Пусть дана последовательность $\{a_n\}$, монотонно сверху сходящаяся к $0$. Тогда ряд $\sum_{i=0}^\infty (-1)^i a_i$ сходится.
    \end{theorem}

    \begin{proof}
        Рассмотрим последовательности 
        \begin{align*}
            \{P_n\}_{n=0}^\infty &:= \{S_{2n}\}_{n=0}^\infty = \left\{\sum_{i=0}^{2n} (-1)^i a_i\right\}_{n=0}^\infty&
            \{Q_n\}_{n=0}^\infty &:= \{S_{2n + 1}\}_{n=0}^\infty = \left\{\sum_{i=0}^{2n + 1} (-1)^i a_i\right\}_{n=0}^\infty
        \end{align*}
        Несложно видеть, что 
        \begin{align*}
            P_{n+1} - P_n &= - a_{2n+1} + a_{2n+2} \leqslant 0&
            Q_{n+1} - Q_n &= a_{2n+2} - a_{2n-3} \geqslant 0\\
            Q_{n} - P_{n} &= - a_{2n+1} \leqslant 0&
            P_{n+1} - Q_{n} &= a_{2n+2} \geqslant 0
        \end{align*}
        Тогда имеем, что $\{P_n\}_{n=0}^\infty$ монотонно убывает, $\{Q_n\}_{n=0}^\infty$ монотонно возрастает, а также
        \[\{P_n\}_{n=0}^\infty \geqslant \{Q_n\}_{n=0}^\infty.\]
        Тогда последовательности $\{P_n\}_{n=0}^\infty$ и $\{Q_n\}_{n=0}^\infty$ сходятся и сходятся к $P$ и $Q$ соответственно. При этом последовательность
        \[\{P_n\}_{n=0}^\infty - \{Q_n\}_{n=0}^\infty = \{P_n - Q_n\}_{n=0}^\infty = a_{2n+1}\]
        тоже сходится по условию и сходится к $0$. Поэтому
        \[P - Q = \lim \{P_n\}_{n=0}^\infty - \lim \{Q_n\}_{n=0}^\infty = 0\]
        значит $P=Q$. Значит и последовательность префиксных сумм тоже сходится к $P=Q$.
    \end{proof}

    \begin{lemma}[преобразование Абеля]
        \[\sum_{k=0}^n a_k b_k = \sum_{k=0}^{n-1} (a_k - a_{k+1})B_k + a_n B_n\]
        где $B_n := \sum_{i=0}^n b_i$.
    \end{lemma}

    \begin{theorem}[признак Дирихле]
        Если даны $\{a_i\}_{i=0}^\infty$ и $\{b_i\}_{i=0}^\infty$, что $\{a_i\}_{i=0}^\infty \searrow 0$, а $\{B_n\}_{n=0}^\infty = \{\sum_{i=0}^n b_i\}_{i=0}^\infty$ ограничена, то ряд $\sum_{i=0}^\infty a_i b_i$ сходится.
    \end{theorem}

    \begin{proof}
        \[S_n = \sum_{i=0}^n a_k b_k = \sum_{i=0}^n (a_k - a_{k+1}) B_k + a_n B_n\]
        Пусть $|B_n| < C$ для всех $n$. Несложно видеть, что 
        \[\lim_{n \to \infty} |a_n B_n| \leqslant \lim a_n C = C \lim a_n = 0,\]
        поэтому $\lim a_n B_n = 0$. Также
        \[|(a_k-a_{k+1}) B_k| < C |a_k - a_{k+1}| = C(a_k - a_{k+1}),\]
        поэтому
        \[|S_n - a_n B_n| \leqslant \sum_{k=0}^{n-1} |(a_k - a_{k+1})B_k| < C \sum_{k=0}^{n-1} (a_k - a_{k+1}) = C(a_1 - a_{n+1}),\]
        что тоже сходится. Поэтому $\{S_n\}_{n=0}^\infty$ сходится, т.е. и ряд сходится.
    \end{proof}

    \subsection{Пределы функций, непрерывность}

    \begin{definition}[по Коши]
        \emph{Предел} функции $f: X \to \RR$ при в точке $x$ --- такое значение $y$, что
        \[\forall \varepsilon > 0\, \exists \delta > 0: f(V_\delta(x) \cap X) = U_\varepsilon(y)\]
        Обозначение: $\lim\limits_{t \to x} f(t) = y$.
    \end{definition}

    \begin{definition}[по Гейне]
        \emph{Предел} функции $f: X \to \RR$ при в точке $x$ --- такое значение $y$, что для любой последовательность $\{x_n\}_{n=0}^\infty$ элементов $X \setminus \{x\}$ последовательность $\{f(x_n)\}_{n=0}^\infty$ сходится к $y$. Обозначение: $\lim\limits_{t \to x} f(t) = y$.
    \end{definition}

    \begin{theorem}
        Определения пределов по Коши и по Гейне равносильны.
    \end{theorem}

    \begin{proof}
        Будем доказывать равносильность отрицаний утверждений, ставимых в определениях.
        \begin{enumerate}
            \item Пусть функция $f: X \to \RR$ не сходится по Коши в $x$ к значению $y$. Значит есть такое $\varepsilon > 0$, что в любой проколотой окрестности $x$ (в множестве $X$) есть точка, значение $f$ в которой не лежит в $\varepsilon$-окрестности. Рассмотрев любую такую проколотую окрестность $I_0 = V_{\delta_0}(x)$, берём в ней любую такую точку $x_0$. Далее рассмотрев $I_1 = V_{\delta_1}(x)$, где $\delta_1 = \min(\delta_0/2, |x-x_0|)$, берём там любую точку $x_1$, где значение $f$ вылетает вне $\varepsilon$-окрестности $y$. Так далее строим последовательность $\{x_n\}_{n=0}^\infty$, сходящуюся к $x$, значения $f$ в которой не лежат в $\varepsilon$-окрестности $y$, что означает, что $\{f(x_n)\}_{n=0}^\infty$ не сходится к $y$, что означает, что $f$ не сходится по Гейне в $x$ к значению $y$.
            \item Пусть функция $f: X \to \RR$ не сходится по Гейне в $x$ к значению $y$. Значит есть последовательность $\{x_n\}_{n=0}^\infty$, сходящаяся к $x$, что последовательность её значений не сходится к $y$. Значит есть $\varepsilon > 0$, что после любого момента в последовательности будет член, значение в котором вылезает вне $\varepsilon$-окрестности $y$. Поскольку для любой проколотой окрестности $x$ есть момент, начиная с которого вся последовательность лежит в этой окрестности, то в любой проколотой окрестности $x$ есть член, значение которого вылезает вне $\varepsilon$-окрестности $y$, что означает, что $f$ не сходится по Коши в $x$ к $y$.
        \end{enumerate}
    \end{proof}

    \begin{statement}
        Функция $f: X \to \RR$ имеет в $x$ предел тогда и только тогда, когда
        \[\forall \varepsilon > 0\; \exists \delta > 0:\; \forall x_1, x_2 \in V_\delta(x)\quad |f(x_1) - f(x_2)| < \varepsilon\]
    \end{statement}

    \begin{proof}
        Такое же как для последовательностей: см. теорему \ref{fundamental_seq_theorem}.
    \end{proof}

    \begin{statement}
        Для функций $f: \RR \to \RR$ и $g: \RR \to \RR$ верно, что
        \begin{enumerate}
            \item $\lim\limits_{x \to a} f(x) + \lim\limits_{x \to a} g(x) = \lim\limits_{x \to a} (f + g)(x)$
            \item $\lim\limits_{x \to a} (-f)(x) = -\lim\limits_{x \to a} f(x)$
            \item $\lim\limits_{x \to a} f(x) \cdot \lim\limits_{x \to a} g(x) = \lim (f \cdot g)(x)$
            \item $\frac{1}{\lim\limits_{x \to a} f(x)} = \lim\limits_{x \to a} (\frac{1}{f})(x)$ (если $\lim\limits_{x \to a} f(x) \neq 0$)
            \item $\lim\limits_{y \to \lim\limits_{x \to a} g(x)} f(y) = \lim\limits_{x \to a} (f \circ g)(x)$
        \end{enumerate}
        и всегда, когда определена левая сторона определена, правая тоже определена.
    \end{statement}

    \begin{remark}
        Утверждения \ref{stupid_seq_statement_1}, \ref{stupid_seq_statement_2} и \ref{stupid_seq_statement_3} верны, если заменить последовательности на функции, пределы последовательностей на пределы функций в некоторой точке $x$, а асимптотические неравенства на неравенства на окрестности $x$.
    \end{remark}

    \begin{definition}
        \emph{Осцелляцией} называется $\osc_E f = \sup_E f - \inf_E f$.
    \end{definition}

    \begin{definition}
        \emph{Верхним пределом} функции $f$ в точке $x_0$ называется
        \[\varlimsup\limits_{x \to x_0} f(x) = \inf_{\delta > 0} (\sup_{V_\delta(x_0)} f)\]
        \emph{Нижним пределом} функции $f$ в точке $x_0$ называется
        \[\varliminf\limits_{x \to x_0} f(x) = \sup_{\delta > 0} (\inf_{V_\delta(x_0)} f)\]
    \end{definition}

    \begin{statement}
        Функция $f: X \to \RR$ имеет в $x$ предел тогда и только тогда, когда $\varlimsup\limits_{t \to x} f(t) = \varliminf\limits_{t \to x} f(t)$.
    \end{statement}

    \begin{definition}
        Функция $f: X \to \RR$ называется \emph{непрерывной в точке} $x$, если $\lim\limits_{t \to x} f(t) = f(x)$. В изолированных точках $f$ всегда непрерывна.
    \end{definition}

    \begin{definition}
        Функция $f: X \to \RR$ называется \emph{непрерывной на множестве} $Y \subseteq X$, если она непрерывна во всех точках $Y$.
    \end{definition}

    \begin{statement}
        Для непрерывных на $X$ функций $f$ и $g$ верно, что
        \begin{itemize}
            \item $f+g$ непрерывна на $X$;
            \item $fg$ непрерывна на $X$;
            \item $\frac{1}{f}$ непрерывна на $X$ (если $f \neq 0$).
        \end{itemize}
    \end{statement}
\end{document}