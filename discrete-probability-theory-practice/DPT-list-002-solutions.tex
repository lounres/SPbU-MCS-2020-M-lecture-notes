\documentclass[12pt,a4paper]{article}
\usepackage{solutions}
\usepackage{float}
\usepackage{multicol}
\usepackage{inkscape}
\usepackage{dsfont}
\usepackage[all]{xy}
\CompileMatrices

\makeatletter
\renewcommand*\env@matrix[1][*\c@MaxMatrixCols c]{%
  \hskip -\arraycolsep
  \let\@ifnextchar\new@ifnextchar
  \array{#1}}
\makeatother

\title{Листочек 1.\\Дискретная теория вероятностей. 1 курс.\\Решения.}
\author{Глеб Минаев @ 102 (20.Б02-мкн)}
% \date{}

\DeclareMathOperator{\sign}{sign}
\newcommand{\DD}{\ensuremath{\mathbb{D}}\xspace}
\renewcommand{\Re}{\qopname\relax o{Re}}
\renewcommand{\Im}{\qopname\relax o{Im}}

\begin{document}
    \maketitle

    \begin{multicols}{2}
        \tableofcontents
    \end{multicols}

    \breaker

    \begin{enumproblem}
        Пусть $f$ --- вероятностная производящая функция ветвящегося процесса $X$. Тогда
        \[f(s) = (1-p-q) + qs + ps^2\]
        Значит процесс будет критическим, когда $f'(1) = 1$, т.е.
        \[1 = f'(1) = q + 2p\]
        А вероятность вырождения является неподвижной точки $f$. Обозначим её за $P$, тогда
        \begin{gather*}
            P = p P^2 + q P + 1 - p - q\\
            0 = p (P^2 - 1) + q (P - 1) - (P - 1)\\
            0 = (P - 1)(p (P + 1) + q - 1)\\
            0 = (P - 1)(P - \frac{1 - p - q}{p})\\
        \end{gather*}
        Таким образом, если $q + 2p \leqslant 1$, то $P = 1$, иначе $P = \frac{1 - p - q}{p}$.
    \end{enumproblem}

    \begin{enumproblem}
        Обозначим в нашем случайном блуждании вероятности (для любых $n \in \NN \cup \{0\}$ и $k \in \ZZ$)
        \begin{align*}
            &p := \PP(S_{n+1} = k+1 \mid S_n = k)&
            &1-p := \PP(S_{n+1} = k-1 \mid S_n = k)
        \end{align*}

        Заметим, что $Y_n$ принимает значения из $\NN \cup \{0\}$. Предположим, что $Y_n = d$.
        
        Если $d > 0$, то $M_n > S_n$. Тогда независимо ни от чего $M_{n+1} = M_n$, а тогда
        \[Y_{n+1} - Y_n = -(S_{n+1} - S_n)\]
        Значит (независимо от значений $Y_m$, $m \in \{0; \dots; n-1\}$)
        \begin{align*}
            &\PP(Y_{n+1} = d-1 \mid Y_n = d) = p&
            &\PP(Y_{n+1} = d+1 \mid Y_n = d) = 1-p
        \end{align*}

        Если же $d = 0$, то при увеличении $S_n$ и $M_n$ увеличится, а при уменьшении не изменится. Следовательно (независимо от значений $Y_m$, $m \in \{0; \dots; n-1\}$)
        \begin{align*}
            &\PP(Y_{n+1} = 0 \mid Y_n = d) = p&
            &\PP(Y_{n+1} = 1 \mid Y_n = d) = 1-p
        \end{align*}

        Таким образом $(Y_n)_{n=0}^\infty$ --- цепь маркова, а матрица переходов задаётся формулой
        \[
            P_{i, j} :=
            \begin{cases}
                p& \text{ если $j = \max(i-1, 0)$}\\
                1-p& \text{ если $j = i+1$}\\
                0& \text{ иначе}
            \end{cases}
        \]
    \end{enumproblem}

    \begin{enumproblem}\ 
        \begin{enumerate}
            \item Заметим, что
                \begin{align*}
                    &\PP\left(X_{n+1+r} = i \mid X_{n+r} = j \wedge \bigwedge_{m=0}^{n-1} X_{m+r} = k_{m+r}\right)\\
                    &\begin{aligned}
                        = \sum_{k_0, \dots, k_{r-1}} &\PP\left(X_{n+1+r} = i \mid X_{n+r} = j \wedge \bigwedge_{m=0}^{n-1+r} X_{m} = k_{m}\right)\\
                        &\PP\left(\bigwedge_{m=0}^{r-1} X_{m} = k_{m} \mid X_{n+r} = j \wedge \bigwedge_{m=0}^{n-1} X_{m+r} = k_{m+r}\right)
                    \end{aligned}\\
                    &\begin{aligned}
                        = \sum_{k_0, \dots, k_{r-1}} &\PP\left(X_{n+1+r} = i \mid X_{n+r} = j\right)\\
                        &\PP\left(\bigwedge_{m=0}^{r-1} X_{m} = k_{m} \mid X_{n+r} = j \wedge \bigwedge_{m=0}^{n-1} X_{m+r} = k_{m+r}\right)
                    \end{aligned}\\
                    &= \PP\left(X_{n+1+r} = i \mid X_{n+r} = j\right) \sum_{k_0, \dots, k_{r-1}} \PP\left(\bigwedge_{m=0}^{r-1} X_{m} = k_{m} \mid X_{n+r} = j \wedge \bigwedge_{m=0}^{n-1} X_{m+r} = k_{m+r}\right)\\
                    &= \PP\left(X_{n+1+r} = i \mid X_{n+r} = j\right) = P_{i, j}
                \end{align*}
                Следовательно $(X_{n+r})_{n=0}^\infty$ является цепью Маркова с той же матрицей переходов.
            
            \item Заметим, что
                \begin{align*}
                    &\PP\left(X_{2(n+1)} = i \mid X_{2n} = j \wedge \bigwedge_{m=0}^{n-1} X_{2m} = k_{m}\right)\\
                    &\begin{aligned}
                        = \sum_{k}\ &\PP\left(X_{2(n+1)} = i \mid X_{2n+1} = k \wedge X_{2n} = j \wedge \bigwedge_{m=0}^{n-1} X_{2m} = k_{m}\right)\\
                        &\PP\left(X_{2n+1} = k \mid X_{2n} = j \wedge \bigwedge_{m=0}^{n-1} X_{2m} = k_{m}\right)
                    \end{aligned}\\
                    &= \sum_{k} \PP\left(X_{2(n+1)} = i \mid X_{2n+1} = k\right) \PP\left(X_{2n+1} = l \mid X_{2n} = j\right)\\
                    &= \sum_{k} P_{i, k} P_{l, j}\\
                    &= \sum_{k} \PP\left(X_{2(n+1)} = i \mid X_{2n+1} = k \wedge X_{2n} = j\right) \PP\left(X_{2n+1} = l \mid X_{2n} = j\right)\\
                    &= \PP\left(X_{2(n+1)} = i \mid X_{2n} = j\right)
                \end{align*}
                Следовательно $(X_{2n})_{n=0}^\infty$ является цепью Маркова с матрицей переходов равной квадрату предыдущей матрицы переходов.

            \item Обозначим $Y_n := (X_n, X_{n+1})$ Заметим, что
                \begin{align*}
                    &\PP\left(Y_{n+1} = (i_1, i_2) \mid Y_n = (j_1, j_2) \wedge \bigwedge_{m=0}^{n-1} Y_m = (k_{m, 1}, k_{m, 2})\right)\\
                    =\ &\PP\left(X_{n+1} = i_1 \mid Y_n = (j_1, j_2) \wedge \bigwedge_{m=0}^{n-1} Y_m = (k_{m, 1}, k_{m, 2})\right)\\
                    &\cdot \PP\left(X_{n+2} = i_2 \mid X_{n+1} = i_1 \wedge Y_n = (j_1, j_2) \wedge \bigwedge_{m=0}^{n-1} Y_m = (k_{m, 1}, k_{m, 2})\right)\\
                    =\ &\PP\left(X_{n+1} = i_1 \mid X_{n+1} = j_2 \wedge X_n = j_1 = k_{n-1, 2} \wedge \bigwedge_{m=1}^{n-1} X_m = k_{m, 1} = k_{m-1, 2} \wedge X_0 = k_{0, 1}\right)\\
                    &\cdot \PP\left(X_{n+2} = i_2 \mid X_{n+1} = i_1 = j_2 \wedge X_n = j_1 = k_{n-1, 2} \wedge \bigwedge_{m=1}^{n-1} X_m = k_{m, 1} = k_{m-1, 2} \wedge X_0 = k_{0, 1}\right)\\
                    =\ &\PP\left(X_{n+1} = i_1 \mid X_{n+1} = j_2 \wedge X_n = j_1\right) \cdot \PP\left(X_{n+2} = i_2 \mid X_{n+1} = i_1 = j_2 \wedge X_n = j_1\right)\\
                    =\ &\PP\left(X_{n+1} = i_1 \mid Y_n = (j_1, j_2)\right) \cdot \PP\left(X_{n+2} = i_2 \mid X_{n+1} = i_1 \wedge Y_n = (j_1, j_2)\right)\\
                    =\ &\PP\left(Y_{n+1} = (i_1, i_2) \mid Y_n = (j_1, j_2)\right)\\
                    =\ &[i_1 = j_2] \cdot P_{i_1, i_2}
                \end{align*}
                (где $[\cdot]$ обозначает \href{https://ru.wikipedia.org/wiki/\%D0\%A1\%D0\%BA\%D0\%BE\%D0\%B1\%D0\%BA\%D0\%B0_\%D0\%90\%D0\%B9\%D0\%B2\%D0\%B5\%D1\%80\%D1\%81\%D0\%BE\%D0\%BD\%D0\%B0}{скобку Айверсона}). Таким образом $(Y_n)_{n=0}^\infty$ является цепью Маркова с матрицей перехода, задаваемой соотношением
                \[
                    P_{(i_1, i_2), (j_1, j_2)} =
                    \begin{cases}
                        P_{j_1, j_2}& \text{ если } i_2 = j_1\\
                        0& \text{ иначе}
                    \end{cases}
                \]
        \end{enumerate}
    \end{enumproblem}

    \begin{enumproblem}
        Заметим, что существование инвариантного распределения равносильно существованию вектора $(\alpha_n)_{n=0}^\infty$, где всякое $\alpha_i \geqslant 0$, сумма $\sum_{n=0}^\infty \alpha_n$ сходится к положительному значению, а сам вектор является собственным для матрицы переходов. Действительно, инвариантное распределение само по себе является таким вектором, а при умножении всего вектора на положительную константу он всё ещё удовлетворяет всем требованиям, значит его можно домножить на такую константу, что $\sum_{n=0}^\infty \alpha_n = 1$. Таким образом будем проверять существование векторов такого вида.

        При этом как мы только что заметили, можно проверить существует ли вектор, у которого $\alpha_0 \in \{0; 1\}$ (домножением на $\alpha_0^{-1}$ мы получаем это требование). При этом заметим, что инвариантность по применению матрицы переходов равносильно тому, что $\alpha_{n+1} = \alpha_n p_n$. Значит любой искомый вектор, где $\alpha_0 = 0$ просто равен нулю, а значит не подходит. Значит $\alpha_0 = 1$, а тогда $\alpha_n = \prod_{k < n} p_k$. Значит искомый вектор существует, когда ряд
        \[\sum_{n=0}^\infty \prod_{k=0}^{n-1} p_k = 1 + p_0 + p_0 p_1 + p_0 p_1 p_2 + \dots\]
        сходится.
    \end{enumproblem}

    \begin{enumproblem}
        Пусть $u_n$ --- вероятность того, что через ровно $n$ шагов мы вернёмся в стартовую точку. Тогда $p$ возвратна тогда и только тогда, когда $\sum_{n=0}^\infty u_n = \infty$.

        Заметим, что $u_n$ равно $0$, если $n \not\divided 3$. Если же $n = 3k$, то
        \[u_n = \binom{3k}{k} p^k (1-p)^{2k} \approx \sqrt{\frac{3}{4 \pi k}} \left(\frac{27}{4}\right)^k (p(1-p)^2)^k = \sqrt{\frac{3}{4 \pi k}} \left(\frac{27}{4}p(1-p)^2\right)^k\]

        Несложно видеть, что $\sum_{n=0}^\infty u_n = \infty$ тогда и только тогда, когда
        \[\sum_{k=0}^\infty \sqrt{\frac{3}{4 \pi k}} \left(\frac{27}{4}p(1-p)^2\right)^k = \infty\]
        Обозначим
        \[\alpha := \frac{27}{4}p(1-p)^2\]
        Если $\alpha > 1$, то ряд выше расходится, если $\alpha < 1$, то сходится, а если $\alpha = 1$, то ряд имеет вид
        \[\sum_{k=0}^\infty \sqrt{\frac{3}{4 \pi k}} = \sqrt{\frac{3}{4 \pi}} \sum_{k=0}^\infty \frac{1}{\sqrt{k}}\]
        а поэтому также рассходится. Значит теперь перед нами стоит вопрос, а когда же $\alpha \geqslant 1$.

        Пусть $f(p) = \frac{27}{4}p(1-p)^2$. Несложно видеть, что корни $f'(p)$ есть $1/3$ --- локальный максимум --- и $1$ --- локальный минимум. При этом $f(1/3) = 1$. Значит $f(p) \leqslant 1$ при $p \in [0; 1]$, и равенство достигается при $p = 1/3$. Поэтому ответ: при $p = 1/3$.
    \end{enumproblem}
\end{document}