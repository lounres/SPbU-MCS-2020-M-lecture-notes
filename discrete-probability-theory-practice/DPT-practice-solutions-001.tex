\documentclass[12pt,a4paper]{article}
\usepackage{solutions}
% \usepackage{float}
\usepackage{inkscape}

\title{Занятие от 18.02.\\Геометрия и топология. 1 курс.\\Решения.}
\author{Глеб Минаев @ 102 (20.Б02-мкн)}
% \date{}

\newcommand{\const}{\ensuremath{\mathrm{const}}\xspace}

\begin{document}
    \maketitle

    \begin{problem}{16}
        Пусть на данной сетке $O$ будет началом отсчёта, а $Z$ имеет координаты $(z_x; z_y)$. Поставим в каждую точку $A = (a_x; a_y)$ значение равное вероятности $p(A)$, что турист дойдёт в неё из $O$. Заметим, что для всяких $a_x, a_y \in \ZZ$
        \[
            p(a_x, a_y) = 
            \begin{cases}
                0& \text{ если $a_x < 0$ или $a_y < 0$}\\
                1& \text{ если $a_x = 0 = a_y$}\\
                \frac{p(a_x - 1, a_y) + p(a_x, a_y - 1)}{2}& \text{ иначе}\\
            \end{cases}
        \]
        Тогда легко видеть, что $p(a_x, a_y) = \binom{a_x + a_y}{a_x} / 2^{a_x + a_y}$.
    \end{problem}

    \begin{problem}{17}
        Было...
    \end{problem}

    \begin{problem}{21}
        Заметим, что событие $B$ распадается на два:
        \begin{enumerate}
            \item среди оставшихся 3 цифр одна является либо ``1'', либо ``5'', две другие является различными цифрами из набора $\{0; 2; 4; 6; 7; 8; 9\}$;
            \item оставшиеся три цифры взяты из набора $\{0; 2; 4; 6; 7; 8; 9\}$, и ровно две из них совпадают.
        \end{enumerate}
        Соответственно
        \[
            \PP(B)
            = \frac{2 \cdot \binom{7}{2} \cdot 3! + \frac{7!}{5!} \cdot 3}{10^3}
            = \frac{7 \cdot 6^2 + 7 \cdot 6 \cdot 3}{1000}
            = \frac{42 \cdot 9}{1000}
            = 0{,}378
        \]

        Событие $C$ является противоположным к событию, когда все цифры различны. Следовательно
        \[
            \PP(C)
            = 1 - \frac{\binom{10}{3} \cdot 3!}{10^3}
            = 1 - \frac{10 \cdot 9 \cdot 8}{10^3}
            = 1 - \frac{72}{10^2}
            = 1 - 0{,}22
            = 0{,}88
        \]
    \end{problem}

    \begin{problem}{26}
        Очевидно, что ответ равен
        \[\sum_{k=0}^{\lfloor n/2 \rfloor} p^{2k} (1-p)^{n-2k} \binom{n}{2k} = \frac{((1-p) + p)^n + ((1-p) - p)^n}{2} = \frac{1 + (1-2p)^n}{2}\]
    \end{problem}

    \begin{problem}{30}
        Рассмотрим для всякого $S \subseteq \{1; \dots; n\}$
        \[B_S := \bigcap_{i \in S} A_i \setminus \bigcup_{j \notin S} A_j\]
        Очевидно, что
        \begin{align*}
            \Omega &= \bigsqcup_{S \subseteq \{1; \dots; n\}} B_S&
            A_i &= \bigsqcup_{\substack{S \subseteq \{1; \dots; n\}\\ i \in S}} B_S
        \end{align*}
        Таким образом
        \[
            \sum_{i=1}^n \PP(A_i)
            = \sum_{i=1}^n \sum_{\substack{S \subseteq \{1; \dots; n\}\\ i \in S}} \PP(B_S)
            = \sum_{S \subseteq \{1; \dots; n\}} \PP(B_S) \cdot |S|
        \]
        
        Предположим $\PP(B_{\{1; \dots; n\}}) = 0$. Тогда
        \[
            \sum_{S \subseteq \{1; \dots; n\}} \PP(B_S) \cdot |S|
            \leqslant \sum_{S \subseteq \{1; \dots; n\}} \PP(B_S) \cdot (n - 1)
            = (n - 1) \sum_{S \subseteq \{1; \dots; n\}} \PP(B_S)
            = n - 1
        \]
        Но $\sum_{i=1}^n A_i > n - 1$ --- противоречие. Таким образом $\PP(S_{\{1; \dots; n\}}) > 0$.
    \end{problem}

    \begin{problem}{23}
        \begin{enumerate}
            \item Рассмотрим равносильную задачу.
                \begin{quotation}
                    В строку $n_1 + \dots + n_N$ раз выписывают случайное число от $1$ до $N$. С какой вероятностью $n_1$ раз будет написано ``$1$'', \dots, $n_N$ раз будет написано ``$N$''?
                \end{quotation}
                
                В таком случае очевидно, что ответ
                \[\frac{\binom{n_1 + \dots + n_N}{n_1, \dots, n_N}}{N^{n_1 + \dots + n_N}}\]

            \item Очевидно, что
                \[p_k = \frac{\binom{n}{k} (N-1)^{n - k}}{N^n}\]

                Заметим, что
                \[
                    \frac{p_{k+1}}{p_k}
                    = \frac{\binom{n}{k+1}}{\binom{n}{k} (N-1)}
                    = \frac{n-k}{(k+1) (N-1)}
                \]
                Соответственно
                \begin{align*}
                    \frac{n-k}{(k+1)(N-1)} &> 1&
                    \frac{n-k}{k+1} &> N - 1&
                    \frac{n+1}{k+1} &> N&
                    \frac{n+1}{N} &> k+1&
                    \frac{n-N+1}{N} &> k&
                \end{align*}
                Следовательно максимум достигается в точке $k = \lceil \frac{n-N+1}{N} \rceil$.

            \item Заметим, что $N = \frac{n}{a}(1 + o(1))$.
                \begin{align*}
                        \lim_{n \to \infty} p_k
                        &= \lim_{n \to \infty} \frac{1}{k!} \left(\frac{N-1}{N}\right)^n \frac{n \cdot \dots \cdot (n-k+1)}{(N-1)^k}\\
                        &= \lim_{n \to \infty} \frac{1}{k!} \left(1 - \frac{a}{n}\right)^n \left(\frac{n}{N}\right)^k\\
                        &= \frac{1}{k!} e^{-a} \frac{1}{a^k} = \frac{1}{k! e^a a^k}\\
                \end{align*}

            \item Заметим, что ответ
                \[\sum_{i=0}^N (-1)^i \binom{N}{i} \frac{(N-i)^n}{N^n}\]

                Действительно. Заметим, что количество исходов, когда конкретные $r$ ячеек пусты равно $(N-r)^n$. Потом эти значения умножаются на $\binom{n}{r}$ и подставляются выше. Поэтому случаи, когда конкретные $r$ ячеек пусты и только они, посчитаны в выражении выше
                \[
                    \sum_{i=0}^r (-1)^i \binom{r}{i}
                    = (1 - 1)^r
                    = \begin{cases}
                        1& \text{ если $r = 0$}\\
                        0& \text{ иначе}
                    \end{cases}
                \]
                раз. Таким образом и остаются только случаи, когда никакие ячейки не пусты.
            
            \item Если рассматривать случаи, когда какие-то конкретные $n-r$ ячейки пусты, то аналогично предыдущему пункту можно получить, что кол-во вариантов, когда только они и пусты, равно
                \[\sum_{i=0}^r (-1)^i \binom{r}{i} (r-i)^n\]
                Следовательно итоговая вероятность равна
                \[\frac{\binom{n}{r}}{N^n}\sum_{i=0}^r (-1)^i \binom{r}{i} (r-i)^n\]
        \end{enumerate}
    \end{problem}

    \begin{problem}{24}
        Заметим, что количество искомых вариантов есть
        \[\sum_{i=0}^n (-1)^i \binom{n}{i} (n-i)!\]
        
        Действительно, пусть фиксированы какие-то $r$ чисел среди набора $\{1; \dots; n\}$. Тогда количество перестановок, где эти $r$ чисел являются неподвижными точками, равно $(n-r)!$, а таких наборов из $r$ чисел $\binom{n}{r}$. Тогда в выражении выше всякая перестановка, сохраняющая на месте ровно какие-то $r$ чисел, посчитана
        \[
            \sum_{i=0}^r (-1)^i \binom{r}{i}
            = (1 - 1)^r
            = \begin{cases}
                1& \text{ если $r = 0$}\\
                0& \text{ иначе}
            \end{cases}
        \]
        раз. Т.е. останутся посчитанными только перестановки без неподвижных точек. Поэтому итоговая вероятность равна
        \[\sum_{i=0}^n (-1)^n \frac{\binom{n}{i} (n-i)!}{n!} = \sum_{i=0}^n \frac{(-1)^i}{i!}\]

        Очевидно, что в пределе (по $n$) значение равно
        \[\sum_{i=0}^\infty \frac{(-1)^i}{i!} = \exp(-1) = \frac{1}{e}\]
    \end{problem}

    \begin{problem}{25}
        WLOG можно считать, что игрок попросил открыть дверь № 1. Тогда у нас имеется 3 случая, где может быть приз. Если приз за дверью № 1, то менять дверь не выгодно; а в остальных двух случаях это наоборот выгодно. Поэтому в таком случае вероятность выигрыша при смене двери будет равна $2/3$, а в противном случае --- $1/3$.
    \end{problem}

    \begin{problem}{31}
        \begin{enumerate}
            \item Очевидно, $\PP(A) = [1 \in A]$ является искомой мерой.
            \item Пусть $\mathcal{P}$ --- множество простых. Тогда для всякого $n \in \NN \setminus \{0\}$
                \[\PP(n) = \PP(A_n) \prod_{p \in \mathcal{P}} (1 - \PP(A_{np}))\]
                Легко видеть, что произведение сходится. При этом если оно сходится к чему-то положительному, то тогда $(\PP(A_{np}))_{p \in \mathcal{P}} \to 0$. В таком случае
                \[
                    \ln(\PP(n))
                    = \ln(\PP(A_n)) + \sum_{p \in \mathcal{P}} \ln(1 - \PP(A_{np}))
                    = \const - \sum_{p \in \mathcal{P}} \PP(A_{np})
                    = \const - \PP(A_n) \sum_{p \in \mathcal{P}} \PP(A_p)
                \]
                Следовательно $\PP(n) > 0$ тогда и только тогда, когда $\PP(A_{np}) > 0$, ни для какого $p \in \mathcal{P}$ $\PP(A_{np}) = 1$ и $\sum_{p \in \mathcal{P}} \PP(A_p)$ сходится.

                Но заметим, что
                \[
                    \sum_{n=1}^\infty \frac{1}{n}
                    = \prod_{p \in \mathcal{P}} \sum_{d=0}^\infty \frac{1}{p^d}
                    = \prod_{p \in \mathcal{P}} \frac{1}{1 - 1/p}
                \]
                Формально говоря, префикс гармонического ряда равен произведению префиксов в геометрических суммах; поэтому из равенства $\sum_{n=1}^\infty \frac{1}{n} = +\infty$ следует, что $\prod_{p \in \mathcal{P}} (1 - \frac{1}{p}) = 0$. И как мы успели понять это значит, что $\sum_{p \in \mathcal{P}} \frac{1}{p} = +\infty$.
                
                Поэтому $\sum_{p \in \mathcal{P}} \PP(A_p) = +\infty$; значит для всякого $n \in \NN \setminus \{0\}$ имеем, что $\PP(n) = 0$ --- противоречие.
        \end{enumerate}
    \end{problem}

    \begin{problem}{14}
        TODO
    \end{problem}
\end{document}