\documentclass[12pt,a4paper]{article}
\usepackage{solutions}
% \usepackage{float}
\usepackage{inkscape}

\title{Занятие от 18.02.\\Геометрия и топология. 1 курс.\\Решения.}
\author{Глеб Минаев @ 102 (20.Б02-мкн)}
% \date{}

\newcommand{\const}{\ensuremath{\mathrm{const}}\xspace}

\begin{document}
    \maketitle

    \begin{problem}{28}
        Пронумеруем школьников последовательно целыми числами так, что выбранный школьник имеет индекс $0$. Посмотрим с какой вероятностью он может подсмотреть решение у соседа справа.

        Заметим, что он сможет подсмотреть решение у соседа справа тогда и только тогда, когда будет выполнено следующее условие. Существует $n \in \NN \cup \{0\}$, что школьники $0$, $1$, \dots, $n$ подсмотрели решение у соседа справа, а школьник $n+1$ смог придумать решение сам. Для всякого $n \in \NN \cup \{0\}$ рассмотрим событие $A_n$ заключающееся в том, что школьники $0$, \dots, $n$ не смогли сами придумать решение, но смогли подсмотреть у соседа справа, а школьник $n+1$ смог придумать его сам. Тогда понятно, что все события $\{A_n\}_{n=0}^\infty$ дизъюнктны, а в объединении дают исходное событие (где также считается, что школьник $0$ не решил задачу сам). Тогда
        \[\PP(A_n) = \left(\left(1 - \frac{1}{2}\right) \cdot \frac{1}{4}\right)^{n+1} \cdot \frac{1}{2} = \frac{1}{16} \cdot \left(\frac{1}{8}\right)^n\]
        Следовательно
        \[\sum_{n=0}^\infty \PP(A_n) = \frac{1}{16} \cdot \frac{1}{1 - \frac{1}{8}} = \frac{1}{14}\]

        Таким образом школьник сможет подсмотреть задачу справа при условии, что не решил её сам, с вероятностью $1/14$, а значит сможет подсмотреть её вообще с вероятностью $1/7$. Следовательно он не сможет подсмотреть её с одной стороны с вероятностью $6/7$, а с обеих --- с вероятностью $36/49$. А значит не получит решение с вероятностью $18/49$.
    \end{problem}

    \begin{problem}{29}
        Пусть вероятности выпадения цифр равны $p_1$, \dots, $p_6$ соответственно ($p_1 + \dots + p_6 = 1$). Следовательно вероятность выпадения совпадающих цифр равна
        \[p_1^2 + \dots + p_6^2 \geqslant \frac{(p_1 + \dots + p_6)^2}{6} = \frac{1}{6}\]
        При этом мы знаем, что соблюдается равенство, следовательно $p_1 = \dots = p_6$.
    \end{problem}

    \begin{problem}{13}
        Мы имеем, что
        \[\frac{\PP(B \cap A)}{\PP(A)} = \PP(B \mid A) = \PP(B \mid \overline{A}) = \frac{\PP(B \cap \overline{A})}{\PP(\overline{A})}\]
        Следовательно
        \[\PP(B \mid A) = \frac{\PP(B \cap A) + \PP(B \cap \overline{A})}{\PP(A) + \PP(\overline{A})} = \frac{\PP(B)}{1} = \PP(B)\]
        Таким образом события независимы.
    \end{problem}

    \begin{problem}{3}
        Представим более общую задачу. Пусть дан ориентированный граф $G = \langle V, E \rangle$ без ориентированных циклов, и в нём выделено две вершины: ``вход'' $s$ и ``выход'' $t$. путешественник выходит из ``входа'' и с заданным распределением вероятностей на рёбрах идёт в следующую вершину (в нашем случае распределение равномерное: все рёбра выходящие из одной вершины имеют равную вероятность). Тогда для всякой вершины $v \in V$ рассмотрим событие $A_v$ попадания в вершину $v$. Также пусть вероятность перехода по ребру $e \in E$ равна $p_e$. Тогда несложно понять, что
        \begin{itemize}
            \item $\PP(s) = 1$;
            \item $\forall v \in V \setminus \{s\} \quad \PP(A_v) = \sum_{u \in V \setminus \{v\}} \PP(A_u) p_{(u; v)}$.
        \end{itemize}
        
        Несложно понять, что по данным правилам единственным образом восстанавливаются все вероятности попадания в вершины, и сделать это можно легко алгоритмически. Таким образом применим этот алгоритм к нашему случаю.

        Несложно видеть, что ответ равен $5/9$.
    \end{problem}

    \begin{problem}{5}
        Заметим, что все события делятся на три случая:
        \begin{enumerate}
            \item первое орудие попало, второе -- промазало;
            \item второе орудие попало, первое -- промазало;
            \item оба орудия попали и их цели совпали.
        \end{enumerate}

        Рассмотрим вероятности данных компонент.
        \begin{enumerate}
            \item Очевидно, что вероятность равна $0.2 \cdot (1 - 0.3) = 0.14$.
            \item Очевидно, что вероятность равна $0.3 \cdot (1 - 0.2) = 0.24$.
            \item Очевидно, что вероятность равна $0.2 \cdot 0.3 \cdot (1/3^2 + 1/3^2 + 1/3^2) = 0.02$.
        \end{enumerate}

        Итоговая вероятность получается равной $0.4$.
    \end{problem}

    \begin{problem}{7}\ 
        
        \textbf{Способ 1.} Заметим, что задача равносильна такой.
        \begin{quotation}
            Есть счётная последовательность. В каждой ячейки последовательности независимо друг от друга с вероятностью $a$ появляется яйцо; а с вероятностью $p$ всякое яйцо вылупляется. С какой вероятностью вылупится ровно $l$ черепашат?
        \end{quotation}
        Тогда получаем, что на всякой ячейке появляется черепашонок с вероятностью $ap$. А значит вероятность того, что черепашат будет ровно $l$ равна $e^{-ap} \frac{(ap)^l}{l!}$.

        \textbf{Способ 2.} Честно посчитаем:
        \begin{align*}
            \sum_{k=l}^{+\infty} \left(e^{-a} \frac{a^k}{k!}\right) p^l (1-p)^{k-l} \binom{k}{l}
            &= \sum_{k=l}^{+\infty} e^{-a} \frac{a^k p^l (1-p)^{k-l}}{l! (k-l)!}\\
            &= e^{-a} \frac{a^l p^l}{l!}\sum_{k=0}^{+\infty} \frac{a^k (1-p)^k}{k!}\\
            &= e^{-a} \frac{a^l p^l}{l!} e^{a(1-p)}\\
            &= e^{-ap} \frac{(ap)^l}{l!}\\
        \end{align*}
    \end{problem}

    \begin{problem}{9}
        Рассмотрим такую задачу.
        \begin{quotation}
            Всё как раньше, но только невеста пропускает первые $A$ женихов и после них берёт первого лучшего чем все предыдущие. С какой вероятностью он будет наилучшим.
        \end{quotation}
        Пронумеруем женихов по хорошести от худшего к лучшему числами от $1$ до $N$. Тогда всего случаев $N!$.
        
        Пусть жених $N$ встретился на месте $M + 1$ ($A \leqslant M < N$). Тогда случай является подходящим, если среди женихов на первых $M$ местах лучший находится среди первых $A$ мест. Заметим, что вероятность этого события не зависит от того, как набор женихов стоит перед женихом $N$; поэтому WLOG там стоят первые $M$ женихов.

        Несложно видеть, что вероятность того, что жених $M$ будет среди первых $A$ равна $A/M$. Также несложно видеть, что жених $N$ может быть на каждом месте с одной и той же вероятностью. Следовательно искомая вероятность равна
        \[\sum_{M = A}^{N-1} \frac{1}{N} \cdot \frac{A}{M} = \frac{A}{N} \cdot \sum_{M=A}^{N-1} \frac{1}{M} = \frac{A}{N} \cdot \sum_{M=A}^{N-1} \frac{1}{M/N} \cdot \frac{1}{N}\]

        Поэтому несложно видеть, что если $N \to \infty$, а $A/N \to a$, то искомый предел равен $a\ln(1/a)$.
    \end{problem}

    \begin{problem}{10}
        Пусть пришло $n$ доноров. Найдём вероятность того, что кровь не подойдёт. Разберём случаи.
        \begin{enumerate}
            \item Если пациент группы I, то вклад в вероятность провала равен $p_\mathrm{I} \cdot (1-p_\mathrm{I})^n$.
            \item Если пациент группы II, то вклад в вероятность провала равен $p_\mathrm{II} \cdot (1-p_\mathrm{I}-p_\mathrm{II})^n$.
            \item Если пациент группы III, то вклад в вероятность провала равен $p_\mathrm{III} \cdot (1-p_\mathrm{I}-p_\mathrm{III})^n$.
            \item Если пациент группы IV, то вклад в вероятность провала равен $0$.
        \end{enumerate}
        Следовательно вероятность удачи равна
        \[
            1 - p_\mathrm{I} \cdot (1-p_\mathrm{I})^n - p_\mathrm{II} \cdot (1-p_\mathrm{I}-p_\mathrm{II})^n - p_\mathrm{III} \cdot (1-p_\mathrm{I}-p_\mathrm{III})^n
            = 1 - 0.337 \cdot 0.663^n - 0.375 \cdot 0.288^n - 0.209 \cdot 0.454^n
        \]

        \begin{enumerate}\ItemedProblem
            \item При $n=1$ вероятность равна $0.573683$.
            \item Вероятность $\geqslant 0.9$ при $n \geqslant 4$.
        \end{enumerate}
    \end{problem}
\end{document}