\documentclass[12pt,a4paper]{article}
\usepackage{../.tex/mcs-notes}
\usepackage{todonotes}
\usepackage{mathrsfs}
\usepackage{bussproofs}
\usepackage{multicol}
\usepackage{stmaryrd}

\settitle
{Основы математической логики.}
{\href{http://www.mi-ras.ru/~speranski/}{Станислав Олегович Сперанский}}
{fundamentals-of-math-logic/main.pdf}
\date{}

% \newcommand{\subsets}{\ensuremath{\mathcal{P}}\xspace}
% \newcommand{\finsubsets}{\ensuremath{\mathcal{P_{\mathrm{fin}}}}\xspace}
% \DeclareMathOperator{\Ind}{Ind}
\DeclareMathOperator{\dom}{dom}
\DeclareMathOperator{\range}{range}
% \DeclareMathOperator{\Min}{Min}
% \DeclareMathOperator{\Seq}{Seq}
% \DeclareMathOperator{\Left}{left}
% \DeclareMathOperator{\Right}{right}
% \DeclareMathOperator{\IS}{IS}
% \DeclareMathOperator{\ord}{ord}
% \DeclareMathOperator{\Ord}{Ord}
% \DeclareMathOperator{\card}{card}
% \DeclareMathOperator{\Card}{Card}
\newcommand{\ZF}{\ensuremath{\mathrm{ZF}}\xspace}
\newcommand{\Caxiom}{\ensuremath{\mathrm{C}}\xspace}
\newcommand{\ZFC}{\ensuremath{\mathrm{ZFC}}\xspace}
\newcommand{\CH}{\ensuremath{\mathrm{CH}}\xspace}
\newcommand{\Prop}{\ensuremath{\mathrm{Prop}}\xspace}
\newcommand{\Formul}{\ensuremath{\mathrm{Form}}\xspace}
\newcommand{\Sub}{\ensuremath{\mathrm{Sub}}\xspace}
\newcommand{\NP}{\ensuremath{\mathrm{NP}}\xspace}
\newcommand{\Pred}{\ensuremath{\mathrm{Pred}}\xspace}
\newcommand{\Func}{\ensuremath{\mathrm{Func}}\xspace}
\newcommand{\Const}{\ensuremath{\mathrm{Const}}\xspace}
\newcommand{\arity}{\ensuremath{\mathrm{arity}}\xspace}
\newcommand{\id}{\ensuremath{\mathrm{id}}\xspace}
\newcommand{\Id}{\ensuremath{\mathrm{Id}}\xspace}
\newcommand{\Img}{\ensuremath{\mathrm{Im}}\xspace}
\newcommand{\Aut}{\ensuremath{\mathrm{Aut}}\xspace}
\newcommand{\least}{\ensuremath{\mathrm{least}}\xspace}
\newcommand{\Var}{\ensuremath{\mathrm{Var}}\xspace}
\newcommand{\Term}{\ensuremath{\mathrm{Term}}\xspace}
\newcommand{\Atom}{\ensuremath{\mathrm{Atom}}\xspace}
\newcommand{\sub}{\ensuremath{\mathrm{sub}}\xspace}
\newcommand{\FV}{\ensuremath{\mathrm{FV}}\xspace}
\newcommand{\Sent}{\ensuremath{\mathrm{Sent}}\xspace}
\newcommand{\Th}{\ensuremath{\mathrm{Th}}\xspace}
\newcommand{\supp}{\ensuremath{\mathrm{supp}}\xspace}
\newcommand{\Eq}{\ensuremath{\mathrm{Eq}}\xspace}
\newcommand{\tildeforall}{\widetilde{\forall}}
\newcommand{\tildeexists}{\widetilde{\exists}}
\newcommand{\Abs}{\ensuremath{\mathrm{Abs}}\xspace}


\begin{document}
    \maketitle

    \listoftodos[TODOs]

    \tableofcontents

    \vspace{2em}
    Материалы лекций: \href{http://www.mi-ras.ru/~speranski/courses/logic-1-2021-spring/materials.html}{ссылка}
    % \todo[color=green!40, inline]{Добавить конспекты теории из упражнений. Слить аккуратно воедино. Добавить ссылку на упражнения.}
    
    % Литература:
    % \begin{itemize}
    %     \item K. Hrbacek and T. Jech. Introduction to Set Theory. 3rd ed., revised and expanded. Marcel Dekker, Inc., 1999.
    %     \item T. Jech. Set Theory. 3rd ed., revised and expanded. Springer, 2002.
    % \end{itemize}

    \subsection{Формальности про алфавиты и слова}

    \begin{definition}
        \emph{Алфавит} $A$ --- множество элементов произвольной природы.

        \emph{$A$-слова} или \emph{слова над алфавитом $A$} --- элементы $A^*$ (т.е. всевозможные конечные последовательности элементов из $A$).

        Для всякого $w \in A^*$ \emph{длиной слова $w$} называется $|w|$ (что также равно $\dom(w)$).
    \end{definition}

    \begin{definition}
        Пусть даны слова $w_1$ и $w_2$ над $A$. Рассмотрим отображение
        \[
            v:
            |w_1| + |w_2| \to A,
            i \mapsto \begin{cases}
                w_1(i)& \text{ если $i < |w_1|$}\\
                w_2(i - |w_1|)& \text{ если $i \geqslant |w_1|$}
            \end{cases}
        \]
        Понятно, что $v \in A^*$. Полученное $v$ называется конкатенацией $w_1$ и $w_2$ и обозначается $w_1w_2$.
    \end{definition}

    \begin{definition}
        Пусть $w, w' \in A^*$. Тогда $w'$ называется \emph{подсловом $w$}, если $w = v_0 w' v_1$ для некоторых $\{v_0; v_1\} \subseteq A^*$. Обозначение: $w' \preccurlyeq w$.

        В этом случае $\langle w'; |v_0|\rangle$ называется \emph{вхождением} $w'$ в $w$.
    \end{definition}

    \begin{definition}
        Пусть $\langle w', k\rangle$ --- вхождение $w'$ в $w$, т.е. $w = v_0 w' v_1$, где $|v_0| = k$. Тогда для всякого $u \in A^*$ можно определить
        \[w [w' / u, k] := v_0 u v_1,\]
        т.е. результат замены данного вхождения $w'$ в $w$ на $u$.
        
        Если никакие два различных вхождения $w'$ в $w$ не пересекаются, то можно определить $w [w' / u]$ как результат одновременной замены всех вхождений $w'$ в $w$ на $u$. 
    \end{definition}

    \subsection{Язык пропозициональной классической логики\\(Propositional Classic Logic, PCL)}

    \begin{definition}
        Пусть (навсегда) зафиксировано некоторое счётное множество \Prop. Будем называть его элементы \emph{пропозициональными переменными} или просто \emph{переменными}.

        Алфавит $\mathscr{L}$ пропозициональной классической логики состоит из элементов \Prop, а также:
        \begin{itemize}
            \item символов связок:
                \begin{itemize}
                    \item ``$\rightarrow$'' --- символ импликации,
                    \item ``$\wedge$'' --- символ конъюнкции,
                    \item ``$\vee$'' --- символ дизъюнкции,
                    \item ``$\neg$'' --- символ отрицания,
                \end{itemize}
            \item и вспомогательных символов: ``('' и ``)''.
        \end{itemize}

        Обозначим за \Formul наименьшее подмножество $\mathscr{L}^*$, замкнутое относительно следующих порождающих правил:
        \begin{itemize}
            \item если $p \in \Prop$, то $p \in \Formul$;
            \item если $\{\varphi, \psi\} \in \Prop$, то $(\varphi \rightarrow \psi) \in \Formul$;
            \item если $\{\varphi, \psi\} \in \Prop$, то $(\varphi \wedge \psi) \in \Formul$;
            \item если $\{\varphi, \psi\} \in \Prop$, то $(\varphi \vee \psi) \in \Formul$;
            \item если $\varphi \in \Formul$, то $\neg \varphi \in \Formul$.
        \end{itemize}
        Элементы \Formul называются \emph{формулами}.
    \end{definition}

    \begin{theorem}
        \Formul существует.
    \end{theorem}

    \begin{proof}
        Рассмотрим семейство $T$ всех подмножеств $\mathscr{L}^*$, удовлетворяющих порождающим правилам выше. Заметим, что $T$ не пусто, так как содержит $\mathscr{L}^*$. Тогда можно рассмотреть $F := \bigcap T$. Несложно убедиться, что оно удовлетворяет всем порождающим правилам, значит лежит в $T$. И при этом меньше по включению всех других множеств в $T$. Значит его и можно взять в качестве $\Formul$.
    \end{proof}

    \begin{definition}
        Будем говорить, что $\varphi$ является \emph{началом $\psi$}, и писать $\varphi \sqsubseteq \psi$, если $\psi = \varphi \tau$ для некоторого $v \in \mathscr{L}^*$.
    \end{definition}

    \begin{lemma}
        Всякое $\varphi \in \Formul$ имеет один из следующих видов:
        \begin{enumerate}
            \item $p$ для некоторого $p \in \Prop$;
            \item $(\theta \circ \chi)$ для некоторых $\{\theta; \chi\} \subseteq \Formul$ и $\circ \in \{\rightarrow; \wedge; \vee\}$;
            \item $\neg \theta$ для некоторого $\theta \in \Formul$.
        \end{enumerate}
    \end{lemma}

    \begin{proof}
        Предположим противное. Тогда рассмотрим $F := \Formul \setminus \{\varphi\}$. Заметим, что $F$ удовлетворяет тем же порождающим правилам, что и $\Formul$. Действительно:
        \begin{itemize}
            \item Если $p \in \Prop$, то $p \in \Formul$. При этом $p \neq \varphi$ по условию леммы. Следовательно $p \in F$.
            \item Если $\{\varphi, \psi\} \in \Prop$, то $(\varphi \rightarrow \psi) \in \Formul$. При этом $(\varphi \rightarrow \psi) \neq \varphi$ по условию леммы. Следовательно $(\varphi \rightarrow \psi) \in F$. Аналогично для $\wedge$ и $\vee$.
            \item Если $\varphi \in \Formul$, то $\neg \varphi \in \Formul$. При этом $\neg \varphi \neq \varphi$ по условию леммы. Следовательно $\neg \varphi \in F$.
        \end{itemize}
        Значит $F$ --- меньшее по включение чем $\Formul$ множество, удовлетворяющее условиям наложенным на $\Formul$ --- противоречие.
    \end{proof}

    \begin{corollary}
        Рассмотрим последовательность множеств $(F_n)_{n=0}^\infty$, что $F_0 = \Prop$, а
        \[
            F_{n+1} = F_n 
            \cup \{(\varphi \circ \chi) \mid \{\theta; \chi\} \subseteq \Formul \wedge \circ \in \{\rightarrow; \wedge; \vee\}\}
            \cup \{\neg \theta \mid \theta \in \Formul\}
        \]
        Тогда
        \begin{enumerate}
            \item всякое $F_n \subseteq \Formul$;
            \item всякое $\varphi \in \Formul$ лежит в некотором $F_n$.
        \end{enumerate}
    \end{corollary}

    \begin{corollary}
        $\bigcup_{n=0}^\infty F_n = \Formul$. 
    \end{corollary}

    \begin{lemma}
        Пусть $\{\varphi; \psi\} \subseteq \Formul$ таковы, что $\psi \sqsubseteq \varphi$. Тогда $\psi = \varphi$.
    \end{lemma}

    \begin{proof}
        Докажем утверждение возвратной индукцией по $|\varphi|$.
        
        Рассмотрим случаи.
        \begin{enumerate}
            \item Если $\varphi \in \Prop$, то, очевидно, $\psi = \varphi$.
            \item Если $\varphi = (\theta \circ \chi)$, где $\{\theta; \chi\} \in \Formul$ и $\circ \in \{\rightarrow; \wedge; \vee\}$, то $\psi$ начинается на ``('', значит имеет вид $(\theta' \circ' \chi')$, где $\{\theta'; \chi'\} \in \Formul$ и $\circ' \in \{\rightarrow; \wedge; \vee\}$. Следовательно либо $\theta \sqsubseteq \theta'$, либо $\theta' \sqsubseteq \theta$. Но $|\theta| < |\varphi| - 3$, и $|\theta'| < |\psi| - 3 \leqslant |\varphi| - 3$. Тогда можно применить предположение индукции и получить, что $\theta = \theta'$. Значит $\circ = \circ'$, а далее по аналогии получаем, что $\chi = \chi'$. Следовательно $\varphi = \psi$.
            \item Если $\varphi = \neg \theta$, то $\psi$ начинается на ``$\neg$'', следовательно $\psi = \neg \theta'$. Тогда $\theta' \sqsubseteq \theta$, а тогда по предположению индукции $\theta' = \theta$, значит $\varphi = \psi$.
        \end{enumerate}
    \end{proof}

    \begin{theorem}[о единственности представления формул]
        Всякая формула в $\Formul \setminus \Prop$ представляется единственным образом в одном из видов
        \begin{itemize}
            \item $(\theta \rightarrow \chi)$,
            \item $(\theta \wedge \chi)$,
            \item $(\theta \vee \chi)$,
            \item $\neg \theta$,
        \end{itemize}
        где $\{\theta; \chi\} \subseteq \Formul$.
    \end{theorem}

    \begin{proof}
        По доказанной лемме всякое $\phi$ имеет такое представление. Пусть тогда их несколько; рассмотрим случаи.
        \begin{enumerate}
            \item Если $\phi$ начинается на ``('', то тогда $\phi = (\theta \circ \chi) = (\theta' \circ' \chi')$. Тогда по доказанной лемме $\theta = \theta'$, $\circ = \circ'$, $\chi = \chi'$. Значит представления совпадают.
            \item Если $\phi$ начинается на ``$\neg$'', то $\phi = \neg \theta = \neg \theta'$. Тогда $\theta = \theta'$, а следовательно представления совпадают.
        \end{enumerate}
    \end{proof}

    \begin{definition}
        Для всякого $\varphi \in \Formul$ определим
        \[\Sub(\varphi) := \{\psi \Formul \mid \psi \preccurlyeq \varphi\}.\]
        Элементы $\Sub(\varphi)$ называют \emph{подформулами $\varphi$}.
    \end{definition}

    \begin{lemma}
        Пусть $\varphi \in \Formul$. Тогда каждое вхождение ``$\neg$'' или ``$($'' в $\varphi$ является началом вхождения некоторой подформулы.
    \end{lemma}

    \begin{proof}
        \todo[inline]{TODO (А надо ли?)}
    \end{proof}

    \begin{theorem}
        Пусть $\varphi \in \Formul$.
        \begin{enumerate}
            \item Если $\varphi \in \Prop$, то $\Sub(\varphi) = \{\varphi\}$.
            \item Если $\varphi = (\theta \circ \chi)$, где $\{\theta; \chi\} \subseteq \Formul$ и $\circ \in \{\rightarrow; \wedge; \vee\}$, то
                \[\Sub(\varphi) = \Sub(\theta) \cup \Sub(\chi) \cup \{\varphi\}.\]
            \item Если $\varphi = \neg \theta$, где $\theta \in \Formul$, то
                \[\Sub(\varphi) = \Sub(\theta) \cup \{\varphi\}.\]
        \end{enumerate}
    \end{theorem}

    \begin{proof}
        \begin{enumerate}
            \item Очевидно.
            \item \todo[inline]{Непонятно. TODO}
            \item \todo[inline]{Непонятно. TODO}
        \end{enumerate}
    \end{proof}

    \subsection{Семантика пропозициональной классической логики}

    \begin{definition}
        Под \emph{оценкой} мы будем понимать произвольную функцию из $\Prop$ в $2$ (т.е. в $\{0; 1\}$). Интуитивно $0$ --- <<ложь>>, а $1$ --- <<правда>>.
    \end{definition}

    \begin{theorem}
        Пусть дана случайная $v: \Prop \to 2$. Тогда существует единственная $v^*: \Formul \to 2$, которая удовлетворяет следующим свойствам:
        \begin{enumerate}
            \item $\forall p \in \Prop\qquad v^*(p) = 1 \quad \Longleftrightarrow \quad v(p) = 1$;
            \item $\forall \{\varphi; \psi\} \subseteq \Formul\qquad v^*(\text{``$(\varphi \rightarrow \psi)$''}) = 1 \quad \Longleftrightarrow \quad v^*(\varphi) = 0 \vee v^*(\psi) = 1$;
            \item $\forall \{\varphi; \psi\} \subseteq \Formul\qquad v^*(\text{``$(\varphi \wedge \psi)$''}) = 1 \quad \Longleftrightarrow \quad v^*(\varphi) = 1 \wedge v^*(\psi) = 1$;
            \item $\forall \{\varphi; \psi\} \subseteq \Formul\qquad v^*(\text{``$(\varphi \vee \psi)$''}) = 1 \quad \Longleftrightarrow \quad v^*(\varphi) = 1 \vee v^*(\psi) = 1$;
            \item $\forall \varphi \in \Formul\qquad v^*(\neg \varphi) = 1 \quad \Longleftrightarrow \quad v^*(\varphi) = 0$.
        \end{enumerate}
    \end{theorem}

    \begin{proof}
        \todo[inline]{TODO}
    \end{proof}

    \begin{definition}
        Если для некоторой оценки $v$ и формулы $\varphi$ верно, что $v^*(\varphi) = 1$, то пишут $v \Vdash \varphi$.
    \end{definition}

    \begin{definition}
        Формулу $\varphi$ называют:
        \begin{itemize}
            \item \emph{выполнимой}, если $v \Vdash \varphi$ для некоторой оценки $v$;
            \item \emph{общезначимой} (или \emph{тождественно истинной}, или \emph{тавтологией}), если $v \Vdash \varphi$ для всякой оценки $v$.
        \end{itemize}
    \end{definition}

    \begin{remark*}
        Очевидно, например, что
        \[\varphi \text{ общезначима}\quad \Longleftrightarrow\quad \neg \varphi \text{ не выполнима}.\]
    \end{remark*}

    \begin{theorem*}[Кук-Левин]
        Проблема выполнимости для пропозициональной классической логики \NP-полна.
    \end{theorem*}

    \begin{definition}
        Пусть $\Gamma \subseteq \Formul$ и $\varphi \in \Formul$. Говорят, что $\varphi$ \emph{семантически следует из $\Gamma$}, если для любой оценки $v$
        \[(\forall \psi \in \Gamma \quad v \Vdash \psi) \qquad \longrightarrow \qquad v \Vdash \varphi.\]
        Обозначение: $\Gamma \vDash \varphi$.

        Вместо $\varnothing \vDash \varphi$ обычно пишут $\vDash \varphi$.
    \end{definition}

    \begin{remark*}
        Очевидно, например, что
        \[\vDash \varphi\quad \Longleftrightarrow\quad \varphi \text{ общезначима}.\]
    \end{remark*}

    \begin{definition}
        Формулы $\varphi$ и $\psi$ называются \emph{семантически эквивалентными}, если $\vDash \varphi \leftrightarrow \psi$. Обозначение: $\varphi \equiv \psi$.
    \end{definition}

    \begin{example}
        Для любых $\{\varphi; \psi; \chi\} \subseteq \Formul$:
        \begin{itemize}
            \item $(\varphi \rightarrow \psi) \equiv \neg \varphi \vee \psi$;
            \item $(\varphi \vee \psi) \wedge \chi \equiv (\varphi \wedge \chi) \vee (\psi \wedge \chi)$;
            \item $(\varphi \wedge \psi) \vee \chi \equiv (\varphi \vee \chi) \wedge (\psi \vee \chi)$;
            \item $\neg (\varphi \wedge \psi) \equiv \neg \varphi \vee \neg \psi$;
            \item $\neg (\varphi \vee \psi) \equiv \neg \varphi \wedge \neg \psi$;
            \item $\neg \neg \varphi \equiv \varphi$.
        \end{itemize}
    \end{example}

    \begin{exercise}
        Всякая формула семантически эквивалентна некоторой ДНФ.
    \end{exercise}

    \subsection{Гильбертовское исчисление для пропозициональной классической логики}

    \begin{definition}
        Рассмотрим следующие аксиомы:
        \begin{description}
            \item[$\mathrm{I1}$.] $\varphi \rightarrow (\psi \rightarrow \varphi)$;
            \item[$\mathrm{I2}$.] $(\varphi \rightarrow (\psi \rightarrow \chi)) \rightarrow ((\varphi \rightarrow \psi) \rightarrow (\varphi \rightarrow \chi))$;
            \item[$\mathrm{C1}$.] $\varphi \wedge \psi \rightarrow \varphi$;
            \item[$\mathrm{C2}$.] $\varphi \wedge \psi \rightarrow \psi$;
            \item[$\mathrm{C3}$.] $\varphi \rightarrow (\psi \rightarrow \varphi \wedge \psi)$;
            \item[$\mathrm{D1}$.] $\varphi \rightarrow \varphi \vee \psi$;
            \item[$\mathrm{D2}$.] $\psi \rightarrow \varphi \vee \psi$;
            \item[$\mathrm{D3}$.] $(\varphi \rightarrow \chi) \rightarrow ((\psi \rightarrow \chi) \rightarrow (\varphi \vee \psi \rightarrow \chi))$;
            \item[$\mathrm{N1}$.] $(\varphi \rightarrow \psi) \rightarrow ((\varphi \rightarrow \neg \psi) \rightarrow \neg \varphi)$;
            \item[$\mathrm{N2}$.] $\neg \varphi \rightarrow (\varphi \rightarrow \psi)$;
            \item[$\mathrm{N3}$.] $\varphi \vee \neg \varphi$.
        \end{description}

        Также имеется ровно одно \emph{правило вывода}, именуемое \emph{``modus ponens''}:
        \begin{prooftree}
            \AxiomC{$\varphi$}
            \AxiomC{$\varphi \rightarrow \psi$}
                \RightLabel{($\mathrm{MP}$)}
                \BinaryInfC{$\psi$}
        \end{prooftree}

        Пусть $\Gamma \subseteq \Formul$. \emph{Вывод из $\Gamma$} в данном гильбертовском исчислении --- конечная последовательность
        \[\varphi_0, \dots, \varphi_n\]
        (где $n \in \NN$) элементов $\Formul$, что для каждого $i \in \{0; \dots; n\}$ верно одно из следующих условий:
        \begin{itemize}
            \item $\varphi_i$ есть аксиома;
            \item $\varphi_i$ является элементом $\Gamma$;
            \item существуют $\{j; k\} \subseteq \{0; \dots; i-1\}$ такие, что $\varphi_k = \varphi_j \rightarrow \varphi_i$.
        \end{itemize}
        При этом $\varphi_n$ называется \emph{заключением} рассматриваемого вывода, а элементы $\Gamma$ --- его \emph{гипотезами}.

        Для $\Gamma \cup \{\varphi\} \subseteq \Formul$ запись $\Gamma \vdash \varphi$ означает, что существует вывод из $\Gamma$ с заключением $\varphi$. Вместо $\varnothing \vdash \varphi$ обычно пишут $\vdash \varphi$.
    \end{definition}

    \begin{lemma}\ 
        \begin{enumerate}
            \item {\bf Монотонность.} Если $\Gamma \subseteq \Delta$ и $\Gamma \vdash \varphi$, то $\Delta \vdash \varphi$.
            \item {\bf Транзитивность.} Если для всякого $\psi \in \Gamma$ верно $\Delta \vdash \psi$ и $\Gamma \vdash \varphi$, то $\Delta \vdash \varphi$.
            \item {\bf Компактность.} Если $\Gamma \vdash \varphi$, то для некоторого конечного $\Delta \subseteq \Gamma$ верно $\Delta \vdash \varphi$.
        \end{enumerate}
    \end{lemma}

    \begin{proof}\ 
        \begin{enumerate}
            \item Рассматривая вывод $\varphi$ из $\Gamma$, сиюминутно получаем вывод $\varphi$ из $\Delta$.
            \item Возьмём вывод $\varphi$ из $\Gamma$. Рассмотрим все использованные утверждения $\Gamma$ в этом выводе; получим конечное множество $\Gamma'$. Далее для всякого $\psi \in \Gamma'$ рассмотрим вывод $\psi$ из $\Delta$, сотрём $\psi$ на конце этого вывода и припишем его в начало ранее рассмотренного вывода $\varphi$. Тогда несложно понять, что мы получаем вывод $\varphi$ из $\Delta$.
            \item Хватает просто взять в качестве $\Delta$ множество всех формул из $\Gamma$, использованных в каком-то конкретном выводе $\varphi$ из $\Gamma$. Тогда очевидно, что $\Delta$ конечно, а рассмотренный вывод станет выводом $\varphi$ из $\Delta$.
        \end{enumerate}
    \end{proof}

    \begin{example}
        Покажем, что $\vdash \varphi \vee \psi \rightarrow \psi \vee \varphi$:
        \begin{center}
            \begin{tabular}{rll}
                1.& $(\varphi \rightarrow \psi \vee \varphi) \rightarrow ((\psi \rightarrow \psi \vee \varphi) \rightarrow (\varphi \vee \psi \rightarrow \psi \vee \varphi))$& $\mathrm{D3}$\\
                2.& $\varphi \rightarrow \psi \vee \varphi$& $\mathrm{D2}$\\
                3.& $\psi \rightarrow \psi \vee \varphi$& $\mathrm{D1}$\\
                4.& $(\psi \rightarrow \psi \vee \varphi) \rightarrow (\varphi \vee \psi \rightarrow \psi \vee \varphi)$& из 2 и 1; $\mathrm{MP}$\\
                5.& $\varphi \vee \psi \rightarrow \psi \vee \varphi$& из 3 и 4; $\mathrm{MP}$\\
            \end{tabular}
        \end{center}
    \end{example}

    \begin{example}
        Покажем, что $\{\varphi \wedge \psi\} \vdash \psi \wedge \varphi$:
        \begin{center}
            \begin{tabular}{rll}
                1.& $\psi \rightarrow (\varphi \rightarrow \psi \wedge \varphi)$& $\mathrm{C3}$\\
                2.& $\varphi \wedge \psi \rightarrow \psi$& $\mathrm{C2}$\\
                3.& $\varphi \wedge \psi \rightarrow \varphi$& $\mathrm{C1}$\\
                4.& $\varphi \wedge \psi$& гипотеза\\
                5.& $\psi$& из 4 и 2; $\mathrm{MP}$\\
                6.& $\varphi$& из 4 и 3; $\mathrm{MP}$\\
                7.& $\varphi \rightarrow \psi \wedge \varphi$& из 5 и 1; $\mathrm{MP}$\\
                8.& $\psi \wedge \varphi$& из 6 и 7; $\mathrm{MP}$\\
            \end{tabular}
        \end{center}
    \end{example}

    \begin{lemma}\label{self-implication-lemma}
        $\vdash \varphi \rightarrow \varphi$.
    \end{lemma}

    \begin{proof}
        \begin{center}
            \begin{tabular}{rll}
                1.& $(\varphi \rightarrow ((\varphi \rightarrow \varphi) \rightarrow \varphi)) \rightarrow ((\varphi \rightarrow (\varphi \rightarrow \varphi)) \rightarrow (\varphi \rightarrow \varphi))$& $\mathrm{I2}$\\
                2.& $\varphi \rightarrow ((\varphi \rightarrow \varphi) \rightarrow \varphi)$& $\mathrm{I1}$\\
                3.& $\varphi \rightarrow (\varphi \rightarrow \varphi)$& $\mathrm{I1}$\\
                4.& $(\varphi \rightarrow (\varphi \rightarrow \varphi)) \rightarrow (\varphi \rightarrow \varphi)$& из 2 и 1\\
                5.& $\varphi \rightarrow \varphi$& из 3 и 4\\
            \end{tabular}
        \end{center}
    \end{proof}

    \begin{remark}
        Важно, что для доказательства были использованы только $\mathrm{I1}$, $\mathrm{I2}$ и $\mathrm{MP}$.
    \end{remark}

    \begin{theorem}[о дедукции в гильбертовском исчислении]
        Для любых $\Gamma \cup \{\varphi; \psi\} \subseteq \Formul$ верно, что
        \[\Gamma \cup \{\varphi\} \vdash \psi \qquad \Longleftrightarrow \qquad \Gamma \vdash \varphi \rightarrow \psi.\]
    \end{theorem}

    \begin{proof}
        \begin{itemize}
            \item[$\Longleftarrow$)] Пусть $\Gamma \vdash \varphi \rightarrow \psi$. Зафиксируем какой-нибудь вывод
                \[\varphi_0, \dots, \varphi_n\]
                из $\Gamma$, где $\varphi_n = \varphi \rightarrow \psi$. Тогда несложно убедиться, что
                \[\varphi_0, \dots, \varphi_n, \varphi, \psi\]
                будет выводом $\psi$ из $\Gamma \cup \{\varphi\}$. Стало быть, $\Gamma \cup \{\varphi\} \vdash \psi$.

            \item[$\Longrightarrow$)] Пусть $\Gamma \cup \{\varphi\} \vdash \psi$. Зафиксируем какой-нибудь вывод
                \[\psi_0, \dots, \psi_n\]
                из $\Gamma \cup \{\varphi\}$, где $\psi_n = \psi$. Покажем по полной индукции по $i$, что $\Gamma \vdash \varphi \rightarrow \psi_i$.

                Рассмотрим возможные случаи.
                \begin{itemize}
                    \item Пусть $\psi_i$ --- аксиома или элемент $\Gamma$. Тогда
                        \begin{center}
                            \begin{tabular}{rll}
                                1.& $\psi_i \rightarrow (\varphi \rightarrow \psi_i)$& $\mathrm{I2}$\\
                                2.& $\psi_i$& гипотеза / элемент $\Gamma$\\
                                3.& $\varphi \rightarrow \psi_i$& из 2 и 1\\
                            \end{tabular}
                        \end{center}
                        будет выводом из $\varnothing$ или $\Gamma$ соответственно. Стало быть, $\Gamma \vdash \varphi \rightarrow \psi_i$.

                    \item Пусть $\psi_i = \varphi$. По лемме \ref{self-implication-lemma} мы имеем, что $\vdash \varphi \rightarrow \varphi$. Стало быть, $\Gamma \vdash \varphi \rightarrow \varphi$.
                    
                    \item Пусть $\psi_i$ получено из предыдущих $\psi_j$ и $\psi_k = \psi_j \rightarrow \psi_i$ по $\mathrm{MP}$. Тогда можно построить следующий ``квазивывод'' из $\Gamma$:
                        \begin{center}
                            \begin{tabular}{rll}
                                1.& $(\varphi \rightarrow (\psi_j \rightarrow \psi_i)) \rightarrow ((\varphi \rightarrow \psi_j) \rightarrow (\varphi \rightarrow \psi_i))$& $\mathrm{I2}$\\
                                2.& $\varphi \rightarrow (\psi_j \rightarrow \psi_i)$& предположение индукции\\
                                3.& $\varphi \rightarrow \psi_j$& предположение индукции\\
                                4.& $(\varphi \rightarrow \psi_j) \rightarrow (\varphi \rightarrow \psi_i)$& из 2 и 1\\
                                5.& $\varphi \rightarrow \psi_i$& из 3 и 4\\
                            \end{tabular}
                        \end{center}
                        Стало быть, $\Gamma \vdash \varphi \rightarrow \psi_i$.
                \end{itemize}

                В частности, при $i := n$ мы имеем, что $\Gamma \vdash \varphi \rightarrow \psi_n$, т.е. $\Gamma \vdash \varphi \rightarrow \psi$.
        \end{itemize}
    \end{proof}

    \begin{remark}
        Важно, что для доказательства были использованы только $\mathrm{I1}$, $\mathrm{I2}$ и $\mathrm{MP}$.
    \end{remark}

    \begin{definition}
        Для удобства введём обозначения
        \[\top := p_\star \rightarrow p_\star \quad \text{ и } \quad \bot := \neg \top,\]
        где $p_\star$ --- фиксированная пропозициональная константа.
    \end{definition}

    \begin{corollary}
        Для любых $\Gamma \cup \{\varphi\} \subseteq \Formul$,
        \[
            \Gamma \vdash \varphi
            \quad \Longleftrightarrow \quad
            \vdash \bigwedge_{i=1}^n \psi_i \rightarrow \varphi \text{ для некоторых } \{\psi_1; \dots; \psi_n\} \subseteq \Gamma.
        \]
        (В случае $n=0$ соответствующая конъюнкция отождествляется с $\top$.)
    \end{corollary}

    \begin{proof}
        Заметим, что для всяких $\{\psi_1; \dots; \psi_n; \varphi\} \subseteq \Gamma$.
        \[
            \vdash \bigwedge_{i=1}^n \psi_i \rightarrow \varphi
            \quad \Longleftrightarrow \quad
            \left\{\bigwedge_{i=1}^n \psi_i\right\} \vdash \varphi
        \]
        и
        \[
            \Gamma \vdash \varphi
            \quad \Longleftrightarrow \quad
            \{\psi_1; \dots; \psi_n\} \vdash \varphi \text{ для некоторых } \{\psi_1; \dots; \psi_n\} \subseteq \Gamma.
        \]
        Соответственно нужно лишь показать, что
        \[
            \{\psi_1; \dots; \psi_n\} \vdash \varphi
            \quad \Longleftrightarrow \quad
            \left\{\bigwedge_{i=1}^n \psi_i\right\} \vdash \varphi
        \]

        По $\mathrm{C1}$ и $\mathrm{C2}$ имеем, что $\left\{\bigwedge_{i=1}^n \psi_i\right\} \vdash \{\psi_i\}_{i=1}^n$, а по $\mathrm{C3}$ --- что $\{\psi_i\}_{i=1}^n \vdash \left\{\bigwedge_{i=1}^n \psi_i\right\}$. Из транзитивности $\vdash$ получаем искомый равносильный переход.
    \end{proof}

    \begin{lemma}
        Всякая аксиома гильбертовского исчисления общезначима.
    \end{lemma}

    \begin{theorem}[о корректности $\vDash$]\label{Hilbert-conclusion-correctness}
        Для всяких $\Gamma \cup \{\varphi\} \subseteq \Formul$
        \[
            \Gamma \vdash \varphi
            \quad \Longrightarrow \quad
            \Gamma \vDash \varphi
        \]
    \end{theorem}

    \begin{proof}
        Пусть $\Gamma \vdash \varphi$. Зафиксируем какой-нибудь вывод
        \[\varphi_1, \dots, \varphi_n\]
        $\varphi$ из $\Gamma$. Рассмотрим произвольную оценку $v$ такую, что $v \Vdash \psi$ для всех $\psi \in \Gamma$. Покажем по индукции по $i \in \{0; \dots; n\}$, что $v \Vdash \varphi_i$. Для этого рассмотрим следующие случаи:
        \begin{itemize}
            \item Если $\varphi_i$ --- аксиома, то $\vDash \varphi_i$, а потому $v \Vdash \varphi_i$.
            \item Если $\varphi_i$ --- элемент $\Gamma$, то тогда, очевидно, $v \Vdash \varphi_i$.
            \item Если $\varphi_i$ получается из предшествующих $\varphi_j$ и $\varphi_k = \varphi_j \rightarrow \varphi_i$ по $\mathrm{MP}$. Ввиду предположения индукции,
            \[
                v \Vdash \varphi_j
                \quad \text{ и } \quad
                v \Vdash \varphi_j \rightarrow \varphi_i,
            \]
            откуда немедленно следует $v \Vdash \varphi_i$.
        \end{itemize}
        В частности при $i := n$ мы имеем $v \Vdash \varphi_n$, т.е. $v \Vdash \varphi$.

        Таким образом $\Gamma \vDash \varphi$.
    \end{proof}

    \begin{definition}
        $\Gamma \subseteq \Formul$ называется \emph{простой теорией}, если оно удовлетворяет следующим условиям:
        \begin{itemize}
            \item $\Gamma \neq \Formul$;
            \item $\{\varphi \in \Formul \mid \Gamma \vdash \varphi\} \subseteq \Gamma$;
            \item для любого $\varphi \vee \psi \in \Gamma$ верно, что $\varphi \in \Gamma$ или $\psi \in \Gamma$.
        \end{itemize}
    \end{definition}

    \begin{lemma}
        Пусть $\Gamma$ --- простая теория. Тогда для любых $\{\varphi; \psi\} \subseteq \Formul$:
        \begin{align*}
            \neg \varphi \in \Gamma \quad &\Longleftrightarrow \quad \varphi \notin \Gamma;\\
            \varphi \wedge \psi \in \Gamma \quad &\Longleftrightarrow \quad \varphi \in \Gamma \text{ и } \psi \in \Gamma;\\
            \varphi \vee \psi \in \Gamma \quad &\Longleftrightarrow \quad \varphi \in \Gamma \text{ или } \psi \in \Gamma;\\
            \varphi \rightarrow \psi \in \Gamma \quad &\Longleftrightarrow \quad \varphi \notin \Gamma \text{ или } \psi \in \Gamma.\\
        \end{align*}
    \end{lemma}

    \begin{proof}
        \begin{itemize}
            \item[$\neg$, $\Rightarrow$)] Пусть $\neg \varphi \in \Gamma$. Предположим, что $\varphi \in \Gamma$. Рассмотрим всякое $\psi \in \Formul$.
                \begin{center}
                    \begin{tabular}{rll}
                        1.& $\neg \varphi \rightarrow (\varphi \rightarrow \psi)$& $\mathrm{N2}$\\
                        2.& $\neg \varphi$& гипотеза\\
                        3.& $\varphi$& гипотеза\\
                        4.& $\varphi \rightarrow \psi$& из 2 и 1\\
                        5.& $\psi$& из 3 и 4\\
                    \end{tabular}
                \end{center}
                То есть $\Gamma \vdash \psi$, а значит $\Gamma \vdash \Formul$. Следовательно по построению $\Gamma = \Formul$ --- противоречие. Значит $\varphi \notin \Gamma$.

            \item[$\neg$, $\Leftarrow$)] Пусть $\varphi \notin \Gamma$. Поскольку $\vdash \varphi \vee \neg \varphi$, тем более $\Gamma \vdash \varphi \vee \neg \varphi$, то $\varphi \vee \neg \varphi \in \Gamma$, откуда по построению $\neg \varphi \in \Gamma$.

            \item[$\wedge$, $\Rightarrow$)] Пусть $\varphi \wedge \psi \in \Gamma$. Используя $\mathrm{C1}$ и $\mathrm{C2}$, получаем, что $\Gamma \vdash \varphi$ и $\Gamma \vdash \psi$, следовательно $\varphi \in \Gamma$ и $\psi \in \Gamma$.

            \item[$\wedge$, $\Leftarrow$)] Пусть $\varphi \in \Gamma$ и $\psi \in \Gamma$. Используя $\mathrm{C3}$, получаем, что $\Gamma \vdash \varphi \wedge \psi$, а следовательно $\varphi \wedge \psi \in \Gamma$.
            
            \item[$\vee$, $\Rightarrow$)] Пусть $\varphi \vee \psi \in \Gamma$. Тогда по построению $\Gamma$ имеем $\varphi \in \Gamma$ или $\psi \in \Gamma$.

            \item[$\vee$, $\Leftarrow$)] Пусть $\varphi \in \Gamma$ или $\psi \in \Gamma$. Тогда применяя $\mathrm{D1}$ или $\mathrm{D2}$, получаем, что $\Gamma \vdash \varphi \vee \psi$, следовательно $\varphi \vee \psi \in \Gamma$.

            \item[$\rightarrow$, $\Rightarrow$)] Пусть $\varphi \rightarrow \psi \in \Gamma$. Следовательно если $\varphi \in \Gamma$, то $\Gamma \vdash \psi$, т.е. $\psi \in \Gamma$. Таким образом $\varphi \notin \Gamma$ или $\psi \in \Gamma$.

            \item[$\rightarrow$, $\Leftarrow$)] Пусть $\varphi \notin \Gamma$ или $\psi \in \Gamma$. В первом случае $\neg \varphi \in \Gamma$, откуда с помощью $\mathrm{N2}$ получаем $\Gamma \vdash \varphi \rightarrow \psi$, а значит $\varphi \rightarrow \psi \in \Gamma$. Во втором случае $\Gamma \vdash \varphi \rightarrow \psi$ можно получить с помощью $\mathrm{I1}$.
        \end{itemize}
    \end{proof}

    \begin{lemma}[о расширении, aka Линдебаума]
        Пусть $\Gamma \cup \{\varphi\} \subseteq \Formul$ таковы, что $\Gamma \nvdash \varphi$. Тогда существует простая теория $\Gamma' \supseteq \Gamma$, что $\Gamma' \nvdash \varphi$.
    \end{lemma}

    \begin{proof}
        Ясно, что $\Formul$ счётно. Поэтому его элементы можно расположить в последовательность $(\psi_n)_{n=0}^\infty$ (т.е. $\Formul = \{\psi_n\}_{n=0}^\infty$). Теперь определим последовательность $(\Gamma_n)_{n=0}^\infty$ подмножеств $\Formul$ по рекурсии следующим образом.
        \begin{itemize}
            \item Если $n=0$, то $\Gamma_n = \Gamma$.
            \item Если $n = m + 1$ и $\Gamma_m \cup \{\psi_m\} \vdash \varphi$, то $\Gamma_n := \Gamma_m$.
            \item Если $n = m + 1$ и $\Gamma_m \cup \{\psi_m\} \nvdash \varphi$, то $\Gamma_n := \Gamma_m \cup \{\psi_m\}$.
        \end{itemize}
        По построению мы имеем $\Gamma_0 \subseteq \Gamma_1 \subseteq \dots$. Кроме того, $\Gamma_n \nvdash \varphi$ для всех $n \in \NN$. Возьмём
        \[\Gamma' := \bigcup_{n \in \NN} \Gamma_n\]

        Заметим следующее.
        \begin{itemize}
            \item Безусловно $\Gamma \subseteq \Gamma'$.
            \item $\Gamma' \nvdash \varphi$. Действительно, иначе есть конечное $\Delta \subseteq \Gamma'$, что $\Delta \vdash \varphi$. Следовательно есть $\Gamma_n \supseteq \Delta$, т.е. $\Gamma_n \vdash \varphi$ --- противоречие с определением $\Gamma_n$.
            \item Для всякого $\psi \in \Formul$
                \[\psi \notin \Gamma' \quad \Longrightarrow \quad \Gamma' \cup \{\psi\} \vdash \varphi\]
                Действительно, $\psi = \psi_n$ для некоторого $n$. Тогда из $\psi \notin \Gamma'$ следует, что $\Gamma_n \cup \{\psi\} \vdash \varphi$, и следовательно $\Gamma' \cup \{\varphi\} \vdash \varphi$.
        \end{itemize}

        Проверим, что $\Gamma'$ --- простая теория.
        \begin{itemize}
            \item Из $\Gamma' \nvdash \varphi$ следует, что $\Gamma' \neq \Formul$.
            \item Пусть $\psi \notin \Gamma'$. Тогда из того, что $\Gamma' \nvdash \varphi$ и $\Gamma' \cup \{\psi\} \vdash \varphi$ следует, что $\Gamma' \nvdash \psi$.
            \item Пусть $\theta \notin \Gamma'$ и $\chi \notin \Gamma'$. Тогда $\Gamma' \cup \{\theta\} \vdash \varphi$ и $\Gamma' \cup \{\chi\} \vdash \varphi$. Следовательно $\Gamma' \vdash \theta \rightarrow \varphi$ и $\Gamma' \vdash \chi \rightarrow \varphi$. Тогда по $\mathrm{D3}$ имеем, что $\Gamma' \vdash \theta \vee \chi \rightarrow \varphi$, т.е. $\Gamma' \cup \{\theta \vee \chi\} \vdash \varphi$, следовательно $\theta \vee \chi \notin \Gamma'$.
        \end{itemize}

        Таким образом $\Gamma'$ --- простая теория, обладающая необходимыми свойствами.
    \end{proof}

    \begin{proof}[Доказательство для любой мощности $\Prop$.]
        Пусть $\kappa := |\Prop|$. Ясно, что $|\Formul| = \kappa$. Поэтому элементы $\Formul$ можно расположить в трансфинитную последовательность длины $\kappa$:
        \[\langle \psi_\alpha: \alpha \in \kappa\rangle\]
        т.е. $\Formul = \{\psi_\alpha\}_{\alpha \in \kappa}$. Определим $\langle \Gamma_\alpha \rangle_{\alpha \in \kappa}$ по трансфинитной рекурсии следующим образом.
        \begin{itemize}
            \item Если $\alpha = 0$, то $\Gamma_\alpha := \Gamma$.
            \item Если $\alpha = \beta + 1$ и $\Gamma_\beta \cup \{\psi_\beta\} \vdash \psi$, то $\Gamma_\alpha := \Gamma_\beta$.
            \item Если $\alpha = \beta + 1$ и $\Gamma_\beta \cup \{\psi_\beta\} \nvdash \psi$, то $\Gamma_\alpha := \Gamma_\beta \cup \{\psi_\beta\}$.
            \item Если $\alpha$ --- предельный ординал, то $\Gamma_\alpha = \bigcup_{\beta \in \alpha} \Gamma_\beta$.
        \end{itemize}
        По построению мы имеем $\Gamma_\beta \subseteq \Gamma_\alpha$ для всяких $\beta \in \alpha \in \kappa$. Кроме того, $\Gamma_\alpha \nvdash \varphi$ для всех $\alpha \in \kappa$. Возьмём
        \[\Gamma' := \bigcup_{\alpha \in \kappa} \Gamma_\alpha\]

        Заметим следующее.
        \begin{itemize}
            \item Безусловно $\Gamma \subseteq \Gamma'$.
            \item $\Gamma' \nvdash \varphi$. Действительно, иначе есть конечное $\Delta \subseteq \Gamma'$, что $\Delta \vdash \varphi$. Следовательно есть $\Gamma_\alpha \supseteq \Delta$, т.е. $\Gamma_\alpha \vdash \varphi$ --- противоречие с определением $\Gamma_\alpha$.
            \item Для всякого $\psi \in \Formul$
                \[\psi \notin \Gamma' \quad \Longrightarrow \quad \Gamma' \cup \{\psi\} \vdash \varphi\]
                Действительно, $\psi = \psi_\alpha$ для некоторого $n$. Тогда из $\psi \notin \Gamma'$ следует, что $\Gamma_\alpha \cup \{\psi\} \vdash \varphi$, и следовательно $\Gamma' \cup \{\varphi\} \vdash \varphi$.
        \end{itemize}

        Проверим, что $\Gamma'$ --- простая теория.
        \begin{itemize}
            \item Из $\Gamma' \nvdash \varphi$ следует, что $\Gamma' \neq \Formul$.
            \item Пусть $\psi \notin \Gamma'$. Тогда из того, что $\Gamma' \nvdash \varphi$ и $\Gamma' \cup \{\psi\} \vdash \varphi$ следует, что $\Gamma' \nvdash \psi$.
            \item Пусть $\theta \notin \Gamma'$ и $\chi \notin \Gamma'$. Тогда $\Gamma' \cup \{\theta\} \vdash \varphi$ и $\Gamma' \cup \{\chi\} \vdash \varphi$. Следовательно $\Gamma' \vdash \theta \rightarrow \varphi$ и $\Gamma' \vdash \chi \rightarrow \varphi$. Тогда по $\mathrm{D3}$ имеем, что $\Gamma' \vdash \theta \vee \chi \rightarrow \varphi$, т.е. $\Gamma' \cup \{\theta \vee \chi\} \vdash \varphi$, следовательно $\theta \vee \chi \notin \Gamma'$.
        \end{itemize}

        Таким образом $\Gamma'$ --- простая теория, обладающая необходимыми свойствами.
    \end{proof}

    \begin{example}
        Пусть $v$ --- оценка. Рассмотрим
        \[\Gamma_v := \{\varphi \in \Formul \mid v \Vdash \varphi\}\]
        Легко убедиться, что $\Gamma_v$ --- простая теория.
    \end{example}

    \begin{definition}
        Для всякой простой теории $\Gamma$ определим оценку
        \[
            v_\Gamma(p) =
            \begin{cases}
                1& \text{ если $p \in \Gamma$}\\
                1& \text{ иначе}
            \end{cases}
        \]
        Иначе говоря, $v_\Gamma$ --- характеристическая функция для $\Prop \cap \Gamma$.
    \end{definition}

    \begin{lemma}
        Пусть $\Gamma$ --- простая теория. Тогда для всякой $\varphi \in \Formul$
        \[v_\Gamma \Vdash \varphi \quad \Longleftrightarrow \quad \varphi \in \Gamma\]
    \end{lemma}

    \begin{proof}
        Докажем это по полной индукции по $|\varphi|$.

        \textbf{Шаг.} Рассмотрим случаи.
        \begin{itemize}
            \item $\varphi \in \Prop$. В таком случае доказываемое утверждение очевидно следует из определения $v_\Gamma$.
            \item $\varphi = (\theta \rightarrow \chi)$.
            \[
                v_\Gamma \Vdash \varphi
                \quad \Longleftrightarrow \quad
                v_\Gamma \nVdash \theta \vee v_\Gamma \Vdash \chi
                \quad \Longleftrightarrow \quad
                \theta \notin \Gamma \vee \chi \in \Gamma
                \quad \Longleftrightarrow \quad
                \varphi \in \Gamma
            \]
            \item $\varphi = (\theta \wedge \chi)$.
            \[
                v_\Gamma \Vdash \varphi
                \quad \Longleftrightarrow \quad
                v_\Gamma \Vdash \theta \wedge v_\Gamma \Vdash \chi
                \quad \Longleftrightarrow \quad
                \theta \in \Gamma \wedge \chi \in \Gamma
                \quad \Longleftrightarrow \quad
                \varphi \in \Gamma
            \]
            \item $\varphi = (\theta \vee \chi)$.
            \[
                v_\Gamma \Vdash \varphi
                \quad \Longleftrightarrow \quad
                v_\Gamma \Vdash \theta \vee v_\Gamma \Vdash \chi
                \quad \Longleftrightarrow \quad
                \theta \in \Gamma \vee \chi \in \Gamma
                \quad \Longleftrightarrow \quad
                \varphi \in \Gamma
            \]
            \item $\varphi = \neg \theta$.
            \[
                v_\Gamma \Vdash \varphi
                \quad \Longleftrightarrow \quad
                v_\Gamma \nVdash \theta
                \quad \Longleftrightarrow \quad
                \theta \notin \Gamma
                \quad \Longleftrightarrow \quad
                \varphi \in \Gamma
            \]
        \end{itemize}
    \end{proof}

    \begin{theorem}[о сильной полноте $\vdash$]
        Для любых $\Gamma \cup \{\varphi\} \subseteq \Formul$
        \[
            \Gamma \vdash \varphi
            \quad \Longleftrightarrow \quad
            \Gamma \vDash \varphi
        \]
    \end{theorem}

    \begin{proof}
        \begin{itemize}
            \item[$\Rightarrow$)] См. \hyperref[Hilbert-conclusion-correctness]{теорему о корректности}.
            \item[$\Leftarrow$)] Допустим, что $\Gamma \nvdash \varphi$. Как нам известно найдётся простая теория $\Gamma' \supseteq \Gamma$, что $\Gamma' \nvdash \varphi$. Очевидно, $\varphi \notin \Gamma'$. Следовательно $v_{\Gamma'} \Vdash \psi$ для всех $\psi \in \Gamma'$, но $v_{\Gamma'} \nVdash \varphi$. В итоге, $\Gamma' \nvDash \varphi$, и тем более $\Gamma \nvDash \varphi$.
        \end{itemize}
    \end{proof}

    \begin{corollary}[теорема о слабой полноте $\vdash$]
        Для любой $\varphi \in \Gamma$
        \[
            \vdash \varphi
            \quad \Longleftrightarrow \quad
            \vDash \varphi
        \]
    \end{corollary}

    \begin{corollary}[теорема о компактности $\vDash$]
        Для любых $\Gamma \cup \{\varphi\} \subseteq \Formul$
        \[
            \Gamma \vDash \varphi
            \quad \Longleftrightarrow \quad
            \Delta \vDash \varphi \text{ для некоторого конечного $\Delta \subseteq \Gamma$}.
        \]
    \end{corollary}

    \subsection{Структуры}

    \begin{definition}
        \emph{Сигнатура} --- четвёрка вида
        \[\sigma = \langle \Pred_\sigma, \Func_\sigma, \Const_\sigma, \arity_\sigma \rangle\]
        где $\Pred_\sigma$, $\Func_\sigma$, $\Const_\sigma$ --- попарно непересекающиеся множества, а $\arity_\sigma$ --- функция из $\Pred_\sigma \cup \Func_\sigma$ в $\NN \setminus \{0\}$.

        Элементы $\Pred_\sigma$, $\Func_\sigma$, $\Const_\sigma$ называются соответственно \emph{предикатными}, \emph{функциональными} и \emph{константными символами} $\sigma$.

        Для данного символа $\varepsilon \in \Pred_\sigma \cup \Func_\sigma$ число $\arity(\varepsilon)$ называют \emph{его местностью}, \emph{арностью} или \emph{валентностью}.

        Когда из контекста понятно, о какой сигнатуре идёт речь, то индекс $\cdot_\sigma$ может опускаться.

        В дальнейшем при записи сигнатур нами будет допускаться определённая свобода. Так, если
        \[
            \Pred_\sigma = \{P_1; \dots; P_i\},
            \qquad
            \Func_\sigma = \{f_1; \dots; f_j\}
            \qquad
            \text{и}
            \qquad
            \Const_\sigma = \{c_1; \dots; c_j\},
        \]
        то $\sigma$ удобно представить как
        \[\langle P_1^{n_1}, \dots, P_i^{n_i}; f_1^{m_1}, \dots, f_j^{m_j}; c_1, \dots, c_k \rangle,\]
        где $n_1$, \dots, $n_i$ и $m_1$, \dots, $m_j$ суть местности соответственно $P_1$, \dots, $P_i$ и $f_1$, \dots, $f_j$.
    \end{definition}

    \begin{example}\hypertarget{strict-POS-signature-definition}{}
        Сигнатура (строгих) ЧУМ --- $\langle {<}^2 \rangle$, абелевых групп --- $\langle {=}^2; {+}^2; {-}^1; 0 \rangle$.
    \end{example}

    \begin{remark*}
        Неформально говоря, элементы $\Pred_\sigma$ играют роль ``отношений'' (например: равенство ``$=$'', упорядоченность ``$<$'' или ``$\leqslant$'', отношение делимости ``$\mid$'' или ``$\divided$''; при этом отношения можно брать не только на парах элементов), $\Func_\sigma$ --- роль функций и операторов (например: сложение (двухместное) ``$+$'', умножение (двухместное) ``$\cdot$'', унарный минус aka отрицание (унарное) ``$-$''), а $\Const_\sigma$ --- роль глобальных констант (например: нейтральные по сложению ``$0$'' и умножению ``$1$'' элементы колец, пространство векторов $V_A$ в аффинном пространстве).
        \todo[inline]{Уточнить про роль константных символов.}
    \end{remark*}

    \begin{definition}
        Пусть дана сигнатура $\sigma$. \emph{$\sigma$-структура} --- пара вида
        \[\mathfrak{A} = \langle A, I_\mathfrak{A} \rangle\]
        где $A$ --- непустое множество, а $I_\mathfrak{A}$ --- функция с областью определения $\Pred_\sigma \cup \Func_\sigma \cup \Const_\sigma$, что:
        \begin{itemize}
            \item для любого $n$-местного $P \in \Pred_\sigma$ верно $I_\mathfrak{A}(P) \subseteq A^n$;
            \item для любого $m$-местного $f \in \Pred_\sigma$ верно $I_\mathfrak{A}(f): A^m \rightarrow A$;
            \item для любого $c \in \Pred_\sigma$ верно $I_\mathfrak{A}(c) \in A$.
        \end{itemize}

        При этом $A$ называется \emph{носителем} или \emph{универсумом $\mathfrak{A}$}, а $I_\mathfrak{A}$ --- \emph{интерпретацией $\sigma$ в $\mathfrak{A}$}. Вместо $I_\mathfrak{A}(P)$, $I_\mathfrak{A}(f)$ и $I_\mathfrak{A}(c)$ часто пишут соответственно $P^\mathfrak{A}$, $f^\mathfrak{A}$ и $c^\mathfrak{A}$.

        Кроме того, если $\sigma$ представляется как
        \[\langle P_1^{n_1}, \dots, P_i^{n_i}; f_1^{m_1}, \dots, f_j^{m_j}; c_1, \dots, c_k \rangle,\]
        то $\mathfrak{A}$ удобно представить как
        \[\langle A; P_1^\mathfrak{A}, \dots, P_i^\mathfrak{A}; f_1^\mathfrak{A}, \dots, f_j^\mathfrak{A}; c_1^\mathfrak{A}, \dots, c_k^\mathfrak{A} \rangle.\]
        Более того, для некоторых стандартных структур $\mathfrak{A}$ даже индекс $\cdot^\mathfrak{A}$ может опускаться, хоть это и чревато некоторой путаницей.
    \end{definition}

    \begin{remark*}
        В предыдущем семестре для ЧУМ мы использовали запись $\langle A, {<}_A \rangle$, но аккуратнее было бы $\mathfrak{A} = \langle A; {<}_\mathfrak{A} \rangle$; вместе с тем, очевидно, далеко не всякая структура в сигнатуре ЧУМ является ЧУМ.
    \end{remark*}

    \begin{example}\hypertarget{R-signature-definition}{}
        Рассмотрим сигнатуру
        \[\sigma := \langle {=}^2; s^1, {+}^2, {\cdot}^2; 0 \rangle.\]
        \hypertarget{R-structure-definition}{}
        Обозначим через $\mathfrak{R}$ $\sigma$-структуру с носителем $\NN$, что:
        \begin{itemize}
            \item $=^\mathfrak{R}$ --- это отношение равенства на $\NN$;
            \item $s^\mathfrak{R}$ --- это функция последователя на $\NN$;
            \item $+^\mathfrak{R}$ --- это обычная функция сложения на $\NN$;
            \item $\cdot^\mathfrak{R}$ --- это обычная функция умножения на $\NN$;
            \item $0^\mathfrak{R}$ --- это настоящий ноль из $\mathfrak{R}$.
        \end{itemize} 
        Эту структуру называют \emph{стандартной моделью арифметики}. 
    \end{example}

    \begin{example}\hypertarget{ring-signature-definition}{}
        Рассмотрим сигнатуру
        \[\sigma := \langle {=}^2; {+}^2, {-}^1, {\cdot}^2; 0, 1 \rangle.\]
        Её называют \emph{сигнатурой колец}, разумеется. Обозначим
        \hypertarget{Z-structure-definition}{}
        \[
            \mathfrak{Z} :=
            \quad
            \begin{gathered}
                \text{$\sigma$-структура с носителем $\ZZ$, в которой все символы}\\
                \text{$\sigma$ интерпретируются естественным образом.}
            \end{gathered}
        \]
        В частности, ${-}^\mathfrak{Z}$ --- это функция взятия обратного по сложению над $\ZZ$. По аналогии вместо $\ZZ$ можно было бы использовать:
        \begin{itemize}
            \item $\ZZ/n\ZZ$ aka $\ZZ_n$, т.е. множество всех целых чисел по модулю $n$;
            \item $M_n(\RR)$, т.е. множество всех матриц порядка $n$ над $\RR$;
            \item $\QQ[x]$, т.е. множество всех многочленов от $x$ с коэффициентами из $\QQ$.
        \end{itemize}
    \end{example}

    \begin{example}
        Рассмотрим сигнатуру
        \[\sigma := \langle {=}^2, {\cong}^4, B^3 \rangle.\]
        \hypertarget{G-structure-definition}{}
        Обозначим через $\mathfrak{G}$ $\sigma$-структуру с носителем $\RR \times \RR$, что:
        \begin{itemize}
            \item ${=}^\mathfrak{G}$ --- это отношение равенства на $\RR \times \RR$;
            \item ${\cong}^\mathfrak{G}$ определяется по правилу
                \[
                    (r_1, r_2, r_3, r_4) \in {\cong}^\mathfrak{G}
                    \quad \Longleftrightarrow \quad
                    \text{отрезки $r_1r_2$ и $r_3r_4$ равны;}
                \]
            \item $B^\mathfrak{G}$ определяется по правилу
                \[
                    (r_1, r_2, r_3) \in B^\mathfrak{G}
                    \quad \Longleftrightarrow \quad
                    \text{$r_2$ лежит между $r_1$ и $r_3$ на прямой.}
                \]
        \end{itemize}
        Её можно назвать \emph{стандартной моделью геометрии}.
    \end{example}

    \begin{example}
        Пусть $\sigma$ --- сигнатура из предыдущего примера. Возьмём
        \[\HH := \{z \in \CC \mid \Img(z) > 0\}.\]
        \hypertarget{H-structure-definition}{}
        Обозначим через $\mathfrak{H}$ $\sigma$-структуру с носителем $\HH$, что:
        \begin{itemize}
            \item ${=}^\mathfrak{H}$ --- это отношение равенства на $\RR \times \RR$;
            \item ${\cong}^\mathfrak{H}$ определяется по правилу
                \[
                    (r_1, r_2, r_3, r_4) \in {\cong}^\mathfrak{G}
                    \quad \Longleftrightarrow \quad
                    \begin{gathered}
                        \text{отрезки $r_1r_2$ и $r_3r_4$ равны}\\
                        \text{в смысле метрики Пуанкаре;}                        
                    \end{gathered}
                \]
            \item $B^\mathfrak{H}$ определяется по правилу
                \[
                    (r_1, r_2, r_3) \in B^\mathfrak{G}
                    \quad \Longleftrightarrow \quad
                    \begin{gathered}
                        \text{$r_2$ лежит между $r_1$ и $r_3$ на полуокружности}\\
                        \text{(или полупрямой), ортогональной вещественной оси.}
                    \end{gathered}
                \]
        \end{itemize}
        Её называют \emph{моделью Пуанкаре геометрии Лобачевского}.
    \end{example}

    \begin{definition}
        Пусть $\mathfrak{A}$ и $\mathfrak{B}$ --- две $\sigma$-структуры. \emph{Гомоморфизм из $\mathfrak{A}$ в $\mathfrak{B}$} --- такое отображение $\xi: A \to B$, что
        \begin{enumerate}
            \item \label{structure-def-predicate-condition} для любого $n$-местного $P \in \Pred_\sigma$ и всех $(a_1, \dots, a_n) \in A^n$
                \[
                    (a_1, \dots, a_n) \in P^\mathfrak{A}
                    \quad \Longrightarrow \quad
                    (\xi(a_1), \dots, \xi(a_n)) \in P^\mathfrak{B};
                \]
            \item для любого $m$-местного $f \in \Pred_\sigma$ и всех $(a_1, \dots, a_m) \in A^m$
                \[
                    \xi(f^\mathfrak{A}(a_1, \dots, a_n)) = f^\mathfrak{B}(\xi(a_1), \dots, \xi(a_m));
                \]
            \item для любого $c \in \Pred_\sigma$
                \[
                    \xi(c^\mathfrak{A}) = c^\mathfrak{B}
                \]
        \end{enumerate}
    \end{definition}

    \begin{lemma}
        Композиция гомоморфизмов --- гомоморфизм.
    \end{lemma}

    \begin{definition}
        \emph{Вложение $\mathfrak{A}$ в $\mathfrak{B}$} --- инъективный гомоморфизм из $\mathfrak{A}$ в $\mathfrak{B}$ с усиленным до равносильности условием \ref{structure-def-predicate-condition}, т.е.
        \begin{center}
            $\xi^{-1}[P^\mathfrak{B}] = P^\mathfrak{A}$\quad для каждого $P \in \Pred_\sigma$.
        \end{center}
    \end{definition}

    \begin{lemma}
        Композиция вложений --- вложение.
    \end{lemma}

    \begin{definition}
        \emph{Изоморфизм из $\mathfrak{A}$ в $\mathfrak{B}$} --- сюръективное вложение $\mathfrak{A}$ в $\mathfrak{B}$.

        Говорят, что $\mathfrak{A}$ и $\mathfrak{B}$ изоморфны, и пишут $\mathfrak{A} \simeq \mathfrak{B}$, если есть хоть какой-то изоморфизм из $\mathfrak{A}$ в $\mathfrak{B}$.
    \end{definition}

    \begin{lemma}
        Композиция изоморфизмов --- изоморфизм.
    \end{lemma}

    \begin{lemma}
        Изоморфность --- ``отношение эквивалентности'', т.е.
        \begin{enumerate}
            \item для всякой структуры $\mathfrak{A}$ верно
                \[\mathfrak{A} \simeq \mathfrak{A};\]
            \item для всяких структур $\mathfrak{A}$ и $\mathfrak{B}$ верно
                \[\mathfrak{A} \simeq \mathfrak{B} \quad \Longleftrightarrow \quad \mathfrak{B} \simeq \mathfrak{A};\]
            \item для всяких структур $\mathfrak{A}$, $\mathfrak{B}$ и $\mathfrak{C}$ верно
                \[\mathfrak{A} \simeq \mathfrak{B} \simeq \mathfrak{C} \quad \Longrightarrow \quad \mathfrak{A} \simeq \mathfrak{C}.\]
        \end{enumerate}
    \end{lemma}

    \begin{lemma}
        Гомоморфизм $\xi$ из $\mathfrak{A}$ в $\mathfrak{B}$ является изоморфизмом тогда и только тогда, когда есть обратный к нему, т.е. гомоморфизм $\eta$ из $\mathfrak{B}$ в $\mathfrak{A}$, что
        \begin{center}
            $\xi \circ \eta = \id_A$ \quad и \quad $\eta \circ \xi = \id_B$.
        \end{center}
    \end{lemma}

    \begin{example}
        Рассмотрим нестрогие ЧУМ $\mathfrak{A}$ и $\mathfrak{B}$, что:
        \begin{itemize}
            \item $A = B = \NN \setminus \{0\}$;
            \item ${\leqslant}^\mathfrak{A}$ --- это отношение делимости на $\NN \setminus \{0\}$;
            \item ${\leqslant}^\mathfrak{B}$ --- это обычный порядок на $\NN \setminus \{0\}$.
        \end{itemize}
        Тогда
        \[\lambda: A \to B, a \mapsto a\]
        будет биективным гомоморфизмом из $\mathfrak{A}$ в $\mathfrak{B}$, но даже не вложением.
    \end{example}

    \begin{example}
        Рассмотрим нестрогие ЧУМ $\mathfrak{A}$ и $\mathfrak{B}$, что:
        \begin{itemize}
            \item $A = \NN$ и $B = \PP$;
            \item ${\leqslant}^\mathfrak{A}$ --- это обычный порядок на $\NN$;
            \item ${\leqslant}^\mathfrak{B}$ --- это обычный порядок на $\PP$.
        \end{itemize}
        Тогда
        \[\lambda: A \to B, a \mapsto \text{``$a$-тое простое число''}\]
        будет изоморфизмом из $\mathfrak{A}$ в $\mathfrak{B}$.
    \end{example}

    \begin{example}
        Зафиксируем $p \in \PP$. Рассмотрим группы $\mathfrak{A}$ и $\mathfrak{B}$, что:
        \begin{itemize}
            \item $A = \ZZ$ и $B = \ZZ_p$;
            \item ${+}^\mathfrak{A}$ --- это сложение на $\ZZ$;
            \item ${+}^\mathfrak{B}$ --- это сложение на $\ZZ_p$.
        \end{itemize}
        Тогда
        \[\lambda: A \to B, a \mapsto a \mathbin{\mathrm{mod}} p\]
        будет сюръективным гомоморфизмом из $\mathfrak{A}$ в $\mathfrak{B}$.
    \end{example}

    \begin{example}
        Рассмотрим группы $\mathfrak{A}$ и $\mathfrak{B}$, что:
        \begin{itemize}
            \item $A = \RR$ и $B = \RR \setminus \{0\}$;
            \item ${+}^\mathfrak{A}$ --- это сложение на $\RR$;
            \item ${+}^\mathfrak{B}$ --- это \underline{умножение} на $\RR \setminus \{0\}$.
        \end{itemize}
        Тогда
        \[\lambda: A \to B, a \mapsto 2^a\]
        будет вложением $\mathfrak{A}$ в $\mathfrak{B}$.
    \end{example}

    \begin{definition}
        \emph{Автоморфизм $\mathfrak{A}$} --- изоморфизм из $\mathfrak{A}$ в $\mathfrak{A}$.

        Множество всех автоморфизмов $\mathfrak{A}$ обозначается за $\Aut(\mathfrak{A})$.
    \end{definition}

    \begin{remark*}
        Интуитивно автоморфизмы --- абстрактный аналог симметрий.

        Также над $\Aut(A)$ можно естественным образом задать структуру группы, где роль бинарной операции играет композиция.
    \end{remark*}

    \begin{example}
        Пусть $\xi$ --- произвольный автоморфизм стандартной \hyperlink{R-structure-definition}{модели $\mathfrak{R}$ арифметики}. Тогда
        \[
            \xi(0) = 0, \quad
            \xi(1) = s^\mathfrak{R}(\xi(0)) = 1, \quad
            \xi(2) = s^\mathfrak{R}(s^\mathfrak{R}(\xi(0))) = 2, \dots
        \]
        Значит $\xi = \id_{\NN}$. Стало быть, $\Aut(\mathfrak{R}) = \{\id_{\NN}\}$.
    \end{example}

    \begin{example}\hypertarget{R_le-structure-definition}{}
        Обозначим через $\mathfrak{R}_{<}$ стандартный ЧУМ с носителем $\NN$. Пусть $\xi$ --- произвольный автоморфизм $\mathfrak{R}_{<}$. Нетрудно понять, что:
        \begin{align*}
            \xi(\least(\NN)) &= \least(\NN);\\
            \xi(\least(\NN \setminus \{0\})) &= \least(\NN \setminus \{\xi(0)\});\\
            \xi(\least(\NN \setminus \{0, 1\})) &= \least(\NN \setminus \{\xi(0), \xi(1)\});\\
            &\hspace{0.5em}\vdots
        \end{align*}
        Отсюда мы получаем:
        \begin{align*}
            \xi(0) &= \least(\NN) = 0;\\
            \xi(1) &= \least(\NN \setminus \{\xi(0)\}) = 1;\\
            \xi(2) &= \least(\NN \setminus \{\xi(0), \xi(1)\}) = 2;\\
            &\hspace{0.5em}\vdots
        \end{align*}
        Значит, $\xi = \id_{\NN}$. Стало быть, $\Aut(\mathfrak{R}_{<}) = \id_{\NN}$.
    \end{example}

    \begin{example}[со схемой решения]\hypertarget{D-structure-definition}{}
        Обозначим через $\mathfrak{D}$ нестрогий ЧУМ с носителем $\NN \setminus \{0\}$, в котором $\leqslant$ интерпретируется как отношение делимости. Заметим, что
        \begin{itemize}
            \item всякий автоморфизм $\mathfrak{D}$ переводит элементы $\PP$ в элементы $\PP$;
            \item каждую биекцию из $\PP$ в $\PP$ можно единственным образом расширить до автоморфизма $\mathfrak{D}$.
        \end{itemize}
        Отсюда нетрудно получить, что $\Aut(\mathfrak{D})$ фактически состоит из перестановок $\PP$, а точнее, из их расширений.
    \end{example}

    \begin{example}[в качестве дополнительного упражнения]\label{G-structure-isomorphisms-example}
        Для стандартной \hyperlink{G-structure-definition}{модели $\mathfrak{G}$ геометрии} всякий автоморфизм представим в виде композиции движения и гомотетии.
    \end{example}

    \subsection{Язык кванторной классической логики\\(Quantifier Classic Logic, QCL)}

    \begin{definition}
        Пусть (навсегда) зафиксировано некоторое счётное множество \Var. Будем называть его элементы \emph{предметными переменными} или просто \emph{переменными}.

        Пусть дана сигнатура $\sigma$. Алфавит $\mathscr{L}_\sigma$ квантовой классической логики над $\sigma$ состоит из элементов $\Pred_\sigma \cup \Func_\sigma \cup \Const_\sigma \cup \Var$, а также:
        \begin{itemize}
            \item символов связок: ``$\rightarrow$'', ``$\wedge$'', ``$\vee$'' и ``$\neg$'',
            \item символов кванторов: ``$\forall$'' и ``$\exists$'',
            \item и вспомогательных символов: ``('', ``)'' и ``,''.
        \end{itemize}
        Для каждого $x \in \Var$ слова ``$\forall x$'' и ``$\exists x$'' называются \emph{кванторами по $x$}.

        Обозначим за $\Term_\sigma$ наименьшее подмножество $\mathscr{L}_\sigma^*$, замкнутое относительно следующих порождающих правил:
        \begin{itemize}
            \item если $x \in \Var$, то $x \in \Term_\sigma$;
            \item если $c \in \Const$, то $c \in \Term_\sigma$;
            \item если $f \in \Func_\sigma$, $\arity_\sigma(f) = n$ и $\{t_1, \dots, t_n\} \subseteq \Term_\sigma$, то $f(t_1, \dots, t_n) \in \Term_\sigma$.
        \end{itemize}
        Элементы $\Term_\sigma$ называются \emph{$\sigma$-термами}.
    \end{definition}

    \begin{example}
        Пусть $\sigma$ --- \hyperlink{R-signature-definition}{сигнатура стандартной модели арифметики}. Тогда
        \[{+}(s({+}(x, y)), {\cdot}(s(x), y))\]
        является $\sigma$-термом, который удобнее записать как $s(x + y) + s(x) \cdot y$.
    \end{example}

    \begin{definition}
        Обозначим за $\Formul_\sigma$ наименьшее подмножество $\mathscr{L}_\sigma^*$, замкнутое относительно следующих порождающих правил:
        \begin{itemize}
            \item если $P \in \Pred_\sigma$, $\arity_\sigma(f) = n$ и $\{t_1, \dots, t_n\} \subseteq \Term_\sigma$, то $f(t_1, \dots, t_n) \in \Formul_\sigma$;
            \item если $\{\Phi, \Psi\} \subseteq \Formul_\sigma$, то $\{(\Phi \rightarrow \Psi), (\Phi \wedge \Psi), (\Phi \vee \Psi), \neg \Phi\} \in \Formul_\sigma$;
            \item если $\Phi \subseteq \Formul_\sigma$ и $x \in \Var$, то $\{\forall x\ \Phi, \exists x\ \Phi\} \in \Formul_\sigma$.
        \end{itemize}
        Элементы $\Formul_\sigma$ называются \emph{$\sigma$-формулами}.

        Под \emph{атомарными $\sigma$-формулами}, или \emph{$\sigma$-атомами}, понимают те, что не содержат ни символов связок ни символов кванторов. Множество всех $\sigma$-атомов обозначается за $\Atom_\sigma$.
    \end{definition}

    \begin{definition}
        Для любого $t \in \Term_\sigma$ определим
        \[\sub(t) := \{s \in \Term_\sigma \mid s \preccurlyeq t\}.\]
        Элементы $\sub(t)$ называются \emph{подтермами $t$}.
        
        Для любого $\Phi \in \Formul_\sigma$ определим
        \[\Sub(\Phi) := \{\Psi \in \Formul_\sigma \mid \Psi \preccurlyeq \Phi\}.\]
        Элементы $\Sub(\Phi)$ называются \emph{подформулами $\Phi$}.
    \end{definition}

    \begin{lemma}
        Пусть $\{s, t\} \subseteq \Term_\sigma$ таковы, что $t \sqsubseteq s$. Тогда $t = s$.
    \end{lemma}

    \begin{theorem}[о единственности представления термов]
        Всякий $t \in \Term_\sigma \setminus (\Var \cup \Const_\sigma)$ можно единственным образом представить в виде
        \[f(t_1, \dots, t_n),\]
        где $f \in \Func_\sigma$, $\arity_\sigma(f) = n$ и $\{t_1, \dots, t_n\} \subseteq \Term_\sigma$.
    \end{theorem}

    \begin{lemma}
        Пусть $t \in \Term_\sigma$ и $f \in \Func_\sigma$. Тогда всякое вхождение $f$ в $t$ является началом вхождения некоторого подтерма.
    \end{lemma}

    \begin{theorem}[о подтермах]
        Пусть $t \in \Term_\sigma$.
        \begin{enumerate}
            \item Если $t \in \Var \cup \Const_\sigma$, то $\sub(t) = \{t\}$.
            \item Если $t = f(t_1, \dots, t_n)$, где $f \in \Func_\sigma$, $\arity_\sigma(f) = n$ и $\{t_1, \dots, t_n\} \subseteq \Term_\sigma$, то
                \[\sub(t) = \sub(t_1) \cup \dots \cup \sub(t_n) \cup \{t\}\]
        \end{enumerate}
    \end{theorem}

    \begin{theorem}[о единственности представления атомов]
        Всякий $\Phi \in \Atom_\sigma$ можно единственным способом представить в виде
        \[P(t_1, \dots, t_n),\]
        где $P \in \Pred_\sigma$, $\arity_\sigma(P) = n$ и $\{t_1, \dots, t_n\} \subseteq \Term_\sigma$.
    \end{theorem}

    \begin{lemma}
        Пусть $\{\Phi, \Psi\} \subseteq \Formul_\sigma$ таковы, что $\Phi \sqsubseteq \Psi$. Тогда $\Phi = \Psi$.
    \end{lemma}

    \begin{theorem}
        Всякую $\Phi \in \Formul_\sigma \setminus \Atom_\sigma$ можно единственным образом представить в виде
        \begin{center}
            $(\Theta \rightarrow \Omega)$, \quad
            $(\Theta \wedge \Omega)$, \quad
            $(\Theta \vee \Omega)$, \quad
            $\neg \Theta$, \quad
            $\forall x\ \Theta$ или
            $\exists x\ \Theta$,            
        \end{center}
        где $\{\Theta; \Omega\} \subseteq \Formul_\sigma$.
    \end{theorem}

    \begin{lemma}
        Пусть $\Phi \in \Formul_\sigma$. Тогда всякое вхождение ``$\neg$'', ``('', ``$\forall$'' или ``$\exists$'' в $\Phi$ является началом вхождения некоторой подформулы.
    \end{lemma}

    \begin{theorem}[о подформулах]
        Пусть $\Phi \in \Formul_\sigma$.
        \begin{enumerate}
            \item Если $\Phi \in \Atom_\sigma$, то $\Sub(\Phi) = \{\Phi\}$.
            \item Если $\Phi = (\Theta \circ \Omega)$, где $\{\Theta, \Omega\} \subseteq \Formul_\sigma$ и ${\circ} \in \{{\rightarrow}, {\wedge}, {\vee}\}$, то
                \[\Sub(\Phi) = \Sub(\Theta) \cup \Sub(\Omega) \cup \{\Phi\}.\]
            \item Если $\Phi = \neg \Theta$, где $\Theta \in \Formul_\sigma$, то
                \[\Sub(\Phi) = \Sub(\Theta) \cup \{\Phi\}.\]
            \item Если $\Phi = Q x\ \Theta$, где $x \in \Var$, $\Theta \in \Formul_\sigma$ и $Q \in \{\forall, \exists\}$, то
                \[\Sub(\Phi) = \Sub(\Theta) \cup \{\Phi\}.\]
        \end{enumerate}
    \end{theorem}

    \begin{definition}
        Пусть $\Phi \in \Formul_\sigma$, $x \in \Var$ и $Q \in \{\forall, \exists\}$.

        Каждое вхождение $Qx$ в $\Phi$ является началом вхождения некоторой подформулы, причём единственного. Его называют \emph{областью действия} данного вхождения $Qx$.

        Вхождение $x$ в $\Phi$ называется \emph{связанным}, если оно входит в область действия какого-нибудь вхождения $\forall x$ или $\exists x$, и \emph{свободными} иначе.

        Далее, говорят, что $x$ является \emph{свободной переменной в $\Phi$}, если у $x$ есть хотя бы одно свободное вхождение в $\Phi$. Множество свободных переменных $\Phi$ обозначается за $\FV(\Phi)$.

        Интуитивно элементы $\FV(\Phi)$ играют роль параметров $\Phi$. Запись $\Phi(x_1, \dots, x_l)$ указывает на то, что $\FV(\Phi) \subseteq \{x_1, \dots, x_l\}$.

        Наконец, обозначим
        \[\Sent_\sigma := \{\Phi \in \Formul_\sigma \mid \FV(\Phi) = \varnothing\}.\]
        Элементы $\Sent_\sigma$ называют \emph{$\sigma$-предложениями}, реже --- \emph{замкнутыми $\sigma$-формулами}. Они могут выступать в качестве нелогических аксиом.
    \end{definition}
        
    \begin{example}    
        Пусть $\sigma$ --- \hyperlink{R-signature-definition}{сигнатура арифметики}. Рассмотрим $\sigma$-формулу
        \[\Phi := \forall x\ \exists y\ x = y + 0 \cdot u \wedge \forall y\ \exists u\ x + u = y.\]
        Тогда $\FV(\Phi) = \{u; x\}$. Тут мы можем написать $\Phi(u, x)$ или $\Phi(x, u)$; также приемлемы ``избыточные'' $\Phi(u, x, y)$ или $\Phi(x, u, v)$ и т.п.
    \end{example}

    \begin{example}
        Пусть $\sigma$ --- \hyperlink{strict-POS-signature-definition}{сигнатура (строгих) ЧУМ}. В таком случае под \emph{аксиомами ЧУМ} понимают следующие предложения:
        \begin{enumerate}
            \item $\forall x\ \neg (x < x)$;
            \item $\forall x\ \forall y\ \forall z\ (x < y \wedge y < z \rightarrow x < z)$.
        \end{enumerate}

        Интуитивно $\sigma$-структура является ЧУМ, если и только если она удовлетворяет этим аксиомам.
    \end{example}

    \begin{definition}
        Пусть $\Phi \in \Formul_\sigma$, $x \in \Var$ и $t \in \Term_\sigma$. Обозначим
        \[
            \Phi(x/t) :=
            \begin{gathered}
                \text{результат одновременной замены всех}\\
                \text{свободных вхождений $x$ в $\Phi$ на $t$.}
            \end{gathered}
        \]
    \end{definition}

    \begin{remark*}
        Применение этой операции порой приводит к весьма нежелательным последствиям. Так, $\sigma$-формулы вида
        \[\forall x\ \Phi \rightarrow \Phi(x/t)\]
        кажутся истинными, но при
        \[
            \sigma := \langle {<}^2 \rangle,
            \quad
            \Phi := \exists y\ x < y
            \quad \text{ и } \quad
            t := y
        \]
        мы получаем
        \[\forall x\ \exists y\ x < y \rightarrow \exists y\ y < y.\]
        Кроме того, $\sigma$-формулы вида
        \[\Phi(x/t) \rightarrow \exists x\ \Phi\]
        кажутся истинными, но при
        \[
            \sigma := \langle {=}^2 \rangle,
            \quad
            \Phi := \forall y\ x = y
            \quad \text{ и } \quad
            t := y
        \]
        мы получаем
        \[\forall y\ y = y \rightarrow \exists x\ \forall y\ x = y.\]
        Поэтому с подстановками нужно быть аккуратнее.
    \end{remark*}

    \begin{definition}
        Мы будем говорить, что \emph{$t$ свободен для (подстановки вместо) $x$ в $\Phi$}, если ни одно из свободных вхождений $x$ в $\Phi$ не находится в области действия квантора по переменной $t$.
    \end{definition}

    \begin{remark}
        В частности, всякая переменная, не присутствующая в $\Phi$, является свободной для подстановки вместо всякой другой переменой в $\Phi$.
    \end{remark}

    \subsection{Семантика кванторной классической логики}

    \begin{definition}
        \emph{Означивание переменных в $\mathfrak{A}$}, или просто \emph{означинвание в $\mathfrak{A}$} --- функция из $\Var$ в $A$.
        
        Каждое означивание $\nu$ в $\mathfrak{A}$ можно расширить до $\overline{\nu}: \Term_\sigma \rightarrow A$ естественным образом:
        \begin{enumerate}
            \item $\forall x \in \Var \qquad \overline{\nu}(x) = \nu(x)$;
            \item $\forall c \in \Const_\sigma \qquad \overline{\nu}(c) = c^\mathfrak{A}$;
            \item Для всякого $n$-местного $f \in \Func_\sigma$ и всяких $\{t_1, \dots, t_n\} \subseteq \Term_\sigma$
                \[
                    \overline{\nu}(\text{``$f(t_1, \dots, t_n)$''})
                    = f^\mathfrak{A}(\overline{\nu}(t_1), \dots, \overline{\nu}(t_n)).
                \]
        \end{enumerate}

        В дальнейшем для любых $x \in \Var$ и $a \in A$ через $\nu_a^x$ будет обозначаться означивание
        \[
            \nu_a^x(y) :=
            \begin{cases}
                \nu(y)& \text{ если $y \neq x$}\\
                a& \text{ если $y  x$}
            \end{cases}
        \]

    \end{definition}

    \begin{definition}
        Пусть дано означивание $\nu$ в $\mathfrak{A}$. Определим $\mathfrak{A} \Vdash \Phi[\nu]$ индукцией по построению $\Phi$:
        \begin{enumerate}
            \item для всякого $n$-местного предиката $P \in \Pred_\sigma$ и всяких $\{t_1, \dots, t_n\} \subseteq \Term_\sigma$
                \[
                    \mathfrak{A} \Vdash P(t_1, \dots, t_n)[\nu]
                    \quad \Longleftrightarrow \quad
                    (\overline{\nu}(t_1), \dots, \overline{\nu}(t_n)) \in P^\mathfrak{A};
                \]
            \item для всяких $\{\Psi, \Theta\} \subseteq \Formul_\sigma$
                \[
                    \mathfrak{A} \Vdash (\Psi \rightarrow \Theta)[\nu]
                    \quad \Longleftrightarrow \quad
                    \mathfrak{A} \nVdash \Psi[\nu] \text{ или } \mathfrak{A} \Vdash \Theta[\nu];
                \]
            \item для всяких $\{\Psi, \Theta\} \subseteq \Formul_\sigma$
                \[
                    \mathfrak{A} \Vdash (\Psi \wedge \Theta)[\nu]
                    \quad \Longleftrightarrow \quad
                    \mathfrak{A} \Vdash \Psi[\nu] \text{ и } \mathfrak{A} \Vdash \Theta[\nu];
                \]
            \item для всяких $\{\Psi, \Theta\} \subseteq \Formul_\sigma$
                \[
                    \mathfrak{A} \Vdash (\Psi \vee \Theta)[\nu]
                    \quad \Longleftrightarrow \quad
                    \mathfrak{A} \Vdash \Psi[\nu] \text{ или } \mathfrak{A} \Vdash \Theta[\nu];
                \]
            \item для всякого $\Psi \in \Formul_\sigma$
                \[
                    \mathfrak{A} \Vdash \neg \Psi[\nu]
                    \quad \Longleftrightarrow \quad
                    \mathfrak{A} \nVdash \Psi[\nu];
                \]
            \item для всякого $\Psi \in \Formul_\sigma$ и всякого $x \in \Var$
                \[
                    \mathfrak{A} \Vdash \exists x\ \Psi[\nu]
                    \quad \Longleftrightarrow \quad
                    \mathfrak{A} \Vdash \Psi[\nu_a^x] \text{ для некоторого } a \in A;
                \]
            \item для всякого $\Psi \in \Formul_\sigma$ и всякого $x \in \Var$
                \[
                    \mathfrak{A} \Vdash \forall x\ \Psi[\nu]
                    \quad \Longleftrightarrow \quad
                    \mathfrak{A} \Vdash \Psi[\nu_a^x] \text{ для всех } a \in A.
                \]
        \end{enumerate}

        Когда $\mathfrak{A} \Vdash \Phi[\nu]$, мы будем говорить, что \emph{$\Phi$ истинно в $\mathfrak{A}$ при $\nu$}, или \emph{$\mathfrak{A}$ удовлетворяет $\Phi$ при $\nu$}.
    \end{definition}

    \begin{remark}
        (Не)верность $\mathfrak{A} \Vdash \Phi[\nu]$ не зависит от того, какие значения $\nu$ сопоставляет элементам $\Var \setminus \FV(\Phi)$.
    \end{remark}

    \begin{definition}
        Если $\Phi$ имеет вид $\Phi(x_1, \dots, x_l)$, т.е. $\FV(\Phi) \subseteq \{x_1, \dots, x_l\}$, то вместо $\mathfrak{A} \Vdash \Phi[\nu]$ нередко пишут
        \[\mathfrak{A} \Vdash \Phi[x_1/\nu(x_1), \dots, x_l/\nu(x_l)],\]
        или же
        \[\mathfrak{A} \Vdash \Phi[\nu(x_1), \dots, \nu(x_l)].\]
        В частности, для $\Phi \in \Sent_\sigma$ обычно используется запись $\mathfrak{A} \Vdash \Phi$.

        Наконец, пусть $\Gamma \subseteq \Sent_\sigma$. Говорят, что \emph{$\mathfrak{A}$ является моделью $\Gamma$}, и пишут $\mathfrak{A} \Vdash \Gamma$, если $\mathfrak{A} \Vdash \Phi$ для всех $\Phi \in \Gamma$.
    \end{definition}

    \begin{theorem}\label{attribution-through-isomorphism-theorem}
        Пусть $\xi$ --- изоморфизм $\mathfrak{A}$ и $\mathfrak{B}$.
        \begin{enumerate}
            \item Для любого означивания $\nu$ в $\mathfrak{A}$ $\nu \circ \xi$ является означиванием $\mathfrak{B}$.
            \item Для каждого $\sigma$-терма $t$ и любого означивания $\nu$ в $\mathfrak{A}$
                \[\overline{\nu \circ \xi}(t) = \xi(\overline{\nu}(t)),\]
                т.е. $\overline{\nu \circ \xi} = \overline{\nu} \circ \xi$.
            \item Для каждой $\sigma$-формулы $\Phi$ и любого означивания $\nu$ в $\mathfrak{A}$
                \[
                    \mathfrak{A} \Vdash \Phi[\nu]
                    \quad \Longleftrightarrow \quad
                    \mathfrak{B} \Vdash \Phi[\nu \circ \xi].
                \]
        \end{enumerate}
    \end{theorem}

    \begin{proof}
        \todo[inline]{TODO (А надо ли?)}
        \begin{enumerate}
            \item Очевидно.
            \item Очевидно получается из простой индукции по построению $t$.
            \item Очевидно получается из простой индукции по построению $\Phi$.
        \end{enumerate}
    \end{proof}

    \begin{definition}
        Для произвольного класса $\sigma$-структур $\mathcal{K}$ положим
        \[\Th(\mathcal{K}) := \{\Phi \in \Sent_\sigma \mid \mathfrak{A} \Vdash \Phi \text{ для всех } \mathfrak{A} \in \mathcal{K}\}\]

        Вместо $\Th(\{\mathfrak{A}\})$ обычно пишут \emph{$\Th(\mathfrak{A})$}. Говорят, что \emph{$\mathfrak{A}$ и $\mathfrak{B}$ элементарно эквивалентны}, если $\Th(\mathfrak{A}) = \Th(\mathfrak{B})$.
    \end{definition}

    \begin{lemma}
        Изоморфные структуры элементарно эквивалентны.
    \end{lemma}

    \begin{definition}
        Пусть дана произвольная $\sigma$-структура $\mathfrak{A}$.

        $S \subseteq A^l$ \emph{определимо в $\mathfrak{A}$}, если существует $\sigma$-формула $\Phi(x_1, \dots, x_l)$, что
        \[S = \{\overrightarrow{a} \in A^l \mid \mathfrak{A} \Vdash \Phi[\overrightarrow{a}]\};\]
        в этом случае ещё говорят, что $\Phi$ определяет $S$ в $\mathfrak{A}$.

        $\xi: A^l \to A$ \emph{определима в $\mathfrak{A}$}, если определим её график. $a \in A$ \emph{определим в $\mathfrak{A}$}, если $\{a\}$ определимо.
    \end{definition}

    \begin{example}
        Отношение делиммости на $\NN$ определимо в \hyperlink{R-structure-definition}{$\mathfrak{R}$} посредством формулы
        \[\Phi(x, y) := x \neq 0 \wedge \exists u\ x \cdot u = y,\]
        а обычный строгий порядок на $\NN$ --- посредством
        \[\Psi(x, y) := \exists u\ (u \neq 0 \wedge x + u = y).\]
    \end{example}

    \begin{example}
        Функция последователя на $\NN$ определима в \hyperlink{R_le-structure-definition}{$\mathfrak{R}_{<}$} посредством
        \[x < y \wedge \neg \exists u\ (x < u \wedge u < y).\]
    \end{example}

    \begin{example}
        В \hyperlink{Z-structure-definition}{стандартном кольце $\mathfrak{Z}$ с носителем $\ZZ$} будет определимо отношение ``быть больше нуля''; тут можно использовать
        \[\Phi(x) := x \neq 0 \wedge \exists u_1, u_2, u_3, u_4\ (x = u_1^2 + u_2^2 + u_3^2 + u_4^2),\]
        а потому обычный строгий порядок на $\ZZ$ определим в $\mathfrak{Z}$ посредством
        \[\Psi(x, y) := \exists u\ (\Phi(u) \wedge x + u = y).\]
    \end{example}

    \begin{example}
        Обычный строгий порядок на $\RR$ определим в стандартном кольце $\mathfrak{R}$ с носителем $\RR$ посредством
        \[\Theta(x, y) := \exists u\ (u \neq 0 \wedge x + u^2 = y).\]
    \end{example}

    \begin{example}
        Рассмотрим \hyperlink{G-structure-definition}{стандартную модель $\mathfrak{G}$ геометрии}. В ней определимы отношения
        \begin{itemize}
            \item ``$x$ лежит на прямой $yz$'' посредством
                \[\Phi(x, y, z) := B(x, y, z) \vee B(z, x, y) \vee B(y, z, x)\]
            \item и ``прямые $xx'$ и $yy'$ параллельны'' посредством
                \[\Psi(x, x', y, y') := x \neq x' \wedge y \neq y' \wedge \neg \exists u\ (\Phi(u, x, x') \wedge \Phi(u, y, y')).\]
        \end{itemize}
        \hypertarget{Euclid_5-axiom-definition}{}При этом аксиому о параллельности можно выразить так:
        \[
            \mathrm{Euclid}_5 := \forall x, y, z\ (x \neq y \wedge \neg \Phi(z, x, y) \rightarrow \forall u, v\ (\Psi(z, u, x, y) \wedge \Psi(z, v, x, y) \rightarrow \Phi(z, u, v)))
        \]
        Она будет истина в $\mathfrak{G}$, но ложна в \hyperlink{H-structure-definition}{модели $\mathfrak{H}$}.
    \end{example}

    \begin{example}[без доказательства]
        Рассмотрим структуру $\langle \NN; \mid; s \rangle$, где $\mid$ интерпретируется как отношение делимости на $\NN$. Ноль определим в этой структуре посредством
        \[\Phi(x) := \neg x \mid x,\]
        а отношение равенства на $\NN$ --- посредством
        \[\Psi(x, y) := (\Phi(x) \wedge \Phi(y)) \vee (x \mid y \wedge y \mid x).\]
        \href{https://en.wikipedia.org/wiki/Julia_Robinson}{Джулией Робинсон} было доказано, что
        \begin{quotation}
            функции сложения и умножения на $\NN$ определимы в $\langle \NN; \mid; s \rangle$.
        \end{quotation}
    \end{example}

    \begin{example}[без доказательства]
        Для каждого $n \in \NN$ обозначим
        \[\supp(n) := \text{множество всех простых делителей $n$}.\]
        \href{https://en.wikipedia.org/wiki/Erd\%C5\%91s\%E2\%80\%93Woods_number}{Гипотеза Эрдёша-Вудса} заключается в следующем.
        \begin{quotation}
            Найдётся такое $N \in \NN$, что для любых $i, j \in \NN$ из
            \[\supp(i+n) = \supp(j+n) \text{ для всех } n \in \{0, \dots, N\}\]
            следует, что $i = j$.
        \end{quotation}
        Это известная открытая проблема.

        Теперь рассмотрим $\langle \NN; \perp; s \rangle$, где $\perp$ интерпретируется как отношение взаимной простоты на $\NN$. Джоном Вудсом было показано, что TFAE:
        \begin{itemize}
            \item отношение равенства на $\NN$ определимо в $\langle \NN; \perp; s \rangle$;
            \item функции сложения и умножения на $\NN$ определимы в $\langle \NN; {=}, \perp; s \rangle$;
            \item верна гипотеза Эрдёша-Вудса.
        \end{itemize}
    \end{example}

    \begin{example}
        Известно, что
        \begin{itemize}
            \item всякое (вычислимо) перечислимое множество определимо в \hyperlink{R-structure-definition}{$\mathfrak{R}$};
            \item всякая частичная вычислимая функция определима в $\mathfrak{R}$.
        \end{itemize}
        Например, $f: n \mapsto 2^n$ оказывается определима в $\mathfrak{R}$.

        Этот пример связан со знаменитыми теоремами Гёделя о неполноте ``достаточно богатых систем''.
    \end{example}

    \begin{theorem}
        Пусть $S$ определимо в $\mathfrak{A}$. Тогда для любого $\xi \in \Aut(\mathfrak{A})$
        \[\xi[S] \subseteq S,\]
        т.е. $S$ замкнуто относительно автоморфизмов $\mathfrak{A}$.
    \end{theorem}

    \begin{proof}
        Пусть $\Phi(x_1, \dots, x_l)$ --- формула, задающая $S$ в $\mathfrak{A}$ и $(a_1, \dots, a_l)$ --- случайный элемент $A^l$. Из теоремы \ref{attribution-through-isomorphism-theorem} следует, что
        \[
            \mathfrak{A} \Vdash \Phi[a_1, \dots, a_l]
            \quad \Longleftrightarrow \quad
            \mathfrak{A} \Vdash \Phi[\xi(a_1), \dots, \xi(a_l)]
        \]
        Следовательно, $(\xi(a_1), \dots, \xi(a_l)) \in S$. Поэтому $\xi[S] \subseteq S$.
    \end{proof}
    
    \begin{corollary}
        В терминах предыдущей теоремы
        \[\xi[S] = S.\]
    \end{corollary}

    \begin{remark}
        Это даёт необходимое, но далеко не достаточное условие определимости. Так если $\Formul_\sigma$ счётно, $A$ бесконечно и $\Aut(\mathfrak{A}) = \{\id_A\}$, то:
        \begin{itemize}
            \item всякое $S \subseteq A^l$ замкнуто относительно автоморфизмов $\mathfrak{A}$;
            \item множество всех определимых в $\mathfrak{A}$ множеств не больше $|\Formul_\sigma|$, т.е. не более чем счётно;
            \item значит, существует $2^{|A|}$ замкнутых относительно автоморфизмов $\mathfrak{A}$, но не определимых в $\mathfrak{A}$ множеств.
        \end{itemize}

        Конкретным примером $\mathfrak{A}$ может служить \hyperlink{R_le-structure-definition}{$\mathfrak{R}_{<}$}.
    \end{remark}

    \begin{example}
        В $\langle \ZZ; =; + \rangle$ неопределимо обычное отношение порядка на $\ZZ$, так как $\xi: a \mapsto -a$ является автоморфизмом данной структуры, нарушающим это отношение.
    \end{example}

    \begin{example}
        С другой стороны, в $\langle \ZZ; {<} \rangle$ уже не определима функция сложения на $\ZZ$, так как $\xi: a \mapsto a + 1$ является автоморфизмом данной структуры, несохраняющим данное эту функцию.
    \end{example}

    \begin{example}
        В \hyperlink{D-structure-definition}{$\mathfrak{D}$} нельзя определить никакой элемент, кроме $1$.
    \end{example}

    \begin{example}[см. упражнение \ref{G-structure-isomorphisms-example}]
        В \hyperlink{G-structure-definition}{$\mathfrak{G}$} нельзя определить:
        \begin{itemize}
            \item никакую конкретную фигуру за исключением $\varnothing$ и $\RR \times \RR$;
            \item отношение ``длина отрезка $xy$ равна единице'';
            \item отношение ``вершины треугольника $xyz$ обходятся против часовой стрелки''.
        \end{itemize}
    \end{example}

    \begin{definition}
        Пусть $=$ содержится в $\Pred_\sigma$, причём $\arity_\sigma({=}) = 2$. $\sigma$-структуру $\mathfrak{A}$ называют \emph{нормальной}, если $=$ интерпретируется в $\mathfrak{A}$ как настоящее равенство, т.е. ${=}^\mathfrak{A}$ совпадает с $\id_A$.
    \end{definition}

    \begin{remark}
        С практической точки зрения, если мы работаем в рамках фиксированного языка, начинает стираться грань между:
        \begin{itemize}
            \item \emph{настоящим равенством}, для разговора о котором необходимо выйти за пределы данного языка;
            \item \emph{неразличимостью средствами языка}.
        \end{itemize}
        В математике роль отношения равенства нередко играет отношение эквивалентности специального типа.
    \end{remark}

    \begin{definition}
        Обозначим через $\Eq_\sigma$ множество, состоящее из $\sigma$-предложений
        \begin{itemize}
            \item $\forall x\ x = x$,
            \item $\forall x\ \forall y\ (x = y \rightarrow y = x)$,
            \item $\forall x\ \forall y\ \forall z\ (x = y \wedge y = z \rightarrow x = z)$,
        \end{itemize}
        а также всех $\sigma$-предложений видов
        \begin{itemize}
            \item $\forall x_1\ \forall y_1\ \dots \forall x_n\ \forall y_n\ (x_1 = y_1 \wedge \dots \wedge x_n = y_n \rightarrow (P(x_1, \dots, x_n) \leftrightarrow P(y_1, \dots, y_n)))$, где $P \in \Pred_\sigma$, $\arity_\sigma(P) = n$,
            \item $\forall x_1\ \forall y_1\ \dots \forall x_m\ \forall y_m\ (x_1 = y_1 \wedge \dots \wedge x_m = y_m \rightarrow f(x_1, \dots, x_m) = f(y_1, \dots, y_m))$, где $f \in \Func_\sigma$, $\arity_\sigma(f) = m$.
        \end{itemize}
        Под \emph{аксиомами равенства для $\sigma$} понимают элементы $\Eq_\sigma$.
    \end{definition}

    \begin{remark}
        Разумеется, $\mathfrak{A} \Vdash \Eq_\sigma$ для всякой нормальной $\sigma$-структуры $\mathfrak{A}$.
    \end{remark}

    \begin{definition}
        Пусть $\mathfrak{A}$ --- произвольная модель $\Eq_\sigma$.

        Очевидно, ${=}^\mathfrak{A}$ будет отношением эквивалентности на $A$. Обозначим через $\mathfrak{A}'$ нормальную $\sigma$-структуру с носителем $A/{=}_\mathfrak{A}$, что:
        \begin{itemize}
            \item для любого $c \in \Const_\sigma$
                \[c^{\mathfrak{A}'} := [c^\mathfrak{A}];\]
            \item для любого $m$-местного $f \in \Func_\sigma$
                \[f^{\mathfrak{A}'}([a_1], \dots, [a_m]) := [f^\mathfrak{A}(a_1, \dots, a_m)];\]
            \item для любого $n$-местного $P \in \Func_\sigma$
                \[([a_1], \dots, [a_m]) \in P^{\mathfrak{A}'} \quad :\Longleftrightarrow \quad (a_1, \dots, a_m) \in f^\mathfrak{A}.\]
        \end{itemize}
        (Здесь $[a]$ --- класс эквивалентности $a$ по ${=}^\mathfrak{A}$.) Корректность данного определения обеспечивают аксиомы равенства для $\sigma$.
    \end{definition}

    \begin{theorem}
        Для любых $\sigma$-формулы $\Phi$ и означивания $\nu$ в $\mathfrak{A}$
        \[
            \mathfrak{A} \Vdash \Phi[\nu]
            \quad \Longleftrightarrow \quad
            \mathfrak{A}' \Vdash \Phi[\nu'],
        \]
        где $\nu': x \mapsto [\nu(x)]$.
    \end{theorem}

    \begin{theorem}%\label{attribution-through-isomorphism-theorem}
        \begin{enumerate}\ 
            \item Для любого означивания $\nu$ в $\mathfrak{A}$ $\nu'$ является означиванием в $\mathfrak{A}'$.
            \item Для каждого $\sigma$-терма $t$ и любого означивания $\nu$ в $\mathfrak{A}$
                \[\overline{\nu'}(t) = [\overline{\nu}(t)].\]
            \item Для каждой $\sigma$-формулы $\Phi$ и любого означивания $\nu$ в $\mathfrak{A}$
                \[
                    \mathfrak{A} \Vdash \Phi[\nu]
                    \quad \Longleftrightarrow \quad
                    \mathfrak{A}' \Vdash \Phi[\nu'].
                \]
        \end{enumerate}
    \end{theorem}

    \begin{proof}
        \todo[inline]{TODO (А надо ли?)}
        \begin{enumerate}
            \item Очевидно.
            \item Очевидно получается из простой индукции по построению $t$.
            \item Очевидно получается из простой индукции по построению $\Phi$.
        \end{enumerate}
    \end{proof}

    \begin{corollary}
        Для каждого $\Gamma \subseteq \Sent_\sigma$ TFAE:
        \begin{itemize}
            \item у $\Gamma$ есть нормальная модель;
            \item у $\Gamma \cup \Eq$ есть модель.
        \end{itemize}
    \end{corollary}

    Отныне будет предполагаться, что все рассматриваемые $\sigma$-структуры нормальны, если явным способом не оговорено обратное.

    \begin{definition}
        $\sigma$-формулу называют:
        \begin{itemize}
            \item \emph{выполнимой}, если $\mathfrak{A} \Vdash \Phi[\nu]$ для некоторых $\mathfrak{A}$ и $\nu$;
            \item \emph{общезначимой}, если $\mathfrak{A} \Vdash \Phi[\nu]$ для всех $\mathfrak{A}$ и $\nu$.
        \end{itemize}
        Здесь подразумевается, что $\mathfrak{A}$ бегает по $\sigma$-структурам, тогда как $\nu$ --- по означиваниям в $\mathfrak{A}$.
    \end{definition}

    \begin{remark}
        Очевидно, что
        \[\Phi \text{ общезначима} \quad \Longleftrightarrow \quad \neg \Phi \text{ не выполнима.}\]
    \end{remark}

    \begin{theorem}[Чёрча; должно быть на старших курсах]
        Проблема выполнимости для кванторной классической логики в сигнатуре $\langle R^2 \rangle$ алгоритмически неразрешима.
    \end{theorem}

    \begin{definition}
        Пусть $\Phi \in \Formul_\sigma$, и $x_1, \dots, x_l$ суть в точности все элементы $\FV(\Phi)$ в порядке появления в $\Phi$. Обозначим
        \[
            \tildeforall \Phi := \forall x_1\ \dots \forall x_l\ \Phi
            \quad \text{ и }\quad
            \tildeexists \Phi := \exists x_1\ \dots \exists x_l\ \Phi
        \]
        Тогда $\tildeforall \Phi$ называют \emph{универсальным замыканием $\Phi$}, а $\tildeexists \Phi$ --- \emph{экзистенциальным замыканием $\Phi$}.
    \end{definition}

    \begin{remark}
        Ясно, что для каждой $\mathfrak{A}$:
        \begin{align*}
            \mathfrak{A} \Vdash \tildeforall \Phi \quad &\Longleftrightarrow \quad \mathfrak{A} \Vdash \Phi[\nu] \text{ для всех } \nu;\\
            \mathfrak{A} \Vdash \tildeexists \Phi \quad &\Longleftrightarrow \quad \mathfrak{A} \Vdash \Phi[\nu] \text{ для некоторого } \nu.
        \end{align*}
        Стало быть, имеют место следующие эквивалентности:
        \begin{align*}
            \Phi \text{ выполнима} \quad &\Longleftrightarrow \quad \mathfrak{A} \Vdash \tildeexists \Phi \text{ для некоторой } \mathfrak{A};\\
            \Phi \text{ общезначима} \quad &\Longleftrightarrow \quad \mathfrak{A} \Vdash \tildeforall \Phi \text{ для всех } \mathfrak{A}.
        \end{align*}
    \end{remark}

    \begin{definition}
        Пусть $\Gamma \subseteq \Sent_\sigma$ и $\Phi \in \Formul_\sigma$. Говорят, что \emph{$\Phi$ семантически следует из $\Gamma$}, и пишут $\Gamma \vDash \Phi$, если для любой $\mathfrak{A}$
        \[\mathfrak{A} \Vdash \Gamma \quad \longrightarrow \quad \mathfrak{A} \Vdash \tildeforall \Phi.\]
        Вместо $\varnothing \vDash \Phi$ обычно пишут $\vDash \Phi$.

        Формулы $\Phi$ и $\Psi$ называются \emph{семантически эквивалентными}, если $\vDash \Phi \leftrightarrow \Psi$; при этом пишут $\Phi \equiv \Psi$.
    \end{definition}

    \begin{remark}
        Очевидно
        \[\vDash \Phi \quad \Longleftrightarrow \quad \Phi \text{ общезначима}.\]
    \end{remark}

    \begin{example}
        Для любых $\Phi \in \Formul_\sigma$ и $x \in \Var$
        \[
            \neg \forall x \Phi \equiv \exists x \neg \Phi
            \quad \text{ и } \quad
            \neg \exists x \Phi \equiv \forall x \neg \Phi.
        \]
    \end{example}

    \begin{definition}
        $\sigma$-формула называется \emph{бескванторной}, если $\forall \not\preccurlyeq \Phi$ и $\exists \not\preccurlyeq \Phi$.

        Под \emph{пренексными нормальными формами (ПНФ)} понимаются $\sigma$-формулы вида
        \[Q_1 x_1\ \dots Q_l x_l\ \Psi,\]
        где $\{Q_1, \dots, Q_l\} \subseteq \{\forall, \exists\}$, $\{x_1, \dots, x_l\} \subseteq \Var$ и $\Psi$ бескванторная.
    \end{definition}

    \begin{exercise}
        Всякая $\sigma$-формула семантически эквивалентна некоторой ПНФ.
    \end{exercise}

    \subsection{Гильбертовское исчисление для кванторной классической логики}

    \begin{definition}
        Рассмотрим следующие аксиомы:
        \begin{description}
            \item[$\mathrm{I1}$.] $\Phi \rightarrow (\Psi \rightarrow \Phi)$;
            \item[$\mathrm{I2}$.] $(\Phi \rightarrow (\Psi \rightarrow \Theta)) \rightarrow ((\Phi \rightarrow \Psi) \rightarrow (\Phi \rightarrow \Theta))$;
            \item[$\mathrm{C1}$.] $\Phi \wedge \Psi \rightarrow \Phi$;
            \item[$\mathrm{C2}$.] $\Phi \wedge \Psi \rightarrow \Psi$;
            \item[$\mathrm{C3}$.] $\Phi \rightarrow (\Psi \rightarrow \Phi \wedge \Psi)$;
            \item[$\mathrm{D1}$.] $\Phi \rightarrow \Phi \vee \Psi$;
            \item[$\mathrm{D2}$.] $\Psi \rightarrow \Phi \vee \Psi$;
            \item[$\mathrm{D3}$.] $(\Phi \rightarrow \Theta) \rightarrow ((\Psi \rightarrow \Theta) \rightarrow (\Phi \vee \Psi \rightarrow \Theta))$;
            \item[$\mathrm{N1}$.] $(\Phi \rightarrow \Psi) \rightarrow ((\Phi \rightarrow \neg \Psi) \rightarrow \neg \Phi)$;
            \item[$\mathrm{N2}$.] $\neg \Phi \rightarrow (\Phi \rightarrow \Psi)$;
            \item[$\mathrm{N3}$.] $\Phi \vee \neg \Phi$;
            \item[$\mathrm{Q1}$.] $\forall x\ \Phi \rightarrow \Phi(x/t)$, где $t$ свободен для $x$ в $\Phi$;
            \item[$\mathrm{Q2}$.] $\Phi(x/t) \rightarrow \exists x\ \Phi$, где $t$ свободен для $x$ в $\Phi$.
        \end{description}

        Кроме того, в случаях, когда ${=}$ содержится в $\Pred_\sigma$, элементы $\Eq_\sigma$ также будут считаться аксиомами нашего исчисления.

        Помимо них у нас имеется правило ``modus ponens'', т.е.
        \begin{prooftree}
            \AxiomC{$\Phi$}
            \AxiomC{$\Phi \rightarrow \Psi$}
                \RightLabel{($\mathrm{MP}$)}
                \BinaryInfC{$\Psi$}
        \end{prooftree}
        и добавляются два новых ``кванторных'' правила вывода:
        \begin{multicols}{3}
            \centering
            \begin{prooftree}
                \AxiomC{$\Phi \rightarrow \Psi$}
                    \RightLabel{($\mathrm{BR}1$)}
                    \UnaryInfC{$\Phi \rightarrow \forall x\ \Psi$}
            \end{prooftree}
            и
            \begin{prooftree}
                \AxiomC{$\Phi \rightarrow \Psi$}
                    \RightLabel{($\mathrm{BR}2$)}
                    \UnaryInfC{$\exists x\ \Phi \rightarrow \Psi$}
            \end{prooftree}
        \end{multicols}
        где $x \notin \FV(\Psi)$. Он традиционно называются \emph{правилами Бернайса}.

        Пусть $\Gamma \subseteq \Sent_\sigma$. \emph{Вывод из $\Gamma$} в данном гильбертовском исчислении --- конечная последовательность
        \[\Phi_0, \dots, \Phi_n\]
        (где $n \in \NN$) элементов $\Formul_\sigma$, что для каждого $i \in \{0; \dots; n\}$ верно одно из следующих условий:
        \begin{itemize}
            \item $\Phi_i$ есть аксиома;
            \item $\Phi_i$ является элементом $\Gamma$;
            \item существуют $\{j; k\} \subseteq \{0; \dots; i-1\}$ такие, что $\Phi_k = \Phi_j \rightarrow \Phi_i$;
            \item существует $j \in \{0; \dots; i-1\}$ такое, что $\Phi_i$ получается из $\Phi_j$ по $\mathrm{BR1}$ или по $\mathrm{BR2}$.
        \end{itemize}
        При этом $\Phi_n$ называется \emph{заключением} рассматриваемого вывода, а элементы $\Gamma$ --- его \emph{гипотезами}.

        Для $\Gamma \subseteq \Sent_\sigma$ и $\Phi \in \Formul_\sigma$ запись $\Gamma \vdash \Phi$ означает, что существует вывод из $\Gamma$ с заключением $\Phi$. Вместо $\varnothing \vdash \Phi$ обычно пишут $\vdash \Phi$.
    \end{definition}

    \begin{definition}
        Пусть даны $\Gamma \subseteq \Sent_\sigma$ и $\Phi \in \Formul_\sigma$.
        \begin{itemize}
            \item \emph{$\Phi$ опровержима в $\Gamma$}, если $\Gamma \vdash \neg \Phi$.
            \item \emph{$\Phi$ независима от $\Gamma$}, если $\Gamma \nvdash \Phi$ и $\Gamma \nvdash \neg \Phi$.
        \end{itemize}
    \end{definition}

    \begin{example}
        Благодаря трудам Коэна и Гёделя мы знаем, что:
        \begin{itemize}
            \item \Caxiom независима от \ZF, а \CH --- от \ZFC;
            \item в частности, $\neg \Caxiom$ не опровержимо в \ZF, а $\neg \CH$ --- в \ZFC.
        \end{itemize}
        Разумеется, тут предполагается непротиворечивость соответственно \ZF и \ZFC, поскольку иначе выводимо всё, что угодно.
    \end{example}

    \begin{lemma}\ 
        \begin{enumerate}
            \item {\bf Монотонность.} Если $\Gamma \subseteq \Delta$ и $\Gamma \vdash \Phi$, то $\Delta \vdash \Phi$.
            \item {\bf Транзитивность.} Если для всякого $\Psi \in \Gamma$ верно $\Delta \vdash \Psi$ и $\Gamma \vdash \Phi$, то $\Delta \vdash \Phi$.
            \item {\bf Компактность.} Если $\Gamma \vdash \Phi$, то для некоторого конечного $\Delta \subseteq \Gamma$ верно $\Delta \vdash \Phi$.
        \end{enumerate}
    \end{lemma}

    \begin{remark*}
        Однако стоит помнить, что $\Gamma$ и $\Delta$ представляют собой множества $\sigma$-предложений, т.е. у их элементов нет свободных переменных.
    \end{remark*}

    \begin{proof}\ 
        \begin{enumerate}
            \item Рассматривая вывод $\Phi$ из $\Gamma$, сиюминутно получаем вывод $\Phi$ из $\Delta$.
            \item Возьмём вывод $\Phi$ из $\Gamma$. Рассмотрим все использованные утверждения $\Gamma$ в этом выводе; получим конечное множество $\Gamma'$. Далее для всякого $\Psi \in \Gamma'$ рассмотрим вывод $\Psi$ из $\Delta$, сотрём $\Psi$ на конце этого вывода и припишем его в начало ранее рассмотренного вывода $\Phi$. Тогда несложно понять, что мы получаем вывод $\Phi$ из $\Delta$.
            \item Хватает просто взять в качестве $\Delta$ множество всех формул из $\Gamma$, использованных в каком-то конкретном выводе $\Phi$ из $\Gamma$. Тогда очевидно, что $\Delta$ конечно, а рассмотренный вывод станет выводом $\Phi$ из $\Delta$.
        \end{enumerate}
    \end{proof}

    \begin{definition}
        Пусть $\xi: \Prop \to \Formul_\sigma$. Для всякой пропозициональной формулы $\varphi$ обозначим
        \[
            \xi \varphi :=
            \begin{gathered}
                \text{результат замены (всех вхождений)}\\
                \text{каждой $p \in \Prop$ в $\varphi$ на $\xi(p)$.}
            \end{gathered}
        \]
        Также для удобства для всякого $\Gamma \subseteq \Formul$ определим $\xi \Gamma := \{\xi \psi \mid \psi \in \Gamma\}$
    \end{definition}

    \begin{remark*}
        $\xi \varphi$ есть $\sigma$-формула.
    \end{remark*}

    \begin{lemma}
        Пусть $\xi: \Prop \rightarrow \Formul_\sigma$ и $\Gamma \cup \{\varphi\} \subseteq \Formul$. Тогда
        \[
            \Gamma \vdash \varphi
            \quad \Longrightarrow \quad
            \xi \Gamma \vdash \xi \varphi.
        \]
    \end{lemma}
    
    \begin{remark*}
        $\Gamma \vdash \varphi$ рассматривается в пропозициональном исчислении, а $\xi \Gamma \vdash \xi \varphi$ --- в кванторном.
    \end{remark*}

    \begin{proof}
        Зафиксируем какой-нибудь вывод
        \[\varphi_0, \dots, \varphi_n\]
        $\varphi$ из $\Gamma$ (в пропозициональном исчислении), и рассмотрим последовательность
        \[\xi \varphi_0, \dots, \xi \varphi_n.\]
        Покажем индукцией по $i \in \{0, \dots, n\}$, что $\xi \varphi_0, \dots, \xi \varphi_i$ является выводом из $\xi \Gamma$ (в кванторном исчислении).

        Возможны следующие случаи.
        \begin{itemize}
            \item $\varphi_i$ --- аксиома. Тогда $\xi \varphi_i$ --- аксиома.
            \item $\varphi_i$ --- элемент $\Gamma$. Тогда $\xi \varphi_i$ --- элемент $\xi \Gamma$.
            \item $\varphi_i$ получается из предшествующих $\varphi_j$ и $\varphi_k = \varphi_j \rightarrow \varphi_i$ по $\mathrm{MP}$. Тогда $\xi \varphi_i$ получается из предшествующих $\xi \varphi_j$ и $\xi \varphi_k = \xi \varphi_j \rightarrow \xi \varphi_i$ по $\mathrm{MP}$.
        \end{itemize}

        Стало быть, $\xi \Gamma \vdash \xi \varphi_i$ для всех $i \in \{0, \dots, n\}$. В частности для $i = n$ мы имеем, что $\xi \Gamma \vdash \xi \varphi$.
    \end{proof}

    \begin{corollary}
        В терминах доказанной теоремы, предполагая $\Gamma = \varnothing$, получаем, что
        \[
            \vdash \varphi
            \quad \Longrightarrow \quad
            \vdash \xi \varphi.
        \]
    \end{corollary}

    \begin{corollary}
        Пусть $\xi: \Prop \to \Formul_\sigma$, $\varphi \in \Formul$ и $\vDash \varphi$ (в смысле проп. логики). Тогда $\vdash \xi \varphi$.
    \end{corollary}

    \begin{example}
        Так, в нашем первопорядковом исчислении окажутся выводимы
        \[\Phi \rightarrow \Phi, \quad \Phi \rightarrow \neg \neg \Phi \quad \text{ и } \quad \neg \neg \Phi \rightarrow \Phi\]
        для произвольной $\sigma$-формулы $\Phi$.
    \end{example}

    \begin{example}
        Пусть переменная $y$ не входит в $\sigma$-формулу $\Phi$. Тогда
        \begin{center}
            \begin{tabular}{rll}
                1.& $\forall x\ \Phi \rightarrow \Phi(x/y)$& $\mathrm{Q1}$\\
                2.& $\forall x\ \Phi \rightarrow \forall y\ \Phi(x/y)$& из 1; $\mathrm{BR1}$
            \end{tabular}
        \end{center}
        будет выводом из $\varnothing$. Кроме того,
        \begin{center}
            \begin{tabular}{rll}
                1.& $\forall y\ \Phi(x/y) \rightarrow \overbrace{\Phi(x/y)(y/x)}^\Phi$& $\mathrm{Q1}$\\
                2.& $\forall y\ \Phi(x/y) \rightarrow \forall x\ \Phi$& из 1; $\mathrm{BR1}$
            \end{tabular}
        \end{center}
        будет выводом из $\varnothing$. Стало быть, $\vdash \forall x\ \Phi \leftrightarrow \forall y\ \Phi(x/y)$.
    \end{example}

    \begin{example}
        Покажем, что $\vdash \exists x\ \neg \Phi \rightarrow \neg \forall x\ \Phi$:
        \begin{center}
            \begin{tabular}{rll}
                1.& $\forall x\ \Phi \rightarrow \overbrace{\Phi(x/x)}^\Phi$& $\mathrm{Q1}$\\
                2.& $(\forall x\ \Phi \rightarrow \Phi) \rightarrow (\neg \Phi \rightarrow \neg \forall x\ \Phi)$& тавтология\\
                3.& $\neg \Phi \rightarrow \neg \forall x\ \Phi$& из 1, 2; $\mathrm{MP}$\\
                4.& $\exists x\ \neg \Phi \rightarrow \neg \forall x\ \Phi$& из 3; $\mathrm{BR2}$
            \end{tabular}
        \end{center}
        Заметим, что с помощью тавтологий из этого можно легко получить $\vdash \forall x\ \Phi \rightarrow \neg \exists x\ \neg \Phi$.
    \end{example}

    \begin{example}
        Пусть $\vdash \Phi \rightarrow \Psi$. Покажем, что тогда $\vdash \forall x\ \Phi \rightarrow \forall x\ \Psi$:
        \begin{center}
            \begin{tabular}{rll}
                1.& $\forall x\ \Phi \rightarrow \Phi$& $\mathrm{Q1}$\\
                2.& $\Phi \rightarrow \Psi$& предположение\\
                3.& $\forall x\ \Phi \rightarrow \Psi$& из 1, 2 и тавтологий; $\mathrm{MP}$\\
                4.& $\forall x\ \Phi \rightarrow \forall x\ \Psi$& из 3; $\mathrm{BR1}$
            \end{tabular}
        \end{center}
        Используя $\mathrm{C1}$, $\mathrm{C2}$, $\mathrm{C3}$, отсюда легко получить:
        \[
            \vdash \Phi \leftrightarrow \Psi
            \quad \Longrightarrow \quad
            \vdash \forall x\ \Phi \leftrightarrow \forall x\ \Psi.
        \]
    \end{example}

    \begin{definition}
        Для удобства введём обозначения
        \[
            \top := \Phi_\star \rightarrow \Phi_\star
            \quad \text{ и } \quad
            \bot := \neg \top,
        \]
        где $\Phi_\star$ --- фиксированное $\sigma$-предложение.
    \end{definition}

    \begin{theorem}
        Для любых $\Gamma \subseteq \Sent_\sigma$, $\Phi \in \Formul_\sigma$ и $x \in \Var$
        \[
            \Gamma \vdash \Phi
            \quad \Longleftrightarrow \quad
            \Gamma \vdash \forall x\ \Phi
        \]
    \end{theorem}

    \begin{proof}
        \begin{itemize}
            \item[$\Rightarrow$)] Пусть $\Gamma \vdash \Phi$.
                \begin{center}
                    \begin{tabular}{rll}
                        1.& $\Phi$& выводится по предложению\\
                        2.& $\Phi \rightarrow (\top \rightarrow \Phi)$& $\mathrm{I1}$\\
                        3.& $\top \rightarrow \Phi$& из 1, 2; $\mathrm{MP}$\\
                        4.& $\top \rightarrow \forall x\ \Phi$& из 3; $\mathrm{BR1}$\\
                        5.& $\forall x\ \Phi$& из 4 и тавтологий; $\mathrm{MP}$\\
                    \end{tabular}
                \end{center}
                
            \item[$\Leftarrow$)] Пусть $\Gamma \vdash \forall x\ \Phi$.
                \begin{center}
                    \begin{tabular}{rll}
                        1.& $\forall x\ \Phi$& выводится по предложению\\
                        2.& $\forall x\ \Phi \rightarrow \overbrace{\Phi(x/x)}^\Phi$& $\mathrm{Q1}$\\
                        3.& $\Phi$& из 1, 2; $\mathrm{MP}$\\
                    \end{tabular}
                \end{center}
        \end{itemize}
    \end{proof}

    \begin{remark}
        Значит, мы можем дополнительно использовать \emph{правило обобщения}:
        \begin{prooftree}
            \AxiomC{$\Phi$}
                \RightLabel{($\mathrm{GR}$)}
                \UnaryInfC{$\forall x\ \Phi$}
        \end{prooftree}
        Оно нередко фигурирует в альтернативных версиях гильбретовского исчисления для классической логики первого порядка.
    \end{remark}

    \begin{corollary}
        Для любых $\Gamma \subseteq \Sent_\sigma$ и $\Phi \in \Formul_\sigma$
        \[
            \Gamma \vdash \Phi
            \quad \Longleftrightarrow \quad
            \Gamma \vdash \tildeforall \Phi.
        \]
    \end{corollary}

    \begin{remark*}
        Как нетрудно убедиться, для $\vDash$ имеет место аналогичное утверждение.
    \end{remark*}

    \begin{theorem}[о дедукции]
        Для любых $\Gamma \cup \{\Phi\} \subseteq \Sent_\sigma$ и $\Psi \in \Formul_\sigma$,
        \[
            \Gamma \cup \{\Phi\} \vdash \Psi
            \quad \Longleftrightarrow \quad
            \Gamma \vdash \Phi \rightarrow \Psi.
        \]
    \end{theorem}

    \begin{proof}
        \begin{itemize}
            \item[$\Leftarrow$)] Тривиально.
            \item[$\Leftarrow$)] Пусть $\Gamma \cup \{\Phi\} \vdash \Psi$. Зафиксируем какой-нибудь вывод
                \[\Psi_0, \dots, \Psi_n\]
                $\Psi$ из $\Gamma \cup \{\Phi\}$. По индукции по $i \in \{0, \dots, n\}$ покажем, что $\Gamma \vdash \Phi \rightarrow \Psi_i$. Ввиду аналогии с пропозициальным исчислением, достаточно разобрать лишь новые случаи, относящиеся к $\mathrm{BR1}$ и $\mathrm{BR2}$.
                \begin{itemize}
                    \item Пусть $\Psi_i = \Theta \rightarrow \forall x\ \Omega$ получается из предшествующего $\Psi_j = \Theta \rightarrow \Omega$ по $\mathrm{BR1}$. Тогда можно построить такой ``квазивывод'' из $\Gamma$:
                        \begin{center}
                            \begin{tabular}{rll}
                                1.& $\Phi \rightarrow \overbrace{(\Theta \rightarrow \Omega)}^{\Psi_j}$& предположение индукции\\
                                2.& $(\Phi \rightarrow (\Theta \rightarrow \Omega)) \rightarrow (\Phi \wedge \Theta \rightarrow \Omega)$& тавтология\\
                                3.& $\Phi \wedge \Theta \rightarrow \Omega$& из 1, 2; $\mathrm{MP}$\\
                                4.& $\Phi \wedge \Theta \rightarrow \forall x\ \Omega$& из 3; $\mathrm{BR1}$\\
                                5.& $(\Phi \wedge \Theta \rightarrow \forall x\ \Omega) \rightarrow (\Phi \rightarrow (\Theta \rightarrow \forall x\ \Omega))$& тавтология\\
                                6.& $\Phi \rightarrow (\Theta \rightarrow \forall x\ \Omega)$& из 4, 5; $\mathrm{MP}$\\
                            \end{tabular}
                        \end{center}
                        Стало быть, $\Gamma \vdash \Phi \rightarrow (\theta \rightarrow \forall x\ \Omega)$.
                    
                        
                    \item Пусть $\Psi_i = \exists x\ \Theta \rightarrow \Omega$ получается из предшествующего $\Psi_j = \Theta \rightarrow \Omega$ по $\mathrm{BR2}$. Тогда можно построить такой ``квазивывод'' из $\Gamma$:
                        \begin{center}
                            \begin{tabular}{rll}
                                1.& $\Phi \rightarrow \overbrace{(\Theta \rightarrow \Omega)}^{\Psi_j}$& предположение индукции\\
                                2.& $(\Phi \rightarrow (\Theta \rightarrow \Omega)) \rightarrow (\Theta \rightarrow (\Phi \rightarrow \Omega))$& тавтология\\
                                3.& $\Theta \rightarrow (\Phi \rightarrow \Omega)$& из 1, 2; $\mathrm{MP}$\\
                                4.& $\exists x\ \Theta \rightarrow (\Phi \rightarrow \Omega)$& из 3; $\mathrm{BR2}$\\
                                5.& $(\exists x\ \Theta \rightarrow (\Phi \rightarrow \Omega)) \rightarrow (\Phi \rightarrow (\exists x\ \Theta \rightarrow \Omega))$& тавтология\\
                                6.& $\Phi \rightarrow (\exists x\ \Theta \rightarrow \Omega)$& из 4, 5; $\mathrm{MP}$\\
                            \end{tabular}
                        \end{center}
                        Стало быть, $\Gamma \vdash \Phi \rightarrow (\exists x\ \Theta \rightarrow \Omega)$.
                \end{itemize}
                В частности, при $i := n$ мы имеем $\Gamma \vdash \Phi \rightarrow \Psi_n$, т.е. $\Gamma \vdash \Phi \rightarrow \Psi$.
        \end{itemize}
    \end{proof}

    \begin{corollary}
        Для любых $\Gamma \subseteq \Sent_\sigma$ и $\Phi \in \Form_\sigma$
        \[
            \Gamma \vdash \Phi
            \quad \Longleftrightarrow \quad
            \vdash \bigwedge_{i=1}^n \Psi_i \rightarrow \Phi \text{ для некоторых } \{\Psi_1; \dots; \Psi_n\} \subseteq \Gamma.
        \]
        (В случае $n=0$ соответствующая конъюнкция отождествляется с $\top$.)
    \end{corollary}
    
    \begin{lemma}
        Пусть $\xi: \Prop \rightarrow \Formul_\sigma$, $\Gamma \cup \{\varphi\} \subseteq \Formul$. Тогда
        \[
            \Gamma \vDash \varphi
            \quad \Longrightarrow \quad
            \xi \Gamma \vDash \xi \varphi
        \]
    \end{lemma}
    
    \begin{proof}
        Рассмотрим произвольные $\sigma$-структуру $\mathfrak{A}$ и означивание $\nu$ в $\mathfrak{A}$. Определим оценку
        \[
            v(p) :=
            \begin{cases}
                1& \text{ если $\mathfrak{A} \Vdash \xi(p) [\nu]$}\\
                0& \text{ иначе.}
            \end{cases}
        \]
        Легко видеть, что для всякой $\psi \in \Formul$
        \[
            v \Vdash \psi
            \quad \Longleftrightarrow \quad
            \mathfrak{A} \Vdash \xi \psi [\nu];
        \]
        это следует из простой индукции по построению $\psi$. Следовательно
        \[
            (\forall \psi \in \Gamma \quad v \Vdash \psi) \rightarrow v \Vdash \varphi
            \quad \Longleftrightarrow \quad
            (\forall \psi \in \Gamma \quad \mathfrak{A} \Vdash \xi \psi [\nu]) \rightarrow \mathfrak{A} \Vdash \xi \varphi [\nu].
        \]
        Но по определению $\varphi$ и $\Gamma$ мы имеем, что левое утверждение выполняется всегда, а значит и правая часть выполняется всегда. Стало быть, $\xi \Gamma \vDash \xi \varphi$.
    \end{proof}

    \begin{lemma}
        Пусть $\Phi$ --- аксиома кванторного исчисления. Тогда $\vDash \Phi$.
    \end{lemma}

    \begin{proof}
        \begin{itemize}
            \item Пусть $\Phi$ --- пропозициональная аксиома. Тогда она имеет вид $\xi \varphi$, где $\varphi$ --- аксиома пропозиционального исчисления, а $\xi$ --- функция из $\Prop$ в $\Formul_\sigma$. Тогда $\vDash \varphi$, а значит и $\vDash \Phi$ по предыдущей лемме.

            \item Пусть $\Phi$ --- кванторная аксиома, т.е. она имеет вид
                \[
                    \forall x\ \Psi \rightarrow \Psi(x/t)
                    \quad \text{ или } \quad
                    \Psi(x/t) \rightarrow \exists x\ \Psi(x),
                \]
                где $t$ свободен для $x$ в $\Psi$. Рассмотрим произвольные $\sigma$-структуру $\mathfrak{A}$ и означение $\nu$ в $\mathfrak{A}$. Можно показать индукцией по построению $\Psi$, что
                \[
                    \mathfrak{A} \vDash \Psi(x/t) [\nu]
                    \quad \Longleftrightarrow \quad
                    \mathfrak{A} \vDash \Psi [\nu_{\overline{\nu}(t)}^x].
                \]
                \todo[inline]{Может быть довести.}
                Отсюда мы сразу получаем $\vDash \Phi$.
        \end{itemize}
    \end{proof}

    \begin{theorem}[о корректности $\vdash$]\label{Hilbert-conclusion-correctness-2}
        Для любых $\Gamma \subseteq \Sent_\sigma$ и $\Phi \in \Formul_\sigma$,
        \[
            \Gamma \vdash \Phi
            \quad \Longrightarrow \quad
            \Gamma \vDash \Phi
        \]
    \end{theorem}

    \begin{proof}
        Пусть $\Gamma \vdash \Phi$. Зафиксируем какой-нибудь вывод
        \[\Phi_0, \dots, \Phi_n\]
        $\Phi$ из $\Gamma$. Пусть $\mathfrak{A}$ --- произвольная модель $\Gamma$. Покажем индукцией по $i \in \{0, \dots, n\}$, что $\mathfrak{A} \Vdash \Phi_i [\nu]$ для всякого означивания $\nu$ в $\mathfrak{A}$.

        Рассмотрим возможные случаи.
        \begin{itemize}
            \item Пусть $\Phi_i$ --- аксиома. Тогда $\vDash \Phi_i$, а потому $\Gamma \vDash \Phi_i$.
            \item Пусть $\Phi_i$ --- элемент $\Gamma$. Тогда, очевидно, $\mathfrak{A} \Vdash \Phi_i [\nu]$ для всех $\nu$.
            \item Пусть $\Phi_i$ получается из предшествующих $\Phi_j$ и $\Phi_k = \Phi_j \rightarrow \Phi_i$ по $\mathrm{MP}$. В силу индукционной гипотезы, для вякого $\nu$
                \[
                    \mathfrak{A} \Vdash \Phi_j [\nu]
                    \quad \text{ и } \quad
                    \mathfrak{A} \Vdash \Phi_j \rightarrow \Phi_i [\nu],
                \]
                откуда немедленно следует $\mathfrak{A} \Vdash \Phi_i [\nu]$.
            \item Пусть $\Phi_i = \Theta \rightarrow \forall x\ \Omega$ получается из предшествующего $\Phi_j = \Theta \rightarrow \Omega$ по $\mathrm{BR1}$. Рассмотрим произвольное означивание $\nu$. Заметим, что
                \[
                    \mathfrak{A} \Vdash \Theta[\nu]
                    \quad \Longleftrightarrow \quad
                    \mathfrak{A} \Vdash \Theta[\nu_a^x] \text{ для всех } a \in A
                \]
                (так как $x \notin \FV(\Theta)$). Кроме того, ввиду предположения индукции мы имеем $\mathfrak{A} \Vdash \Theta \rightarrow \Omega [\nu_a^x]$. Итак
                \begin{align*}
                    \mathfrak{A} \Vdash \Theta [\nu]\quad
                    &\Longrightarrow \quad \mathfrak{A} \Vdash \Theta [\nu_a^x] \text{ для всех } a \in A\\
                    &\Longrightarrow \quad \mathfrak{A} \Vdash \Omega [\nu_a^x] \text{ для всех } a \in A\\
                    &\Longrightarrow \quad \mathfrak{A} \Vdash \forall x\ \Omega [\nu].
                \end{align*}
                Стало быть, $\mathfrak{A} \Vdash \Phi_i [\nu]$.
            \item Случай $\mathfrak{BR2}$ по существу аналогичен случаю $\mathfrak{BR1}$.
        \end{itemize}

        В частности, $\mathfrak{A} \Phi [\nu]$ для всякого означивания $\nu$ в $\mathfrak{A}$. Таким образом $\Gamma \vDash \Phi$.
    \end{proof}

    \begin{corollary}
        Для любой $\Phi \in \Formul_\sigma$, если $\vdash \Phi$, то $\vDash \Phi$.
    \end{corollary}

    \begin{definition}
        $\Gamma \subseteq \Sent_\sigma$ называют \emph{противоречивым}, если $\Gamma \vdash \Phi$ и $\Gamma \vdash \neg \Phi$ для некоторой $\Phi \in \Formul_\sigma$, и \emph{непротиворечивым} иначе.
    \end{definition}

    \begin{lemma}
        Для $\Gamma \subseteq \Formul_\sigma$ TFAE:
        \begin{itemize}
            \item $\vdash \Psi$ для всех $\Psi \in \Formul_\sigma$;
            \item $\Gamma$ противоречиво;
            \item $\Gamma \vdash \bot$.
        \end{itemize}
    \end{lemma}

    \begin{corollary}
        Пусть $\Gamma \subseteq \Sent_\sigma$ и $\Phi \in \Sent_\sigma$.
        \begin{itemize}
            \item Если у $\Gamma$ есть модель, то $\Gamma \nvdash \bot$.
            \item Если у $\Gamma \cup \{\Phi\}$ есть модель, то $\Gamma \nvdash \neg \Phi$.
            \item Если у $\Gamma \cup \{ \neg \Phi\}$ есть модель, то $\Gamma \nvdash \Phi$.
        \end{itemize}
    \end{corollary}

    \begin{remark*}
        Это, пожалуй, самый базовый метод доказательства непротиворечивости теорий независимости предложений от теории.
    \end{remark*}

    \begin{example}
        Пусть в качестве $\sigma$ выступает $\langle =^2; s^1; 0 \rangle$, а $\Gamma$ состоит из
        \begin{itemize}
            \item $\forall x\ s(x) \neq 0$,
            \item $\forall x\ \forall y\ (s(x) = s(y) \rightarrow x = y)$ и
            \item $\forall x\ (x \neq 0 \rightarrow \exists y\ x = s(y))$.
        \end{itemize}
        Обозначим через $\mathfrak{A}$ и $\mathfrak{B}$ естественные $\sigma$-структуры с носителями $\NN$ и $\NN \cup \{\infty\}$ соответственно (считая $s^\mathfrak{B}(\infty) = \infty$). Возьмём
        \[
            \Phi := \forall x\ s(x) \neq x.
        \]
        Тогда $\mathfrak{A} \Vdash \Gamma \cup \{\Phi\}$ и $\mathfrak{B} \Vdash \Gamma \cup \{\neg \Phi\}$, а значит, $\Phi$ независима от $\Gamma$.
    \end{example}

    \begin{example}[без деталей]
        Пусть $\sigma$ --- это сигнатура структур \hyperlink{G-structure-definition}{$\mathfrak{G}$} и \hyperlink{H-structure-definition}{$\mathfrak{H}$}, т.е. $\langle {=}^2; {\simeq}^4, B^3 \rangle$. $\sigma$-пре\-дло\-же\-ния, формализующие постулаты геометрии Евклида без постулата о единственности параллельной прямой, обычно именуют \emph{аксиомами абсолютной геометрии}; обозначим их через $\Abs$. Тогда
        \[
            \mathfrak{G} \Vdash \Abs \cup \{\mathrm{Euclid}_5\}
            \quad \text{ и } \quad
            \mathfrak{H} \Vdash \Abs \cup \{\neg \mathrm{Euclid}_5\}.
        \]
        Значит, \hyperlink{Euclid_5-axiom-definition}{$\mathrm{Euclid}_5$} (аксиома о параллельных) независима от $\Abs$.
    \end{example}

    \begin{definition}
        В дальнейшем, когда это не приводит к путанице, мы будем нередко отождествлять с $\sigma$ с $\Pred_\sigma \cup \Func_\sigma \cup \Const_\sigma$ (учитывая роли символов и их местности). В частности:
        \begin{itemize}
            \item запись $\varepsilon \in \sigma$ является сокращением для $\varepsilon \in \Pred_\sigma \cup \Func_\sigma \cup \Const_\sigma$;
            \item запись $\sigma \subseteq \sigma'$ означает, что
                \begin{align*}
                    &\Pred_\sigma \subseteq \Pred_{\sigma'},&
                    &\Func_\sigma \subseteq \Func_{\sigma'}&
                    &\text{ и }&
                    &\Const_\sigma \subseteq \Const_{\sigma'},
                \end{align*}
                причём $\arity_\sigma$ совпадает с сужением $\arity_{\sigma'}$ на $\Pred_\sigma \cup \Func_\sigma$;
            \item под $|\sigma|$ подразумевается $|\Pred_\sigma \cup \Func_\sigma \cup \Const_\sigma|$.
        \end{itemize}

        Пусть даны $\sigma$-структура $\mathfrak{A}$ и $\sigma'$-структура $\mathfrak{A}'$, причём $\sigma'$ включает $\sigma$. Говорят, что $\mathfrak{A}$ является \emph{$\sigma$-обеднением} $\mathfrak{A}'$, а $\mathfrak{A}'$ --- \emph{$\sigma'$-обогащением}, если $A = A'$ и $\varepsilon^\mathfrak{A} = \varepsilon^{\mathfrak{A}'}$ для всех $\varepsilon \in \sigma$.
    \end{definition}

    \begin{theorem}[о сильной полноте $\vdash$]
        Для любых $\Gamma \subseteq \Sent_\sigma$ и $\Phi \in \Formul_\sigma$
        \[
            \Gamma \vdash \Phi
            \quad \Longleftrightarrow \quad
            \Gamma \vDash \Phi
        \]
    \end{theorem}

    \begin{proof}[Идея доказательства.]
        \begin{itemize}
            \item[$\Rightarrow$)] См. \hyperref[Hilbert-conclusion-correctness-2]{теорему о корректности}.
            \item[$\Leftarrow$)] Допустим, что $\Gamma \nvdash \Phi$. Хотим показать, что $\Gamma \nvDash \Phi$.

                Заметим, что $\Gamma \nvdash \Phi$ равносильно $\Gamma \nvdash \tildeforall \Phi$, а $\Gamma \nvDash \Phi$ --- $\Gamma \nvDash \tildeforall \Phi$. Теперь действуем по аналогии с пропозициональной логикой:
                \begin{itemize}
                    \item Построим ``насыщенную теорию'' $\Gamma' \supseteq \Gamma$ такую, что $\Gamma' \nvdash \tildeforall \Phi$; при этом $\Gamma' \nvdash \tildeforall \Phi$ будет равносильно $\tildeforall \Phi \notin \Gamma'$.
                    \item Далее, с помощью $\Gamma'$ построим структуру $\mathfrak{A}_{\Gamma'}$ такую, что для любого предложения $\Psi$,
                        \[
                            \mathfrak{A}_{\Gamma'} \Vdash \Psi
                            \quad \Longleftrightarrow \quad
                            \Psi \in \Gamma'.
                        \]
                        Мы получим $\mathfrak{A}_{\Gamma'} \Vdash \Gamma$ и $\mathfrak{A}_{\Gamma'} \nVdash \tildeforall \Phi$. Стало быть, $\Gamma \nvDash \tildeforall \Phi$.
                \end{itemize}
        \end{itemize}
    \end{proof}

    \begin{remark*}
        Для удобства обозначим
        \[\Term^\circ_\sigma := \{t \in \Term_\sigma \mid \sub(t) \cap \Var = \varnothing\}.\]
        Элементы $\Term^\circ_\sigma$ называются \emph{замкнутыми $\sigma$-термами}. В определении ``насыщенности'' в логике первого порядка используется естественный кванторный аналог дизъюнктивного свойства:
        \begin{quotation}
            Для любого $\exists x\ \Phi \in \Gamma$ существует $t \in \Term^\circ_\sigma$ такой, что $\Phi(x/t) \in \Gamma$.
        \end{quotation}
        Однако реализовать данное свойство в исходной сигнатуре $\sigma$ порой невозможно. Так, если $\Const_\sigma = \varnothing$, то $\Term^\circ_\sigma = \varnothing$, а потому ``насыщенных $\sigma$-теорий'' вообще не существует. Значит, придётся обогащать $\sigma$, добавляя новые константы.
    \end{remark*}

    \begin{remark}
        Под \emph{мощностью $\mathfrak{A}$} традиционно понимают мощность носителя $\mathfrak{A}$, т.е. $|A|$. Далее мы убедимся, что теорема о сильной полноте $\vdash$ остаётся верной, если ограничиться рассмотрением $\sigma$-структур мощности $\leqslant |\Formul_\sigma|$.
    \end{remark}

    \begin{definition}
        Изначально $\vdash$ определено для фиксированной структуры $\sigma$. По этой причине правильнее говорить о \emph{выводимости над $\sigma$}, а не просто о выводимости (без указания сигнатуры), и писать $\vdash_\sigma$ вместо $\vdash$.
    \end{definition}

    \begin{theorem}[о консервативности]
        Пусть $\sigma \subseteq \sigma'$. Тогда для любых $\Gamma \subseteq \Sent_\sigma$ и $\Phi \in \Formul_\sigma$
        \[
            \Gamma \vdash_\sigma \Phi
            \quad \Longleftrightarrow \quad
            \Gamma \vdash_{\sigma'} \Phi.
        \]
    \end{theorem}

    \begin{proof}
        Как легко убедиться, $\Gamma \vDash_\sigma \Phi$ равносильно $\Gamma \vDash_{\sigma'} \Phi$, где $\vDash_\sigma$ и $\vDash_{\sigma'}$ --- это семантическое следование над $\sigma$ и $\sigma'$ соответственно. Значит
        \[
            \Gamma \vdash_\sigma \Phi
            \quad \Longleftrightarrow \quad
            \Gamma \vDash_\sigma \Phi
            \quad \Longleftrightarrow \quad
            \Gamma \vDash_{\sigma'} \Phi
            \quad \Longleftrightarrow \quad
            \Gamma \vdash_{\sigma'} \Phi
        \]
        (ввиду теоремы и сильной полноте для $\vdash_\sigma$ и $\vdash_{\sigma'}$).
    \end{proof}

    \begin{remark}
        Тут хватит и слабой полноты, так как $\vdash$ компактно и монотонно.
    \end{remark}

    \begin{theorem}[о слабой полноте $\vdash$]
        Для любой $\Phi \in \Formul_\sigma$
        \[
            \vdash \Phi
            \quad \Longleftrightarrow \quad
            \vDash \Phi,
        \]
        т.е. выводимость из $\varnothing$ равносильна общезначимости.
    \end{theorem}

    \begin{theorem}[о компактности $\vDash$, aka локальная теорема Гёделя-Мальцева]
        Для любых $\Gamma \subseteq \Sent_\sigma$ и $\Phi \in \Formul_\sigma$
        \[
            \Gamma \vDash \Phi
            \quad \Longleftrightarrow \quad
            \Delta \vDash \Phi \text{ для некоторого конечного } \Delta \subseteq \Gamma.
        \]
    \end{theorem}

    \begin{corollary}
        Для любого $\Gamma \subseteq \Sent_\sigma$
        \[
            \Gamma \nvDash \bot
            \quad \Longleftrightarrow \quad
            \Delta \nvDash \bot \text{ для всех конечных } \Delta \subseteq \Gamma.
        \]
        Иначе говоря, $\Gamma$ выполнимо тогда и только тогда, когда всякое конечное подмножество $\Gamma$ выполнимо.
    \end{corollary}

    \begin{remark}
        Всякое $\Gamma \subseteq \Sent_\sigma$ называют \emph{локально выполнимым}, если всякое конечное подмножество $\Gamma$ выполнимо. Стало быть
        \[
            \Gamma \text{ выполнимо}
            \quad \Longleftrightarrow \quad
            \Gamma \text{ локально выполнимо};
        \]
        отсюда ``локальная'' в альтернативном назывании. К слову, локальная выполнимость влечёт выполнимость влечёт непротиворечивость по теореме о корректности.
    \end{remark}

    \begin{remark*}
        С помощью теоремы о компактности для $\vDash$ можно получить немало интересных результатов. Например:
    \end{remark*}

    \begin{statement}
        Пусть у $\Gamma$ есть модели сколь угодно большой конечной мощности. Тогда у $\Gamma$ есть бесконечная модель.
    \end{statement}

    \begin{proof}
        Мы будем считать, что $\Pred_\sigma$ содержит $=$; если его нет, можно воспользоваться неразличимостью некоторых элементов, после чего построить бесконечную модель, но тут нужно глубже вдаваться в подробности. Для каждого $n \in \NN \setminus \{0, 1\}$ положим
        \[\Phi_n := \exists x_1\ \dots \exists x_n\ \bigwedge_{i=1}^{n-1} \bigwedge_{j=i+1}^n \neg x_i = x_j.\]
        Очевидно, для любой $\sigma$-структуры $\mathfrak{A}$,
        \[
            \mathfrak{A} \Vdash \Phi_n
            \quad \Longleftrightarrow \quad
            |A| \geqslant n.
        \]
        Пусть $\Gamma \subseteq \Sent_\sigma$ удовлетворяет условию утверждения. Рассмотрим
        \[\Gamma' := \Gamma \cup \{\Phi_n\}_{n \in \NN \setminus \{0, 1\}}.\]
        Разумеется, $\Gamma'$ локально выполнимо. Значит оно выполнимо, т.е. у $\Gamma'$ есть модель $\mathfrak{A}$. Поскольку в $\mathfrak{A}$ выполнены все $\Phi_n$, то $A$ бесконечно.
    \end{proof}

    \begin{definition}
        Пусть $\mathfrak{A}$ и $\mathfrak{B}$ --- $\sigma$-структуры. Говорят, что $\mathfrak{A}$ является \emph{подструктурой} $\mathfrak{B}$, а $\mathfrak{B}$ --- \emph{расширением} $\mathfrak{A}$, и пишут $\mathfrak{A} \leqslant \mathfrak{B}$, если $A \subseteq B$ и $\id_A$ является вложением $\mathfrak{A}$ в $\mathfrak{B}$. Итак, $\mathfrak{A} \leqslant \mathfrak{B}$ тогда и только тогда, когда:
        \begin{itemize}
            \item $c^\mathfrak{A} = c^\mathfrak{B}$ для каждого $c \in \Const_\sigma$;
            \item $f^\mathfrak{A} = f^\mathfrak{B} \upharpoonright_{A^m}$ для каждого $m$-местного $f \in \Func_\sigma$;
            \item $P^\mathfrak{A} = P^\mathfrak{B} \cap A^n$ для каждого $n$-местного $P \in \Pred_\sigma$.
        \end{itemize}
        Понятно, что $S \subseteq B$ является носителем (некоторой) подструктуры $\mathfrak{B}$, если и только если:
        \begin{itemize}
            \item $c^\mathfrak{B} \in S$ для каждого $c \in \Const_\sigma$;
            \item $f^\mathfrak{B}[S^m] \subseteq S$ для каждого $m$-местного $f \in \Func_\sigma$.
        \end{itemize}
    \end{definition}

    \begin{remark*}
        В случае, когда $S \subseteq B$ удовлетворяет описанным выше условиям, соответствующая подструктура определяется однозначно; поэтому подструктуры нередко отождествляют с их носителями при условии, что объемлющая $\mathfrak{B}$ фиксирована.
    \end{remark*}

    \begin{example}
        Пусть $\mathfrak{A}$ --- ЧУМ. Тогда все подструктуры $\mathfrak{A}$ также есть ЧУМ. Кроме того, такие свойства как линейность и фундированность будет наследоваться при переходе к подструктурам.
    \end{example}

    \begin{example}
        Пусть $\mathfrak{A}$ --- абелева группа в сигнатуре $\langle =; +^2, -^1; 0 \rangle$. Тогда подструктуры $\mathfrak{A}$ суть в точности подгруппы $\mathfrak{A}$. Без $-$ в сигнатуре, однако, это было бы неверно, поскольку могут отсутствовать обратные.
    \end{example}

    \begin{definition}
        Говорят, что $\mathfrak{A}$ является \emph{элементарной подструктурой} $\mathfrak{B}$, а $\mathfrak{B}$ --- \emph{элементарным расширением} $\mathfrak{A}$, и пишут $\mathfrak{A} \preccurlyeq \mathfrak{B}$, если $\mathfrak{A} \leqslant \mathfrak{B}$ и для любых $\Phi(x_1, \dots, x_n) \in \Formul_\sigma$ и $(a_1, \dots, a_k) \in A^k$,
        \[
            \mathfrak{A} \Vdash \Phi[\vec{x}/\vec{a}]
            \quad \Longleftrightarrow \quad
            \mathfrak{B} \Vdash \Phi[\vec{x}/\vec{a}]
        \]
        Очевидно, $\mathfrak{A} \preccurlyeq \mathfrak{B}$ влечёт $\Th(\mathfrak{A}) = \Th(\mathfrak{B})$.
    \end{definition}

    \begin{theorem}[Лёвенгейма-Сколема, о понижении мощности]
        У всякой $\sigma$-структуры есть элементарная подструктура мощности $\leqslant |\Formul_\sigma|$.
    \end{theorem}

    \begin{proof}
        Возьмём $\kappa := |\Formul_\sigma|$. Не ограничивая общности, мы будем считать, что $\Pred_\sigma$ содержит $=$.

        Пусть $\mathfrak{A}$ --- произвольная $\sigma$-структура. Для любых $\Phi(x_1, \dots, x_k, y) \in \Formul_\sigma$ и $(a_1, \dots, a_k) \in A^k$ положим
        \[
            \llbracket \Phi(\vec{a}, y) \rrbracket := \{a \in A \mid \mathfrak{A} \Vdash \Phi[\vec{x}/\vec{a}, y/a]\}
        \]
        Определим последовательность $\langle S_n : n \in \NN \rangle$ подмножеств $A$ следующим образом.
        \begin{itemize}
            \item Если $n = 0$, то $S_n$ --- некоторое фиксированное подмножество $A$ мощности $\leqslant \kappa$.
            \item Если $n = m + 1$, то $S_n := S_m \cup \range \eta_m$, где $\eta_m$ --- какая-нибудь функция выбора для
                \[\{\llbracket \Phi(\vec{a}, y) \rrbracket \mid \vec{a} \in S^*_m \wedge \mathfrak{A} \Vdash \exists y\ \Phi[\vec{x}/\vec{a}]\}\]
        \end{itemize}
        По построению мы имеем $S_0 \subseteq S_1 \subseteq S_2 \subseteq \dots$. Возьмём
        \[S := \bigcup_{n \in \NN} S_n.\]
        Понятно, что для любых $\Phi(x_1, \dots, x_k, y) \in \Formul_\sigma$ и $(a_1, \dots, a_k) \in S^k$
        \begin{equation}
            \mathfrak{A} \Vdash \exists y\ \Phi[\vec{x}/\vec{a}]
            \quad \Longrightarrow \quad
            \mathfrak{A} \Vdash \Phi[\vec{x}/\vec{a}, y/a] \text{ для некоторого } a \in S.
            \label{Löwenheim–Skolem-theorem-eqution}
        \end{equation}
        Далее, $S$ является носителем некоторой подструктуры $\mathfrak{A}$:
        \begin{itemize}
            \item для всякого $c \in \Const_\sigma$ верно $\mathfrak{A} \Vdash \exists y\ c = y$, откуда $c^\mathfrak{A} \in S$;
            \item для всякого $m$-местного $f \in \Func_\sigma$ и любого $\vec{a} \in S^m$ верно $\mathfrak{A} \Vdash \exists y\ f(\vec{x}) = y [\vec{x}/\vec{a}]$, откуда $f^\mathfrak{A}(\vec{a}) \in S$.
        \end{itemize}
        
        Обозначим через $\mathfrak{G}$ подструктуру $\mathfrak{A}$ с носителем $S$. Заметим, что
        \[
            |S_n| \leqslant \kappa \text{ для всех } n \in \NN.
        \]
        Это легко установить по индукции:
        \begin{itemize}
            \item очевидно, $|S_0| \leqslant \kappa$;
            \item если $|S_n| \leqslant \kappa$, то $|S_{n+1}| \leqslant |S_n| + |\Formul_\sigma| \cdot |S^*_n| \leqslant \kappa + \kappa \cdot \kappa = \kappa$.
        \end{itemize}
        Отсюда $|S| = |\bigcup_{n \in \NN} S_n| \leqslant \aleph_0 \cdot \kappa = \kappa$.

        Наконец, используя \ref{Löwenheim–Skolem-theorem-eqution}, нетрудно показать, что индукцией по построению $\Phi(x_1, \dots, x_k) \in \Formul_\sigma$, что для любых $(a_1, \dots, a_k) \in S^k$
        \[
            \mathfrak{G} \Vdash \Phi[\vec{x}/\vec{a}]
            \quad \Longleftrightarrow \quad
            \mathfrak{A} \Vdash \Phi[\vec{x}/\vec{a}].
        \]
        Таким образом $\mathfrak{G} \preccurlyeq \mathfrak{A}$.
    \end{proof}

    \begin{corollary}
        Для всякого $\Gamma \subseteq \Sent_\sigma$
        \[
            \text{у $\Gamma$ есть модель}
            \quad \Longleftrightarrow \quad
            \text{у $\Gamma$ есть модель мощности не более чем $|\Formul_\sigma|$.}
        \]
    \end{corollary}

    \begin{exercise}
        \[
            |\Formul_\sigma|
            = \max \{|\Pred_\sigma|, |\Func_\sigma|, |\Const_\sigma|, \aleph_0\}
            = \max \{|\sigma|, \aleph_0\}
        \]
    \end{exercise}


\end{document}