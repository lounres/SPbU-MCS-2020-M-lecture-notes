\documentclass[12pt,a4paper]{article}
\usepackage{solutions}
\usepackage{mathrsfs}
\usepackage{multicol}
\usepackage{bussproofs}

\title{Основы математической логики.\\ Практика. 1 курс.\\Решения.}
\author{Глеб Минаев @ 102 (20.Б02-мкн)}
% \date{}

% \newcommand{\subsets}{\ensuremath{\mathcal{P}}\xspace}
% \newcommand{\finsubsets}{\ensuremath{\mathcal{P_{\mathrm{fin}}}}\xspace}
% \DeclareMathOperator{\Ind}{Ind}
\DeclareMathOperator{\dom}{dom}
% \DeclareMathOperator{\range}{range}
% \DeclareMathOperator{\Min}{Min}
% \DeclareMathOperator{\Seq}{Seq}
% \DeclareMathOperator{\Left}{left}
% \DeclareMathOperator{\Right}{right}
% \DeclareMathOperator{\IS}{IS}
% \DeclareMathOperator{\ord}{ord}
% \DeclareMathOperator{\Ord}{Ord}
% \DeclareMathOperator{\card}{card}
% \DeclareMathOperator{\Card}{Card}
% \newcommand{\ZF}{\ensuremath{\mathrm{ZF}}\xspace}
% \newcommand{\ZFC}{\ensuremath{\mathrm{ZFC}}\xspace}
% \newcommand{\CH}{\ensuremath{\mathrm{CH}}\xspace}
\newcommand{\Prop}{\ensuremath{\mathrm{Prop}}\xspace}
\newcommand{\Formul}{\ensuremath{\mathrm{Form}}\xspace}
\newcommand{\Sub}{\ensuremath{\mathrm{Sub}}\xspace}
\newcommand{\NP}{\ensuremath{\mathrm{NP}}\xspace}

\newcommand{\reset}{\setcounter{enumprb}{0}}

\begin{document}
    \maketitle

    % \begin{multicols}{2}
    %     \tableofcontents
    % \end{multicols}

    \section{Формальные слова}

    \begin{enumproblem}
        Очевидное следствие следующего упражнения.
    \end{enumproblem}

    \begin{enumproblem}
        \begin{lemma}
            Пусть даны $\{\varphi; \psi\} \subseteq \Formul$ и $\chi \in \mathscr{L}^*$, что $\varphi$ и $\psi$ входят в $\chi$, и некоторые их вхождения в $\chi$ пересекаются и в объединении дают всё $\chi$. Тогда $\varphi \preccurlyeq \psi$ или $\psi \preccurlyeq \varphi$.
        \end{lemma}

        \begin{proof}
            WLOG вхождение $\varphi$ в $\chi$, рассмотренное в условии, имеет начало $0$.

            Если рассмотренное в условии вхождение $\psi$ в $\chi$ тоже имеет начало $0$, то тогда понятно, что они совпадают (теорема из лекции). Тогда предположим противное.

            Если же вхождение $\psi$ заканчивается не позже чем вхождение $\varphi$, то лемма сиюминутно доказана. Поэтому покажем, что оставшегося произойти не могло.
            
            Докажем лемму по индукции по $|\varphi|$.

            \textbf{База.} $|\varphi| = 1$. Очевидно, что тогда из-за пересекаемости $\psi$ придётся включить в себя $\varphi$.

            \textbf{Шаг.} Рассмотрим тип $\varphi$.
            \begin{enumerate}
                \item Если $\varphi \in \Prop$, то см. базу.
                \item Если $\varphi = \neg \varphi'$, то понятно, что условие леммы применимо к $\varphi'$ и $\psi$.
                    \begin{itemize}
                        \item Если $\psi \preccurlyeq \varphi'$, то $\psi \preccurlyeq \varphi$.
                        \item Если $\varphi' \preccurlyeq \psi$, то $\varphi' = \psi$, следовательно $\psi \preccurlyeq \varphi$.
                    \end{itemize}
                \item Если $\varphi = (\tau \circ \sigma)$, где $\{\tau; \sigma\} \subseteq \Formul$, а $\circ \in \{\wedge; \vee; \rightarrow\}$. Понятно, что $\psi$ будет содержать последний символ $\varphi$ --- ``)''. Тогда несложно видеть из баланса скобок, что $\psi$ придётся включить в себя всё $\varphi$.
            \end{enumerate}
        \end{proof}
    \end{enumproblem}

    \section{PCL: естественная дедукция I}\reset

    \begin{enumproblem}\ 
        \begin{description}
            \item[$\mathrm{I1}$.]\ 
                \begin{prooftree}
                    \AxiomC{$[\varphi]^1$}
                        \RightLabel{(2)}
                        \UnaryInfC{$\psi \rightarrow \varphi$}
                            \RightLabel{(1)}
                            \UnaryInfC{$\varphi \rightarrow (\psi \rightarrow \varphi)$}
                \end{prooftree}

            \item[$\mathrm{I2}$.]\ 
                \begin{prooftree}
                    \AxiomC{$[\varphi]^3$}
                    \AxiomC{$[\varphi \rightarrow \psi]^2$}
                        \BinaryInfC{$\psi$}
                    \AxiomC{$[\varphi]^3$}
                    \AxiomC{$[\varphi \rightarrow (\psi \rightarrow \chi)]^1$}
                        \BinaryInfC{$\psi \rightarrow \chi$}
                            \BinaryInfC{$\chi$}
                                \RightLabel{(3)}
                                \UnaryInfC{$\varphi \rightarrow \chi$}
                                    \RightLabel{(2)}
                                    \UnaryInfC{$(\varphi \rightarrow \psi) \rightarrow (\varphi \rightarrow \chi)$}
                                        \RightLabel{(1)}
                                        \UnaryInfC{$(\varphi \rightarrow (\psi \rightarrow \chi)) \rightarrow ((\varphi \rightarrow \psi) \rightarrow (\varphi \rightarrow \chi))$}
                \end{prooftree}

            \item[$\mathrm{C1}$.]\ 
                \begin{prooftree}
                    \AxiomC{$\varphi \wedge \psi$}
                        \UnaryInfC{$\varphi$}
                \end{prooftree}
                
            \item[$\mathrm{C2}$.]\ 
                \begin{prooftree}
                    \AxiomC{$\varphi \wedge \psi$}
                        \UnaryInfC{$\psi$}
                \end{prooftree}
                
            \item[$\mathrm{C3}$.]\ 
                \begin{prooftree}
                    \AxiomC{$[\varphi]^1$}
                    \AxiomC{$[\psi]^2$}
                        \BinaryInfC{$\varphi \wedge \psi$}
                            \RightLabel{(2)}
                            \UnaryInfC{$\psi \rightarrow \varphi \wedge \psi$}
                                \RightLabel{(1)}
                                \UnaryInfC{$\varphi \rightarrow (\psi \rightarrow \varphi \wedge \psi)$}
                \end{prooftree}

            \item[$\mathrm{D1}$.]\ 
                \begin{prooftree}
                    \AxiomC{$\varphi$}
                        \UnaryInfC{$\varphi \vee \psi$}
                \end{prooftree}
                
            \item[$\mathrm{D2}$.]\ 
                \begin{prooftree}
                    \AxiomC{$\psi$}
                        \UnaryInfC{$\varphi \vee \psi$}
                \end{prooftree}
                
            \item[$\mathrm{D3}$.]\ 
                \begin{prooftree}
                    \AxiomC{$[\varphi \vee \psi]^3$}
                    \AxiomC{$[\varphi]^4$}
                    \AxiomC{$[\varphi \rightarrow \chi]^1$}
                        \BinaryInfC{$\chi$}
                    \AxiomC{$[\psi]^4$}
                    \AxiomC{$[\psi \rightarrow \chi]^2$}
                        \BinaryInfC{$\chi$}
                            \RightLabel{(4)}
                            \TrinaryInfC{$\chi$}
                                \RightLabel{(3)}
                                \UnaryInfC{$\varphi \vee \psi \rightarrow \chi$}
                                    \RightLabel{(2)}
                                    \UnaryInfC{$(\psi \rightarrow \chi) \rightarrow (\varphi \vee \psi \rightarrow \chi)$}
                                        \RightLabel{(1)}
                                        \UnaryInfC{$(\varphi \rightarrow \chi) \rightarrow ((\psi \rightarrow \chi) \rightarrow (\varphi \vee \psi \rightarrow \chi))$}
                \end{prooftree}
                
            \item[$\mathrm{N1}$.]\ 
                \begin{prooftree}
                    \AxiomC{$[\varphi]^3$}
                    \AxiomC{$[\varphi \rightarrow \psi]^1$}
                        \BinaryInfC{$\psi$}
                    \AxiomC{$[\varphi]^3$}
                    \AxiomC{$[\varphi \rightarrow \neg \psi]^2$}
                        \BinaryInfC{$\neg \psi$}
                            \RightLabel{(3)}
                            \BinaryInfC{$\neg \varphi$}
                                \RightLabel{(2)}
                                \UnaryInfC{$(\varphi \rightarrow \neg \psi) \rightarrow \neg \varphi$}
                                    \RightLabel{(1)}
                                    \UnaryInfC{$(\varphi \rightarrow \psi) \rightarrow ((\varphi \rightarrow \neg \psi) \rightarrow \neg \varphi)$}
                \end{prooftree}
                
            \item[$\mathrm{N2}$.]\ 
                \begin{prooftree}
                    \AxiomC{$[\varphi]^2$}
                    \AxiomC{$[\neg \varphi]^1$}
                        \RightLabel{(3)}
                        \BinaryInfC{$\psi$}
                            \RightLabel{(2)}
                            \UnaryInfC{$\varphi \rightarrow \psi$}
                                \RightLabel{(1)}
                                \UnaryInfC{$\neg \varphi \rightarrow (\varphi \rightarrow \psi)$}
                \end{prooftree}
                
            \item[$\mathrm{N3}$.]\ 
                \begin{prooftree}
                    \AxiomC{$[\neg (\varphi \vee \neg \varphi)]^1$}
                    \AxiomC{$[\neg \varphi]^3$}
                        \UnaryInfC{$\varphi \vee \neg \varphi$}
                            \RightLabel{(3)}
                            \BinaryInfC{$\varphi$}
                    \AxiomC{$[\neg (\varphi \vee \neg \varphi)]^1$}
                    \AxiomC{$[\varphi]^2$}
                        \UnaryInfC{$\varphi \vee \neg \varphi$}
                            \RightLabel{(2)}
                            \BinaryInfC{$\neg \varphi$}
                                \RightLabel{(1)}
                                \BinaryInfC{$\varphi \vee \neg \varphi$}
                \end{prooftree}
        \end{description}
    \end{enumproblem}

    \begin{enumproblem}\ 
        \begin{description}
            \item[$\mathrm{IT}$.]\ 
                \begin{prooftree}
                    \AxiomC{$[\varphi]^3$}
                    \AxiomC{$[\varphi \rightarrow \psi]^1$}
                        \BinaryInfC{$\psi$}
                    \AxiomC{$[\psi \rightarrow \chi]^2$}
                            \BinaryInfC{$\chi$}
                                \RightLabel{(3)}
                                \UnaryInfC{$\varphi \rightarrow \chi$}
                                    \RightLabel{(2)}
                                    \UnaryInfC{$(\psi \rightarrow \chi) \rightarrow (\varphi \rightarrow \chi)$}
                                        \RightLabel{(1)}
                                        \UnaryInfC{$(\varphi \rightarrow \psi) \rightarrow ((\psi \rightarrow \chi) \rightarrow (\varphi \rightarrow \chi))$}
                \end{prooftree}
                
            \item[$\mathrm{PE}$.]\ 
                \begin{prooftree}
                    \AxiomC{$[\psi]^2$}
                    \AxiomC{$[\varphi]^3$}
                    \AxiomC{$[\varphi \rightarrow (\psi \rightarrow \chi)]^1$}
                        \BinaryInfC{$\psi \rightarrow \chi$}
                            \BinaryInfC{$\chi$}
                                \RightLabel{(3)}
                                \UnaryInfC{$\varphi \rightarrow \chi$}
                                    \RightLabel{(2)}
                                    \UnaryInfC{$\psi \rightarrow (\varphi \rightarrow \chi)$}
                                        \RightLabel{(1)}
                                        \UnaryInfC{$(\varphi \rightarrow (\psi \rightarrow \chi)) \rightarrow (\psi \rightarrow (\varphi \rightarrow \chi))$}
                \end{prooftree}
        \end{description}
    \end{enumproblem}

    \begin{enumproblem}\ 
        \begin{description}
            \item[$\mathrm{2N}$.]\ 
                \begin{prooftree}
                    \AxiomC{$[\varphi]^1$}
                    \AxiomC{$[\neg \varphi]^2$}
                        \RightLabel{(2)}
                        \BinaryInfC{$\neg \neg \varphi$}
                            \RightLabel{(1)}
                            \UnaryInfC{$\varphi \rightarrow \neg \neg \varphi$}
                \end{prooftree}

            \item[$\mathrm{Co}$.]\ 
                \begin{prooftree}
                    \AxiomC{$[\varphi]^3$}
                    \AxiomC{$[\varphi \rightarrow \psi]^1$}
                        \BinaryInfC{$\psi$}
                    \AxiomC{$[\neg \psi]^2$}
                            \RightLabel{(3)}
                            \BinaryInfC{$\neg \varphi$}
                                \RightLabel{(2)}
                                \UnaryInfC{$\neg \psi \rightarrow \neg \varphi$}
                                    \RightLabel{(1)}
                                    \UnaryInfC{$(\varphi \rightarrow \psi) \rightarrow (\neg \psi \rightarrow \neg \varphi)$}
                \end{prooftree}
        \end{description}
    \end{enumproblem}

    \begin{enumproblem}\ 
        \begin{description}
            \item[\underline{$\mathrm{2N}$}.]\ 
                \begin{prooftree}
                    \AxiomC{$[\neg \varphi]^2$}
                    \AxiomC{$[\neg \neg \varphi]^1$}
                        \RightLabel{(2)}
                        \BinaryInfC{$\varphi$}
                            \RightLabel{(1)}
                            \UnaryInfC{$\neg \neg \varphi \rightarrow \varphi$}
                \end{prooftree}

            \item[\underline{$\mathrm{Co}$}.]\ 
                \begin{prooftree}
                    \AxiomC{$[\neg \psi]^3$}
                    \AxiomC{$[\neg \psi \rightarrow \neg \varphi]^1$}
                        \BinaryInfC{$\neg \varphi$}
                    \AxiomC{$[\varphi]^2$}
                            \RightLabel{(3)}
                            \BinaryInfC{$\psi$}
                                \RightLabel{(2)}
                                \UnaryInfC{$\varphi \rightarrow \psi$}
                                    \RightLabel{(1)}
                                    \UnaryInfC{$(\neg \psi \rightarrow \neg \varphi) \rightarrow (\varphi \rightarrow \psi)$}
                \end{prooftree}

            \item[$\mathrm{NE}$.]\ 
                \begin{prooftree}
                    \AxiomC{$[\neg \varphi]^3$}
                    \AxiomC{$[\neg \varphi \rightarrow \psi]^1$}
                        \BinaryInfC{$\psi$}
                    \AxiomC{$[\neg \varphi]^3$}
                    \AxiomC{$[\neg \varphi \rightarrow \neg \psi]^2$}
                        \BinaryInfC{$\neg \psi$}
                            \RightLabel{(3)}
                            \BinaryInfC{$\varphi$}
                                \RightLabel{(2)}
                                \UnaryInfC{$(\neg \varphi \rightarrow \neg \psi) \rightarrow \varphi$}
                                    \RightLabel{(1)}
                                    \UnaryInfC{$(\neg \varphi \rightarrow \psi) \rightarrow ((\neg \varphi \rightarrow \neg \psi) \rightarrow \varphi)$}
                \end{prooftree}
        \end{description}
    \end{enumproblem}

    \section{PCL: естественная дедукция II}\reset

    \begin{enumproblem}\ItemedProblem\ 
        \begin{enumerate}
            \item Итог: $4/4$. Без $\neg\mathrm{Elim}$.
                \begin{prooftree}
                    \AxiomC{$[p \vee q]^1$}
                    \AxiomC{$[p]^2$}
                        \UnaryInfC{$q \vee p$}
                    \AxiomC{$[q]^2$}
                        \UnaryInfC{$q \vee p$}
                            \RightLabel{(2)}
                            \TrinaryInfC{$q \vee p$}
                                \RightLabel{(1)}
                                \UnaryInfC{$(p \vee q) \rightarrow (q \vee p)$}
                \end{prooftree}
            
            \item Итог: $8/8$. Без $\neg\mathrm{Elim}$.
                \begin{prooftree}
                    \AxiomC{$[(p \vee q) \vee r]^1$}
                    \AxiomC{$[p \vee q]^2$}
                    \AxiomC{$[p]^3$}
                        \UnaryInfC{$p \vee (q \vee r)$}
                    \AxiomC{$[q]^3$}
                        \UnaryInfC{$q \vee r$}
                            \UnaryInfC{$p \vee (q \vee r)$}
                                \RightLabel{(3)}
                                \TrinaryInfC{$p \vee (q \vee r)$}
                    \AxiomC{$[r]^2$}
                        \UnaryInfC{$q \vee r$}
                            \UnaryInfC{$p \vee (q \vee r)$}
                                    \RightLabel{(2)}
                                    \TrinaryInfC{$p \vee (q \vee r)$}
                                        \RightLabel{(1)}
                                        \UnaryInfC{$((p \vee q) \vee r) \rightarrow (p \vee (q \vee r))$}
                \end{prooftree}
        \end{enumerate}
    \end{enumproblem}

    \begin{enumproblem}\ItemedProblem\ 
        \begin{enumerate}
            \item Итог: $9/9$. Без $\neg\mathrm{Elim}$.
                \begin{prooftree}
                    \AxiomC{$[p \wedge (q \vee r)]^1$}
                        \UnaryInfC{$q \vee r$}
                    \AxiomC{$[p \wedge (q \vee r)]^1$}
                        \UnaryInfC{$p$}
                    \AxiomC{$[q]^2$}
                            \BinaryInfC{$p \wedge q$}
                                \UnaryInfC{$(p \wedge q) \vee (p \wedge r)$}
                    \AxiomC{$[p \wedge (q \vee r)]^1$}
                        \UnaryInfC{$p$}
                    \AxiomC{$[r]^2$}
                            \BinaryInfC{$p \wedge r$}
                                \UnaryInfC{$(p \wedge q) \vee (p \wedge r)$}
                                    \RightLabel{(2)}
                                    \TrinaryInfC{$(p \wedge q) \vee (p \wedge r)$}
                                        \RightLabel{(1)}
                                        \UnaryInfC{$(p \wedge (q \vee r)) \rightarrow ((p \wedge q) \vee (p \wedge r))$}
                \end{prooftree}
            
            \item Итог: $10/10$. Без $\neg\mathrm{Elim}$.
                \begin{prooftree}
                    \AxiomC{$[(p \wedge q) \vee (p \wedge r)]^1$}
                    \AxiomC{$[p \wedge q]^2$}
                        \UnaryInfC{$p$}
                    \AxiomC{$[p \wedge r]^2$}
                        \UnaryInfC{$p$}
                            \RightLabel{(2)}
                            \TrinaryInfC{$p$}
                    \AxiomC{$[(p \wedge q) \vee (p \wedge r)]^1$}
                    \AxiomC{$[p \wedge q]^3$}
                        \UnaryInfC{$q$}
                            \UnaryInfC{$q \vee r$}
                    \AxiomC{$[p \wedge r]^3$}
                        \UnaryInfC{$r$}
                            \UnaryInfC{$q \vee r$}
                                \RightLabel{(3)}
                                \TrinaryInfC{$p$}
                                    \BinaryInfC{$p \wedge (q \vee r)$}
                                        \RightLabel{(1)}
                                        \UnaryInfC{$((p \wedge q) \vee (p \wedge r)) \rightarrow (p \wedge (q \vee r))$}
                \end{prooftree}
            
            \item Итог: $10/10$. Без $\neg\mathrm{Elim}$.
                \begin{prooftree}
                    \AxiomC{$[p \vee (q \wedge r)]^1$}
                    \AxiomC{$[p]^2$}
                        \UnaryInfC{$p \vee q$}
                    \AxiomC{$[q \wedge r]^2$}
                        \UnaryInfC{$q$}
                            \UnaryInfC{$p \vee q$}
                                \RightLabel{(2)}
                                \TrinaryInfC{$p \vee q$}
                    \AxiomC{$[p \vee (q \wedge r)]^1$}
                    \AxiomC{$[p]^3$}
                        \UnaryInfC{$p \vee r$}
                    \AxiomC{$[q \wedge r]^3$}
                        \UnaryInfC{$r$}
                            \UnaryInfC{$p \vee r$}
                                \RightLabel{(3)}
                                \TrinaryInfC{$p \vee r$}
                                    \BinaryInfC{$(p \vee q) \wedge (p \vee r)$}
                                        \RightLabel{(1)}
                                        \UnaryInfC{$(p \vee (q \wedge r)) \rightarrow ((p \vee q) \wedge (p \vee r))$}
                \end{prooftree}
            
            \item Итог: $9/9$. Без $\neg\mathrm{Elim}$.
                \begin{prooftree}
                    \AxiomC{$[(p \vee q) \wedge (p \vee r)]^1$}
                        \UnaryInfC{$p \vee q$}
                    \AxiomC{$[p]^2$}
                        \UnaryInfC{$p \vee (q \wedge r)$}
                    \AxiomC{$[(p \vee q) \wedge (p \vee r)]^1$}
                        \UnaryInfC{$p \vee r$}
                    \AxiomC{$[p]^3$}
                        \UnaryInfC{$p \vee (q \wedge r)$}
                    \AxiomC{$[q]^2$}
                    \AxiomC{$[r]^3$}
                        \BinaryInfC{$q \wedge r$}
                            \UnaryInfC{$p \vee (q \wedge r)$}
                                \RightLabel{(3)}
                                \TrinaryInfC{$p \vee (q \wedge r)$}
                                    \RightLabel{(2)}
                                    \TrinaryInfC{$p \vee (q \wedge r)$}
                                        \RightLabel{(1)}
                                        \UnaryInfC{$((p \vee q) \wedge (p \vee r)) \rightarrow (p \vee (q \wedge r))$}
                \end{prooftree}
        \end{enumerate}
    \end{enumproblem}

    \begin{enumproblem}\ItemedProblem\ 
        \begin{enumerate}
            \item Итог: $6/6$. Без $\neg\mathrm{Elim}$.
                \begin{prooftree}
                    \AxiomC{$[p]^2$}
                        \UnaryInfC{$p \vee q$}
                    \AxiomC{$[\neg (p \vee q)]^1$}
                            \RightLabel{(2)}
                            \BinaryInfC{$\neg p$}
                    \AxiomC{$[q]^3$}
                        \UnaryInfC{$p \vee q$}
                    \AxiomC{$[\neg (p \vee q)]^1$}
                            \RightLabel{(3)}
                            \BinaryInfC{$\neg q$}
                                \BinaryInfC{$\neg p \wedge \neg q$}
                                    \RightLabel{(1)}
                                    \UnaryInfC{$\neg (p \vee q) \rightarrow (\neg p \wedge \neg q)$}
                \end{prooftree}
            
            \item Итог: $7/7$. Без $\neg\mathrm{Elim}$.
                \begin{prooftree}
                    \AxiomC{$[\neg p \wedge \neg q]^1$}
                    \AxiomC{$[p \vee q]^2$}
                    \AxiomC{$[p]^3$}
                    \AxiomC{$[\neg p \wedge \neg q]^4$}
                        \UnaryInfC{$\neg p$}
                            \RightLabel{(4)}
                            \BinaryInfC{$\neg (\neg p \wedge \neg q)$}
                    \AxiomC{$[q]^3$}
                    \AxiomC{$[\neg p \wedge \neg q]^5$}
                        \UnaryInfC{$\neg q$}
                            \RightLabel{(5)}
                            \BinaryInfC{$\neg (\neg p \wedge \neg q)$}
                                \RightLabel{(3)}
                                \TrinaryInfC{$\neg (\neg p \wedge \neg q)$}
                                    \RightLabel{(2)}
                                    \BinaryInfC{$\neg (p \vee q)$}
                                        \RightLabel{(1)}
                                        \UnaryInfC{$(\neg p \wedge \neg q) \rightarrow \neg (p \vee q)$}
                \end{prooftree}
            
            \item Итог: $7/7$. С $\neg\mathrm{Elim}$.
                \begin{prooftree}
                    \AxiomC{$[\neg (p \wedge q)]^1$}
                    \AxiomC{$[\neg p]^3$}
                        \UnaryInfC{$\neg p \vee \neg q$}
                    \AxiomC{$[\neg (\neg p \vee \neg q)]^2$}
                            \RightLabel{(3)}
                            \BinaryInfC{$p$}
                    \AxiomC{$[\neg q]^4$}
                        \UnaryInfC{$\neg p \vee \neg q$}
                    \AxiomC{$[\neg (\neg p \vee \neg q)]^2$}
                            \RightLabel{(4)}
                            \BinaryInfC{$q$}
                                \BinaryInfC{$p \wedge q$}
                                    \RightLabel{(2)}
                                    \BinaryInfC{$\neg p \vee \neg q$}
                                        \RightLabel{(1)}
                                        \UnaryInfC{$\neg (p \wedge q) \rightarrow (\neg p \vee \neg q)$}
                \end{prooftree}
            
            \item Итог: $6/6$. Без $\neg\mathrm{Elim}$.
                \begin{prooftree}
                    \AxiomC{$[\neg p \vee \neg q]^1$}
                    \AxiomC{$[p \wedge q]^3$}
                        \UnaryInfC{$p$}
                    \AxiomC{$[\neg p]^2$}
                            \RightLabel{(3)}
                            \BinaryInfC{$\neg (p \wedge q)$}
                    \AxiomC{$[p \wedge q]^4$}
                        \UnaryInfC{$q$}
                    \AxiomC{$[\neg q]^2$}
                            \RightLabel{(4)}
                            \BinaryInfC{$\neg (p \wedge q)$}
                                \RightLabel{(2)}
                                \TrinaryInfC{$\neg (p \wedge q)$}
                                    \RightLabel{(1)}
                                    \UnaryInfC{$(\neg p \vee \neg q) \rightarrow \neg (p \wedge q)$}
                \end{prooftree}
        \end{enumerate}
    \end{enumproblem}

    \begin{enumproblem}\ItemedProblem\ 
        \begin{enumerate}
            \item Итог: $4/4$. С $\neg\mathrm{Elim}$.
                \begin{prooftree}
                    \AxiomC{$[\neg p \vee q]^1$}
                    \AxiomC{$[p]^2$}
                    \AxiomC{$[\neg p]^3$}
                        \RightLabel{(4)}
                        \BinaryInfC{$q$}
                    \AxiomC{$[q]^3$}
                            \RightLabel{(3)}
                            \TrinaryInfC{$q$}
                                \RightLabel{(2)}
                                \UnaryInfC{$p \rightarrow q$}
                                    \RightLabel{(1)}
                                    \UnaryInfC{$(\neg p \vee q) \rightarrow (p \rightarrow q)$}
                \end{prooftree}
            
            \item Итог: $6/6$. С $\neg\mathrm{Elim}$.
                \begin{prooftree}
                    \AxiomC{$[\neg (\neg p \vee q)]^2$}
                    \AxiomC{$[\neg (\neg p \vee q)]^2$}
                    \AxiomC{$[\neg p]^3$}
                        \UnaryInfC{$\neg p \vee q$}
                            \RightLabel{(3)}
                            \BinaryInfC{$p$}
                    \AxiomC{$[p \rightarrow q]^1$}
                                \BinaryInfC{$q$}
                                    \UnaryInfC{$\neg p \vee q$}
                                        \RightLabel{(2)}
                                        \BinaryInfC{$\neg p \vee q$}
                                            \RightLabel{(1)}
                                            \UnaryInfC{$(p \rightarrow q) \rightarrow (\neg p \vee q)$}
                \end{prooftree}
            
            \item Итог: $5/5$. С $\neg\mathrm{Elim}$.
                \begin{prooftree}
                    \AxiomC{$[\neg p]^2$}
                    \AxiomC{$[p]^3$}
                        \RightLabel{(4)}
                        \BinaryInfC{$q$}
                            \RightLabel{(3)}
                            \UnaryInfC{$p \rightarrow q$}
                    \AxiomC{$[(p \rightarrow q) \rightarrow p]^1$}
                                \BinaryInfC{$p$}
                    \AxiomC{$[\neg p]$}
                                    \RightLabel{(2)}
                                    \BinaryInfC{$p$}
                                        \RightLabel{(1)}
                                        \UnaryInfC{$((p \rightarrow q) \rightarrow p) \rightarrow p$}
                \end{prooftree}
        \end{enumerate}
    \end{enumproblem}

    \section{PCL: гильбертовское исчисление}\reset

    \begin{enumproblem}\ 
        \begin{description}
            \item[$\mathrm{IT}$.] Покажем, что $\{(\varphi \rightarrow \psi); (\psi \rightarrow \chi)\} \vdash (\varphi \rightarrow \chi)$.
                \begin{center}
                    \begin{tabular}{rll}
                        1.& $(\psi \rightarrow \chi) \rightarrow (\varphi \rightarrow (\psi \rightarrow \chi))$& $\mathrm{I1}$\\
                        2.& $(\psi \rightarrow \chi)$& гипотеза\\
                        3.& $\varphi \rightarrow (\psi \rightarrow \chi)$& из 1 и 2\\
                        4.& $(\varphi \rightarrow (\psi \rightarrow \chi)) \rightarrow ((\varphi \rightarrow \psi) \rightarrow (\varphi \rightarrow \chi))$& $\mathrm{I2}$\\
                        5.& $(\varphi \rightarrow \psi) \rightarrow (\varphi \rightarrow \chi)$& из 3 и 4\\
                        6.& $(\varphi \rightarrow \psi)$& гипотеза\\
                        7.& $(\varphi \rightarrow \chi)$& из 5 и 6\\
                    \end{tabular}
                \end{center}
                Убирая по одной гипотезы, получаем из теоремы дедукции (а она выводится благодаря аксиомам $\mathrm{I1}$ и $\mathrm{I2}$) утверждение $(\varphi \rightarrow \psi) \rightarrow ((\psi \rightarrow \chi) \rightarrow (\varphi \rightarrow \chi))$.

            \item[$\mathrm{PE}$.] Покажем, что $\{\varphi \rightarrow (\psi \rightarrow \chi); \phi; \psi\} \vdash \psi \rightarrow (\varphi \rightarrow \chi)$.
                \begin{center}
                    \begin{tabular}{rll}
                        1.& $\varphi \rightarrow (\psi \rightarrow \chi)$& гипотеза\\
                        2.& $\varphi$& гипотеза\\
                        3.& $\psi \rightarrow \chi$& из 1 и 2\\
                        4.& $\psi$& гипотеза\\
                        5.& $\chi$& из 3 и 4\\
                    \end{tabular}
                \end{center}
                Убирая по одной в правильном порядке гипотезы, получаем из теоремы дедукции (а она выводится благодаря аксиомам $\mathrm{I1}$ и $\mathrm{I2}$) утверждение $(\varphi \rightarrow (\psi \rightarrow \chi)) \rightarrow (\psi \rightarrow (\varphi \rightarrow \chi))$.
        \end{description}
    \end{enumproblem}

    \begin{enumproblem}\ 
        \begin{description}
            \item[$\mathrm{2N}$.] Покажем, что $\{\varphi\} \vdash \neg \neg \varphi$.
                \begin{center}
                    \begin{tabular}{rll}
                        1.& $\varphi \rightarrow (\neg \varphi \rightarrow \varphi)$& $\mathrm{I1}$\\
                        2.& $\varphi$& гипотеза\\
                        3.& $\neg \varphi \rightarrow \varphi$& из 1 и 2\\
                        4.& $(\neg \varphi \rightarrow \varphi) \rightarrow ((\neg \varphi \rightarrow \neg \varphi) \rightarrow \neg \neg \varphi)$& $\mathrm{N1}$\\
                        5.& $(\neg \varphi \rightarrow \neg \varphi) \rightarrow \neg \neg \varphi$& из 3 и 4\\
                        6.& $\neg \varphi \rightarrow \neg \varphi$& лемма ($\mathrm{I1}$ и $\mathrm{I2}$)\\
                        7.& $\neg \neg \varphi$& из 5 и 6\\
                    \end{tabular}
                \end{center}
                Убирая по одной гипотезы, получаем из теоремы дедукции (а она выводится благодаря аксиомам $\mathrm{I1}$ и $\mathrm{I2}$) утверждение $\varphi \rightarrow \neg \neg \varphi$.

            \item[$\mathrm{Co}$.] Покажем, что $\{\varphi \rightarrow \psi; \neg \psi\} \vdash \neg \varphi$.
                \begin{center}
                    \begin{tabular}{rll}
                        1.& $\neg \psi \rightarrow (\varphi \rightarrow \neg \psi)$& $\mathrm{I1}$\\
                        2.& $\neg \psi$& гипотеза\\
                        3.& $\varphi \rightarrow \neg \psi$& из 1 и 2\\
                        4.& $(\varphi \rightarrow \psi) \rightarrow ((\varphi \rightarrow \neg \psi) \rightarrow \neg \varphi)$& $\mathrm{N1}$\\
                        5.& $\varphi \rightarrow \psi$& гипотеза\\
                        6.& $(\varphi \rightarrow \neg \psi) \rightarrow \neg \varphi$& из 4 и 5\\
                        7.& $\neg \varphi$& из 3 и 6\\
                    \end{tabular}
                \end{center}
                Убирая по одной в правильном порядке гипотезы, получаем из теоремы дедукции (а она выводится благодаря аксиомам $\mathrm{I1}$ и $\mathrm{I2}$) утверждение $(\varphi \rightarrow \psi) \rightarrow (\neg \psi \rightarrow \neg \varphi)$.
        \end{description}
    \end{enumproblem}

    \begin{enumproblem}\ 
        \begin{description}
            \item[\underline{$\mathrm{2N}$}.] Покажем, что $\{\neg \neg \varphi\} \vdash \varphi$.
                \begin{center}
                    \begin{tabular}{rll}
                        1.& $\neg \neg \varphi \rightarrow (\neg \varphi \rightarrow \varphi)$& $\mathrm{N2}$\\
                        2.& $\neg \neg \varphi$& гипотеза\\
                        3.& $\neg \varphi \rightarrow \varphi$& из 1 и 2\\
                        4.& $\varphi \rightarrow \varphi$& лемма ($\mathrm{I1}$ и $\mathrm{I2}$)\\
                        5.& $(\varphi \rightarrow \varphi) \rightarrow ((\neg \varphi \rightarrow \varphi) \rightarrow (\varphi \vee \neg \varphi \rightarrow \varphi))$& $\mathrm{D3}$\\
                        6.& $(\neg \varphi \rightarrow \varphi) \rightarrow (\varphi \vee \neg \varphi \rightarrow \varphi)$& из 4 и 5\\
                        7.& $\varphi \vee \neg \varphi \rightarrow \varphi$& из 3 и 6\\
                        8.& $\varphi \vee \neg$& $\mathrm{N3}$\\
                        9.& $\varphi$& из 7 и 8\\
                    \end{tabular}
                \end{center}
                Убирая по одной гипотезы, получаем из теоремы дедукции (а она выводится благодаря аксиомам $\mathrm{I1}$ и $\mathrm{I2}$) утверждение $\neg \neg \varphi \rightarrow \varphi$.

            \item[\underline{$\mathrm{Co}$}.] Покажем, что $\{\neg \psi \rightarrow \neg \varphi\} \vdash \varphi \rightarrow \psi$.
                \begin{center}
                    \begin{tabular}{rll}
                        1.& $\psi \rightarrow (\varphi \rightarrow \psi)$& $\mathrm{I1}$\\
                        2.& $\neg \varphi \rightarrow (\varphi \rightarrow \psi)$& $\mathrm{N2}$\\
                        3.& $\neg \psi \rightarrow \neg \varphi$& гипотеза\\
                        4.& $(\neg \psi \rightarrow \neg \varphi) \rightarrow ((\neg \varphi \rightarrow (\varphi \rightarrow \psi)) \rightarrow (\neg \psi \rightarrow (\varphi \rightarrow \psi)))$& $\mathrm{IT}$ ($\mathrm{I1}$ и $\mathrm{I2}$)\\
                        5.& $(\neg \varphi \rightarrow (\varphi \rightarrow \psi)) \rightarrow (\neg \psi \rightarrow (\varphi \rightarrow \psi))$& из 3 и 4\\
                        6.& $\neg \psi \rightarrow (\varphi \rightarrow \psi)$& из 2 и 5\\
                        7.& $(\psi \rightarrow (\varphi \rightarrow \psi)) \rightarrow ((\neg \psi \rightarrow (\varphi \rightarrow \psi)) \rightarrow (\psi \vee \neg \psi \rightarrow (\varphi \rightarrow \psi)))$& $\mathrm{D3}$\\
                        8.& $(\neg \psi \rightarrow (\varphi \rightarrow \psi)) \rightarrow (\psi \vee \neg \psi \rightarrow (\varphi \rightarrow \psi))$& из 1 и 7\\
                        9.& $\psi \vee \neg \psi \rightarrow (\varphi \rightarrow \psi)$& из 6 и 8\\
                        10.& $\psi \vee \neg \psi$& $\mathrm{N3}$\\
                        11.& $\varphi \rightarrow \psi$& из 9 и 10\\
                    \end{tabular}
                \end{center}
               Убирая по одной в правильном порядке гипотезы, получаем из теоремы дедукции (а она выводится благодаря аксиомам $\mathrm{I1}$ и $\mathrm{I2}$) утверждение $(\neg \psi \rightarrow \neg \varphi) \vdash (\varphi \rightarrow \psi)$.
        \end{description}
    \end{enumproblem}

    \begin{enumproblem}\ 
        \begin{enumerate}
            \item Покажем, что $\{\neg \varphi \rightarrow \psi; \neg \varphi \rightarrow \neg \psi\} \vdash \varphi$.
                \begin{center}
                    \begin{tabular}{rll}
                        1.& $(\neg \varphi \rightarrow \psi) \rightarrow ((\neg \varphi \rightarrow \neg \psi) \rightarrow \neg \neg \varphi)$& $\mathrm{N1}$\\
                        2.& $\neg \varphi \rightarrow \psi$& гипотеза\\
                        3.& $(\neg \varphi \rightarrow \neg \psi) \rightarrow \neg \neg \varphi$& из 1 и 2\\
                        4.& $\neg \varphi \rightarrow \neg \psi$& гипотеза\\
                        5.& $\neg \neg \varphi$& из 3 и 4\\
                        6.& $\neg \neg \varphi \rightarrow \varphi$& $\overline{\mathrm{2N}}$\\
                        7.& $\varphi$& из 5 и 6\\
                    \end{tabular}
                \end{center}
                Убирая по одной гипотезы, получаем из теоремы дедукции (а она выводится благодаря аксиомам $\mathrm{I1}$ и $\mathrm{I2}$) утверждение $(\neg \varphi \rightarrow \psi) \rightarrow ((\neg \varphi \rightarrow \neg \psi) \rightarrow \varphi)$.

            \item\ 
                \begin{description}
                    \item[\underline{$\mathrm{2N}$}.] Покажем, что $\{\neg \neg \varphi\} \vdash \varphi$.
                        \begin{center}
                            \begin{tabular}{rll}
                                1.& $(\neg \varphi \rightarrow \neg \varphi) \rightarrow ((\neg \varphi \rightarrow \neg \neg \varphi) \rightarrow \varphi)$& $\mathrm{NE}$\\
                                2.& $\neg \varphi \rightarrow \neg \varphi$& лемма ($\mathrm{I1}$ и $\mathrm{I2}$)\\
                                3.& $(\neg \varphi \rightarrow \neg \neg \varphi) \rightarrow \varphi$& из 1 и 2\\
                                4.& $\neg \neg \varphi \rightarrow (\neg \varphi \rightarrow \neg \neg \varphi)$& $\mathrm{I1}$\\
                                5.& $\neg \neg \varphi$& гипотеза\\
                                6.& $\neg \varphi \rightarrow \neg \neg \varphi$& из 4 и 5\\
                                7.& $\varphi$& из 3 и 6\\
                            \end{tabular}
                        \end{center}
                        Убирая по одной в правильном порядке гипотезы, получаем из теоремы дедукции (а она выводится благодаря аксиомам $\mathrm{I1}$ и $\mathrm{I2}$) утверждение $\neg \neg \varphi \rightarrow \varphi$.
                    
                    \item[$\mathrm{N1}$.] Покажем, что $\{\varphi \rightarrow \psi; \varphi \rightarrow \neg \psi\} \vdash \neg \varphi$.
                        \begin{center}
                            \begin{tabular}{rll}
                                1.& $\neg \neg \varphi \rightarrow \varphi$& $\underline{\mathrm{2N}}$ ($\mathrm{I1}$, $\mathrm{I2}$ и $\mathrm{NE}$)\\
                                2.& $(\neg \neg \varphi \rightarrow \varphi) \rightarrow ((\varphi \rightarrow \psi) \rightarrow (\neg \neg \varphi \rightarrow \psi))$& $\mathrm{IT}$ ($\mathrm{I1}$ и $\mathrm{I2}$)\\
                                3.& $(\neg \neg \varphi \rightarrow \varphi) \rightarrow ((\varphi \rightarrow \neg \psi) \rightarrow (\neg \neg \varphi \rightarrow \neg \psi))$& $\mathrm{IT}$ ($\mathrm{I1}$ и $\mathrm{I2}$)\\
                                4.& $(\varphi \rightarrow \psi) \rightarrow (\neg \neg \varphi \rightarrow \psi)$& из 1 и 2\\
                                5.& $(\varphi \rightarrow \neg \psi) \rightarrow (\neg \neg \varphi \rightarrow \neg \psi)$& из 1 и 3\\
                                6.& $\varphi \rightarrow \psi$& гипотеза\\
                                7.& $\varphi \rightarrow \neg \psi$& гипотеза\\
                                8.& $\neg \neg \varphi \rightarrow \psi$& из 4 и 6\\
                                9.& $\neg \neg \varphi \rightarrow \neg \psi$& из 5 и 7\\
                                10.& $(\neg \neg \varphi \rightarrow \psi) \rightarrow ((\neg \neg \varphi \rightarrow \neg \psi) \rightarrow \neg \varphi)$& $\mathrm{NE}$\\
                                11.& $(\neg \neg \varphi \rightarrow \neg \psi) \rightarrow \neg \varphi$& из 8 и 10\\
                                12.& $\neg \varphi$& из 9 и 11\\
                            \end{tabular}
                        \end{center}
                        Убирая по одной в правильном порядке гипотезы, получаем из теоремы дедукции (а она выводится благодаря аксиомам $\mathrm{I1}$ и $\mathrm{I2}$) утверждение $(\varphi \rightarrow \psi) \rightarrow ((\varphi \rightarrow \neg \psi) \rightarrow \neg \varphi)$.
                \end{description}
        \end{enumerate}
    \end{enumproblem}

    \begin{enumproblem}\ 
        \begin{description}
            \item[$\mathrm{N2}$.] Покажем, что $\{\varphi; \neg \varphi\} \vdash \psi$.
                \begin{center}
                    \begin{tabular}{rll}
                        1.& $\varphi \rightarrow (\neg \psi \rightarrow \varphi)$& $\mathrm{I1}$\\
                        2.& $\varphi$& гипотеза\\
                        3.& $\neg \psi \rightarrow \varphi$& из 1 и 2\\
                        4.& $\neg \varphi \rightarrow (\neg \psi \rightarrow \neg \varphi)$& $\mathrm{I1}$\\
                        5.& $\neg \varphi$& гипотеза\\
                        6.& $\neg \psi \rightarrow \neg \varphi$& из 4 и 5\\
                        7.& $(\neg \psi \rightarrow \varphi) \rightarrow ((\neg \psi \rightarrow \neg \varphi) \rightarrow \psi)$& $\mathrm{NE}$\\
                        8.& $(\neg \psi \rightarrow \neg \varphi) \rightarrow \psi$& из 3 и 7\\
                        9.& $\psi$& из 6 и 8\\
                    \end{tabular}
                \end{center}
                Убирая по одной в правильном порядке гипотезы, получаем из теоремы дедукции (а она выводится благодаря аксиомам $\mathrm{I1}$ и $\mathrm{I2}$) утверждение $\neg \varphi \rightarrow (\varphi \rightarrow \psi)$.
            
            \item[$\mathrm{N3}$.]\ 
                \begin{center}
                    \begin{tabular}{rll}
                        1.& $\varphi \rightarrow (\varphi \vee \neg \varphi)$& $\mathrm{D1}$\\
                        2.& $(\varphi \rightarrow (\varphi \vee \neg \varphi)) \rightarrow ((\varphi \rightarrow \neg (\varphi \vee \neg \varphi)) \rightarrow \neg \varphi)$& $\mathrm{N1}$ ($\mathrm{I1}$, $\mathrm{I2}$ и $\mathrm{NE}$)\\
                        3.& $(\varphi \rightarrow \neg (\varphi \vee \neg \varphi)) \rightarrow \neg \varphi$& из 1 и 2\\
                        4.& $\neg (\varphi \vee \neg \varphi) \rightarrow (\varphi \rightarrow \neg (\varphi \vee \neg \varphi))$& $\mathrm{I1}$\\
                        5.& $(\neg (\varphi \vee \neg \varphi) \rightarrow (\varphi \rightarrow \neg (\varphi \vee \neg \varphi))) \rightarrow$& $\mathrm{IT}$ ($\mathrm{I1}$ и $\mathrm{I2}$)\\
                        &$(((\varphi \rightarrow \neg (\varphi \vee \neg \varphi)) \rightarrow \neg \varphi) \rightarrow (\neg (\varphi \vee \neg \varphi) \rightarrow \neg \varphi))$&\\
                        6.& $((\varphi \rightarrow \neg (\varphi \vee \neg \varphi)) \rightarrow \neg \varphi) \rightarrow (\neg (\varphi \vee \neg \varphi) \rightarrow \neg \varphi)$& из 4 и 5\\
                        7.& $\neg (\varphi \vee \neg \varphi) \rightarrow \neg \varphi$& из 3 и 6\\
                        8.& $\neg \varphi \rightarrow (\varphi \vee \neg \varphi)$& $\mathrm{D2}$\\
                        9.& $(\neg \varphi \rightarrow (\varphi \vee \neg \varphi)) \rightarrow ((\neg \varphi \rightarrow \neg (\varphi \vee \neg \varphi)) \rightarrow \varphi)$& $\mathrm{NE}$\\
                        10.& $(\neg \varphi \rightarrow \neg (\varphi \vee \neg \varphi)) \rightarrow \varphi$& из 8 и 9\\
                        11.& $\neg (\varphi \vee \neg \varphi) \rightarrow (\neg \varphi \rightarrow \neg (\varphi \vee \neg \varphi))$& $\mathrm{I1}$\\
                        12.& $(\neg (\varphi \vee \neg \varphi) \rightarrow (\neg \varphi \rightarrow \neg (\varphi \vee \neg \varphi))) \rightarrow$& $\mathrm{IT}$ ($\mathrm{I1}$ и $\mathrm{I2}$)\\
                        &$(((\neg \varphi \rightarrow \neg (\varphi \vee \neg \varphi)) \rightarrow \varphi) \rightarrow (\neg (\varphi \vee \neg \varphi) \rightarrow \varphi))$&\\
                        13.& $((\neg \varphi \rightarrow \neg (\varphi \vee \neg \varphi)) \rightarrow \varphi) \rightarrow (\neg (\varphi \vee \neg \varphi) \rightarrow \varphi)$& из 11 и 12\\
                        14.& $\neg (\varphi \vee \neg \varphi) \rightarrow \varphi$& из 10 и 13\\
                        15.& $(\neg (\varphi \vee \neg \varphi) \rightarrow \varphi) \rightarrow ((\neg (\varphi \vee \neg \varphi) \rightarrow \neg \varphi) \rightarrow (\varphi \vee \neg \varphi))$& $\mathrm{NE}$\\
                        16.& $(\neg (\varphi \vee \neg \varphi) \rightarrow \neg \varphi) \rightarrow (\varphi \vee \neg \varphi)$& из 14 и 15\\
                        17.& $\varphi \vee \neg \varphi$& из 7 и 16\\
                    \end{tabular}
                \end{center}
        \end{description}
    \end{enumproblem}

    \begin{enumproblem}
        TODO
    \end{enumproblem}

    \begin{enumproblem}\ 
        \begin{enumerate}
            \item Покажем, что $\{(\varphi \rightarrow \psi); (\psi \rightarrow \chi)\} \vdash (\varphi \rightarrow \chi)$.
                \begin{center}
                    \begin{tabular}{rll}
                        1.& $(\psi \rightarrow \chi) \rightarrow (\varphi \rightarrow (\psi \rightarrow \chi))$& $\mathrm{I1}$\\
                        2.& $(\psi \rightarrow \chi)$& гипотеза\\
                        3.& $\varphi \rightarrow (\psi \rightarrow \chi)$& из 1 и 2\\
                        4.& $(\varphi \rightarrow (\psi \rightarrow \chi)) \rightarrow ((\varphi \rightarrow \psi) \rightarrow (\varphi \rightarrow \chi))$& $\mathrm{I2}$\\
                        5.& $(\varphi \rightarrow \psi) \rightarrow (\varphi \rightarrow \chi)$& из 3 и 4\\
                        6.& $(\varphi \rightarrow \psi)$& гипотеза\\
                        7.& $(\varphi \rightarrow \chi)$& из 5 и 6\\
                    \end{tabular}
                \end{center}
                Убирая по одной гипотезы, получаем из теоремы дедукции (а она выводится благодаря аксиомам $\mathrm{I1}$ и $\mathrm{I2}$) утверждение $(\varphi \rightarrow \psi) \rightarrow ((\psi \rightarrow \chi) \rightarrow (\varphi \rightarrow \chi))$.

            \item Покажем, что $\{\varphi\} \vdash \neg \neg \varphi$.
                \begin{center}
                    \begin{tabular}{rll}
                        1.& $\varphi \rightarrow (\neg \varphi \rightarrow \varphi)$& $\mathrm{I1}$\\
                        2.& $\varphi$& гипотеза\\
                        3.& $\neg \varphi \rightarrow \varphi$& из 1 и 2\\
                        4.& $(\neg \varphi \rightarrow \varphi) \rightarrow ((\neg \varphi \rightarrow \neg \varphi) \rightarrow \neg \neg \varphi)$& $\mathrm{N1}$\\
                        5.& $(\neg \varphi \rightarrow \neg \varphi) \rightarrow \neg \neg \varphi$& из 3 и 4\\
                        6.& $\neg \varphi \rightarrow \neg \varphi$& лемма ($\mathrm{I1}$ и $\mathrm{I2}$)\\
                        7.& $\neg \neg \varphi$& из 5 и 6\\
                    \end{tabular}
                \end{center}
                Убирая по одной гипотезы, получаем из теоремы дедукции (а она выводится благодаря аксиомам $\mathrm{I1}$ и $\mathrm{I2}$) утверждение $\varphi \rightarrow \neg \neg \varphi$.
        \end{enumerate}
    \end{enumproblem}

\end{document}