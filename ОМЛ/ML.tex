\documentclass[12pt,a4paper]{article}
\usepackage{../.tex/mcs-notes}
\usepackage{todonotes}
\usepackage{mathrsfs}

\settitle
{Основы математической логики.}
{\href{http://www.mi-ras.ru/~speranski/}{Станислав Олегович Сперанский}}
{\%D0\%9E\%D0\%9C\%D0\%9B\%2FML.pdf}
\date{}

% \newcommand{\subsets}{\ensuremath{\mathcal{P}}\xspace}
% \newcommand{\finsubsets}{\ensuremath{\mathcal{P_{\mathrm{fin}}}}\xspace}
% \DeclareMathOperator{\Ind}{Ind}
\DeclareMathOperator{\dom}{dom}
% \DeclareMathOperator{\range}{range}
% \DeclareMathOperator{\Min}{Min}
% \DeclareMathOperator{\Seq}{Seq}
% \DeclareMathOperator{\Left}{left}
% \DeclareMathOperator{\Right}{right}
% \DeclareMathOperator{\IS}{IS}
% \DeclareMathOperator{\ord}{ord}
% \DeclareMathOperator{\Ord}{Ord}
% \DeclareMathOperator{\card}{card}
% \DeclareMathOperator{\Card}{Card}
% \newcommand{\ZF}{\ensuremath{\mathrm{ZF}}\xspace}
% \newcommand{\ZFC}{\ensuremath{\mathrm{ZFC}}\xspace}
% \newcommand{\CH}{\ensuremath{\mathrm{CH}}\xspace}
\newcommand{\Prop}{\ensuremath{\mathrm{Prop}}\xspace}
\newcommand{\Formul}{\ensuremath{\mathrm{Form}}\xspace}
\newcommand{\Sub}{\ensuremath{\mathrm{Sub}}\xspace}
\newcommand{\NP}{\ensuremath{\mathrm{NP}}\xspace}


\begin{document}
    \maketitle

    \listoftodos[TODOs]

    \tableofcontents

    \vspace{2em}
    Материалы лекций: \href{http://www.mi-ras.ru/~speranski/courses/logic-1-2021-spring/materials.html}{ссылка}
    % \todo[color=green!40, inline]{Добавить конспекты теории из упражнений. Слить аккуратно воедино. Добавить ссылку на упражнения.}
    
    % Литература:
    % \begin{itemize}
    %     \item K. Hrbacek and T. Jech. Introduction to Set Theory. 3rd ed., revised and expanded. Marcel Dekker, Inc., 1999.
    %     \item T. Jech. Set Theory. 3rd ed., revised and expanded. Springer, 2002.
    % \end{itemize}

    \subsection{Формальности про алфавиты и слова}

    \begin{definition}
        \emph{Алфавит} $A$ --- множество элементов произвольной природы.

        \emph{$A$-слова} или \emph{слова над алфавитом $A$} --- элементы $A^*$ (т.е. всевозможные конечные последовательности элементов из $A$).

        Для всякого $w \in A^*$ \emph{длиной слова $w$} называется $|w|$ (что также равно $\dom(w)$).
    \end{definition}

    \begin{definition}
        Пусть даны слова $w_1$ и $w_2$ над $A$. Рассмотрим отображение
        \[
            v:
            |w_1| + |w_2| \to A,
            i \mapsto \begin{cases}
                w_1(i)& \text{ если $i < |w_1|$}\\
                w_2(i - |w_1|)& \text{ если $i \geqslant |w_1|$}
            \end{cases}
        \]
        Понятно, что $v \in A^*$. Полученное $v$ называется конкатенацией $w_1$ и $w_2$ и обозначается $w_1w_2$.
    \end{definition}

    \begin{definition}
        Пусть $w, w' \in A^*$. Тогда $w'$ называется \emph{подсловом $w$}, если $w = v_0 w' v_1$ для некоторых $\{v_0; v_1\} \subseteq A^*$. Обозначение: $w' \preccurlyeq w$.

        В этом случае $\langle w'; |v_0|\rangle$ называется \emph{вхождением} $w'$ в $w$.
    \end{definition}

    \begin{definition}
        Пусть $\langle w', k\rangle$ --- вхождение $w'$ в $w$, т.е. $w = v_0 w' v_1$, где $|v_0| = k$. Тогда для всякого $u \in A^*$ можно определить
        \[w [w' / u, k] := v_0 u v_1,\]
        т.е. результат замены данного вхождения $w'$ в $w$ на $u$.
        
        Если никакие два различных вхождения $w'$ в $w$ не пересекаются, то можно определить $w [w' / u]$ как результат одновременной замены всех вхождений $w'$ в $w$ на $u$. 
    \end{definition}

    \subsection{Язык пропозициональной классической логики\\(Proposal Classic Logic, PCL)}

    \begin{definition}
        Пусть (навсегда) зафиксировано некоторое счётное множество \Prop. Будем называть его элементы \emph{пропозициональными переменными} или просто \emph{переменными}.

        Алфавит $\mathscr{L}$ классической пропозициональной классической логики состоит из элементов \Prop, а также:
        \begin{itemize}
            \item символов связок:
                \begin{itemize}
                    \item $\rightarrow$ --- символ импликации,
                    \item $\wedge$ --- символ конъюнкции,
                    \item $\vee$ --- символ дизъюнкции,
                    \item $\neg$ --- символ отрицания,
                \end{itemize}
            \item и вспомогательных символов: ( и ).
        \end{itemize}

        Обозначим за \Formul наименьшее подмножество $\mathscr{L}^*$, замкнутое относительно следующих порождающих правил:
        \begin{itemize}
            \item если $p \in \Prop$, то $p \in \Formul$;
            \item если $\{\varphi, \psi\} \in \Prop$, то $(\varphi \rightarrow \psi) \in \Formul$;
            \item если $\{\varphi, \psi\} \in \Prop$, то $(\varphi \wedge \psi) \in \Formul$;
            \item если $\{\varphi, \psi\} \in \Prop$, то $(\varphi \vee \psi) \in \Formul$;
            \item если $\varphi \in \Formul$, то $\neg \varphi \in \Formul$.
        \end{itemize}
        Элементы \Formul называются формулами.
    \end{definition}

    \begin{theorem}
        \Formul существует.
    \end{theorem}

    \begin{proof}
        Рассмотрим семейство $T$ всех подмножеств $\mathscr{L}^*$, удовлетворяющих порождающим правилам выше. Заметим, что $T$ не пусто, так как содержит $\mathscr{L}^*$. Тогда можно рассмотреть $F := \bigcap T$. Несложно убедиться, что оно удовлетворяет всем порождающим правилам, значит лежит в $T$. И при этом меньше по включению всех других множеств в $T$. Значит его и можно взять в качестве $\Formul$.
    \end{proof}

    \begin{definition}
        Будем говорить, что $\varphi$ является \emph{началом $\psi$}, и писать $\varphi \sqsubseteq \psi$, если $\psi = \varphi \tau$ для некоторого $v \in \mathscr{L}^*$.
    \end{definition}

    \begin{lemma}
        Всякое $\varphi \in \Formul$ имеет один из следующих видов:
        \begin{enumerate}
            \item $p$ для некоторого $p \in \Prop$;
            \item $(\theta \circ \chi)$ для некоторых $\{\theta; \chi\} \subseteq \Formul$ и $\circ \in \{\rightarrow; \wedge; \vee\}$;
            \item $\neg \theta$ для некоторого $\theta \in \Formul$.
        \end{enumerate}
    \end{lemma}

    \begin{proof}
        Предположим противное. Тогда рассмотрим $F := \Formul \setminus \{\varphi\}$. Заметим, что $F$ удовлетворяет тем же порождающим правилам, что и $\Formul$. Действительно:
        \begin{itemize}
            \item Если $p \in \Prop$, то $p \in \Formul$. При этом $p \neq \varphi$ по условию леммы. Следовательно $p \in F$.
            \item Если $\{\varphi, \psi\} \in \Prop$, то $(\varphi \rightarrow \psi) \in \Formul$. При этом $(\varphi \rightarrow \psi) \neq \varphi$ по условию леммы. Следовательно $(\varphi \rightarrow \psi) \in F$. Аналогично для $\wedge$ и $\vee$.
            \item Если $\varphi \in \Formul$, то $\neg \varphi \in \Formul$. При этом $\neg \varphi \neq \varphi$ по условию леммы. Следовательно $\neg \varphi \in F$.
        \end{itemize}
        Значит $F$ --- меньшее по включение чем $\Formul$ множество, удовлетворяющее условиям наложенным на $\Formul$ --- противоречие.
    \end{proof}

    \begin{corollary}
        Рассмотрим последовательность множеств $(F_n)_{n=0}^\infty$, что $F_0 = \Prop$, а
        \[
            F_{n+1} = F_n 
            \cup \{(\varphi \circ \chi) \mid \{\theta; \chi\} \subseteq \Formul \wedge \circ \in \{\rightarrow; \wedge; \vee\}\}
            \cup \{\neg \theta \mid \theta \in \Formul\}
        \]
        Тогда
        \begin{enumerate}
            \item всякое $F_n \subseteq \Formul$;
            \item всякое $\varphi \in \Formul$ лежит в некотором $F_n$.
        \end{enumerate}
    \end{corollary}

    \begin{corollary}
        $\bigcup_{n=0}^\infty F_n = \Formul$. 
    \end{corollary}

    \begin{lemma}
        Пусть $\{\varphi; \psi\} \subseteq \Formul$ таковы, что $\psi \sqsubseteq \varphi$. Тогда $\psi = \varphi$.
    \end{lemma}

    \begin{proof}
        Докажем утверждение возвратной индукцией по $|\varphi|$.
        
        Рассмотрим случаи.
        \begin{enumerate}
            \item Если $\varphi \in \Prop$, то, очевидно, $\psi = \varphi$.
            \item Если $\varphi = (\theta \circ \chi)$, где $\{\theta; \chi\} \in \Formul$ и $\circ \in \{\rightarrow; \wedge; \vee\}$, то $\psi$ начинается на ``('', значит имеет вид $(\theta' \circ' \chi')$, где $\{\theta'; \chi'\} \in \Formul$ и $\circ' \in \{\rightarrow; \wedge; \vee\}$. Следовательно либо $\theta \sqsubseteq \theta'$, либо $\theta' \sqsubseteq \theta$. Но $|\theta| < |\varphi| - 3$, и $|\theta'| < |\psi| - 3 \leqslant |\varphi| - 3$. Тогда можно применить предположение индукции и получить, что $\theta = \theta'$. Значит $\circ = \circ'$, а далее по аналогии получаем, что $\chi = \chi'$. Следовательно $\varphi = \psi$.
            \item Если $\varphi = \neg \theta$, то $\psi$ начинается на ``$\neg$'', следовательно $\psi = \neg \theta'$. Тогда $\theta' \sqsubseteq \theta$, а тогда по предположению индукции $\theta' = \theta$, значит $\varphi = \psi$.
        \end{enumerate}
    \end{proof}

    \begin{theorem}[о единственности представления формул]
        Всякая формула в $\Formul \setminus \Prop$ представляется единственным образом в одном из видов
        \begin{itemize}
            \item $(\theta \rightarrow \chi)$,
            \item $(\theta \wedge \chi)$,
            \item $(\theta \vee \chi)$,
            \item $\neg \theta$,
        \end{itemize}
        где $\{\theta; \chi\} \subseteq \Formul$.
    \end{theorem}

    \begin{proof}
        По доказанной лемме всякое $\phi$ имеет такое представление. Пусть тогда их несколько; рассмотрим случаи.
        \begin{enumerate}
            \item Если $\phi$ начинается на ``('', то тогда $\phi = (\theta \circ \chi) = (\theta' \circ' \chi')$. Тогда по доказанной лемме $\theta = \theta'$, $\circ = \circ'$, $\chi = \chi'$. Значит представления совпадают.
            \item Если $\phi$ начинается на ``$\neg$'', то $\phi = \neg \theta = \neg \theta'$. Тогда $\theta = \theta'$, а следовательно представления совпадают.
        \end{enumerate}
    \end{proof}

    \begin{definition}
        Для всякого $\varphi \in \Formul$ определим
        \[\Sub(\varphi) := \{\psi \Formul \mid \psi \preccurlyeq \varphi\}.\]
        Элементы $\Sub(\varphi)$ называют \emph{подформулами $\varphi$}.
    \end{definition}

    \begin{lemma}
        Пусть $\varphi \in \Formul$. Тогда каждое вхождение ``$\neg$'' или ``$($'' в $\varphi$ является началом вхождения некоторой подформулы.
    \end{lemma}

    \begin{proof}
        \todo[inline]{TODO (А надо ли?)}
    \end{proof}

    \begin{theorem}
        Пусть $\varphi \in \Formul$.
        \begin{enumerate}
            \item Если $\varphi \in \Prop$, то $\Sub(\varphi) = \{\varphi\}$.
            \item Если $\varphi = (\theta \circ \chi)$, где $\{\theta; \chi\} \subseteq \Formul$ и $\circ \in \{\rightarrow; \wedge; \vee\}$, то
                \[\Sub(\varphi) = \Sub(\theta) \cup \Sub(\chi) \cup \{\varphi\}.\]
            \item Если $\varphi = \neg \theta$, где $\theta \in \Formul$, то
                \[\Sub(\varphi) = \Sub(\theta) \cup \{\varphi\}.\]
        \end{enumerate}
    \end{theorem}

    \begin{proof}
        \begin{enumerate}
            \item Очевидно.
            \item \todo[inline]{Непонятно. TODO}
            \item \todo[inline]{Непонятно. TODO}
        \end{enumerate}
    \end{proof}

    \subsection{Семантика пропозициональной классической логики}

    \begin{definition}
        Под \emph{оценкой} мы будем понимать произвольную функцию из $\Prop$ в $2$ (т.е. в $\{0; 1\}$). Интуитивно $0$ --- <<ложь>>, а $1$ --- <<правда>>.
    \end{definition}

    \begin{theorem}
        Пусть дана случайная $v: \Prop \to 2$. Тогда существует единственная $v^*: \Formul \to 2$, которая удовлетворяет следующим свойствам:
        \begin{enumerate}
            \item $\forall p \in \Prop\qquad v^*(p) = 1 \Leftrightarrow v(p) = 1$.
            \item $\forall \{\varphi; \psi\} \subseteq \Formul\qquad v^*(\text{``$(\varphi \rightarrow \psi)$''}) = 1 \Leftrightarrow v^*(\varphi) = 0 \vee v^*(\psi) = 1$.
            \item $\forall \{\varphi; \psi\} \subseteq \Formul\qquad v^*(\text{``$(\varphi \wedge \psi)$''}) = 1 \Leftrightarrow v^*(\varphi) = 1 \wedge v^*(\psi) = 1$.
            \item $\forall \{\varphi; \psi\} \subseteq \Formul\qquad v^*(\text{``$(\varphi \vee \psi)$''}) = 1 \Leftrightarrow v^*(\varphi) = 1 \vee v^*(\psi) = 1$.
            \item $\forall \varphi \in \Formul\qquad v^*(\neg \varphi) = 1 \Leftrightarrow v^*(\varphi) = 0$.
        \end{enumerate}
    \end{theorem}

    \begin{proof}
        \todo[inline]{TODO}
    \end{proof}

    \begin{definition}
        Если для некоторой оценки $v$ и формулы $\varphi$ верно, что $v^*(\varphi) = 1$, то пишут $v \Vdash \varphi$.
    \end{definition}

    \begin{definition}
        Формулу $\varphi$ называют:
        \begin{itemize}
            \item \emph{выполнимой}, если $v \Vdash \varphi$ для некоторой оценки $v$;
            \item \emph{общезначимой} (или \emph{тождественно истинной}, или \emph{тавтологией}), если $v \Vdash \varphi$ для всякой оценки $v$.
        \end{itemize}
    \end{definition}

    \begin{remark*}
        Очевидно, например, что
        \[\varphi \text{ общезначима}\quad \Longleftrightarrow\quad \neg \varphi \text{ не выполнима}.\]
    \end{remark*}

    \begin{theorem*}[Кук-Левин]
        Проблема выполнимости для пропозициональной классической логики \NP-полна.
    \end{theorem*}

    \begin{definition}
        Пусть $\Gamma \subseteq \Formul$ и $\varphi \in \Formul$. Говорят, что $\varphi$ \emph{семантически следует из $\Gamma$}, если для любой оценки $v$
        \[(\forall \psi \in \Gamma \quad v \Vdash \psi) \qquad \longrightarrow \qquad v \Vdash \varphi.\]
        Обозначение: $\Gamma \vDash \varphi$.

        Вместо $\varnothing \vDash \varphi$ обычно пишут $\vDash \varphi$.
    \end{definition}

    \begin{remark*}
        Очевидно, например, что
        \[\vDash \varphi\quad \Longleftrightarrow\quad \varphi \text{ общезначима}.\]
    \end{remark*}

    \begin{definition}
        Формулы $\varphi$ и $\psi$ называются \emph{семантически эквивалентными}, если $\vDash \varphi \leftrightarrow \psi$. Обозначение: $\varphi \equiv \psi$.
    \end{definition}

    \begin{example}
        Для любых $\{\varphi; \psi; \chi\} \subseteq \Formul$:
        \begin{itemize}
            \item $(\varphi \rightarrow \psi) \equiv \neg \varphi \vee \psi$;
            \item $(\varphi \vee \psi) \wedge \chi \equiv (\varphi \wedge \chi) \vee (\psi \wedge \chi)$;
            \item $(\varphi \wedge \psi) \vee \chi \equiv (\varphi \vee \chi) \wedge (\psi \vee \chi)$;
            \item $\neg (\varphi \wedge \psi) \equiv \neg \varphi \vee \neg \psi$;
            \item $\neg (\varphi \vee \psi) \equiv \neg \varphi \wedge \neg \psi$;
            \item $\neg \neg \varphi \equiv \varphi$.
        \end{itemize}
    \end{example}

    \begin{exercise}
        Всякая формула семантически эквивалентна некоторой ДНФ.
    \end{exercise}
\end{document}