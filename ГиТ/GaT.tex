\documentclass[12pt,a4paper]{article}
\usepackage{../.tex/mcs-notes}
\usepackage{todonotes}

\settitle
{Геометрия и топология.}
{Евгений Анатольевич Фоминых}
{\%D0\%93\%D0\%B8\%D0\%A2/GaT.pdf}
\date{}

\DeclareMathOperator{\diam}{diam}
\DeclareMathOperator{\Int}{Int}
\DeclareMathOperator{\Ext}{Ext}
\DeclareMathOperator{\Cl}{Cl}
\DeclareMathOperator{\Fr}{Fr}
\DeclareMathOperator{\SCl}{SCl}
\DeclareMathOperator{\Homeo}{Homeo}
\DeclareMathOperator*{\bigtimes}{\text{\raisebox{-6pt}{\scalebox{3}{$\times$}}}}
\newcommand{\FAC}{\ensuremath{\mathrm{FAC}}\xspace}
\newcommand{\SAC}{\ensuremath{\mathrm{SAC}}\xspace}
\newcommand{\T}{\ensuremath{\mathrm{T}}\xspace}

\begin{document}
    \maketitle

    \listoftodos[TODOs]

    \tableofcontents

    \vspace{2em}
    \todo[inline]{Разделы!}

    Литература:
    \begin{itemize}
        \item Виро О.Я., Иванов О.А., Нецветаев Н.Ю., Харламов В.М., ``Элементарная топология'', М.:МЦНМО, 2012.
        \item Коснёвски Чес, ``Начальный курс алгебраической топологии'', М.:Мир, 1983.
        \item Ю.Г. Борисович, Н.М. Близняков, Я.А. Израилевич, Т.Н. Фоменко, ``Введение в топологию'', М.:Наука. Физматлит, 1995.
        \item James Munkres, Topology.
    \end{itemize}

    \begin{definition}
        Функция $d: X \times X \to \RR_+$ называется \emph{метрикой} (или \emph{расстоянием}) в множестве $X$, если:
        \begin{itemize}
            \item $d(x, y) = 0 \Leftrightarrow x = y$;
            \item $d(x, y) = d(y, x)$;
            \item $d(x, z) \leqslant d(x, y) + d(y, z)$ (``неравенство треугольника'').
        \end{itemize}
        Пара $(X, d)$, где $d$ --- метрика в $X$, называется \emph{метрическим пространством}.
    \end{definition}

    \begin{example}
        Пусть $X$ --- произвольное множество. Тогда метрика
        \[
            d(x, y) := 
                \begin{cases}
                    1& \text{если $x \neq y$}\\
                    0& \text{если $x = y$}
                \end{cases}
        \]
        называется \emph{дискретной} метрикой на множестве $X$.
    \end{example}

    \begin{example}\ 
        \begin{itemize}
            \item $X := \RR$, тогда $d(x, y) := |x-y|$ --- метрика.
            \item $X := \RR^n$, $x = (x_1, \dots, x_n)$, $y = (y_1, \dots, y_n)$. Тогда
                \[d(x, y) := \sqrt{(x_1 - y_1)^2 + \dots + (x_n + y_n)^2}\]
                называется \emph{евклидовой} метрикой.
            \item $X := \RR^n$, $d(x, y) := \max_{i = 1}^n |x_i - y_i|$
            \item $X := \RR^n$, $d(x, y) := \sum_{i = 1}^n |x_i - y_i|$
            \item $X := C[0; 1]$, $d(x(t), y(t)) = \max_{t \in [0; 1]} |x(t) - y(t)|$. $(X, d)$ называют \emph{пространством непрерывных функций}.
        \end{itemize}
    \end{example}

    \begin{definition}
        Пусть $(X, d)$ --- метрическое пространство. Сужение функции $d$ на $Y \times Y$ является метрикой в $Y$. Метрическое пространство $(Y, d|_{Y\times Y})$ называется \emph{подпространством} пространства $(X, d)$.
    \end{definition}

    \begin{theorem}\label{generalised_metric_spaces_multiplication_theorem}
        Пусть дана $g: \RR_+ \times \RR_+ \to \RR_+$, что
        \begin{itemize}
            \item $\forall x, y \in \RR_+\quad g(x, y) = 0 \leftrightarrow x = y = 0$;
            \item $\forall x, y, d \in \RR_+\quad g(x + d, y) \geqslant g(x, y) \wedge g(x, y + d) \geqslant g(x, y)$;
            \item $\forall x_1, y_1, x_2, y_2 \in \RR_+ \quad g(x_1 + x_2, y_1 + y_2) \leqslant g(x_1, y_1) + g(x_2, y_2)$.
        \end{itemize}
        Тогда для любых метрических пространств $(X, d_X)$ и $(Y, d_Y)$ функция
        \[d_{X \times Y}((x_1, y_1), (x_2, y_2)) := g(d_X(x_1, x_2), d_Y(y_1, y_2))\]
        будет метрикой на $X \times Y$.
    \end{theorem}

    \begin{proof}
        Проверим, что $d_{X \times Y}$ --- метрика.
        \begin{itemize}
            \item $\forall x_1, x_2 \in X, y_1, y_2 \in Y$
                \begin{align*}
                    d_{X \times Y}((x_1, y_1), (x_2, y_2)) = 0\quad
                    &\longleftrightarrow\quad g(d_X(x_1, x_2), d_Y(y_1, y_2)) = 0\\
                    &\longleftrightarrow\quad d_X(x_1, x_2) = 0 \wedge d_Y(y_1, y_2) = 0\\
                    &\longleftrightarrow\quad x_1 = x_2 \wedge y_1 = y_2
                \end{align*}
            \item $\forall x_1, x_2 \in X, y_1, y_2 \in Y$
                \begin{multline*}
                    d_{X \times Y}((x_1, y_1), (x_2, y_2))\\
                    = g(d_X(x_1, x_2), d_Y(y_1, y_2)) = g(d_X(x_2, x_1), d_Y(y_2, y_1))\\
                    = d_{X \times Y}((x_2, y_2), (x_1, y_1))
                \end{multline*}
            \item $\forall x_1, x_2, x_3 \in X, y_1, y_2, y_3 \in Y$
                \begin{multline*}
                    d_{X \times Y}((x_1, y_1), (x_3, y_3))\\
                    = g(d_X(x_1, x_3), d_Y(y_1, y_3))\\
                    \leqslant g(d_X(x_1, x_2) + d_X(x_2, x_3), d_Y(y_1, y_2) + d_Y(y_2, y_3))\\
                    \leqslant g(d_X(x_1, x_2), d_Y(y_1, y_2)) + g(d_X(x_2, x_3), d_Y(y_2, y_3))\\
                    = d_{X \times Y}((x_1, y_1), (x_2, y_2)) + d_{X \times Y}((x_2, y_2), (x_3, y_3))
                \end{multline*}
        \end{itemize}
    \end{proof}

    \begin{corollary}
        Для любых метрических пространств $(X, d_X)$ и $(Y, d_Y)$ пара $(X \times Y, d_{X \times Y})$, где
        \[d_{X \times Y} := \sqrt{d_X(x_1, x_2)^2 + d_Y(y_1, y_2)^2}\]
        есть метрическое пространство.
    \end{corollary}

    \begin{proof}
        Необходимо лишь проверить, что $g(x, y) := \sqrt{x^2 + y^2}$ удовлетворяет условиям теоремы.
        \begin{itemize}
            \item $\forall x, y \in \RR_+\quad \sqrt{x^2 + y^2} \leftrightarrow x^2 + y^2 = 0 \leftrightarrow x = 0 = y$.
            \item $\forall x, y, d \in \RR_+\quad x+d \geqslant x \Rightarrow (x+d)^2 \geqslant x^2 \Rightarrow (x+d)^2 + y^2 \geqslant x^2 + y^2 \Rightarrow \sqrt{(x+d)^2 + y^2} \geqslant \sqrt{x^2 + y^2}$; для $y$ аналогично.
            \item $\forall x_1, y_1, x_2, y_2 \in \RR_+$ по неравенству Коши-Буняковского-Шварца
            \begin{gather*}
                (x_1y_2 - x_2y_1)^2 \geqslant 0\\
                x_1^2y_2^2 + x_2^2y_1^2 \geqslant 2x_1x_2y_1y_2\\
                (x_1^2 + y_1^2)(x_2^2 + y_2^2) \geqslant (x_1x_2 + y_1y_2)^2\\
                (x_1^2 + y_1^2) + 2\sqrt{(x_1^2 + y_1^2)(x_2^2 + y_2^2)} + (x_2^2 + y_2^2) \geqslant (x_1^2 + y_1^2) + 2(x_1x_2 + y_1y_2) + (x_2^2 + y_2^2)\\
                \left(\sqrt{x_1^2 + y_1^2} + \sqrt{x_2^2 + y_2^2}\right)^2 \geqslant (x_1 + x_2)^2 + (y_1 + y_2)^2\\
                \sqrt{x_1^2 + y_1^2} + \sqrt{x_2^2 + y_2^2} \geqslant \sqrt{(x_1 + x_2)^2 + (y_1 + y_2)^2}
            \end{gather*}
        \end{itemize}
    \end{proof}

    \begin{remark}
        Если $g$ ассоциативна (например, $g(x, y) := \sqrt{x^2 + y^2}$; она заодно коммутативна), то аналогично можно определить метрику на $X_1 \times X_2 \times \dots \times X_n = (X_1 \times (X_2 \times (\dots \times X_n)\dots))$.

        Таким образом евклидова метрика есть метрика, так как её можно получить, применяя $g(x, y) := \sqrt{x^2 + y^2}$ к пространствам $(X_i, d_i) = (\RR, d_{\RR})$ (где $d_{\RR}(x, y) = |x-y|$).
    \end{remark}

    \begin{definition}
        Пусть 
        Для $g(x, y) := \sqrt{x^2 + y^2}$ из последней теоремы пространство $(X \times Y, d_{X \times Y})$ называется \emph{(декартовым) произведением} метрических пространств $(X, d_X)$ и $(Y, d_Y)$. Аналогично определяется произведение конечного числа пространств.
    \end{definition}

    \begin{remark}
        На роль $g(x, y)$ подходят следующие функции:
        \begin{itemize}
            \item $(x^\alpha + y^\alpha)^{1/\alpha}$ для всех $\alpha \geqslant 1$;
            \item $\max(x, y)$.
        \end{itemize}
        А следующие функции уже не подходят:
        \begin{itemize}
            \item $(x^\alpha + y^\alpha)^{1/\alpha}$ для всех $\alpha < 1$ (даже для отрицательных);
            \item $\min(x, y)$;
            \item $x \cdot y$ и $x / y$.
        \end{itemize}
    \end{remark}

    \begin{definition}
        Пусть $(X, d)$ --- метрическое пространство, $a \in X$, $r \in \RR$, $r > 0$. Тогда:
        \begin{itemize}
            \item $B_r(a) := \{x \in X \mid d(a, x) < r\}$ --- \emph{(открытый) шар пространства $(X, d)$ с центром в точке $a$ и радиусом $r$};
            \item $\overline{B}_r(a) = D_r(a) := \{x \in X \mid d(a, x) \leqslant r\}$ --- \emph{замкнутой шар пространства $(X, d)$ с центром в точке $a$ и радиусом $r$};
            \item $S_r(a) := \{x \in X \mid d(a, x) = r\}$ --- \emph{сфера пространства $(X, d)$ с центром в точке $a$ и радиусом $r$}.
        \end{itemize}
    \end{definition}

    \begin{definition}
        Пусть $(X, d)$ --- метрическое пространство, $A \subseteq X$. Множество $A$ называется \emph{открытым} в метрическом пространстве, если
        \[\forall a \in A\ \exists r > 0: B_r(a) \subseteq A\]
    \end{definition}

    \begin{theorem}В любом метрическом пространстве $(X, d)$
        \begin{enumerate}
            \item $\varnothing$ и $X$ открыты;
            \item для всяких $a \in X$ и $r > 0$ открытый шар $B_r(a)$ открыт; 
            \item объединение любого семейства открытых множеств открыто;
            \item пересечение конечного семейства открытых множеств открыто.
        \end{enumerate}
    \end{theorem}

    \begin{proof}\ 
        \begin{enumerate}
            \item Очевидно.
            \item Для всякого $x \in B_r(a)$ верно, что $B_{r-d(x, a)}(x) \subseteq B_r(a)$, откуда утверждение очевидно следует.
            \item Пусть дано семейство открытых множеств $\Sigma$. Пусть также $I = \bigcup \Sigma$. Для любого $x \in I$ верно, что существует $J \in \Sigma$, что $x \in J$, а значит есть $r > 0$, что $B_r(x) \subseteq J \subseteq I$, т.е. $x$ --- внутренняя точка $I$. Таким образом $I$ открыто.
            \item Пусть $I = \bigcap_{i = 1}^n I_i$. Тогда для любого $x \in I$ верно, что существуют $r_1, \dots, r_n > 0$, что $B_{r_i}(x) \subseteq I_n$, значит $B_{\min r_i} \subseteq I$, значит $x$ --- внутренняя точка $I$. Таким образом $I$ открыто.
        \end{enumerate}
    \end{proof}

    \begin{definition}
        Пусть $X$ --- некоторое множество. Рассмотрим набор $\Omega$ его подмножеств, для которого:
        \begin{enumerate}
            \item $\varnothing, X \in \Omega$;
            \item объединение любого семейства множеств из $\Omega$ лежит в $\Omega$;
            \item пересечение любого конечного семейства множеств, принадлежащих $\Omega$, также принадлежит $\Omega$.
        \end{enumerate}
        В таком случае:
        \begin{itemize}
            \item $\Omega$ --- \emph{топологическая структура} или просто \emph{топология} в множестве $X$;
            \item множество $X$ с выделенной топологической структурой $\Omega$ (т.е.пара $(X, \Omega)$) называется \emph{топологическим пространством};
            \item элементы множества $\Omega$ называются \emph{открытыми множествами} пространства $(X, \Omega)$.
        \end{itemize}
    \end{definition}

    \begin{example}\ 
        \begin{itemize}
            \item Если $\Omega$ --- множество открытых множеств в метрическом пространстве $(X, d)$, то $(X, \Omega)$ --- топологическое пространство. Таким образом любое метрическое пространство можно отождествлять с соответствующим топологическим пространством.
            \item Топология, индуцированная евклидовой метрикой в $\RR^n$, называется \emph{стандартной}.
            \item $\Omega := 2^X$ --- \emph{дискретная} топология на произвольном множестве $X$. Именно она порождается дискретной метрикой на $X$.
            \item $\Omega := \{\varnothing, X\}$ --- \emph{антидискретная} топология на произвольном множестве $X$.
            \item $X := \RR$, $\Omega := \{(a; +\infty) : a \in \RR\} \cup \{\RR\} \cup \{\varnothing\}$. Такая топология называется \emph{стрелкой}.
            \item $\Omega = \{\varnothing\} \cup \{A \in X: |X\setminus A| \in \NN\}$ --- топология \emph{конечных дополнений} на произвольном множестве $X$.
        \end{itemize}
    \end{example}

    \begin{definition}
        Множество $F \subseteq X$ \emph{замкнуто} в топологическом пространстве $(X, \Sigma)$, если его дополнение $X \setminus F$ открыто (т.е. если $X \setminus F \in \Sigma$).
    \end{definition}

    \begin{theorem}
        В любом топологическом пространстве $X$
        \begin{itemize}
            \item $\varnothing$ и $X$ --- замкнуты;
            \item объединение конечного набора замкнутых множеств замкнуто;
            \item пересечение любого набора замкнутых множеств замкнуто.
        \end{itemize}
    \end{theorem}

    \begin{theorem}
        Пусть $U$ --- открыто, а $V$ --- замкнуто в $(X, \Omega)$. Тогда:
        \begin{itemize}
            \item $U \setminus V$ открыто;
            \item $V \setminus U$ замкнуто.
        \end{itemize}
    \end{theorem}

    \begin{definition}
        Пусть $(X, \Omega)$ --- топологическое пространство и $A \subseteq X$. Тогда \emph{внутренностью} множества $A$ называется
        объединение всех открытых подмножеств $A$:
        \[\Int(A) := \bigcup_{\substack{U \in \Omega\\U \subseteq A}} U\]
    \end{definition}

    \begin{theorem}\ 
        \begin{itemize}
            \item $\Int(A)$ --- открытое множество.
            \item $\Int(A) \subseteq A$.
            \item $B\text{ --- открыто} \wedge B \subseteq A \Rightarrow B \subseteq \Int(A)$.
            \item $A = \Int(A) \Leftrightarrow A \text{ --- открыто}$.
            \item $\Int(\Int(A)) = \Int(A)$.
            \item $A \subseteq B \Rightarrow \Int(A) \subseteq \Int(B)$.
            \item $\Int(\bigcap_{k=1}^n A_k) = \bigcap_{k=1}^n \Int(A_k)$.
            \item $\Int(\bigcup_{A \in \Sigma} A) \supseteq \bigcup_{A \in \Sigma} \Int(A)$.
        \end{itemize}
    \end{theorem}

    \begin{definition}
        \emph{Окрестность} точки $a$ в топологическом пространстве $X$ --- открытое множество в $X$, содержащее $a$.

        Точка $a$ топологического пространства $X$ называется \emph{внутренней точкой} множества $A \subseteq X$, если $A$ содержит как подмножество некоторую окрестность $a$.
    \end{definition}

    \begin{theorem}\ 
        \begin{itemize}
            \item Множество открыто тогда и только тогда, когда все его точки внутренние.
            \item Внутренность множества есть множество всех его внутренних точек.
        \end{itemize}
    \end{theorem}

    \begin{proof}\ 
        \begin{itemize}
            \item
                \begin{itemize}
                    \item[($\Rightarrow$)] Пусть $A$ открыто, а $a \in A$. Тогда $A$ --- та самая окрестность $a$, которая является подмножеством $A$, поэтому $a$ --- внутренняя точка $A$.
                    \item[($\Leftarrow$)] Пусть каждая точка $A$ внутренняя. Тогда для каждого $a \in A$ определим окрестность $I_a$, лежащую в $A$ как подмножество (такая есть по определению). Тем самым $A = \bigcup_{a \in A} I_a$, т.е. $A$ есть объединение открытых множеств, следовательно открытое множество.
                \end{itemize}
            \item
                \begin{itemize}
                    \item[($\subseteq$)] Пусть $a \in \Int(A)$. Вспомним, что $\Int(A)$ --- открытое подмножество $A$. Следовательно, $a$ --- внутренняя точка $A$.
                    \item[($\supseteq$)] Пусть $a$ --- внутренняя точка $A$. Следовательно есть открытое $I$, что $a \in I \subseteq A$, следовательно $I \subseteq \Int(A)$, а значит $a \in \Int(A)$.
                \end{itemize}
        \end{itemize}
    \end{proof}

    \begin{definition}
        Пусть $(X, \Omega)$ --- топологическое пространство, а $A \subseteq X$. \emph{Замыканием} множества $A$ называется пересечение всех замкнутых пространств, содержащих $A$ как подмножество:
        \[\Cl(A) := \bigcap_{\substack{X \setminus V \in \Omega\\V\supseteq A}} V\]
    \end{definition}

    \begin{theorem}\ 
        \begin{itemize}
            \item $\Cl(A)$ --- замкнутое множество.
            \item $\Cl(A) \supseteq A$.
            \item $B\text{ --- замкнуто} \wedge B \supseteq A \Rightarrow B \supseteq \Cl(A)$.
            \item $A = \Cl(A) \Leftrightarrow A \text{ --- замкнуто}$.
            \item $\Cl(\Cl(A)) = \Cl(A)$.
            \item $A \subseteq B \Rightarrow \Cl(A) \subseteq \Cl(B)$.
            \item $\Cl(\bigcup_{k=1}^n A_k) = \bigcup_{k=1}^n \Cl(A_k)$.
            \item $\Cl(\bigcap_{A \in \Sigma} A) \subseteq \bigcap_{A \in \Sigma} \Cl(A)$.
            \item $\Cl(A) \sqcup \Int(X \setminus A) = X$.
        \end{itemize}
    \end{theorem}

    \begin{definition}
        Пусть $X$ --- топологическое пространство, $A \subseteq X$ и $b \in X$. Точка $b$ называется \emph{точкой прикосновения} множества $A$, если всякая её окрестность пересекается с $A$.
    \end{definition}

    \begin{theorem}\ 
        \begin{itemize}
            \item Множество замкнуто тогда и только тогда, когда оно является множеством своих точек прикосновения.
            \item Замыкание множества есть множество всех его точек прикосновения.
        \end{itemize}
    \end{theorem}

    \begin{definition}
        Пусть $X$ --- топологическое пространство, $A \subseteq X$ и $a \in X$.

        \emph{Граница} множества $A$ --- разность замыкания и внутренности $A$: $\Fr(A) := \Cl(A) \setminus \Int(A)$.

        Точка $a$ --- \emph{граничная точка} множества $A$, если всякая её окрестность пересекается с $A$ и с $X \setminus A$.
    \end{definition}

    \begin{theorem}
        Граница множества совпадает с множеством его граничных точек.
    \end{theorem}

    \begin{theorem}\ 
        \begin{itemize}
            \item $\Fr(A)$ замкнуто.
            \item $\Fr(A) = \Fr(X \setminus A)$.
            \item $A\text{ замкнуто} \Leftrightarrow A \supseteq \Fr(A)$.
            \item $A\text{ открыто} \Leftrightarrow A \cap \Fr(A) = \varnothing$.
        \end{itemize}
    \end{theorem}

    \begin{definition}
        Пусть $X$ --- топологическое пространство, $A \subseteq X$ и $a \in X$.

        $a$ --- \emph{предельная точка} $A$, если в любой окрестности $a$ есть точка $A \setminus \{a\}$.

        $a$ --- \emph{изолированная точка} $A$, если $a \in A$ и есть окрестность $a$ без точка $A \setminus \{a\}$.
    \end{definition}

    \begin{theorem}\ 
        \begin{itemize}
            \item $b\text{ --- предельная} \Rightarrow b\text{ --- точка прикосновения}$.
            \item $\Cl(A) = \{\text{внутренние точки $A$}\} \sqcup \{\text{граничные точки $A$}\}$.
            \item $\Cl(A) = \{\text{предельные точки $A$}\} \sqcup \{\text{изолированные точки $A$}\}$.
        \end{itemize}
    \end{theorem}

    \begin{definition}
        Пусть $\Omega_1$ и $\Omega_2$ --- топологии на $X$. Тогда если $\Omega_1 \subseteq \Omega_2$, то говорят, что $\Omega_1$ \emph{слабее (грубее)} $\Omega_2$, а $\Omega_2$ \emph{сильнее (тоньше)} $\Omega_1$.
    \end{definition}

    \begin{example}
        Из всех топологий на $X$ антидискретная --- самая грубая, а дискретная --- самая тонкая.
    \end{example}

    \begin{theorem}\label{metric_generated_topologies_comparation_theorem}
        Топология метрики $d_1$ грубее топологии метрики $d_2$ тогда и только тогда, когда в любом шаре метрики $d_1$ содержится шар метрики $d_2$ с тем же центром.
    \end{theorem}

    \begin{proof}
        \begin{itemize}
            \item[($\Rightarrow$)] Пусть топология метрики $d_1$ грубее топологии метрики $d_2$. Тогда любой шар $B_r^{d_1}(a)$ открыт в $d_2$, следовательно по определению открытости есть шар $B_q^{d_2}(a) \subseteq B_r^{d_1}(a)$.
            \item[($\Leftarrow$)] Пусть в любом шаре метрики $d_1$ содержится шар метрики $d_2$ с тем же центром. Возьмём любое открытое в $d_1$ множество $U$. Тогда для всякой точки $a \in U$ есть шар $B_r^{d_1}(a) \subseteq U$. При этом есть шар $B_q^{d_2}(a) \subseteq B_r^{d_1}(a)$, таким образом $a$ --- внутренняя точка $U$ в $d_2$. Следовательно $U$ открыто в $d_2$.
        \end{itemize}
    \end{proof}

    \begin{corollary}
        Если $d_1$ и $d_2$ --- метрики на $X$ и $d_1 \leqslant d_2$, то топология $d_1$ грубее топологии $d_2$. 
    \end{corollary}

    \begin{definition}
        Две метрики на одном множестве называются \emph{эквивалентными}, если они порождают одну топологию.
    \end{definition}

    \begin{lemma}
        Пусть $(X, d)$ --- метрическое пространство. Тогда для всякого $C > 0$ функция $C \cdot d$ --- метрика на $X$, эквивалентная $d$. 
    \end{lemma}

    \begin{corollary}
        Если для метрик $d_1$ и $d_2$ на $X$ есть такое $C > 0$, что $d_1 \leqslant C d_2$, то $d_1$ грубее $d_2$.
    \end{corollary}

    \begin{definition}
        Метрики $d_1$ и $d_2$ на одном множестве называются \emph{липшицево эквивалентными}, если существуют $c, C > 0$, что $c \cdot d_1 \leqslant d_2 \leqslant C \cdot d_1$.
    \end{definition}

    \begin{theorem}
        Липшицево эквивалентные метрики просто эквивалентны.
    \end{theorem}

    \begin{definition}
        Топологическое пространство \emph{метризуемо}, если есть метрика, её порождающая.
    \end{definition}

    \begin{definition}
        \emph{База} топологии $\Omega$ --- такое семейство $\Sigma$ открытых множеств, что всякое открытое $U$ представимо в виде объединения множеств из $\Sigma$.
        \[\Sigma \subseteq \Omega\text{ --- база} \Longleftrightarrow \forall U \in \Omega\; \exists \Lambda \subseteq \Sigma:\quad U = \bigcup_{W \in \Lambda} W\]
    \end{definition}

    \begin{definition}
        Множество $\Gamma$ подмножеств множества $X$ называются его \emph{покрытием}, если $X := \bigcup_{A \in \Gamma} A$. Часто покрытие записывают в виде $\Gamma = \{A_i\}_{i \in I}$.
    \end{definition}

    \begin{theorem}[второе определение базы]
        Пусть $(X, \Omega)$ --- топологическое пространство и $\Sigma \subseteq \Omega$. Тогда $\Sigma$ --- база топологии $\Omega$ тогда и только тогда, когда для любой точки $a$ любого открытого множества $U$ есть окрестность из $\Sigma$, лежащая в $U$ как подмножество.
    \end{theorem}

    \begin{definition}
        Пусть $(X, \Omega)$ --- топологическое пространство, $a \in X$ и $\Lambda \subseteq \Omega$. $\Lambda$ называется \emph{базой топологии (базой окрестности) в точке $a$}, если:
        \begin{enumerate}
            \item $\forall U \in \Lambda\; a \in U$;
            \item $\forall\text{ окрестности }U\text{ точки }a\; \exists V_a \in \Lambda:\; V_a \subseteq U$.
        \end{enumerate}
    \end{definition}

    \begin{theorem}\ 
        \begin{itemize}
            \item Если $\Sigma$ --- база топологии, то для всякой точки $a \in X$ множество $\Sigma_a := \{U \in \Sigma \mid a \in U\}$ --- база топологии в точке $a$.
            \item Пусть для каждой точки $a \in X$ определена база топологии $\Sigma_a$ в ней. Тогда $\bigcup_{a \in X} \Sigma_a$ --- база топологии.
        \end{itemize}
    \end{theorem}

    \begin{theorem}
        Пусть $\Sigma$ --- семейство подмножеств $X$. Тогда есть не более одной топологии, для которой $\Sigma$ является базой.
    \end{theorem}

    \begin{proof}
        Предположим противное: пусть $\Omega_1$ и $\Omega_2$ --- различные топологии на $X$, для которых $\Sigma$ является базой. По определению базы для всякого $U \in \Omega_1$ есть семейство $\Gamma \subseteq \Sigma$, что $U = \bigcup_{A \in \Gamma} A$; но поскольку $\Gamma \subseteq \Sigma \subseteq \Omega_2$, то всякое $A \in \Gamma$ лежит в $\Omega_2$, а значит $U$ тоже лежит в $\Omega_2$. Таким образом $\Omega_1 \subseteq \Omega_2$; аналогично наоборот, следовательно $\Omega_1 = \Omega_2$ --- противоречие.

        Таким образом для всякого $\Sigma$ будет не более одной топологии, где для которой оно будет базой.
    \end{proof}

    \begin{corollary}
        Пусть $\Sigma_1$ и $\Sigma_2$ --- базы топологий $\Omega_1$ и $\Omega_2$ на одном и том же множестве. Тогда если $\Sigma_1 = \Sigma_2$, то и $\Omega_1 = \Omega_2$.
    \end{corollary}

    \begin{theorem}[критерий базы]
        Пусть $X$ --- произвольное множество, а $\Sigma$ --- его покрытие. $\Sigma$ --- база некоторой топологии на $X$ тогда и только тогда, когда для всяких $A, B \in \Sigma$ есть семейство $\Lambda \subseteq \Sigma$, что $A \cap B = \bigcup_{S \in \Lambda} S$.
    \end{theorem}

    \begin{proof}
        \begin{itemize}
            \item[($\Rightarrow$)] Если $\Sigma$ --- база, то для всяких $A, B \in \Sigma$ множество $A \cap B$ открыто, а поэтому представляется как объединение некоторого подсемейства $\Sigma$.
            
            \item[($\Leftarrow$)] Рассмотрим топологию $\Omega$, образованную всевозможными объединениями множеств из $\Sigma$, т.е.
            \[\Omega := \left\{\bigcup_{S \in \Lambda} S \mid \Lambda \subseteq \Sigma\right\}\]
            Проверим, что это действительно топология.
            \begin{enumerate}
                \item $\Sigma$ --- покрытие, поэтому $X = \bigcup_{S \in \Sigma} S \in \Omega$. Также рассматривая $\Lambda = \varnothing$, получаем, что $\bigcup_{S \in \Lambda} S = \varnothing \in \Omega$.
                \item Пусть $\Phi \subseteq \Omega$. Тогда для каждого $S \in \Phi$ есть семейство $\Lambda_S \subseteq \Sigma$, его образующее, т.е. $S = \bigcup_{T \in \Lambda_S} T$. В таком случае $\Lambda := \bigcup_{S \in \Phi} \Lambda_S$ является подмножеством $\Sigma$, а тогда
                \[\bigcup_{S \in \Phi} S = \bigcup_{S \in \Phi} \bigcup_{T \in \Lambda_S} T = \bigcup_{T \in \Lambda} T \in \Omega\]
                \item Пусть $U, V \in \Omega$. Тогда существуют $M, N \subseteq \Sigma$, что $U = \bigcup_{S \in M} S$ и $V = \bigcup_{S \in N} S$. Также для каждой $P = (A, B) \in M \times N$ существует $\Lambda_P \subseteq \Sigma$, что $A \cap B = \bigcup_{S \in \Lambda_P}$. Пусть $\Lambda := \bigcup_{P \in M \times N} \Lambda_S$. Понятно, что $\Lambda \subseteq \Sigma$. Следовательно
                \[U \cap V = \left(\bigcup_{A \in M} A\right) \cap \left(\bigcup_{B \in N} B\right) = \bigcup_{(A, B) \in M \times N} A \cap B = \bigcup_{P \in M \times N} \bigcup_{S \in \Lambda_P} S = \bigcup_{S \in \Lambda} S \in \Omega\]
            \end{enumerate}
        \end{itemize}
    \end{proof}

    \begin{definition}
        \emph{Предбаза} --- семейство $\Delta$ открытых множеств в пространстве $(X, \Omega)$, что $\Omega$ --- наименьшая топология по включению топология, содержащая $\Delta$.
    \end{definition}

    \begin{theorem}
        Любое семейство $\Delta$ подмножеств множества $X$ является предбазой некоторой топологии.
    \end{theorem}

    \begin{proof}
        Определим
        \[\Sigma := \{X\} \cup \left\{\bigcap_{A \in W} A \mid W \subseteq \Delta \wedge |W| \in \NN\right\}\]
        Заметим, что $\Delta \subseteq \Sigma$. Действительно, для всякого $A \in \Delta$ семейство $W := \{A\}$ является подмножеством $\Delta$, следовательно $A = \bigcap_{T \in W} T \in \Sigma$.

        Покажем, что любая топология, которая содержит как подмножество $\Delta$, содержит и $\Sigma$ как подмножество. Действительно, пусть $A \in \Sigma$ (будем считать, что $A$ --- не $X$ и не $\varnothing$; иначе утверждение очевидно). Тогда есть конечное семейство $W \subseteq \Delta$, что $A = \bigcap_{T \in W} T$. Пусть $\Omega$ --- любая топология, содержащая $\Delta$ как подмножество. Тогда $W \subseteq \Omega$, а следовательно $A = \bigcap_{T \in W} T \in \Omega$. Таким образом $\Sigma \subseteq \Omega$. Поэтому для топология, для которой $\Sigma$ будет предбазой, $\Delta$ тоже будет предбазой.

        Покажем, что $\Sigma$ удовлетворяет критерию базы.
        \begin{itemize}
            \item $X \in \Sigma$, значит $\Sigma$ --- покрытие $X$.
            \item Пусть $A, B \in \Sigma$. Если $A = X$, то $A \cap B = B = \bigcup_{T \in W} T$, где $W := \{B\} \subseteq \Sigma$. Если $A = \varnothing$, то $A \cap B = \varnothing = \bigcup_{T \in W} T$, где $W := \varnothing \subseteq \Sigma$. Аналогично, если $B$ есть $X$ или $\varnothing$. Иначе есть непустые $V, U \subseteq \Delta$, что $A = \bigcap_{T \in V} T$, а $B = \bigcap_{T \in U} T$. Следовательно $A \cap B = \bigcap_{T \in V \cup U} T$. Но поскольку $V \cup U \subseteq \Delta$, то $A \cap B \in \Sigma$. Таким образом $A \cap B = \bigcup_{T \in W} T$, где $W := \{A \cap B\} \subseteq \Sigma$.
        \end{itemize}

        Рассмотрим
        \[\Omega := \left\{\bigcup_{S \in \Lambda} S \mid \Lambda \subseteq \Sigma\right\}\]
        По теореме о критерии базы $\Omega$ --- топология, где $\Sigma$ --- база. С другой стороны $\Omega$ --- множество, которое содержится как подмножество в любой топологии, которая содержит как подмножество $\Sigma$. Следовательно $\Omega$ --- минимальное топология, содержащая как подмножество $\Sigma$, а значит и $\Delta$. Поэтому $\Delta$ --- предбаза в $\Omega$. 
    \end{proof}

    \begin{theorem}
        Пусть $(X, \Omega)$ --- топологическое пространство, а $A \subseteq X$. Тогда множество
        \[\Omega_A := \{U \cap A \mid U \in \Omega\}\]
        есть топология на $A$.
    \end{theorem}

    \begin{definition}
        Пусть $(X, \Omega)$ топологическое пространство, а $A \subseteq X$. Тогда
        \[\Omega_A := \{U \cap A \mid U \in \Omega\}\]
        --- топология, \emph{индуцированная} множеством $A$, а $(A, \Omega_A)$ --- подпространство $(X, \Omega)$.
    \end{definition}

    \begin{theorem}\ 
        \begin{itemize}
            \item Множества, открытые в подпространстве, не обязательно открыты в объемлющем пространстве.
            \item Открытые множества открытого подпространства открыты и во всём пространстве.
            \item Если $\Sigma$ --- база топологии $\Omega$, то
                \[\Sigma_A := \{U \cap A \mid U \in \Sigma\}\]
                --- база индуцированной топологии.
            \item Пусть $(X, \Omega)$ --- топологическое пространство и $B \subseteq A \subseteq X$. Тогда $(\Omega_A)_B = \Omega_B$, т.е. топология, которая индуцируется в $B$ топологией, индуцированной в $A$, совпадает с топологией,индуцированной непосредственно из $X$.
        \end{itemize}
    \end{theorem}

    \begin{definition}
        Пусть $X$, $Y$ --- топологические пространства. Отображение $f: X \to Y$ называется \emph{непрерывным}, если прообраз всякого открытого множества из $Y$ открыт в $X$.
    \end{definition}

    \begin{theorem}\ 
        \begin{itemize}
            \item Отображение непрерывно тогда и только тогда, когда прообраз замкнутого замкнут.
            \item Композиция непрерывных отображений непрерывно.
            \item Пусть $Z$ --- подпространство $X$, а $f: X \to Y$ непрерывно. Тогда $f{\mid}_Z: Z \to Y$ непрерывно.
            \item Пусть $Z$ --- подпространство $Y$, $f: X \to Y$ и $f(X) \subseteq Z$. Пусть $\widetilde f: X \to Z, x \mapsto f(x)$. Тогда $f$ непрерывна тогда и только тогда, когда $\widetilde f$ непрерывна.
        \end{itemize}
    \end{theorem}

    \begin{definition}
        Отображение $f: X \to Y$ называется непрерывным в точке $a \in X$, если для любой окрестности $U$ точки $f(a)$ существует такая окрестность $V$ точки $a$, что $f(V) \subseteq U$.
    \end{definition}

    \begin{theorem}
        Отображение $f: X \to Y$ непрерывно тогда и только тогда, когда оно непрерывно в каждой точке пространства $X$.
    \end{theorem}

    \begin{proof}
        \begin{itemize}
            \item[($\Rightarrow$)] Очевидно, $V = f^{-1}(U)$.
            \item[($\Leftarrow$)] Пусть $U \in \Omega_Y$. Тогда для всякого $a \in f^{-1}(U)$ есть окрестность $V_a$ точки $a$, что $V_a \subseteq f^{-1}(U)$. Следовательно любая точка $f^{-1}(U)$ внутренняя, а значит $f^{-1}(U)$ открыто.
        \end{itemize}
    \end{proof}

    \begin{theorem}
        Пусть $X$ и $Y$ --- топологические пространства, $a \in X$, $f: X \to Y$, $\Sigma_a$ --- база окрестностей в точке $a$ и $\Lambda_{f(a)}$ --- база окрестностей в точке $f(a)$. Тогда $f$ непрерывна в точке $a$ тогда и только тогда, когда для всякого $U \in \Lambda_{f(a)}$ есть $V \in \Sigma_a$, что $f(V) \subseteq U$.
    \end{theorem}

    \begin{proof}\ 
        \begin{itemize}
            \item[($\Rightarrow$)] Пусть $f$ непрерывна в $a$. Рассмотрим любое $U \in \Lambda_{f(a)}$. $U$ --- окрестность $f(a)$, соответственно есть $W$ --- окрестность $a$, что $f(W) \subseteq U$. Но тогда есть $V \in \Sigma_a$, что $V \subseteq W$. Тогда $V \in \Sigma_a$ и $f(V) \subseteq U$.
            \item[($\Leftarrow$)] Пусть для всякого $U \in \Lambda_{f(a)}$ найдётся $V \in \Sigma_a$, что $f(V) \subseteq U$. Рассмотрим любую окрестность $U$ точки $f(a)$. Тогда есть семейство $W \in \Lambda_{f(a)}$, что $W \subseteq U$. Следовательно найдётся $V \in \Sigma_a$, что $f(V) \subseteq W$, а следовательно $V$ --- окрестность $a$, и $f(V) \subseteq U$.
        \end{itemize}
    \end{proof}

    \begin{corollary}
        Пусть $X$, $Y$ --- метрические пространства, $a \in X$, $f: X \to Y$. Тогда
        \begin{enumerate}
            \item $f$ непрерывно в точке $a$ тогда и только тогда, когда
                \[\forall \varepsilon > 0\; \exists \delta > 0:\quad f(B_\delta(a)) \subseteq B_\varepsilon(f(a))\]
            \item $f$ непрерывно в точке $a$ тогда и только тогда, когда
                \[\forall \varepsilon > 0\; \exists \delta > 0:\quad d_X(x, a) < \delta \rightarrow d_Y(f(x), f(a)) < \varepsilon\]
        \end{enumerate}
    \end{corollary}

    \begin{definition}
        Пусть $X$, $Y$ --- метрические пространства. Отображение $f: X \to Y$ называется \emph{липшицевым}, если:
        \[\exists C > 0:\; \forall a, b \in X\quad d_Y(f(a), f(b)) \leqslant C \cdot d_X(a, b)\]
        Значение $C$ называют \emph{константой Липшица} отображения $f$.
    \end{definition}

    \begin{theorem}
        Всякое липшицево отображение непрерывно.
    \end{theorem}

    \begin{proof}
        Действительно,
        \[\forall \varepsilon > 0\qquad \delta := \frac{\varepsilon}{C}\quad \Longrightarrow\quad \Bigl(d_X(x, a) < \delta\quad \longrightarrow\quad d_Y(f(x), f(a)) \leqslant C \cdot d_X(x, a) < C \cdot \delta = \varepsilon \Bigr)\]
    \end{proof}

    \begin{example}\ 
        \begin{itemize}
            \item Пусть фиксирована точка $x_0$ в метрическом пространстве $(X, d)$. Тогда отображение
                \[f: X \to \RR,\; a \mapsto d(a, x0),\]
                непрерывно.
            \item Пусть $A$ --- непустое подмножество метрического пространства $(X, d)$. \emph{Расстоянием от точки $x \in X$ до множества $A$} называется число
                \[d(x, A) := \inf\{d(x, a): a \in A\}.\]
                Отображение
                \[f: X \to \RR,\; x \mapsto d(x, A),\]
                непрерывно.
            \item Метрика $d$ на множестве $X$ является непрерывным отображением $X \times X \to \RR$.
        \end{itemize}
    \end{example}

    \begin{definition}
        Покрытие $\Gamma$ топологического пространства $X$ называется \emph{фундаментальным}, если
        \[\forall U \subseteq X:\quad \Bigl(\forall A \in \Gamma\quad U \cap A \text{ открыто в } A\Bigr) \quad \longrightarrow \quad \Bigl(U \text{ открыто в } X\Bigr)\]
    \end{definition}

    \begin{lemma}
        Покрытие $\Gamma$ топологического пространства $X$ фундаментально тогда и только тогда, когда
        \[\forall V \subseteq X \quad \Bigl(\forall A \in \Gamma\quad V \cap A \text{ замкнуто в } A\Bigr) \quad \longrightarrow \quad \Bigl(U \text{ замкнуто в } X\Bigr)\]
    \end{lemma}

    \begin{proof}
        \begin{itemize}
            \item[($\Rightarrow$)] Пусть $\Gamma$ фундаментально. Рассмотрим $V \subseteq X$, что для всякого $A \in \Gamma$ множество $V \cap A$ замкнуто в $A$. Следовательно $(X \setminus V) \cap A$ открыто в $A$, а тогда по фундаментальности $\Gamma$ множество $X \setminus V$ открыто, а значит всё $V$ замкнуто.
            \item[($\Leftarrow$)] Аналогично, поменяв местами слова "открыто" и "замкнуто".
        \end{itemize}
    \end{proof}

    \begin{theorem}
        Пусть $X$, $Y$ --- топологические пространства, $\Gamma$ --- фундаментальное покрытие $X$ и $f: X \to Y$. Если сужение $f$ на всякое $A \in \Gamma$ непрерывно, то и само $f$ непрерывно.
    \end{theorem}

    \begin{proof}
        Рассмотрим любое открытое в $Y$ множество $U$. Если $A \in \Gamma$, то $f^{-1}(U) \cap A = (f{\mid}_A)^{-1}(U)$ открыто. А в таком случае из фундаментальности $\Gamma$ следует, что $f^{-1}(U)$ открыто. Таким образом $f$ непрерывно.
    \end{proof}

    \begin{definition}
        Покрытие топологического пространства называется
        \begin{itemize}
            \item \emph{открытым}, если оно состоит из открытых множеств;
            \item \emph{замкнутым} --- если из замкнутых;
            \item \emph{локально конечным} --- если каждая точка пространства обладает окрестностью, пересекающейся лишь с конечным числом элементов покрытия.
        \end{itemize}
    \end{definition}

    \begin{theorem}\ 
        \begin{enumerate}
            \item Всякое открытое покрытие фундаментально.
            \item Всякое конечное замкнутое покрытие фундаментально.
            \item Всякое локально конечное замкнутое покрытие фундаментально.
        \end{enumerate}
    \end{theorem}

    \begin{proof}
        Пусть $\Gamma$ --- данное покрытие.
        \begin{enumerate}
            \item Пусть дано $U \subseteq X$, что для всякого $A \in \Gamma$ множество $U \cap A$ открыто в $A$, а значит открыто в $X$. Тогда
                \[U = U \cap X = \bigcup_{A \in \Gamma} U \cap A\]
                есть объединение открытых множеств, а значит само открыто. Таким образом $\Gamma$ фундаментально.

            \item Пусть дано $U \subseteq X$, что для всякого $A \in \Gamma$ множество $U \cap A$ замкнуто в $A$, а значит замкнуто в $X$. Тогда
                \[U = U \cap X = \bigcup_{A \in \Gamma} U \cap A\]
                есть конечное объединение замкнутых множеств, а значит само замкнуто. Таким образом $\Gamma$ фундаментально.

            \item Пусть дано $U \subseteq X$, что для всякого $A \in \Gamma$ множество $U \cap A$ открыто в $A$. Рассмотрим некоторую точку $u \in U$ и её окрестность $V_u$, которая пересекается с конечным набором $\Gamma_u$ элементов из $\Gamma$. Тогда для всякого $A \in \Gamma_u$ множество
                \[U \cap A \cap V = (U \cap A) \cap (A \cap V)\]
                открыто в $V \cap A$. При этом
                \[\{V \cap A \mid A \in \Gamma_u\}\]
                --- конечное замкнутое покрытие, а значит $U \cap V$ открыто в $V$, а значит и в $X$. Таким образом $U \cap V$ --- окрестность $u$, а значит $u$ --- внутренняя точка $U$. Значит $U$ открыто.
        \end{enumerate}
    \end{proof}

    \begin{theorem}
        Пусть $(X, \Omega_X)$ и $(Y, \Omega_Y)$ --- топологические пространства. Тогда
        \[\Sigma := \{U \times V \mid U \in \Omega_X \wedge V \in \Omega_Y\}\]
        является базой топологии на $X \times Y$.
    \end{theorem}

    \begin{proof}
        Проверим критерий базы:
        \begin{enumerate}
            \item $X \in \Omega_X$, $Y \in \Omega_Y$, следовательно $X \times Y \in \Sigma$. Таким образом $\Sigma$ --- покрытие $X \times Y$.
            \item Пусть $U_1 \times V_1, U_2 \times V_2 \in \Sigma$. Тогда $U_1 \cap U_2 \in X$, $V_1 \cap V_2 \in Y$, а значит $(U_1 \times V_1) \cap (U_2 \times V_2) = (U_1 \cap U_2) \times (V_1 \cap V_2) \in \Sigma$.
        \end{enumerate}
        Таким образом $\Sigma$ --- база.
    \end{proof}

    \begin{definition}
        Пусть $(X, \Omega_X)$ и $(Y, \Omega_Y)$ --- топологические пространства, а $\Omega_{X \times Y}$ --- топология, порождённая базой $\Sigma$ из предыдущей теоремы. Тогда $(X \times Y, \Omega_{X \times Y})$ называется \emph{произведением} топологических пространств, а сама $\Omega_{X \times Y}$ называется \emph{стандартной} топологией.
    \end{definition}

    \begin{remark}
        По аналогии если $\Sigma_X$ и $\Sigma_Y$ --- базы топологий $\Omega_X$ и $\Omega_Y$ соответственно, то
        \[\Lambda := \{U \times V \mid U \in \Sigma_X \wedge V \in \Sigma_Y\}\]
        также являются базой стандартной топологии на $X \times Y$.
    \end{remark}

    \begin{definition}
        Обозначения:
        \begin{itemize}
            \item $X = \prod_{i \in I} X_i$– произведение топологических пространств.
            \item Элементами $X$ являются такие функции $x: I \to \bigcup_{i \in I} X_i$, что $x(i) \in X_i$.
            \item $p_i: X \to X_i$ --- координатная проекция, где $p_i(x) := x(i)$.
        \end{itemize}
    \end{definition}

    \begin{definition}
        Пусть $\{(X_i, \Omega_i)\}_{i \in I}$ --- семейство топологических пространств. \emph{Тихоновская топология} на $X = \prod_{i \in I} X_i$ задаётся предбазой, состоящей из всевозможных множеств вида $p_i^{-1}(U)$, где $i \in I$, а $U \subseteq \Omega_i$.
    \end{definition}

    \begin{remark}
        В случае конечного произведения тихоновская топология совпадает со стандартной.
    \end{remark}

    \begin{theorem}
        Пусть $(X, d_X)$ и $(Y, d_Y)$ --- метрические пространства, $\Omega_X$ и $\Omega_Y$ --- топологии в данных метрических пространствах. Рассмотрим две топологии:
        \begin{itemize}
            \item $\Omega_{X \times Y}$ --- топология-произведение топологий $\Omega_X$ и $\Omega_Y$;
            \item $\Omega_{\max}$ --- топология, порождённая произведением метрик по функции $g := \max$ (см. теорему \ref{generalised_metric_spaces_multiplication_theorem}).
        \end{itemize}
        Тогда эти топологии совпадают.
    \end{theorem}

    \begin{proof}
        Определим
        \[d_{\max}: (X \times Y) \times (X \times Y) \to \RR, ((x_1, y_1), (x_2, y_2)) \mapsto \max(d_X(x_1, x_2), d_Y(y_1, y_2))\]
        Таким образом $d_{\max}$ --- метрика, порождающая $\Omega_{\max}$.

        \begin{thlemma}
            \[B_r^{d_{\max}}((x, y)) = B_r^{d_X}(x) \times B_r^{d_Y}(y)\]
        \end{thlemma}

        \begin{proof}
            Очевидно.
        \end{proof}

        Вспомним, что
        \[
            \Sigma_X := \{B_r^{d_X}(x) \mid r > 0 \wedge x \in X\}
            \qquad \text{ и } \qquad
            \Sigma_Y := \{B_r^{d_Y}(y) \mid r > 0 \wedge y \in Y\}
        \]
        являются базами $\Omega_X$ и $\Omega_Y$. Следовательно
        \[\Sigma_{X \times Y} := \{U_X \times U_Y \mid U_X \in \Sigma_X \wedge U_Y \in \Sigma_Y\}\]
        является базой $\Omega_{X \times Y}$. Также заметим, что
        \[\Sigma_{\max} := \{B_r^{d_{\max}}((x, y)) \mid r > 0 \wedge x \in X \wedge y \in Y\} = \{B_r^{d_X}(x) \times B_r^{d_Y}(y) \mid r > 0 \wedge x \in X \wedge y \in Y\}\]
        является базой $\Omega_{\max}$. При этом несложно видеть, что $\Sigma_{\max} \subseteq \Sigma_{X \times Y}$, следовательно $\Omega_{\max}$ грубее $\Omega_{X \times Y}$. Осталось показать, что $\Sigma_{\max}$ порождает $\Sigma_{X \times Y}$, т.е. всякое $U \in \Sigma_{X \times Y}$ представимо в виде объединения некоторых множеств из $\Sigma_{\max}$.

        Пусть $U$ --- некоторый элемент $\Sigma_{X \times Y}$. Тогда есть некоторые $r_X, r_Y > 0$ и $(x, y) \in X \times Y$, что $U = B_{r_X}^{d_X}(x) \times B_{r_Y}^{d_Y}(y)$. Пусть $(x', y') \in U$, тогда $x' \in B_{r_X}^{d_X}(x)$. Следовательно $q_X := r_X - d_X(x, x') > 0$, а $B_{q_X}^{d_X}(x') \subseteq B_{r_X}^{d_X}(x)$; аналогично для $Y$. Пусть $q := \min(q_X, q_Y) > 0$. Тогда
        \[V := B_q^{d_X}(x') \times B_q^{d_Y}(y')\]
        --- окрестность $(x', y')$. При этом $V \subseteq U$. Значит $U$ представляется в виде объединения всех таких окрестностей для каждой точки $(x', y')$ из него. Но $V \in \Sigma_{\max}$, поэтому $\Sigma_{\max}$ порождает $\Sigma_{X \times Y}$. Значит топология, которая порождает $\Sigma_{\max}$, --- $\Omega_{\max}$ --- содержит как подмножество топологию, которую порождает $\Sigma_{X \times Y}$, --- $\Omega_{X \times Y}$.

        Таким образом $\Omega_{\max} = \Omega_{X \times Y}$.
    \end{proof}

    \begin{theorem}
        Пусть дана $g: \RR_+ \times \RR_+ \to \RR_+$, что
        \begin{itemize}
            \item $\forall x, y \in \RR_+\quad g(x, y) = 0 \leftrightarrow x = y = 0$;
            \item $\forall x, y, d \in \RR_+\quad g(x + d, y) \geqslant g(x, y) \wedge g(x, y + d) \geqslant g(x, y)$;
            \item $\forall x_1, y_1, x_2, y_2 \in \RR_+ \quad g(x_1 + x_2, y_1 + y_2) \leqslant g(x_1, y_1) + g(x_2, y_2)$;
            \item $\forall \alpha > 0\, \exists x, y > 0:\quad 0 < g(x, 0) < \alpha \wedge 0 < g(0, y) < \alpha$. 
        \end{itemize}
        Тогда для любых метрических пространств $(X, d_X)$ и $(Y, d_Y)$ функция
        \[
            d_g((x_1, y_1), (x_2, y_2)) = g(d_X(x_1, x_2), d_Y(y_1, y_2))
        \]
        является метрикой, эквивалентной метрике
        \[
            d_{\max}((x_1, y_1), (x_2, y_2)) = \max(d_X(x_1, x_2), d_Y(y_1, y_2))
        \]
    \end{theorem}

    \begin{proof}
        Заметим, что по теореме \ref{generalised_metric_spaces_multiplication_theorem} функция $d_{\max}$ является метрикой. С помощью теоремы \ref{metric_generated_topologies_comparation_theorem} имеем, что нужно показать, что в каждом шаре по одной метрик $d_{\max}$ и $d_g$ есть шар с тем же центром по другой метрики.

        Рассмотрим шар $B_r^{d_g}((x, y))$. Тогда по свойству $g$ есть $q_X > 0$, что $0 < g(q_X, 0) < r/2$; аналогично для $Y$. Следовательно для всех точек $x' \in B_{q_X}^{d_X}(x)$ и $y' \in B_{q_Y}^{d_Y}(y)$ верно, что
        \begin{multline*}
            d_g((x', y'), (x, y))\\
            = g(d_X(x', x), d_Y(y', y))\\
            \leqslant g(d_X(x', x), 0) + g(0, d_Y(y', y))\\
            \leqslant g(q_X, 0) + g(0, q_Y)\\
            < \frac{r}{2} + \frac{r}{2} = r
        \end{multline*}
        Пусть $q := \min(q_X, q_Y)$. Тогда
        \[B_q^{d_{\max}}((x, y)) = B_q^{d_X}(x) \times B_q^{d_Y}(y) \subseteq B_{q_X}^{d_X}(x) \times B_{q_Y}^{d_Y}(y) \subseteq B_r^{d_g}((x, y))\]
        Т.е. для каждого шара по $d_g$ нашёлся подшар по $d_{\max}$.

        \begin{thlemma}
            Для всякого $r > 0$ есть такое $q_X > 0$, что
            \[\forall x \in \RR_+\qquad g(x, 0) < q_X \longrightarrow x < r\]
            Аналогично для $Y$.
        \end{thlemma}

        \begin{proof}
            Рассмотрим $q_X := g(r, 0) > 0$. Тогда если $x \geqslant r$, то $g(x, 0) \geqslant g(r, 0) = q_X$. Следовательно
            \[\forall x \in \RR_+\qquad g(x, 0) < q_X \longrightarrow x < r\]
            Аналогично для $Y$.
        \end{proof}

        Рассмотрим шар $B_r^{d_{\max}}((x, y))$. Тогда определим $q_X$ и $q_Y$ по прошлой лемме для $r$ и координат $X$ и $Y$ соответственно. Пусть также $q := \min(q_X, q_Y)$ Тогда
        \begin{multline*}
            \forall (x', y') \in B_q^{d_g}((x, y))\\
                \left\{
                    \begin{aligned}
                        &g(d_X(x', x), 0) \leqslant g(d_X(x', x), d_Y(y', y)) = d_g((x', y'), (x, y)) < q \leqslant q_X\\
                        &g(0, d_Y(y', y)) \leqslant g(d_X(x', x), d_Y(y', y)) = d_g((x', y'), (x, y)) < q \leqslant q_Y
                    \end{aligned}
                \right.\\
            \Longrightarrow
            \left\{
                \begin{aligned}
                    &d_X(x', x) < r\\
                    &d_Y(y', y) < r
                \end{aligned}
            \right.\\
            \Longrightarrow d_{\max}((x', y'), (x, y)) = \max(d_X(x', x), d_Y(y', y)) < r\\
            \Longrightarrow (x', y') \in B_r^{d_{\max}}((x, y))
        \end{multline*}
    \end{proof}

    \begin{corollary}
        Произведения метрических пространств по функции $g(x, y) := (x^\alpha + y^\alpha)^{1/\alpha}$ для всякого $\alpha \geqslant 1$ даёт такую же топологию, что и произведение стандартных топологий на метрических пространствах. В случае $\alpha = 2$ мы имеем стандартное произведение пространств.
    \end{corollary}

    \begin{theorem}
        Пусть $X = \prod_{i \in I} X_i$ --- произведение топологических пространств. Тогда координатные проекции $p_i: X \to X_i$ непрерывны.
    \end{theorem}

    \begin{proof}
        Для всякого открытого в $X_i$ множества $U$ множество $p_i^{-1}(U)$ --- элемент предбазы тихоновской топологии (по определению), поэтому $p_i^{-1}(U)$ открыто, а значит $p_i$ непрерывно.
    \end{proof}

    \begin{definition}[отображение в $X \times Y$]
        Пусть $X$, $Y$, $Z$– топологические пространства. Любое отображение $f: Z \to X \times Y$ имеет вид
        \[f(z) = (f_1(z), f_1(z)),\qquad \text{для всех $z \in Z$},\]
        где $f_1: Z \to X$, $f_2: Z \to Y$ --- некоторые отображения, называемые \emph{компонентами} отображениями $f$.
    \end{definition}

    \begin{definition}[отображение в $\prod_{i \in I} X_i$]
        Пусть $Z$ и $\{X_i\}_{i \in I}$ --- топологические пространства. \emph{Компонентами} отображения $f: Z \to \prod_{i \in I} X_i$ называются отображения $f_i: Z \to X_i$, задаваемые формулами
        \[f_i := p_i \circ f_i\]
    \end{definition}

    \begin{theorem}[о покоординатной непрерывности]
        Пусть $Z$ и $\{X_i\}_{i \in I}$ --- топологические пространства, $X = \prod_{i \in I} X_i$ --- тихоновское произведение.Тогда отображение $f: Z \to \prod_{i \in I} X_i$ непрерывно тогда и только тогда, когда каждая его компонента $f_i$ непрерывна.
    \end{theorem}

    \begin{proof}
        \begin{itemize}
            \item[($\Rightarrow$)] $f_i = p_i \circ f$, при этом $p_i$ и $f$ непрерывны, следовательно и $f_i$ непрерывно.
            \item[($\Leftarrow$)] Пусть $U$ --- элемент предбазы тихоновской топологии. Тогда существуют $i \in I$ и $V \in \Omega_i$, что $U = p_i^{-1}(V)$, следовательно
            \[f^{-1}(U) = f^{-1}(p_i^{-1}(V)) = (p_i \circ f)^{-1}(V) = f_i^{-1}(V)\]
            --- открытое множество.

            Теперь заметим, что для всякого открытого в $X$ множества $W$ существует семейство $\Sigma$ конечных наборов открытых множеств предбазы, что
            \[W = \bigcup_{\Lambda \in \Sigma} \bigcap_{T \in \Lambda} T\]
            Следовательно
            \[
                f^{-1}(W)
                = f^{-1}\left(\bigcup_{\Lambda \in \Sigma} \bigcap_{T \in \Lambda} T\right)
                = \bigcup_{\Lambda \in \Sigma} f^{-1}\left(\bigcap_{T \in \Lambda} T\right)
                = \bigcup_{\Lambda \in \Sigma} \bigcap_{T \in \Lambda} f^{-1}(T)
            \]
            является открытым, поскольку каждое $f^{-1}(T)$ открыто (т.к. $T$ --- элемент предбазы, для него уже показали), а каждое $\Lambda$ конечно.
        \end{itemize}
    \end{proof}

    \begin{remark}
        Также для проверки на непрерывность $f: X \to Y$ достаточно проверить открытость $f^{-1}(U)$ для всякого $U$ из какой-либо базы или предбазы $Y$.
    \end{remark}

    \begin{remark}
        Развёрнутое утверждение неверно: неверно, что если $f: \prod_{i \in I} X_i \to Y$ непрерывно по каждой координате, от непрерывно и в итоге. Для этого несложно проверить, что подходит
        \[f: \RR^2 \to \RR, (x, y) \mapsto \begin{cases}
            0& \text{если $(x, y) = (0, 0)$}\\
            \frac{2xy}{x^2 + y^2}& \text{иначе}
        \end{cases}\]
    \end{remark}

    \begin{definition}
        Пусть $X$, $Y$ --- топологические пространства. Отображение $f: X \to Y$ называется \emph{гомеоморфизмом}, если
        \begin{enumerate}
            \item $f$ --- биекция,
            \item $f$ --- непрерывно,
            \item $f^{-1}$ --- непрерывно.
        \end{enumerate}
    \end{definition}

    \begin{definition}
        Если существует гомеморфизм между $X$ и $Y$, то $X$ и $Y$ \emph{гомеморфны}. Обозначение: $X \simeq Y$.
    \end{definition}

    \begin{theorem}
        Гомеоморфность --- ``отношение эквивалентности'' на топологических пространствах.
    \end{theorem}

    \begin{proof}\ 
        \begin{itemize}
            \item Тождественное отображение (любого топологического пространства) есть гомеоморфизм. Следовательно $A \simeq A$.
            \item Отображение, обратное гомеоморфизму, есть гомеоморфизм. Следовательно $A \simeq B \leftrightarrow B \simeq A$.
            \item Композиция гомеоморфизмов есть гомеоморфизм. Следовательно $A \simeq B \simeq C \rightarrow A \simeq C$.
        \end{itemize}
    \end{proof}

    \begin{remark}\ 
        \begin{itemize}
            \item Гомеоморфизм задаёт биекцию между открытыми множествами в $X$ и $Y$.
            \item Гомеоморфные пространства неотличимы с точки зрения топологии.
        \end{itemize}
    \end{remark}

    \begin{definition}
        \emph{Топологическое свойство} --- свойство топологического пространства, которое сохраняется при гомеоморфизмах.

        \emph{Топологический инвариант} --- характеристика топологического пространства (например, число, группа и т.д.), сохраняющаяся при гомеоморфизмах.
    \end{definition}

    \begin{remark*}
        Для доказательства \emph{негомеоморфности} двух топологических пространств, как правило, находят топологическое свойство или инвариант, который их различает.
    \end{remark*}

    \begin{remark*}
        С этого момента \emph{счётным множеством} называется всякое множество $X$, что есть инъекция $X \to \NN$. 
    \end{remark*}

    \begin{definition}
        Топологическое пространство удовлетворяет
        \begin{itemize}
            \item \emph{первой аксиоме счётности (1АС или \FAC , first axiom of countability)}, если оно обладает счётными базами во всех своих точках (такое пространство называется ``first-countable space'');
            \item \emph{второй аксиоме счётности (2АС или \SAC , second axiom of countability)}, если оно имеет счётную базу (такое пространство называется ``second-countable space'').
        \end{itemize}
    \end{definition}

    \begin{theorem}\label{AC_conseq_theorem}
        $\SAC \Rightarrow \FAC$.
    \end{theorem}

    \begin{proof}
        Если $\Sigma$ --- база топологии, то для всякого $a \in X$ множество
        \[\Sigma_a := \{U \in \Sigma \mid a \in U\}\]
        --- база в точке $a$. При этом $|\Sigma_a| \leqslant |\Sigma|$, следовательно выполнена $\FAC$.
    \end{proof}

    \begin{theorem}
        Всякое метрическое пространство удовлетворяет $\FAC$.
    \end{theorem}

    \begin{proof}
        Множество
        \[\{B_{\frac{1}{n}}(a)\}_{n=1}^\infty\]
        является счётной базой топологии в точке $a$.
    \end{proof}

    \begin{definition}
        Топологическое свойство называется \emph{наследственным}, если из того, что пространство $X$ обладает этим свойством, следует, что любое подпространство пространства $X$ тоже им обладает.

        Топологическое свойство называется \emph{наследственным при произведении}, если из того, что пространства $X$ и $Y$ обладают этим свойством, следует, что пространство $X \times Y$ тоже им обладает.
    \end{definition}

    \begin{theorem}
        $\SAC$ наследственна и наследственна при произведении.
    \end{theorem}

    \begin{proof}
        \begin{itemize}
            \item Пусть $Y$ --- подпространство пространства $X$, удовлетворяющего $\SAC$, а $\Sigma$ --- счётная база $X$ (существует по $\SAC$). Тогда
                \[\Sigma_Y := \{U \cap Y\mid U \in \Sigma\}\]
                --- база $Y$. При этом $|\Sigma_Y| \leqslant |\Sigma|$, следовательно $Y$ удовлетворяет $\SAC$.
            \item Пусть $\Sigma_X$ и $\Sigma_Y$ --- базы топологических пространств $X$ и $Y$, удовлетворяющих $\SAC$. Тогда
                \[\Sigma := \{U \times V \mid U \in \Sigma_X \wedge V \in \Sigma_Y\}\]
                --- база $X \times Y$, при этом
                \[|\Sigma| \leqslant |\Sigma_X| \times |\Sigma_Y| \leqslant |\NN| \times |\NN| = |\NN|\]
                т.е. $\Sigma$ счётно. Следовательно $X \times Y$ удовлетворяет $\SAC$.
        \end{itemize}
    \end{proof}

    \begin{theorem}[Линдлёфа, \href{https://en.wikipedia.org/wiki/Lindel\%C3\%B6f\%27s_lemma}{вики}]\label{countable_supcover_by_SAC_theorem}
        Если пространство удовлетворяет \SAC, то из всякого его открытого покрытия можно выделить счётное подпокрытие.
    \end{theorem}

    \begin{proof}
        Пусть $\Gamma$ --- открытое покрытие $X$. По \SAC есть счётная база $\Sigma$. Рассмотрим
        \[\Lambda := \{V \in \Sigma \mid \exists U \in \Gamma:\; V \subseteq U\}\]
        Поскольку всякое $U$ из $\Gamma$ является открытым, то представляется в виде объединения элементов из $\Sigma$, следовательно $\Lambda$ непусто. По этой же причине $\Lambda$ является покрытием $X$, так как всякая точка $X$ покрывается некоторым $U \in \Gamma$, которое является объединением элементов из $\Sigma$; но все эти элементы лежат в $\Gamma$, значит $\Gamma$ покрывает $U$, а значит и выбранную точку.

        Теперь для каждого $U \in \Lambda$ рассмотрим $V_U \in \Gamma$, в котором оно содержится. Определим
        \[\Gamma' := \{V_U \mid U \in \Lambda\}\]
        Тогда $\Gamma'$ --- покрытие, поскольку $\Gamma$ является покрытием; $\Gamma' \subseteq \Gamma$; $|\Gamma'| = |\Lambda| \leqslant |\Sigma| \leqslant |\NN|$. Таким образом $\Gamma'$ --- счётное подпокрытие покрытия $\Gamma$.
    \end{proof}

    \begin{definition}
        $A \subseteq X$ называется \emph{всюду плотным}, если $\Cl(A) = X$.
    \end{definition}

    \begin{lemma}
        TFAE:
        \begin{itemize}
            \item $A$ --- всюду плотно.
            \item $\Int(X \setminus A) = \varnothing$.
            \item Всякое непустое открытое множество в $X$ пересекается с $A$.
            \item Всякая точка $X$ является точкой прикосновения $A$.
        \end{itemize}
    \end{lemma}

    \begin{proof}
        $A$ всюду плотно тогда и только тогда, когда $\Cl(A) = X$, т.е. $\Int(X \setminus A) = \varnothing$.

        $\Int(X \setminus A) = \varnothing$ тогда и только тогда, когда нет открытых подмножеств у $X \setminus A$ кроме $\varnothing$, что равносильно тому, что всякое непустое открытое множество содержит точки вне $X \setminus A$, т.е. пересекается с $A$.

        Если всякое непустое открытое множество пересекается с $A$, то в любой окрестности любой точки будут точки $A$, поэтому всякая точка $X$ является точкой прикосновения. Если же есть непустое открытое множество, непересекающееся с $A$, то оно является окрестностью любой своей точки, а значит все его точки не являются точками прикосновения.
    \end{proof}

    \begin{definition}
        Топологическое пространство \emph{сепарабельно}, если оно содержит счётное всюду плотное множество.
    \end{definition}

    \begin{theorem}\label{separable_space_properties}\ 
        \begin{enumerate}
            \item Если топологическое пространство удовлетворяет \SAC, то оно сепарабельно.
            \item Метрическое сепарабельное пространство удовлетворяет \SAC.
        \end{enumerate}
    \end{theorem}

    \begin{proof}\ 
        \begin{enumerate}
            \item По \SAC есть счётная база $\Sigma$. Рассмотрим $A$ --- множество представителей семейства $\Sigma$, т.е. множество выделенных элементов в каждом из множеств в $\Sigma$. Тогда $A$ всюду плотно, но $|A| \leqslant |\Sigma| \leqslant |\NN|$.

            \item Пусть $A$ --- счётное, всюду плотное множество. Рассмотрим
                \[\Sigma := \{B_{\frac{1}{n}}(a) \mid a \in A \wedge n \in \NN \setminus \{0\}\}\]
                Пусть $U$ --- некоторое открытое множество, а $x$ --- некоторая его точка. Тогда в $U$ лежит как подмножество некоторый шар $B_\varepsilon(x)$. Рассмотрим некоторое $\delta \in (0; \varepsilon)$, что
                \[\frac{1}{\delta} - \frac{1}{\varepsilon - \delta} \geqslant 1\]
                (при $\delta \to 0^+$ левая сторона стремится к $+\infty$, следовательно найдётся достаточно маленькое $\delta$, что неравенство будет выполнено). Заметим, что в $B_\delta(x)$ есть некоторая точка $a \in A$ (по свойству $A$). При этом есть $n \in \NN \setminus \{0\}$, что
                \[\frac{1}{\delta} \geqslant n \geqslant \frac{1}{\varepsilon - \delta}\]
                т.е.
                \[\delta \leqslant \frac{1}{n} \leqslant \varepsilon - \delta\]
                Тогда $d(x, a) < \delta \leqslant \frac{1}{n}$, следовательно $x \in B_{\frac{1}{n}}(a)$; но с другой стороны $\frac{1}{n} \leqslant \varepsilon - \delta < \varepsilon - d(a, x)$, поэтому $B_\frac{1}{n}(a) \subseteq B_\varepsilon(x)$. Так можно для всякой точки $x \in U$ предоставить шар из $\Sigma$, лежащий в $U$ как подмножество и покрывающий $x$, значит $U$ порождается объединением шаров из $\Sigma$. А значит $\Sigma$ --- база.

                При этом $|\Sigma| \leqslant |A| \times |\NN \setminus \{0\}| \leqslant |\NN| \times |\NN| = |\NN|$.
        \end{enumerate}
    \end{proof}

    \begin{definition}
        Топологическое пространство удовлетворяет \emph{первой аксиоме отделимости} $\T_1$, если каждая из любых двух различных точек пространства обладает окрестностью, не содержащей другую из этих точек.
    \end{definition}

    \begin{theorem}
        $X$ удовлетворяет $\T_1$ тогда и только тогда, когда все одноточечные множества замкнуты.
    \end{theorem}

    \begin{proof}
        \begin{itemize}
            \item[($\Rightarrow$)] Пусть $x$ --- случайная точка $X$. По $\T_1$ для всякой точки $a \in X \setminus \{x\}$ есть окрестность $U_a$ точки $a$, не содержащая $x$. Следовательно
                \[U := \bigcup_{a \in X \setminus \{x\}} U_a\]
                --- открытое множество, содержащее каждую точку $X \setminus \{x\}$ и не содержащее $x$. Следовательно $X \setminus \{x\} = U$ --- открыто, значит $\{x\}$ замкнуто.

            \item[($\Leftarrow$)] Если $\{x\}$ замкнуто, то $X \setminus \{x\}$ открыто. Значит для всяких $x$ и $y$ множество $X \setminus \{x\}$ будет окрестностью $y$, не содержащей $x$. Таким образом выполнена $\T_1$.
        \end{itemize}
    \end{proof}

    \begin{definition}
        Топологическое пространство удовлетворяет \emph{второй аксиоме отделимости} $\T_2$, если любые две различные точки пространства обладают непересекающимися окрестностями.

        Пространства, удовлетворяющие аксиоме $\T_2$, называются \emph{хаусдорфовыми}.
    \end{definition}

    \begin{remark*}
        Всякое метрическое пространство хаусдорфово.
    \end{remark*}

    \begin{theorem}
        $X$ хаусдорфово тогда и только тогда, когда множество $\{(a, a) \mid a \in X\}$ замкнуто в $X \times X$.
    \end{theorem}

    \begin{proof}
        Обозначим
        \[\Delta := \{(a, a) \mid a \in X\}\]
        \begin{itemize}
            \item[($\Rightarrow$)] Покажем, что $(X \times X) \setminus \Delta$ открыто. Пусть $(b, c) \notin \Delta$. Тогда по $\T_2$ есть окрестности $U_b$ и $U_c$ точек $b$ и $c$ в $X$, что $U_b \cap U_c = \varnothing$. Следовательно $(U_b \times U_c) \cap \Delta = \varnothing$, тогда $U_b \times U_c$ --- окрестность $(b, c)$, лежащая в $(X \times X) \setminus \Delta$ как подмножество.

            \item[($\Leftarrow$)] Пусть $b$ и $c$ --- различные точки $X$. Тогда $(b, c) \notin \Delta$. Поскольку $\Delta$ замкнуто, то $(X \times X) \setminus \Delta$ открыто. Поскольку $\{U \times V \mid U, V \in \Omega_X\}$ --- база $X \times X$, то есть некоторые открытые в $X$ множества $U$ и $V$, что
            \[(b, c) \in U \times V \subseteq (X \times X) \setminus \Delta.\]
            Следовательно $(U \times V) \cap \Delta = \varnothing$, а значит $U \cap V = \varnothing$. При этом $b \in U$, а $c \in V$. Значит $U$ и $V$ --- непересекающиеся окрестности $b$ и $c$. Поскольку $b$ и $c$ случайны, то выполнена $\T_2$. 
        \end{itemize}
    \end{proof}

    \begin{definition}
        Топологическое пространство удовлетворяет \emph{третьей аксиоме отделимости} $\T_3$, если в нём любое замкнутое множество и любая не содержащаяся в этом множестве точка обладают непересекающимися окрестностями.

        Пространства, одновременно удовлетворяющие аксиомам $\T_1$ и $\T_3$, называются \emph{регулярными}.
    \end{definition}

    \begin{theorem}
        $X$ удовлетворяет $\T_3$ тогда и только тогда, когда для любой окрестности $U_a$ любой точки $a$ есть такая окрестность $V_a$ точки $a$, что $\Cl(V_a) \subseteq U_a$.
    \end{theorem}

    \begin{proof}
        \begin{itemize}
            \item[($\Rightarrow$)]
                Пусть $U_a$ --- некоторая окрестность некоторой точки $a$ в $X$. Тогда $X \setminus U_a$ замкнуто. По $\T_3$ у $X \setminus U_a$ и $a$ есть непересекающиеся окрестности $W_a$ и $V_a$ соответственно. Тогда $X \setminus W_a$ замкнуто; при этом $W_a \supseteq X \setminus U_a$, следовательно $X \setminus W_a \subseteq U_a$; аналогично имеем, что $V_a \subseteq X \setminus W_a$. Следовательно
                \[\Cl(V_a) \subseteq X \setminus W_a \subseteq U_a.\]
                Таким образом мы нашли искомую окрестность $V_a$.

            \item[($\Leftarrow$)]
                Пусть даны замкнутое $F$ и точка $a$ вне него. Тогда $U_a := X \setminus F$ --- окрестность $a$. Тогда есть окрестность $V_a$ точки $a$, что $\Cl(V_a) \subseteq U_a$. Следовательно $\Int(X \setminus V_a) \supseteq X \setminus U_a = F$. Значит $\Int(X \setminus V_a)$ и $V_a$ --- непересекающиеся окрестности $F$ и $a$.
        \end{itemize}
    \end{proof}

    \begin{definition}
        Топологическое пространство удовлетворяет \emph{четвёртой аксиоме отделимости} $\T_4$, если в нём любые два непересекающихся замкнутых множества обладают непересекающимися окрестностями.

        Пространства, одновременно удовлетворяющие аксиомам $\T_1$ и $\T_4$, называются \emph{нормальными}.
    \end{definition}

    \begin{theorem}
        $X$ удовлетворяет $\T_4$ тогда и только тогда, когда для любой окрестности $U_A$ любого замкнутого множества $A$ есть такая окрестность $V_A$ множества $A$, что $\Cl(V_A) \subseteq U_A$.
    \end{theorem}

    \begin{proof}
        \begin{itemize}
            \item[($\Rightarrow$)]
                Пусть $U_A$ --- некоторая окрестность некоторого замкнутого множества $A$. Тогда $X \setminus U_a$ замкнуто. По $\T_4$ у $X \setminus U_A$ и $A$ есть непересекающиеся окрестности $W_A$ и $V_A$ соответственно. Тогда $X \setminus W_A$ замкнуто; при этом $W_A \supseteq X \setminus U_a$, следовательно $X \setminus W_A \subseteq U_A$; аналогично имеем, что $V_A \subseteq X \setminus W_A$. Следовательно
                \[\Cl(V_A) \subseteq X \setminus W_A \subseteq U_A.\]
                Таким образом мы нашли искомую окрестность $V_A$.

            \item[($\Leftarrow$)]
                Пусть даны замкнутые непересекающиеся $F$ и $G$ вне него. Тогда $U_G := X \setminus F$ --- окрестность $G$. Тогда есть окрестность $V_G$ множества $G$, что $\Cl(V_G) \subseteq U_G$. Следовательно $\Int(X \setminus V_G) \supseteq X \setminus U_G = F$. Значит $\Int(X \setminus V_G)$ и $V_G$ --- непересекающиеся окрестности $F$ и $G$.
        \end{itemize}
    \end{proof}

    \begin{theorem}
        ``$X$ нормально'' $\Rightarrow$ ``$X$ регулярно'' $\Rightarrow$ ``$X$ хаусдорфово'' $\Rightarrow$ ``$X$ удовлетворяет $\T_1$''.
    \end{theorem}

    \begin{proof}
        По $\T_1$ любое одноточечное множество замкнуто. Следовательно рассматривая как замкнутое множество конкретную точку можно получить следствия $\T_4 \Rightarrow \T_3 \Rightarrow \T_2$. Последнее же следствие теоремы очевидно: нужно всего лишь выкинуть аксиому $\T_2$.
    \end{proof}

    \begin{theorem}
        Всякое метрическое пространство нормально.
    \end{theorem}
    
    \begin{proof}
        Очевидно, что всякое метрическое пространство удовлетворяет $\T_1$. Значит осталось проверить $\T_4$.

        Пусть даны замкнутые непересекающиеся множества $A$ и $B$. Тогда $X \setminus B$ --- окрестность $A$. Значит для сякого $x \in A$ есть $r_x > 0$, что $B_{r_x}(x) \subseteq X \setminus B$, т.е. $B_{r_x}(x) \cap B = \varnothing$. Рассмотрим
        \[U_A := \bigcup_{x \in A} B_{r_x/2}(x);\]
        аналогично определим $U_B$. Очевидно, что $U_A$ и $U_B$ --- окрестности $A$ и $B$. Покажем, что $U_A \cap U_B = \varnothing$.

        Предположим противное, т.е. есть $a \in A$ и $b \in B$, что $B_{r_a/2}(a) \cap B_{r_b/2}(b)$ содержит некоторую точку $x$. Тогда
        \[d(a, b) \leqslant d(a, x) + d(x, b) < \frac{r_a}{2} + \frac{r_b}{2}\]
        WLOG $r_a \geqslant r_b$. Тогда
        \[d(a, b) < \frac{r_a}{2} + \frac{r_b}{2} \leqslant r_a,\]
        т.е. $b \in B_{r_a}(a)$. Но мы знаем, что $B_{r_a}(a) \cap B = \varnothing$ --- противоречие. Значит $U \cap V = \varnothing$.

        Таким образом для случайных непересекающихся замкнутых $A$ и $B$ мы построили их непересекающиеся окрестности. Значит выполнена $\T_4$.
    \end{proof}

    \begin{lemma}\ 
        \begin{enumerate}
            \item Аксиома $\T_1$, хаусдорфовость и регулярность наследуются подпространствами и произведениями.
            \item Существует нормальное пространство $X$ и подпространство $Y$ в нём, не являющееся нормальным.
            \item Существуют нормальные пространства $X$ и $Y$ такие, что $X \times Y$ не является нормальным.
        \end{enumerate}
    \end{lemma}

    \begin{proof}\ 
        % \begin{enumerate}
        %     \item 
        % \end{enumerate}
        \todo[inline]{Доказать. Пока лень...}
    \end{proof}

    \begin{definition}
        Топологическое пространство \emph{связно}, если его нельзя разбить на два непустых открытых множества.
    \end{definition}

    \begin{theorem}
        TFAE
        \begin{itemize}
            \item $X$ связно.
            \item $X$ нельзя разбить на два непустых замкнутых множества.
            \item Любое подмножество $X$, открытое и замкнутое одновременно, либо пусто, либо совпадает со всем пространством $X$.
            \item Не существует сюръективного непрерывного отображения из $X$ в $\{0; 1\}$ с дискретной топологией.
        \end{itemize}
    \end{theorem}

    \begin{proof}
        \begin{itemize}
            \item $X$ связно тогда и только тогда, когда его нельзя разбить на два непустых открытых множества. Заменяя множества разбиения на их дополнения, получаем, что $X$ нельзя разбить на два непустых открытых множества тогда и только тогда, когда $X$ нельзя разбить на два несовпадающих с $X$ замкнутых множества, что равносильно разбиению на два непустых замкнутых множества.

            \item $X$ нельзя разбить на два непустых открытых множества тогда и только тогда, когда всякое непустое открытое множество не имеет непустого открытого дополнения в $X$. Т.е. всякое открытое множество либо совпадает с $\varnothing$ или $X$, либо не является замкнутым, что равносильно тому, что всякое открытое замкнутое множество является либо $\varnothing$, либо $X$.

            \item Сюръективное непрерывное отображение из $X$ в $\{0; 1\}$ с дискретной топологией равносильно разложению $X$ на два открытых непустых множества. Так как прообразы $0$ и $1$ являются множествами, дополняющими друг друга до $X$; при этом сюръективность равносильна непустоте обоих, а непрерывность --- открытости обоих.
        \end{itemize}
    \end{proof}

    \begin{remark*}
        Когда говорят, что какое-то множество связно, всегда имеют в виду, что множество лежит в некотором топологическом пространстве (в каком именно --- должно быть ясно из контекста) и что с индуцированной этим включением топологией оно является связным пространством.
    \end{remark*}

    \begin{theorem}
        Пусть $X \subseteq \RR$. TFAE
        \begin{itemize}
            \item $X$ связно.
            \item $X$ выпукло, т.е. для всяких $a, b \in X$, что $a < b$ отрезок $[a; b] \subseteq X$.
            \item $X$ есть интервал (в широком смысле), точка или $\varnothing$.
        \end{itemize}
    \end{theorem}

    \begin{proof}
        \begin{itemize}
            \item Пусть $X$ связно. Пусть есть такие $a, b \in X$, что $[a; b] \nsubseteq X$, значит есть $c \in (a; b)$, что $c \notin X$. Заметим, что $(-\infty; c)$ и $(c; +\infty)$ открыты. При этом
                \[X = \bigl(X \cap (-\infty; c)\bigr) \sqcup \bigl(X \cap (c; +\infty)\bigr)\]
                Заметим, что $X \cap (-\infty; c)$ и $X \cap (c; +\infty)$ открыты в $X$, значит $X$ несвязно --- противоречие.

            \item Пусть $X$ выпукло. Тогда $X \supseteq (\inf X; \sup X)$, где $\inf$ и $\sup$ могут принимать значения $\pm \infty$. Если $\inf X < \sup X$, то $X$ --- интервал с концами $\inf X$ и $\sup X$ (каким интервалом $X$ является --- вопрос про то, лежат ли $\inf X$ и $\sup X$ в самом $X$); иначе $X$ --- точка или $\varnothing$.

            \item Пусть $X$ --- интервал (в широком смысле), точка или $\varnothing$. Если $X$ --- точка или $\varnothing$, то очевидно, что $X$ --- связно. Поэтому покажем, что если $X$ --- интервал в широком смысле, то оно связно.

                Пусть $X$ раскладывается в объединение двух непустых открытых $A$ и $B$. Заметим, что ни одно из $A$ и $B$ не могут состоять только из концов $X$ (так как должны содержать и некоторую окрестность). Значит $X'$ --- $X$ без своих концов --- раскладывается в объединение двух непустых открытых $A' := A \cap X'$ и $B' := B \cap X'$. Значит $A'$ и $B'$ являются объединением непересекающихся интервалов. Пусть $I$ --- некоторый интервал из разложения $A'$, а $t$ --- его конец. Понятно, что $A'$ открыто в $\RR$, значит $t \notin A'$. Если $t \in X'$, то $t \in B'$, значит некоторая окрестность $t$ лежит в $B'$, а тогда $B'$ и $I$ пересекаются, следовательно $A'$ и $B'$ тоже --- противоречие. Таким образом никакой конец $I$ не лежит в $X'$, значит концы $I$ совпадают с концами $X'$, т.е. $I = X'$; следовательно $A' = X'$, $B' = \varnothing$ --- противоречие. Значит $X'$ и $X$ связны.
        \end{itemize}
    \end{proof}

    \begin{theorem}[Непрерывный образ связного пространства связен]
        Если $f: X \to Y$ --- непрерывное отображение и пространство $X$ связно, то и множество $f(X)$ связно.
    \end{theorem}

    \begin{proof}
        Предположим противное: пусть $f(X)$ несвязно. Тогда $f(X) = U \cup V$, $U \cap V = \varnothing$, где $U$, $V$ непусты и открыты. Следовательно, мы имеем разбиение пространства $X$ на два непустых открытых множества --- $f^{-1}(U)$ и $f^{-1}(V)$, что противоречит связности $X$.
    \end{proof}

    \begin{corollary}
        \emph{Связность} --- топологическое свойство.
    \end{corollary}

    \begin{theorem}[о промежуточном значении]
        Пусть $f: X \to \RR$ --- непрерывное отображение, а $X$ связно. Тогда для любых $a, b \in f(X)$ множество $f(X)$ содержит все числа между $a$ и $b$.
    \end{theorem}

    \begin{proof}
        $f(X)$ связно, значит выпукло, значит содержит $[a; b]$.
    \end{proof}

    \begin{definition}
        \emph{Компонентой связности} пространства $X$ называется всякое его связное подмножество, не содержащееся ни в каком другом (строго большем) связном подмножестве пространства $X$. (Компонента связности пространства $X$ --- максимальное по включению связное множество в $X$.)
    \end{definition}

    \begin{lemma}
        Объединение любого семейства попарно пересекающихся связных множеств связно.
    \end{lemma}

    \begin{proof}
        Пусть $\Sigma$ --- семейство попарно пересекающихся связных множеств в $X$. Определим
        \[Y := \bigcup_{A \in \Sigma} A\]
        Предположим противное: $Y$ раскладывается в объединение непересекающихся открытых в $Y$ множеств $U$ и $V$. Несложно видеть, что для всякого $A \in \Sigma$ множества $U \cap A$ и $V \cap A$ открыты в $A$, не пересекаются, а в объединении дают $A$; следовательно одно из них совпадает с $A$, а другое с $\varnothing$. Т.е. $A$ является подмножеством одного из $U$ и $V$, а с другим не пересекается.

        Пусть $A, B \in \Sigma$. Пусть $A \subseteq U$. Тогда $B$ пересекается с $U$, так как пересекается с $A$. Значит $B \subseteq U$, а $B \cap V = \varnothing$. Таким образом если одно из $U$ и $V$ содержит как подмножество какой-то элемент $\Sigma$, то содержит как подмножества все элементы $\Sigma$, а значит и $Y$; следовательно другое пусто --- противоречие.

        Таким образом $Y$ связно.
    \end{proof}

    \begin{theorem}\ 
        \begin{enumerate}
            \item Каждая точка пространства $X$ содержится в некоторой компоненте связности.
            \item Различные компоненты связности пространства $X$ не пересекаются.
        \end{enumerate}
    \end{theorem}

    \begin{proof}
        \begin{enumerate}
            \item Пусть $x$ --- некоторая точка $X$. Пусть $A_x$ --- объединение всех связных множеств, содержащих $x$ (при этом $A$ определено корректно, так как $\{x\}$ связно). Таким образом $A_x$ является максимальным по включению связным множеством, так как если есть некоторое связное $B$, что $B \supsetneq A_x$, то $B$ --- связное множество, содержащее $x$, а тогда $B \subseteq A_x$ --- противоречие. Значит $A_x$ --- компонента связности, содержащая $x$.

            \item Если $U$ и $V$ --- различные компоненты связности $X$ --- пересекаются, то $U \subsetneq U \cup V$, а $U \cup V$ --- компонента связности по доказанной теореме. Таким образом $U$ не максимальное по включению, но связное множество --- противоречие с определением компоненты связности.
        \end{enumerate}
    \end{proof}

    \begin{corollary}
        Компоненты связности составляют разбиение топологического пространства. (Напомним, что разбиение множества --- это его покрытие попарно непересекающимися подмножествами.)
    \end{corollary}

    \begin{corollary}\ 
        \begin{enumerate}
            \item Любое связное множество содержится в некоторой связной компоненте пространства как подмножество.
            \item Две точки содержатся в одной компоненте связности тогда и только тогда, когда они содержатся в одном связном множестве.
            \item Пространство несвязно тогда и только тогда, когда оно имеет как минимум две компоненты связности.
        \end{enumerate}
    \end{corollary}

    \begin{corollary}
        Число компонент связности является топологическим инвариантом.
    \end{corollary}

    \begin{theorem}
        Замыкание связного множества связно.
    \end{theorem}

    \begin{proof}
        Пусть дано связное множество $A$ в пространстве $X$. Предположим противное: $\Cl(A)$ разбивается на два замкнутых в $\Cl(A)$ непустых множествах $U$ и $V$. Поскольку $\Cl(A)$ замкнуто, то $U$ и $V$ замкнуты в $X$, следовательно $U \cap A$ и $V \cap A$ замкнуты в $A$. Из связности $A$ следует, что WLOG $U \cap A = A$, $V \cap A = \varnothing$, т.е. $A \subseteq U$, $A \cap V = \varnothing$. Соответственно из замкнутости $U$ следует, что $\Cl(A) \subseteq U$. Следовательно $U = \Cl(A)$, а $V = \varnothing$ --- противоречие.
    \end{proof}

    \begin{corollary}
        Компоненты связности замкнуты.
    \end{corollary}

    \begin{definition}
        \emph{Путём} в топологическом пространстве $X$ называется непрерывное отображение $\alpha: [0,1] \to X$. Началом пути $\alpha$ называется точка $\alpha(0)$, концом --- точка $\alpha(1)$. При этом говорят, что путь $\alpha$ соединяет точку $\alpha(0)$ сточкой $\alpha(1)$.
    \end{definition}

    \begin{definition}
        Топологическое пространство называется \emph{линейно связным}, если в нём любые две точки можно соединить путём.
    \end{definition}

    \begin{remark*}
        \emph{Линейно связным множеством} называют подмножество топологического пространства (какого именно, должно быть ясно из контекста), линейно связное как пространство с топологией, индуцированной из объемлющего пространства.
    \end{remark*}

    \begin{theorem}
        Пусть даны линейно связное пространство $X$ и непрерывное отображение $f: X \to Y$. Тогда и пространство $f(X)$ линейно связно.
    \end{theorem}

    \begin{proof}
        Если $\alpha$ --- путь, соединяющий точки $a$ и $b$ из $X$, то $f \circ \alpha$ --- путь, соединяющий точки $f(a)$ и $f(b)$ из $f(X)$.
    \end{proof}

    \begin{corollary}
        Линейная связность --- топологическое свойство.
    \end{corollary}
    
    \begin{corollary}
        Число компонент линейной связности является топологическим инвариантом.
        \todo[inline,color=green!40]{Предупреждение: ``немного опережая события''.}
    \end{corollary}

    \begin{lemma}
        Соединимость путём --- отношение эквивалентности на множестве точек пространства.
    \end{lemma}

    \begin{proof}
        \begin{itemize}
            \item (\emph{Рефлексивность.}) Для всякой точки $a \in X$ путь \[\alpha: [0; 1] \to X, t \mapsto a\] соединяет $a$ с собой.
            \item (\emph{Симметричность.}) Для всякого пути $\alpha$ из точки $a$ в точку $b$ отображение \[\beta: [0; 1] \to X, t \mapsto \alpha(1 - t)\] является путём из $b$ в $a$.
            \item (\emph{Транзитивность.}) Для всякого пути $\alpha$ из $a$ в $b$ и всякого пути $\beta$ из $b$ в $c$ отображение
                \[\gamma: [0; 1] \to X, t \mapsto
                    \begin{cases}
                        \alpha(2t)& \text{если $t \in [0; \frac{1}{2}]$}\\
                        \beta(2t-1)& \text{если $t \in [\frac{1}{2}; 1]$}\\
                    \end{cases}
                \]
                --- путь из $a$ в $c$.
        \end{itemize}
    \end{proof}

    \begin{definition}
        \emph{Компонентой линейной связности} пространства $X$ называется класс эквивалентности отношения соединимости путём.
    \end{definition}

    \begin{exercise}\ 
        \begin{enumerate}
            \item Объединение любого семейства попарно пересекающихся линейно связных множеств линейно связно.
            \item Приведите пример линейно связного множества, замыкание которого не является линейно связным.
            \item Приведите пример незамкнутой компоненты линейной связности.
        \end{enumerate}
    \end{exercise}

    \begin{theorem}
        В топологическом пространстве, каждая точка которого имеет линейно связную окрестность,
        \begin{enumerate}
            \item компоненты линейной связности открыты;
            \item компонентны линейной связности совпадают с компонентами связности.
        \end{enumerate}
    \end{theorem}

    \begin{proof}
        \begin{enumerate}
            \item Пусть $W$ --- компонента линейной связности, $a \in W$ и $U$ --- линейно связная окрестность точки $a$.Тогда $U \subseteq W$, что влечёт открытость $W$.
            \item Пусть $\Sigma$ --- компоненты линейной связности пространства. По предыдущему пункту, каждое $W$ из $\Sigma$ открыто. Пусть $A$ --- компонента связности. В силу связности, $A$ не может пересекать несколько разных элементов $\Sigma$, так ка иначе будет иметь разбиение на открытые множества. Значит $A$ содержится в некотором $W$ из $\Sigma$. Отсюда, $W = A$.
        \end{enumerate}
    \end{proof}

    \begin{lemma}
        Пусть $X$, $Y$ --- топологические пространства, а $f: X \to Y$ --- гомеоморфизм. Тогда для любой точки $a \in X$ пространства $X \setminus \{a\}$ и $Y \setminus \{f(a)\}$ гомеоморфны.
    \end{lemma}

    \begin{theorem}
        Следующие пространства попарно негомеоморфны: $[0; 1]$, $[0; 1)$, $\RR$, $S^1$.
    \end{theorem}

    \begin{proof}
        У $[0; 1]$ можно удалить максимум 2 точки, чтобы оно осталось связным, у $[0; 1)$ и $S^1$ --- по одной, а у $\RR$ --- ноль. Следовательно если какие-то из этих пространств гомеоморфны, то только $[0; 1)$ и $S^1$. Но у $S^1$ какую точку ни удали, оно останется связным, а у $[0; 1)$ --- только $0$; следовательно $[0; 1)$ негомеоморфно $S^1$.
    \end{proof}

    \begin{theorem}
        $\RR^2$ негомеоморфно никакому интервалу (в широком смысле) и $S^1$.
    \end{theorem}

    \begin{proof}
        Если из $\RR^2$ выколоть любое конечное множество точек, то множество останется связным. С другой стороны этим свойством не обладают ни интервалы в широком смысле, ни $S^1$.
    \end{proof}

    \begin{definition}
        Топологическое пространство \emph{компактно}, если из любого его открытого покрытия можно выделить конечное подпокрытие.
    \end{definition}

    \begin{remark*}
        Когда говорят, что какое-то множество \emph{компактно}, всегда имеют в виду, что это множество лежит в топологическом пространстве и что, будучи наделено индуцированной топологией, оно является компактным пространством.
    \end{remark*}

    \begin{remark*}
        При определении компактности множества можно использовать два эквивалентных подхода. Первый подход --- рассматривать открытые множества в подпространстве. Второй --- рассматривать открытые множества в исходном пространстве.
    \end{remark*}

    \begin{theorem}
        Отрезок $[0; 1]$ компактен.
    \end{theorem}

    \begin{proof}
        Пусть дано некоторое открытое покрытие $\Sigma$ отрезка $[0; 1]$. Обозначим $I_0 := [0; 1]$.

        Построим индуктивно последовательность $(I_n)_{n=0}^\infty$ отрезков, которые не покрываются конечным подпокрытием $\Sigma$. $I_0$ уже определён. Если $I_n$ построен, то разделим его пополам; если оба отрезка-половины покрываются конечными подпокрытиями $\Sigma$, значит и $I_n$ покрывается. Таким образом одна из ``половин'' $I_n$ не покрывается: её и обозначим за $I_{n+1}$.

        Так мы получили последовательность вложенных отрезков, значит по аксиоме полноты есть точка $c$, лежащая во всех них. Заметим, что $c$ покрывается $\Sigma$, значит есть некоторый элемент $U$ покрытия $\Sigma$, который покрывает $c$. Но поскольку $\Sigma$ --- открытое покрытие, то $U$ открыто и, следовательно, покрывает некоторую окрестность $c$, а с ней и все отрезки последовательности $(I_n)_{n=0}^\infty$, начиная с некоторого --- противоречие с непокрываемостью конечным подпокрытием $\Sigma$.
    \end{proof}

    \begin{theorem}
        Пусть $X$ --- компактное пространство и $A$ --- замкнутое подмножество. Тогда $A$ компактно.
    \end{theorem}

    \begin{proof}
        Пусть $\Sigma$ --- открытое в $X$ покрытие $A$. Поскольку $X \setminus A$ --- открытое, то $\Sigma \cup \{X \setminus A\}$ --- открытое покрытие $X$, следовательно из него можно выделить конченое подпокрытие. Удалив из него, если нужно, $X \setminus A$, получим конечное подпокрытие $\Sigma$ множества $A$.
    \end{proof}

    \begin{theorem}
        Пусть $X$, $Y$ --- компактные пространства. Тогда и пространство $X \times Y$ компактно.
    \end{theorem}

    \begin{proof}
        Пусть $\Sigma$ --- некоторые покрытие $X \times Y$. Заметим, что, заменив всякое открытое в $\Sigma$ на элементы базы $X \times Y$ в качестве объединения которых оно раскладывается, можно свести задачу поиска конечного подпокрытия к новому покрытию. Восстановление подпокрытия для старого покрытия просто: нужно просто для каждого элемента конечного подпокрытия нового покрытия найти тот элемент старого покрытия, который содержит его как подмножество. Тогда получится конечное подпокрытие старого покрытия.

        Для каждой точки $x$ заметим, что $\Sigma$ является покрытием слоя $\{x\} \times Y$. Несложно понять, что этот слой компактен, и выделить из него конечное подпокрытие $\Lambda_x$. Рассмотрим
        \[W_x := \bigcap_{\substack{U \times V \in \Lambda_x\\U \subseteq X\\V \subseteq Y}} U_x\]
        Поскольку $W_x$ открыто, то $\{W_x\}_{x \in X}$ --- покрытие. Тогда мы можем из него выделить конечное подпокрытие $\{W_{x_i}\}_{i = 1}^n$. Тогда $\bigcup_{i=1}^n \Lambda_{x_i}$ --- конечное подпокрытие $\Sigma$ пространства $X \times Y$.
    \end{proof}

    \begin{theorem}[Тихонова]
        Пусть $\{X_i\}_{i \in i}$ --- семейство компактных топологических пространств. Тогда тихоновское произведение $X = \prod_{i \in I} X_i$ компактно.
    \end{theorem}

    \begin{proof}
        Пусть $\Sigma$ --- покрытие $X$. WLOG можно считать, что $\Sigma$ --- подмножество базы тихоновской топологии, причём в качестве базы мы возьмём всевозможные конечные пересечения стандартной предбазы этой же топологии. Несложно видеть, что данная база выглядит как
        \[\bigcup_{\stackrel{J \subseteq I}{|J| < |\NN|}} \left\{\left(\bigtimes_{i \in I \setminus J} X_i\right) \times \left(\bigtimes_{j \in J} U_j\right) \mid \forall j \in J\quad U_j \in \Omega_j\right\},\]
        т.е. произведение открытых множеств топологических пространств из конечного подсемейства $\{X_i\}_{i \in I}$ и остальных топологических пространств. (Для конечного множества пространств $X_i$ верно, что в них есть точка $x_i$, не имеющая соответствующего координатного прообраза ($p_i^{-1}(x_i) \cap U = \varnothing$) в данном открытом множестве $U$.)

        Заметим, что \dots
        \todo[inline,color=blue!40]{Просто попытка. Не получилось. Нужна трансфинитная индукция или рекурсия... Сама теорема --- анонс.}
    \end{proof}

    % \newpage
    % Let $I$ be set of indices (it can be any finite set, $\NN$, any ordinal or cardinal etc.), $\{X_i\}_{i \in I}$ be a family of sets and $\{\Omega_i\}_{i \in I}$ be another family, where $\Omega_i$ is set of some subsets of $X_i$ for each $i \in I$. Look at
    % \[\bigcup_{\stackrel{J \subseteq I}{|J| < |\NN|}} \left\{\left(\bigtimes_{i \in I \setminus J} X_i\right) \times \left(\bigtimes_{j \in J} U_j\right) \mid \forall j \in J\quad U_j \in \Omega_j\right\}\]
    % \begin{enumerate}
    %     \item Can you understand last formula?
    %     \item How can you describe by human speech what sets are in this object?
    % \end{enumerate}

    \begin{theorem}
        Пусть $X$ --- хаусдорфово пространство, а $A \subseteq X$ --- компакт. Тогда $A$ замкнуто в $X$.
    \end{theorem}

    \begin{proof}
        Пусть $b$ --- некоторая точка $X \setminus A$. Покажем, что $b$ является внутренней для $X \setminus A$.

        Для всякой точки $a \in A$ найдутся непересекающиеся окрестности $U_a$ и $V_a$ точек $a$ и $b$. Тогда $\{U_a\}_{a \in A}$ --- открытое покрытие $A$, значит найдётся конченое подпокрытие $\{U_{a_1}; \dots; U_{a_n}\}$. Получим, что
        \[V := \bigcap_{i=1}^n V_{a_i}\]
        --- окрестность $b$, непересекающаяся с $\bigcup_{i=1}^n U_{a_i}$ --- окрестностью $A$. Таким образом $V \subseteq X \setminus A$. Следовательно $b$ внутренняя точка $X \setminus A$. Значит $A$ замкнуто в $X$.
    \end{proof}

    \begin{theorem}
        Если пространство $X$ хаусдорфово и компактно, то оно нормально.
    \end{theorem}

    \begin{proof}
        Покажем, что $X$ удовлетворяет $\T_3$; $\T_1$ следует из $\T_2$.
        
        Пусть $A$ замкнуто в $X$ и $b$ --- некоторая точка $X \setminus A$. Поскольку $A$ --- замкнутое подмножество компакта, то само является компактом.

        Для каждой точки $a$ множества $A$ выделим непересекающиеся окрестности $U_a$ и $V_a$ точек $a$ и $b$ (они существуют по хаусдорфовости). Тогда $\{U_a\}_{a \in A}$ --- покрытие $A$, значит по компактности из него можно выделить конечное подпокрытие $\{U_{a_i}\}_{i=1}^n$. Таким образом
        \begin{align*}
            U &:= \bigcup_{i=1}^n U_{a_i}&
            &\text{ и }&
            V &:= \bigcap_{i=1}^n V_{a_i}&
        \end{align*}
        являются непересекающимися окрестностями $A$ и $b$. Поскольку $A$ и $b$ случайны, то $\T_3$ выполнена.

        Теперь так же покажем выполняемость $\T_4$. Пусть $A$ и $B$ --- непересекающиеся замкнутые множества и, как следствие, компактны. Для каждой точки $a$ множества $A$ рассмотрим непересекающиеся окрестности $U_a$ и $V_a$ точки $a$ и множества $B$ (они существуют по $\T_3$). Тогда $\{U_a\}_{a \in A}$ --- покрытие $A$, значит по компактности из него можно выделить конечное подпокрытие $\{U_{a_i}\}_{i=1}^n$. Таким образом
        \begin{align*}
            U &:= \bigcup_{i=1}^n U_{a_i}&
            &\text{ и }&
            V &:= \bigcap_{i=1}^n V_{a_i}&
        \end{align*}
        являются непересекающимися окрестностями $A$ и $B$. Поскольку $A$ и $B$ случайны, то $\T_4$ выполнена.

        Таким образом выполнены $\T_3$ и $\T_1$, и следовательно $X$ нормально.
    \end{proof}

    \begin{definition}
        Пусть дано метрическое пространство $(X, d)$. Множество $A \subseteq X$ называется \emph{ограниченным}, если оно содержится в некотором шаре пространства $X$.
    \end{definition}

    \begin{definition}
        Пусть дано метрическое пространство $(X, d)$. \emph{Диаметр} множества $A \subseteq X$ --- величина
        \[\diam(A) := \sup \{d(x,y) \mid x, y \in A\}.\]
    \end{definition}

    \begin{lemma}
        Пусть дано метрическое пространство $(X, d)$. Тогда для всякого множества $A \subseteq X$ верно, что оно ограничено тогда и только, когда $\diam(A) < +\infty$.
    \end{lemma}

    \begin{theorem}
        Компактное метрическое пространство ограничено.
    \end{theorem}

    \begin{proof}
        Возьмём любую точку $x$ нашего пространства $X$ и рассмотрим покрытие его всевозможными шарами $B_r(x)$, $r > 0$. По компактности будет конечное подпокрытие $\{B_{r_i}(x)\}_{i=1}^n$. Значит всё пространство покрывается шаром $B_r(x)$, где $r = max(r_1, \dots, r_n)$, т.е. $X$ ограничено.
    \end{proof}

    \begin{corollary}
        Компактное множество в метрическом пространстве замкнуто и ограничено.
    \end{corollary}

    \begin{proof}
        Метрическое пространство хаусдорфово, а компакт в хаусдорфовом пространстве замкнут.
    \end{proof}

    \begin{theorem}
        Множество в $\RR^n$ компактно тогда и только тогда, когда оно замкнуто и ограничено.
    \end{theorem}

    \begin{proof}\ 
        \begin{itemize}
            \item[($\Rightarrow$)] Очевидно по предыдущему следствию.
            \item[($\Leftarrow$)] Множество $A$ ограничено в $\RR^n$, следовательно содержится в кубе $[-a; a]^n$. Поскольку каждый из отрезков $[-a; a]$ компактен, то их произведение --- куб $[-a; a]^n$ --- компактно. Следовательно $A$ --- замкнутое подмножество компакта, а значит само компактно.
        \end{itemize}
    \end{proof}

    \begin{definition}
        Набор подмножеств множества $X$ \emph{центрирован}, если пересечение любого его конечный поднабора множеств непусто.
    \end{definition}

    \begin{theorem}
        $X$ компактно тогда и только тогда, когда любой центрированный набор замкнутых множеств в $X$ имеет непустое пересечение.
    \end{theorem}

    \begin{proof}
        Заметим, что $\{X \setminus A_i\}_{i \in I}$ --- покрытие $X$ тогда и только тогда, когда $\bigcap_{i \in I} A_i = \varnothing$.

        \begin{itemize}
            \item[($\Rightarrow$)] Пусть $\{A_i\}_{i \in I}$ --- центрированный набор замкнутых множеств. Тогда $\{B_i\}_{i \in I} := \{X \setminus A_i\}_{i \in I}$ --- набор открытых множеств, что никакое их конечное подмножество не является покрытием $X$. Следовательно по компактности $X$ и весь набор $\{B_i\}_{i \in I}$ не является покрытием. Значит пересечение $A_i$, $i \in I$ непусто.

            \item[($\Leftarrow$)] Пусть $\{A_i\}_{i \in I}$ --- покрытие $X$. Следовательно $\{B_i\}_{i \in I} := \{X \setminus A_i\}_{i \in I}$ --- набор замкнутых множеств с пустым общим пересечением. Значит оно не центрировано, что значит, что есть конечный набор $\{B_{i_k}\}_{k=1}^n$ у которого пустое пересечение. Следовательно $\{A_{i_k}\}_{k=1}^n$ является конечным подпокрытием изначального покрытия.
        \end{itemize}
    \end{proof}

    \begin{corollary}
        Пусть $X$ --- топологическое пространство, а $\{A_i\}_{i \in I}$ --- центрированный набор замкнутых множеств в $X$, хотя бы одно из которых компактно. Тогда $\bigcap_{i \in I} A_i$ непусто.
    \end{corollary}

    \begin{corollary}
        Пусть $X$ --- топологическое пространство, $\{A_i\}_{i \in I}$ --- линейно упорядоченный по включению набор непустых замкнутых множеств в $X$, хотя бы одно из которых компактно. Тогда $\bigcap_{i \in I} A_i$ непусто.
    \end{corollary}

    \begin{theorem}
        Пусть даны непрерывное отображение $f: X \to Y$ и компактное пространство $X$. Тогда и пространство $f(X)$ компактно.
    \end{theorem}

    \begin{proof}
        Пусть $\Sigma$ --- открытое покрытие $f(X)$. Тогда
        \[\Lambda := \{f^{-1}(U) \mid U \in \Sigma\}\]
        --- покрытие $X$. По компактности $X$ у него есть конечное подпокрытие $\Lambda'$. Значит
        \[\Sigma' := \{f(V) \mid V \in \Lambda'\}\]
        будет конечным подпокрытием $\Sigma$. Следовательно $f(X)$ компактно.
    \end{proof}

    \begin{corollary}
        Компактность --- топологическое свойство.
    \end{corollary}

    \begin{theorem}[Вейерштрассса]
        Если $f: X \to \RR$ --- непрерывная функция и пространство $X$ компактно, то $f(x)$ достигает наибольшего и наименьшего значений.
    \end{theorem}

    \begin{proof}
        $f(X)$ компактно. Следовательно замкнуто и ограничено. Значит содержит свои инфимум и супремум.
    \end{proof}

    \begin{theorem}
        Если $f: X \to Y$ --- непрерывная биекция компактного пространства $X$ на хаусдорфово пространство $Y$, то $f$ --- гомеоморфизм.
    \end{theorem}

    \begin{proof}
        Для гомеоморфизма $f$ не хватает только обратной непрерывности. Покажем, что образ всякого замкнутого замкнут, и тогда обратная непрерывность будет обеспечена.

        Пусть $V$ --- замкнутое подмножество компакта $X$. Значит $V$ --- компакт. Следовательно $f(V)$ --- компакт как непрерывный образ компакта. И тогда $f(V)$ замкнуто, так как является компактом в хаусдорфовом пространстве.
    \end{proof}

    \begin{definition}
        Отображение $f: X \to Y$ называется \emph{вложением}, если $f$ --- гомеоморфизм между $X$ и $f(X)$. Иначе говоря, $f$ --- вложение, если
        \begin{itemize}
            \item $f$ непрерывно;
            \item $f$ --- инъекция;
            \item $f^{-1}$ непрерывно на области определения.
        \end{itemize}
    \end{definition}

    \begin{corollary}
        Если $f: X \to Y$ --- непрерывная инъекция компактного пространства $X$ в хаусдорфово пространство $Y$, то $f$ --- вложение.
    \end{corollary}

    \begin{lemma}[Лебега]
        Пусть даны компактное метрическое пространство $X$ и его открытое покрытие $\Sigma$. Тогда существует такое $r > 0$, что любой шар радиуса $r$ содержится в одном элементе покрытия.
    \end{lemma}

    \begin{definition}
        Число $r$ называется \emph{числом Лебега} покрытия $\Sigma$.
    \end{definition}

    \begin{proof}
        Для всякого $x \in X$ есть некоторое $r_x > 0$, что шар $B_{r_x}(x)$ содержится как подмножество некоторого элемента $\Sigma$.

        Понятно, что $\{B_{r_x/2}(x)\}_{x \in X}$ --- открытое покрытие $X$. Следовательно у него есть конечное подпокрытие $\{B_{r_{x_i}/2}(x_i)\}_{i=1}^n$. Тогда определим $r := \min \{r_{x_i}/2\}_{i=1}^n$.

        Если $y$ --- некоторая точка $X$, то $y$ лежит в некотором шаре $B_{r_{x_k}/2}(x_k)$. Следовательно
        \[B_r(y) \subseteq B_{r_{x_k}}(x_k),\]
        т.е. шар $B_r(y)$ является подмножеством некоторого элемента $\Sigma$. Поскольку утверждение не зависит от $y$, то $r$ является числом Лебега покрытия $\Sigma$.
    \end{proof}

    \begin{corollary}
        Пусть даны компактное метрическое пространство $X$, топологическое пространство $Y$, непрерывное $f: X \to Y$ и открытое покрытие $\Sigma$ множества $Y$. Тогда существует $r > 0$, что для всякой точки $a$ из $X$ множество $f(B_r(a))$ содержится как подмножество в одном из элементов $\Sigma$.
    \end{corollary}

    \begin{proof}
        Применим лемму Лебега к покрытию $\{f^{-1}(U) \mid U \in \Sigma\}$.
    \end{proof}

    \begin{definition}
        Пусть даны метрические пространства $(X, d_X)$ и $(Y, d_Y)$. Отображение $f: X \to Y$ называется \emph{равномерно непрерывным}, если
        \[\forall \varepsilon > 0\; \exists \delta > 0:\; \forall a, b \in X\qquad d_X(a, b) < \delta \longrightarrow d_Y(f(a), f(b)) < \varepsilon\]
    \end{definition}

    \begin{theorem}
        Пусть даны метрические пространства $X$ и $Y$. Тогда если $X$ компактно, то любое непрерывное $f: X \to Y$ будет равномерно непрерывным.
    \end{theorem}

    \begin{proof}
        Применим лемму Лебега для отображения $f$ и покрытия пространства $Y$ шарами радиуса $\varepsilon/2$.
    \end{proof}

    \begin{definition}
        Пусть $(a_n)_{n=0}^\infty$ --- последовательность точек топологического пространства $X$. Точка $b \in X$ называется её \emph{пределом}, если для всякой окрестности $U$ точки $b$ есть $N \in \NN$ такое, что $a_n \in U$ для всех $n > N$.
        
        Если $b$ --- предел последовательности $(a_n)_{n=0}^\infty$, то говорят, что $(a_n)_{n=0}^\infty$ \emph{сходится} к $b$ ($(a_n)_{n=0}^\infty \to b$, $b = \lim\, (a_n)_{n=0}^\infty$).
    \end{definition}

    \begin{theorem}
        В хаусдорфовом пространстве ни одна последовательность не может иметь более одного предела.
    \end{theorem}

    \begin{definition}
        Пусть $A$ --- подмножество топологического пространства $X$. Совокупность пределов всевозможных последовательностей точек множества $A$ называются \emph{секвенциальным замыканием} этого множества. Обозначение: $\SCl(A)$.
    \end{definition}

    \begin{theorem}
        $\SCl(A) \subseteq \Cl(A)$.
    \end{theorem}

    \begin{proof}
        Предел последовательности точек из $A$ --- точка прикосновения множества $A$.
    \end{proof}

    \begin{theorem}
        Если пространство $X$ удовлетворяет первой аксиоме счётности, то для любого $A \subseteq X$ верно $\SCl(A) = \Cl(A)$.
    \end{theorem}

    \begin{proof}
        Пусть $b \in \Cl(A)$. Если $\{V_i\}_{i = 0}^\infty$ --- счетная база в точке $b$, то $U_n = \bigcap_{i = 0}^n V_i$ --- убывающая база в точке $b$ ($U_0 \supseteq U_1 \supseteq U_2 \supseteq \dots$). Для всякого $n \in \NN \cup \{0\}$ выбираем $a_n \in U_n \cap A$. Тогда $(a_n)_{n=0}^\infty \to b$.
    \end{proof}

    \begin{definition}
        Последовательность $(a_n)_{n=0}^\infty$ в метрическом пространстве $(X, d)$ называется \emph{фундаментальной}, если $\forall \varepsilon > 0 \exists N \in \NN: \forall n, m > N d(a_n, a_m) < \varepsilon$. Другие названия: \emph{``последовательность Коши''}, \emph{``сходящаяся в себе''}.
    \end{definition}

    \begin{lemma}\ 
        \begin{enumerate}
            \item Всякая сходящаяся последовательность фундаментальна.
            \item Всякая фундаментальная последовательность ограничена.
            \item Всякая фундаментальная последовательность, содержащая сходящуюся подпоследовательность, сходится к тому же значению.
        \end{enumerate}
    \end{lemma}

    \begin{definition}
        Метрическое пространство называется \emph{полным}, если всякая его фундаментальная последовательность имеет предел.
    \end{definition}

    \begin{theorem}
        $\RR^n$ полно.
    \end{theorem}

    \begin{proof}
        Пусть $(a_k)_{k=0}^\infty$ --- фундаментальная последовательность точек $\RR^n$. Пусть также $a_k = (a_{k,1}, \dots, a_{k, n})$. Тогда для всякого $i \in \{1; \dots; n\}$ последовательность $(a_{k,i})_{k=0}^\infty$ фундаментальна, значит сходится к некоторому $A_i$. Тогда $(a_k)_{k=0}^\infty$ сходится к $A := (A_1, \dots, A_n)$.
    \end{proof}

    \begin{theorem}
        Пусть даны полное пространство $X$ и его замкнутое подпространство $Y$. Тогда $Y$ полно.
    \end{theorem}

    \begin{proof}
        Пусть $(a_n)_{n=0}^\infty$ --- фундаментальная последовательность в $Y$. Так как $X$ полно, то у неё есть предел $a$. Таким образом $a$ является предельной точкой $Y$, а тогда по замкнутости $Y$ имеем, что $a \in Y$.
    \end{proof}

    \begin{example}\ 
        \begin{enumerate}
            \item $[0;1]$ --- полное подпространство.
            \item $(0;1)$ --- неполное подпространство.
        \end{enumerate}
    \end{example}

    \begin{corollary}
        Полнота --- не топологическое свойство, так как $\RR \simeq (0; 1)$, но $\RR$ полно, а $(0; 1)$ неполно.
    \end{corollary}

    \begin{theorem}
        Метрическое пространство является полным тогда и только тогда, когда любая убывающая последовательность его замкнутых шаров с радиусами, стремящимися к нулю, обладает непустым пересечением.
    \end{theorem}

    \begin{proof}\ 
        \begin{itemize}
            \item[($\Rightarrow$)] Пусть $D_{r_0} \supseteq D_{r_1} \supseteq \dots$ --- убывающая последовательность замкнутых шаров, причём $(r_n)_{n=0}^\infty \to 0$. В каждом $D_{r_n}$ выберем точку $a_n$. Поскольку $(r_n)_{n=0}^\infty \to 0$, то $(a_n)_{n=0}^\infty$ фундаментальна. Тогда по полноте $X$ следует, что у неё есть предел $a$.
            
            Заметим, что $a_k \in D_{r_n}$ для всяких $k \geqslant n \geqslant 0$, а $D_{r_n}$ замкнуто, значит $a \in D_{r_n}$. Таким образом $a \in \bigcap_{n=0}^\infty D_{r_n}$.

            \item[($\Leftarrow$)] Пусть $(a_n)_{n=0}^\infty$ --- фундаментальная последовательность. Заметим, что для всякого $n \in \NN \cup \{0\}$ есть $N_n \in \NN \cup \{0\}$, что для всяких $k, l \geqslant N_n$ верно, что $d(a_k, a_l) \leqslant \frac{1}{2^n}$ и $N_{n+1} \geqslant N_n$. Значит $a_k \in D_{1/2^n}(a_{N_n})$ для всех $k \geqslant N_n$.
            
            Таким образом получим последовательность шаров $D_{1}(a_{N_0}) \supseteq D_{1/2}(a_{N_1}) \supseteq D_{1/4}(a_{N_2}) \supseteq \dots$. Тогда в их пересечении есть точка $a$. Несложно понять, что $a$ --- предел $(a_n)_{n=0}^\infty$.
        \end{itemize}
    \end{proof}

    \begin{definition}
        Внешность множества $A$ --- $\Ext(A) := \Int(X \setminus A)$.
    \end{definition}

    \begin{lemma}\ 
        \begin{enumerate}
            \item $\Ext(A)$ открыто.
            \item $X := \Int(A) \sqcup \Fr(A) \sqcup \Ext(A)$.
        \end{enumerate}
    \end{lemma}

    \begin{definition}
        Подножество $A$ топологического пространства $X$ называется \emph{нигде не плотным}, если $\Int(\Cl(A)) = \varnothing$.
    \end{definition}

    \begin{lemma}
        TFAE
        \begin{enumerate}
            \item $A$ нигде не плотно.
            \item $\Ext(A)$ всюду плотно.
            \item Любое непустое открытое $U \subseteq X$ содержит как подмножество непустое открытое $V \subseteq U$, что $V \cap A = \varnothing$.
        \end{enumerate}
    \end{lemma}

    \begin{proof}
        \begin{itemize}
            \item $(1) \Leftrightarrow \Int(\Cl(A)) = \varnothing \Leftrightarrow X \setminus \Int(\Cl(A)) = X \Leftrightarrow \Cl(\Int(X \setminus A)) = X \Leftrightarrow (2) \Leftrightarrow \Cl(\Ext(A)) = X$.
            \item Заметим, что $V \cap A = \varnothing \Leftrightarrow V \subseteq X \setminus A \Leftrightarrow V \subseteq \Ext(A)$. Поэтому $(2 \leftrightarrow 3)$ равносильно тому, что открытое $A$ всюду плотно тогда и только тогда, когда для всякого непустого открытого $U$ есть непустое открытое подмножество $V$, что $V \subseteq A$.

            Заметим, что $U \cap A$ открыто. Поэтому искомое $V$ существует $\Leftrightarrow$ $U \cap A \neq \varnothing$ $\Leftrightarrow$ $U \nsubseteq X \setminus A$ $\Leftrightarrow$ $A$ всюду плотно.
        \end{itemize}
    \end{proof}

    \begin{theorem}[Бэра]
        Полное пространство нельзя покрыть счётным набором нигде неплотных множеств.
    \end{theorem}

    \begin{proof}
        Предположим противное. Пусть $\{A_i\}_{i=0}^\infty$ --- счётное покрытие $X$ нигде не плотными множествами.

        Построим последовательность вложенных закрытых шаров $(D_n)_{n=0}^\infty$ с радиусами $(r_n)_{n=0}^\infty$ следующим образом. $D_0$ --- любой шар (ненулевого радиуса). Шар $D_{n+1}$ строится так. $\Int(D_n) \cap \Ext(A_n)$ непусто и открыто, значит содержит открытый шар $B$, а он содержит закрытый шар $D_{n+1}$ чей радиус $r_{n+1} \leqslant r_n/2$. Значит $D_{n+1} \subseteq D_n$ и $D_{n+1} \cap A_n = \varnothing$.
        
        Поскольку мы построили уменьшающуюся последовательность шаров, что их радиусы сходятся у нулю, то в их пересечении лежит некоторая точка $a$. Так как для всякого $n \in \NN \cup \{0\}$ верно, что $a \in D_{n+1}$, то $a \notin A_n$, значит $a \notin \bigcup_{n=0}^\infty A_n = X$ --- противоречие.
    \end{proof}

    \begin{corollary}
        Полное пространство без изолированных точек несчётно.
    \end{corollary}

    \begin{proof}
        Предположим противное: $X$ счётно. Покроем его точками из него самого. Так как каждая точка не изолирована, то она нигде не плотна. Противоречие с теоремой Бэра.
    \end{proof}

    \begin{corollary}
        Пусть $X$ --- полное пространство, $A$ --- объединение счётного набора нигде не плотных множеств. Тогда $\Int(A) = \varnothing$.
    \end{corollary}

    \begin{proof}
        Пусть по условию $A := \bigcup_{n=0}^\infty A_n$. Если $\Int(A)$ непусто, то $A$ содержит некоторый закрытый шар $D$. Тогда $\{A_n \cap D\}_{n=0}^\infty$ --- есть покрытие $D$ нигде не плотными множествами. При этом $D$ замкнут, значит является полным. Получаем противоречие с теоремой Бэра.
    \end{proof}

    \begin{definition}
        Пусть $X$ --- метрическое пространство. \emph{Пополнение} $X$ --- такое метрическое пространство $X$, что
        \begin{itemize}
            \item $\overline{X}$ полно;
            \item $X$ --- подпространство $\overline{X}$;
            \item $X$ всюду плотно в $\overline{X}$.
        \end{itemize}
    \end{definition}

    \begin{theorem}
        У любого метрического пространства есть пополнение.
    \end{theorem}

    \begin{proof}
        Рассмотрим множество $Y_0$ фундаментальных последовательностей в пространстве $X$ и определим функцию
        \[d_{Y_0}: Y_0 \times Y_0 \to [0; + \infty), ((a_k)_{k=0}^\infty, (b_l)_{l=0}^\infty) \mapsto \lim_{n \to \infty} d_X(a_n, b_n)\]
        Сначала поймём, почему $d_{Y_0}$ корректно определено, т.е. почему предел в его определении существует и неотрицателен. Заметим, что для всяких $n$ и $m$
        \[d_X(a_n, b_n) - d_X(a_n, a_m) - d_X(b_n, b_m) \leqslant d_X(a_m, b_m) \leqslant d_X(a_n, b_n) + d_X(a_n, a_m) + d_X(b_n, b_m)\]
        Следовательно $|d_X(a_n, b_n) - d_X(a_m, b_m)| \leqslant d_X(a_n, a_m) + d_X(b_n, b_m)$. Поэтому из фундаментальности $(a_n)_{n=0}^\infty$ и $(b_n)_{n=0}^\infty$ следует фундаментальность $(d_X(a_n, b_n))_{n=0}^\infty$. Значит
        \[\lim (d_X(a_n, b_n))_{n=0}^\infty\]
        определён и неотрицателен, так как все члены последовательности неотрицательны.

        Заметим также, что
        \begin{itemize}
            \item $d_{Y_0}((a_k)_{k=0}^\infty, (b_l)_{l=0}^\infty) = d((b_l)_{l=0}^\infty, (a_k)_{k=0}^\infty)$. Это очевидно, так как $d_X(a_n, b_n) = d_X(b_n, a_n)$.
            \item $d_{Y_0}((a_k)_{k=0}^\infty, (b_l)_{l=0}^\infty) + d((b_l)_{l=0}^\infty, (c_m)_{m=0}^\infty) \leqslant d((a_k)_{k=0}^\infty, (c_m)_{m=0}^\infty)$. Это верно, поскольку
            \[d_X(a_n, b_n) + d_X(b_n, c_n) \leqslant d_X(a_n, c_n),\]
            значит верно и в пределе.
        \end{itemize}

        У нас получилось, что $d_{Y_0}$ до метрики не хватает только того, что $d_{Y_0}(A, B) = 0 \leftrightarrow A = B$, что очевидно неверно для $Y_0$; например ``расстояние'' по $d_{Y_0}$ между $(x, x, x, \dots)$ и $(y, x, x, x, \dots)$ для всяких $x, y \in X$ равно $0$, но сами последовательности, очевидно, не совпадают.

        Давайте рассмотрим отношение $\sim$, определяемое так: $A \sim B \leftrightarrow d_{Y_0}(A, B) = 0$. Заметим, что
        \begin{itemize}
            \item $A \sim A$ --- очевидно.
            \item $A \sim B \leftrightarrow B \sim A$. Так как $d_{Y_0}(A, B) = d_{Y_0}(B, A)$.
            \item $A \sim B \sim C \rightarrow A \sim C$. Поскольку $d_{Y_0}(A, C) \leqslant d_{Y_0}(A, B) + d_{Y_0}(B, C) = 0$, значит $d_{Y_0}(A, C) = 0$.
        \end{itemize}
        Т.е. $\sim$ --- отношение эквивалентности. Тогда рассмотрим $Y_1 := Y_0/{\sim}$. Определим на нём функцию
        \[d_{Y_1}: Y_1 \times Y_1 \to [0; + \infty), ([A], [B]) \mapsto d_{Y_0}(A, B)\]
        Заметим, что если $A_1 \sim A_2$, то $d_{Y_0}(A_1, B) = d_{Y_0}(A_2, B)$; поэтому $d_{Y_1}$ определено корректно (значение в результате не меняется при замене представителей классов эквивалентности). При этом $d_{Y_1}$ наследует от $d_{Y_0}$ следующие свойства:
        \begin{itemize}
            \item $d_{Y_1}([A], [B]) = d_{Y_1}([B], [A])$;
            \item $d_{Y_1}([A], [B]) + d_{Y_1}([B], [C]) \geqslant d_{Y_1}([A], [C])$.
        \end{itemize}
        При этом от отношения эквивалентности $\sim$ оно наследует то, что $d_{Y_1}([A], [B]) = 0 \leftrightarrow [A] = [B]$. Значит $d_{Y_1}$ --- метрика на $Y_1$.

        Заметим, что $d_{Y_0}((x_1)_{n=0}^\infty, (x_2)_{n=0}^\infty) = d_X(x_1, x_2)$ для любых $x_1, x_2 \in X$, а значит
        \[d_{Y_1}([(x_1)_{n=0}^\infty], [(x_2)_{n=0}^\infty]) = d_X(x_1, x_2).\]
        Таким образом множество
        \[Y' := \{[(x)_{n=0}^\infty] \mid x \in X\} \subseteq Y_1\]
        с индуцированной на нём метрикой пространства $Y_1$ изоморфно пространству $X$. Т.е. $X$ изоморфно некоторому подпространству $Y_1$. Заметим, что если в $Y_1$ заменить элементы из $Y'$ на соответствующие элементы $X$ и оставить метрику как есть, то получим искомое $\overline{X}$ за исключением того, что не показано, почему $\overline{X}$ полно, а $X$ в нём всюду плотно.
        
        Поэтому обозначим данное пространство за $\overline{X}$ и докажем требуемые свойства. Также будем обозначать $[(x)_{n=0}^\infty]$ за просто $[x]$.

        \begin{thlemma}
            Пусть $A = (a_n)_{n=0}^\infty$ --- фундаментальная последовательность в $X$, что для всяких $n, m \in \NN \cup \{0\}$ верно, что $d_X(n, m) \leqslant \lambda$ для некоторого данного $\lambda$. Тогда для всякого $n \in \NN \cup \{0\}$
            \[d_{\overline{X}}([A], [a_n]) \leqslant \lambda\]
        \end{thlemma}

        \begin{proof}
            По определению
            \[d_{\overline{X}}([A], [a_n]) = d_{Y_0}(A, (a_n)_{k=0}^\infty) = \lim_{k \to \infty} d_X(a_k, a_n) \leqslant \lambda\]
        \end{proof}

        \begin{thlemma}
            Пусть $A = (a_n)_{n=0}^\infty$ --- фундаментальная последовательность в $X$. Тогда
            \[\bigl((a_n)_{k=0}^\infty\bigr)_{0}^\infty\]
            --- фундаментальная последовательность в $Y_0$, сходящаяся к $A$.
        \end{thlemma}

        \begin{proof}
            Заметим, что для всякого $\varepsilon > 0$ есть $N \in \NN \cup \{0\}$, что для всяких $n, m \geqslant N$ верно, что $d_X(a_n, a_m) \leqslant \varepsilon$. Значит
            \[d_{Y_0}((a_n)_{k=0}^\infty, (a_m)_{k=0}^\infty) = d_X(a_n, a_m) \leqslant \varepsilon,\]
            поэтому $\bigl((a_n)_{k=0}^\infty\bigr)_{0}^\infty$ фундаментальна. Также
            \[d_{Y_0}((a_n)_{k=0}^\infty, A) = \lim_{k \to \infty} d_X(a_n, a_k) \leqslant \varepsilon,\]
            что значит, что
            \[\lim_{n \to \infty} d_{Y_0}((a_n)_{k=0}^\infty, A) = 0,\]
            т.е. $A$ --- предел $\bigl((a_n)_{k=0}^\infty\bigr)_{0}^\infty$.
        \end{proof}

        \begin{thlemma}
            Пусть $A = (a_n)_{n=0}^\infty$ --- фундаментальная последовательность в $X$. Тогда $([a_n])_{0}^\infty$ --- фундаментальная последовательность в $\overline{X}$, сходящаяся к $[A]$.
        \end{thlemma}

        \begin{proof}
            Несложно следует из предыдущей леммы c повторением её же доказательства.
        \end{proof}

        Это значит, что $X$ всюду плотно в $\overline{X}$, так как $\SCl(X) = \overline{X}$.

        Тогда покажем, что $\overline{X}$ полно. Пусть $(A_n)_{n=0}^\infty$ --- фундаментальная последовательность в $\overline{X}$. Для всякого $n \in \NN \cup \{0\}$ в $U_{1/2^n}(A_n)$ возьмём элемент $a_n$ из $X$. Тогда $d_{\overline{X}}([a_n], A_n) < 1/2^n$. Значит $L = (a_n)_{n=0}^\infty$ --- фундаментальная последовательность в $X$. Значит $[L] \in \overline{X}$. Также понятно, что $(A_n)_{n=0}^\infty$ сходится к $[L]$.
    \end{proof}

    \begin{definition}
        Топологическое пространство \emph{секвенциально компактно}, если любая последовательность его точек содержит сходящуюся подпоследовательность.
    \end{definition}

    \begin{definition}
        Точка $b$ называется \emph{точкой накопления} множества $A$, если любая её окрестность содержит бесконечное число точек этого множества.
    \end{definition}

    \begin{theorem}
        В компактном пространстве всякое бесконечное множество имеет точку накопления.
    \end{theorem}

    \begin{proof}
        Пусть $S$ --- бесконечное подмножество $X$. Предположим противное: у всякой точки $x \in X$ есть окрестность $U_x$, что $U_x \cap S$ конечно. $\{U_x\}_{x \in X}$ --- открытое покрытие компактного $X$, значит есть конечное подпокрытие $\{U_{x_k}\}_{k=1}^n$. Значит
        \[|S| = \left|\bigcup_{k=1}^n S \cap U_k\right| \leqslant \sum_{k=1}^n |S \cap U_k| \in \NN\]
        --- противоречие с бесконечностью $S$.
    \end{proof}

    \begin{theorem}\label{compact_tern_part12_theorem}
        Всякое компактное метрическое пространство секвенциально компактно.
    \end{theorem}

    \begin{proof}
        Пусть дана последовательность $(a_n)_{n=0}^\infty$ в пространстве $X$.
        
        Если в данной последовательности лишь конечное число точек, то можно выделить константную последовательность, которая, очевидно, сходится.
        
        Иначе она содержит бесконечное количество точек, а значит по прошлой теореме оно имеет точку накопления $a$. Тогда можно построить последовательность $(a_{k_n})_{n=0}^\infty$, где $a_{k_n} \in U_{1/2^n}(a)$ и $k_{n+1} > k_n$; она будет подпоследовательностью $(a_n)_{n=0}^\infty$. Значит $X$ секвенциально замкнуто.
    \end{proof}

    \begin{theorem}[обобщение]\label{compact+FAC=>sequentially_compact_theorem}
        Если топологическое пространство $X$ компактно и удовлетворяет $\FAC$, то оно секвенциально компактно.
    \end{theorem}

    \begin{proof}
        Пусть дана последовательность $(a_n)_{n=0}^\infty$ в пространстве $X$.
        
        Если в данной последовательности лишь конечное число точек, то можно выделить константную последовательность, которая, очевидно, сходится.
        
        Иначе она содержит бесконечное количество точек, а значит по прошлой теореме оно имеет точку накопления $a$. Рассмотрим счётную базу $\{U_n\}_{n=0}^\infty$ в точке $b$; сделаем из неё другую счётную базу
        \[\{V_n\}_{n=0}^\infty := \left\{\bigcap_{k=0}^n U_k\right\}_{n=0}^\infty\]
        в точке $a$. Тогда можно построить последовательность $(a_{k_n})_{n=0}^\infty$, где $a_{k_n} \in V_n$ и $k_{n+1} > k_n$; она будет подпоследовательностью $(a_n)_{n=0}^\infty$. Значит $X$ секвенциально замкнуто.
    \end{proof}
    
    \begin{definition}
        Подмножество $A$ метрического пространства $X$ называется его \emph{$\varepsilon$-сетью} (для $\varepsilon > 0$), если $d(b, A) < \varepsilon$ для всякой точки $b \in X$.
    \end{definition}

    \begin{definition}
        Пространство $X$ \emph{вполне ограничено}, если для всякого $\varepsilon > 0$ существует конечная $\varepsilon$-сеть.
    \end{definition}

    \begin{exercise}
        Докажите, что у вского метрического пространства $X$ для всякого $\varepsilon > 0$ есть $\varepsilon$-сеть $A$, что для всяких $a_1, a_2 \in A$ верно, что $d(a_1, a_2) \geqslant \varepsilon$.
    \end{exercise}

    \begin{theorem}\label{compact_tern_theorem}
        Для метрического пространства $X$ TFAE:
        \begin{enumerate}
            \item $X$ компактно.
            \item $X$ секвенциально компактно.
            \item $X$ полно и вполне ограничено.
        \end{enumerate}
    \end{theorem}
    
    \begin{proof}\ 
        \begin{itemize}
            \item ($1 \Rightarrow 2$) См. теорему \ref{compact_tern_part12_theorem}.
            \item ($1 \Rightarrow X$ вполне ограничено) Пусть $\varepsilon > 0$. Выберем конечное подпокрытие из всех шаров радиуса $\varepsilon$. Тогда центры выбранных шаров дадут конечную $\varepsilon$-сеть.
            \item ($2 \Rightarrow X$ вполне ограничено) Предположим противное: для какого-то $\varepsilon > 0$ нет $\varepsilon$-сети. Тогда построим последовательность $(a_n)_{n=0}^\infty$ следующим образом. $a_0$ --- любая точка. $a_{n+1}$ --- любая точка, что для всех $k = 0, \dots, n$ верно, что $d(a_k, a_n) \geqslant \varepsilon$ (такое существует вследствие отсутствия всякой конечной $\varepsilon$-сети). Тогда получим последовательность, где всякие два члена удалены друг от друга хотя бы на $\varepsilon$, значит у неё не может быть сходящейся подпоследовательности --- противоречие с секвенциальной компактностью.
            \item ($2 \Rightarrow X$ полно) Всякая фундаментальная последовательность имеет по секвенциальной компактности сходящуюся подпоследовательность. А тогда из фундаментальности выходит, что вся последовательность сходится.
            \item ($3 \Rightarrow 1$) Предположим противное: есть минимальное по включению бесконечное открытое покрытие $\Sigma$ пространства $X$. Тогда построим последовательность замкнутых шаров $(C_n)_{n=-1}^\infty$, что у вского из них нет конечного подпокрытия покрытия $\Sigma$.
            
            Положим $C_{-1} := X$. Для всякого $n \in \NN \cup \{0\}$ рассмотрим конечную $1/2^n$-сеть $A_n$. Тогда $\{D_{1/2^n}(a) \cap C_{n-1}\}_{a \in A}$ --- конечное покрытие $C_{n-1}$. Тогда из несуществования конечного подпокрытия множества $C_{n-1}$ следует, что какое-то из множеств $D_{1/2^n}(a) \cap C_{n-1}$, где $a \in A_n$, не имеет конечного подпокрытия покрытия $\Sigma$. Тогда возьмём это множество за $C_n$.

            При этом мы получаем, что $C_n \subseteq D_{1/2^n}(a)$ для некоторой точки $a \in X$, значит $\diam(C_n) \leqslant 1/2^{n-1}$. Также $C_n \subseteq C_{n-1}$, и каждое $C_n$ замкнуто и не имеет конечного подпокрытия. Значит $\bigcap_{n=0}^\infty C_n = \{a\}$. Тогда $a$ покрывается каким-то $U \in \Sigma$. Следовательно для какого-то $N \in \NN \cup \{0\}$ верно, что $D_{1/2^N}(a) \subseteq U$, а тогда $C_{N+1} \subseteq U$, значит $\{U\}$ --- конечное подпокрытие множества $C_{N+1}$ --- противоречие. Значит всё-таки у $\Sigma$ есть конечное подпокрытие.
        \end{itemize}
    \end{proof}

    \begin{theorem}
        Всякое вполне ограниченное метрическое пространство имеет счетную базу (удовлетворяет \SAC).
    \end{theorem}

    \begin{proof}
        Рассмотрим для всякого $n \in \NN \cup \{0\}$ любую конечную $1/2^n$-сеть $A_n$ и множество $A := \bigcup_{n=0}^\infty A_n$. Тогда $A$ счётно и всюду плотно, а значит $X$ сепарабельно. Следовательно по теореме \ref{separable_space_properties} пространство $X$ удовлетворяет \SAC.
    \end{proof}

    \begin{corollary}
        Всякое компактное метрическое пространство имеет счетную базу.
    \end{corollary}

    \begin{proof}
        По теореме \ref{compact_tern_theorem} $X$ вполне ограничено, а значит удовлетворяет \SAC.
    \end{proof}

    \begin{theorem}
        Пусть топологическое пространство $X$ удовлетворяет \SAC. Тогда TFAE:
        \begin{enumerate}
            \item $X$ компактно.
            \item $X$ секвенциально компактно.
        \end{enumerate}
    \end{theorem}

    \begin{proof}\ 
        \begin{itemize}
            \item ($1 \Rightarrow 2$) По теореме \ref{AC_conseq_theorem} $\SAC \Rightarrow \FAC$. По теореме \ref{compact+FAC=>sequentially_compact_theorem} $X$ компактно и $FAC$ $\Rightarrow$ $X$ секвенциально компактно.
            \item ($2 \Rightarrow 1$) Пусть $\Sigma$ --- покрытие $X$. По теореме \ref{countable_supcover_by_SAC_theorem} из \SAC следует, что есть счётное подпокрытие $\{U_n\}_{n=0}^\infty$. Предположим, что у него нет конечного подпокрытия.

            Заметим, что $F_n := X \setminus U_n$ замкнуто, а значит $W_n := \bigcap_{i=0}^n F_i = X \setminus \bigcup_{i=0}^n U_i \neq \varnothing$ и $W_n$ замкнуто. Также $W_0 \supseteq W_1 \supseteq W_2 \supseteq \dots$

            Построим последовательность $(a_n)_{n=0}^\infty$, взяв из каждого $W_n$ по представителю $a_n$. Тогда по секвенциальной компактности выделим подпоследовательность $(a_{k_n})_{n=0}^\infty$, сходящуюся к некоторому $a$.

            $\SCl(W_{k_n}) \subseteq \Cl(W_{k_n}) = W_{k_n} \Rightarrow \forall n\; a \in W_{k_n} \Rightarrow \forall n\; a \in W_n \Rightarrow a \in \bigcap_{n=0}^\infty W_n = \bigcap_{n=0}^\infty F_n = X \setminus \bigcup_{n=0}^\infty U_n = \varnothing$ --- противоречие.
        \end{itemize}
    \end{proof}

    \begin{definition}
        \emph{Разбиение множества} --- это его покрытие попарно непересекающимися подмножествами.
    \end{definition}

    \begin{remark*}\ 
        \begin{itemize}
            \item С каждым разбиением $S$ множества $X$ связано отношение эквивалентности: $x \sim y \leftrightarrow$ $x$ и $y$ лежат в одном из множеств разбиения $S$.
            \item Обратно, с каждым отношением эквивалентности в множестве $X$ связано разбиение $S$ этого множества на классы эквивалентных элементов.
        \end{itemize}
        Во всех случаях $S$ играет роль фактормножества.
    \end{remark*}

    \begin{definition}
        \emph{Фактормножество} множества $X$ по его разбиению $S$ --- это множество, элементами которого являются подмножествами $X$, составляющие разбиение $S$. Обозначение: $X/S$.
        
        На втором языке, фактормножество $X/{\sim}$ множество классов эквивалентности.
    \end{definition}

    \begin{definition}
        \emph{Каноническая проекция} $X$ на $X/S$ --- это отображение $p: X \to X/S$, относящее каждой точке $x \in X$ содержащий её элемент разбиения $S$. Другое название --- \emph{отображение факторизации}.
    \end{definition}

    \begin{remark*}
        Если использовать фактор $X/{\sim}$, то отображение $p: X \to X/{\sim}$ сопоставляет каждой точке $x \in X$ её класс эквивалентности $[x]$.
    \end{remark*}

    \begin{definition}
        Пусть $X$ --- топологическое пространство. Тогда фактормножество $X/S$ наделяется естественной топологией: множество $U \subseteq X/S$ открыто в $X/S$ тогда и только тогда, когда его прообраз $p^{-1}(U)$ открыт в $X$.

        Эта топологическая структура называется \emph{фактортопологией}, а множество $X/S$, наделённое ею, называется \emph{факторпространством} пространства $X$ по разбиению $S$.
    \end{definition}

    \begin{remark*}
        Каноническая проекция является непрерывным отображением.
    \end{remark*}

    \begin{lemma}\ 
        \begin{enumerate}
            \item Факторпространство связного пространства связно.
            \item Факторпространство линейно связного пространства линейно связно.
            \item Факторпространство сепарабельного пространства сепарабельно.
            \item Факторпространство компактного пространства компактно.
        \end{enumerate}
    \end{lemma}

    \begin{proof}
        Все эти свойства сохраняются при непрерывных отображениях.
    \end{proof}

    \begin{definition}
        Частные случаи факторпространств.
        \begin{enumerate}
            \item \emph{Стягивание подмножества в точку.} Пусть $A \subseteq X$, тогда можно рассмотреть разбиение $S$, где $A$ стягивается в одну точку, а все остальные точки не трогаются.

            \item
            \begin{itemize}
                \item \emph{Несвязное объединение.} Пусть $X$, $Y$ --- топологические пространства. Тогда их \emph{несвязное объединение} --- множество $X \sqcup Y$ с топологией, где всякое подмножество $U$ открыто тогда и только тогда, когда $U \cap X$ открыто в $X$ и $U \cap Y$ открыто в $Y$.
                \item Аналогично можно рассматривать не только два пространство, а всякое семейство пространств. Если есть семейство топологических пространств $\Sigma$, то можно рассмотреть пространство $\bigsqcup_{X \in \Sigma} X$, где всякое подмножество $U$ открыто тогда и только тогда, когда $U \cap X$ открыто в $X$ для всех $X \in \Sigma$.
                \item \emph{Приклеивание по отображению.} Пусть даны топологические пространства $X$, $Y$, множество $A \subseteq X$ и непрерывное отображение $f: A \to Y$. Рассмотрим факторпространства несвязного объединения $X \sqcup Y$, где стягиваются множества $\{b\} \cup f^{-1}(b)$ для каждого $b \in f(A)$, а остальные точки остаются как есть. Это пространство обозначается как $X \sqcup_f Y$.
            \end{itemize}
            \begin{example}
                Если $X = Y = S^1$, $A = \{x\}$, где $x \in X$, а $f$ --- любое, то $X \sqcup_f Y$ --- ``восьмёрка'' (две окружности, склеенные по точке) со стандартной метрической топологией.
            \end{example}

            \item \emph{Склеивание частей одного пространства.} Пусть даны топологическое пространство $X$, множество $A \subseteq X$ и непрерывная функция $f: A \to X$. Тогда можем рассмотреть разбиение $S$ на минимальные множества, что для всякого $a \in f(A)$ точка $a$ и элементы $f^{-1}(a)$ лежат в одном множестве; в случае, если $A \cap f(A) = \varnothing$ неодноточечеными множествами разбиения $S$ будут $\{a\} \cup f^{-1}(a)$ для каждого $a \in f(A)$. В таком случае $X/S$ есть склейка $X$ по функции $f$. Обозначение: $X/f$.
            \begin{example}
                Пусть $X = [0; 1] \times [0; 1]$, $A = \{0\} \times [0; 1]$, $f: A \to X, (0, t) \mapsto (1, t)$. Тогда $X/f \simeq S^1 \times [0; 1]$ --- боковая поверхность цилиндра.
            \end{example}

            \item \emph{Фактор по действию группы.} Пусть даны топологическое пространство $X$ и подгруппа $\Gamma$ группы $\Homeo(X)$. Рассмотрим отношение эквивалентности $\sim$, где $x \sim y$ тогда и только тогда, когда $\exists g \in \Gamma:\; g(x) = y$. Тогда $X/{\sim}$ обозначается как $X/\Gamma$.
            \begin{example}
                Пусть $X = \RR$, а $\Gamma = \{f: X \to X, x \mapsto x + a \mid a \in \ZZ\}$ (в таком случае $\Gamma \cong \ZZ^+$). Тогда $X/\Gamma \simeq S^1$.
            \end{example}
        \end{enumerate}
    \end{definition}

    \begin{theorem}[о пропускании отображения через фактор]
        Пусть даны топологические пространства $X$ и $Y$, отношение эквивалентности $\sim$ на $X$, каноническая проекция $p: X \to X/{\sim}$ и отображение $f: X \to Y$, что для всяких $x_1, x_2 \in X$ верно, что $x_1 \sim x_2 \rightarrow f(x_1) = f(x_2)$. Тогда
        \begin{enumerate}
            \item Существует единственное отображение $\overline{f}: X/{\sim} \to Y$, что $f = \overline{f} \circ p$.
            \item $f$ непрерывно тогда и только тогда, когда $\overline{f}$ непрерывно.
        \end{enumerate}
    \end{theorem}

    \begin{proof}\ 
        \begin{enumerate}
            \item Заметим, что для всякого $T \in X/{\sim}$ верно, что:
            \begin{itemize}
                \item для каждого $x \in T$ значение $f(x)$ одно и то же;
                \item для всякого $x \in T$ верно, что $f(x) = \overline{f}(p(x)) = \overline{f}(T)$.
            \end{itemize}
            Из этого следует, что для всякого $T \in X/{\sim}$ значение $\overline{f}$ определено строго единственным образом, значит $\overline{f}$ существует и единственно.
            
            \item
            \begin{itemize}
                \item[($\Leftarrow$)] Очевидно, поскольку тогда $f$ является композицией непрерывных отображение, а значит само непрерывно.
                \item[($\Rightarrow$)] Пусть $U$ --- открытое множество в $Y$. Тогда $p^{-1}(\overline{f}^{-1}(U)) = f^{-1}(U)$ открыто в $X$. Следовательно $\overline{f}^{-1}(U)$ тоже открыто по определению топологии на $X/{\sim}$. Значит $\overline{f}$ непрерывно.
            \end{itemize}
        \end{enumerate}
    \end{proof}

    
\end{document}