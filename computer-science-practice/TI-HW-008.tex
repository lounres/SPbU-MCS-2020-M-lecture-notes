\documentclass[12pt,a4paper]{article}
\usepackage{solutions}
\usepackage{multicol}
\usepackage{float}
% \usepackage{todonotes}

\title{Домашнее задание от 26.10.\\Теоретическая информатика. 2 курс.\\Решения.}
\author{Глеб Минаев @ 204 (20.Б04-мкн)}
% \date{}

\begin{document}
    \maketitle

    \begin{multicols}{2}
        \tableofcontents
    \end{multicols}

    \begin{problem}{2}
        Пусть даны грамматика $G = (\Sigma, N, R, S)$ и ДКА $A = (\Sigma, Q, q_0, \delta, F)$. Давайте рассмотрим следующий алгоритм.

        Будем всё время хранить некоторое семейство $\Omega = \{T_{q, A}\}_{\substack{q \in Q\\ A \in N}}$ подмножеств $Q$ (т.е. $T_{q, A} \subseteq Q$). Мы хотим сделать так, чтобы
        \[T_{q, A} = \{p \in Q \mid \exists w \in L_G(A) \colon \; \delta^*(q, w) = p\}.\]
        Для этого рассмотрим $\Omega_0$, где
        \[T^0_{q, A} := \{\delta^*(q, w) \mid w \in \Sigma^* \wedge (A \to w) \in R\}.\]
        Далее по каждому $\Omega_n$ будем строить $\Omega_{n+1}$ по правилу
        \[
            T^{n+1}_{q, A} := \left\{p \, \left| \,
            \begin{gathered}
                \exists u_0, \dots, u_m \in \Sigma^* \colon\\
                \exists B_1, \dots, B_m \in N \colon\\
                \exists q_0, p_0, \dots, q_m, p_m \in Q \colon\\
                (A \to u_0 B_1 \dots B_m u_m) \in R\\
                {} \wedge \forall i \; \delta^*(q_i, u_i) = p_i\\
                {} \wedge \forall i\; q_{i+1} \in T^n_{p_i, B_{i+1}}\\
                {} \wedge q_0 = q \wedge p_m = p
            \end{gathered}
            \right.\right\}.
        \]

        \begin{lemma}
            $T^n_{q, A} \subseteq T^{n+1}_{q, A}$.
        \end{lemma}

        \begin{proof}
            Докажем по индукции по $n$.

            Если $n = 0$, то для всякого $p \in T^n_{q, A}$ есть $w \in \Sigma^*$, что $(A \to w) \in R$ и $\delta^*(q, w) = p$. Следовательно, $(A \to u_0) \in R$, где $u_0 = w$, $m = 0$, $q_0 = q$, $p_0 = p$, $\delta^*(q_0, u_0) = p_0$. Таким образом $p \in T^{n+1}_{q, A}$. Следовательно, $T^n_{q, A} \subseteq T^{n+1}_{q, A}$.

            Если $n > 0$, то для всякого $p \in T^n_{q, A}$ есть $(A \to u_0 B_1 \dots B_m u_m) \in R$, где $u_0, \dots, u_m \in \Sigma^*$, $B_1, \dots, B_m \in N$, и $q_0, p_0, \dots, q_m, p_m \in Q$, что для вского $i$ верно $\delta^*(q_i, u_i) = p_i$ и $q_{i+1} \in T^{n-1}_{p_i, B_{i+1}}$, а также $q_0 = q$, $p_m = p$. Так как $T^{n-1}_{p_i, B_{i+1}} \subseteq T^n_{p_i, B_{i+1}}$ по предположению индукции, то $q_{i+1} \in T^n_{p_i, B_{i+1}}$, а значит $p \in T^{n+1}_{q, A}$. Следовательно, $T^n_{q, A} \subseteq T^{n+1}_{q, A}$.
        \end{proof}

        \begin{lemma}
            Если $\Omega_n = \Omega_{n+1}$, то $\Omega_m = \Omega_{m+1}$ для всякого $m \geqslant n$.
        \end{lemma}

        \begin{proof}
            Докажем утверждение по индукции по $m$.

            Если $m = n$, то утверждение вырождается в условие. Тогда $m > n$. Тогда по предположению индукции $T^m_{p, B} = T^{m-1}_{p, B}$ для всех $p \in Q$ и $B \in N$. Тогда для всякого $p \in T^{m+1}_{q, A}$ есть $(A \to u_0 B_1 \dots B_k u_k) \in R$, где $u_0, \dots, u_k \in \Sigma^*$, $B_1, \dots, B_k \in N$, и $q_0, p_0, \dots, q_k, p_k \in Q$, что для вского $i$ верно $\delta^*(q_i, u_i) = p_i$ и $q_{i+1} \in T^m_{p_i, B_{i+1}}$, а также $q_0 = q$, $p_k = p$. Но тогда $q_{i+1} \in T^{m-1}_{p_i, B_{i+1}}$. Следовательно, $p \in T^m_{q, A}$. Т.е. $T^{m+1}_{q, A} \subseteq T^m_{q, A}$ для всех $q \in Q$ и $A \in N$.
        \end{proof}

        \begin{corollary}
            Если $\Omega_n = \Omega_{n+1}$, то $\Omega_m = \Omega_n$ для всех $m \geqslant n$.
        \end{corollary}

        \begin{lemma}
            $T_{q, A} = \bigcup_{n=0}^\infty T^n_{q, A}$.
        \end{lemma}

        \begin{proof}
            Сначала покажем по индукции по $n$, что $T^n_{q, A} \subseteq T_{q, A}$.

            Если $n = 0$, то утверждение очевидно. Если $n > 0$, то для всякого $p \in T^n_{q, A}$ есть $(A \to u_0 B_1 \dots B_m u_m) \in R$, где $u_0, \dots, u_m \in \Sigma^*$, $B_1, \dots, B_m \in N$, и $q_0, p_0, \dots, q_m, p_m \in Q$, что для вского $i$ верно $\delta^*(q_i, u_i) = p_i$ и $q_{i+1} \in T^{n-1}_{p_i, B_{i+1}}$, а также $q_0 = q$, $p_m = p$. При этом $q_{i+1} \in T^{n-1}_{p_i, B_{i+1}} \subseteq T_{p_i, B_{i+1}}$. Значит есть $v_i \in L_G(B_{i+1})$, что $\delta^*(p_i, v_i) = q_{i+1}$. Следовательно,
            \[p_m = \delta^*(q_0, u_0 v_1 \dots v_m u_m).\]
            При этом $u_0 v_1 \dots v_m u_m \in L_G(A)$. Таким образом $p \in T_{q, A}$. Следовательно, $T^n_{q, A} \subseteq T_{q, A}$.

            Следовательно, $\bigcup_{n=0}^\infty T^n_{q, A} \subseteq T_{q, A}$.

            Теперь покажем по индукции по размеру дерева разбора $w \in L_G(A)$, что $\delta(q, w) \in \bigcup_{n=0}^\infty T^n_{q, A}$.

            Рассмотрим первую подстановку у $w$: $A \to u_0 B_1 \dots B_m u_m$. Следовательно, $w = u_0 v_1 \dots v_m u_m$, где $v_i \in L_G(B_i)$. Определим $q_i := \delta^*(q, u_0 v_1 \dots v_i)$, $p_i := \delta^*(q, u_0 v_1 \dots v_i u_i)$. Тогда $p_i = \delta^*(q_i, u_i)$, $q_{i+1} = \delta^*(p_i, v_{i+1})$. Так как деревья разборов всех $v_i$ меньше изначального, то по предположению индукции $q_{i+1} \in \bigcup_{n=0}^\infty T^n_{p_i, B_{i+1}}$, а значит $q_{i+1} \in T^{n_i}_{p_i, B_{i+1}}$ для некоторого $n_i$. Пусть $N := \max_i n_i$. Тогда $q_{i+1} \in T^N_{p_i, B_{i+1}}$. Следовательно $\delta^*(q, w) \in T^{N+1}_{q, A}$. А тогда $\delta^*(q, w) \in \bigcup_{n=0}^\infty T^n_{q, A}$.

            Следовательно, $T_{q, A} \subseteq \bigcup_{n=0}^\infty T^n_{q, A}$.
        \end{proof}

        Заметим, что $\Omega_0$ строится алгоритмически, а $\Omega_{n+1}$ строится по $\Omega_n$ алгоритмически. Заметим, что последовательность $(\Omega_n)_{n=0}^\infty$ стабилизируется, так как при переходе от $\Omega_n$ к $\Omega_{n+1}$ каждое $T^n_{q, A}$ сохраняет все старые элементы и, возможно, подбирает новые, значит бесконечно ``расти'' данная последовательность не может. Поэтому чтобы построить $\Omega$ достаточно начать строить последовательность $(\Omega_n)_{n=0}^\infty$ и строить, пока последние два члена не совпадут. Тогда получится последовательность $(\Omega_n)_{n=0}^k$, что с $\Omega_{k-1}$ бесконечная последовательность стабилизируется. Тогда понятно, что
        \[T_{q, A} = \bigcup_{n=0}^{k-1} T^n_{q, A}.\]
        Таким образом можно алгоритмически построить $\Omega$.

        Как только мы построили $\Omega$, вся задача заключается в проверке того, что $T_{q_0, S} \subseteq F$. Эта задача, очевидно, алгоритмически разрешима, а значит и вся задача алгоритмически разрешима.
    \end{problem}

    \begin{problem}{4}\ 
        \begin{enumerate}
            \renewcommand{\theenumi}{\alph{enumi}}
            \renewcommand{\labelenumi}{(\theenumi)}
            \item Предъявим алгоритм, который распознаёт свойство $L(G_1) \neq L(G_2)$.

                Действительно, давайте просто будем перебирать все слова подряд (это, понятно, алгоритмически разрешимо) и для каждого слова запустим алгоритм алгоритм Кокка---Касами---Янгера сначала на $G_1$, а потом на $G_2$. Если на каком-то слове ответы алгоритма Кокка---Касами---Янгера не совпали, то скажем "да". Иначе будем идти дальше.
                
                Если $L(G_1) \neq L(G_2)$, то есть слово $w \in L(G_1) \triangle L(G_2)$. Тогда алгоритм рано или поздно дойдёт до него, получит разные ответы алгоритма Кокка---Касами---Янгера на нём и каждой из грамматик $G_1$ и $G_2$ и скажет "да". Если же $L(G_1) = L(G_2)$, то для всех слов ответы будут совпадать, а значит алгоритм просто не закончит свою работу.

                Следовательно, свойство $L(G_1) \neq L(G_2)$ распознаётся. Следовательно, свойство $L(G_1) = L(G_2)$ не распознаётся, так как иначе бы $L(G_1) = L(G_2)$ было бы разрешимым. Но мы доказали на лекции обратное.
            
            \item Предъявим алгоритм, который распознаёт свойство $L(G_1) \cap L(G_2) \neq \varnothing$.

                Действительно, давайте просто будем перебирать все слова подряд (это, понятно, алгоритмически разрешимо) и для каждого слова запустим алгоритм алгоритм Кокка---Касами---Янгера сначала на $G_1$, а потом на $G_2$. Если на каком-то слове ответы алгоритма Кокка---Касами---Янгера оба будут "да", то скажем "да". Иначе будем идти дальше.

                Если $L(G_1) \cap L(G_2) \neq \varnothing$, то есть слово $w \in L(G_1) \cap L(G_2)$. Тогда алгоритм рано или поздно дойдёт до него, получит оба ответа "да" и скажет "да". Если же $L(G_1) \cap L(G_2) = \varnothing$, то алгоритм не будет находить такого слова и не закончит свою работу.

            \item Предъявим алгоритм, который распознаёт свойство неоднозначности $G_1$.

                Действительно, давайте просто будем перебирать все слова подряд (это, понятно, алгоритмически разрешимо) и для каждого слова запустим алгоритм алгоритм Кокка---Касами---Янгера на $G_1$. Далее по полученной таблице каждого слова будем восстанавливать все возможные деревья. Если на каком-то слове получилось два различных дерева разбора, то скажем "да". Иначе будем идти дальше.
                
                Если $G_1$ неоднозначно, то есть слово $w$, которое неоднозначно задаётся грамматикой $G_1$. Тогда алгоритм рано или поздно дойдёт до него, получит два разных дерева по модификации алгоритма Кокка---Касами---Янгера на нём и грамматике $G_1$ и скажет "да". Если же $G_1$ однозначна, то для всех слов двух деревьев не найдётся, а значит алгоритм просто не закончит свою работу.

                Следовательно, свойство неоднозначности $G_1$ распознаётся. Следовательно, свойство однозначности $G_1$ не распознаётся, так как иначе бы однозначаность $G_1$ была бы разрешима. Но мы доказали на лекции обратное.
        \end{enumerate}
    \end{problem}

    \begin{problem}{5}
        Пусть дана грамматика $G = (\Sigma, N, R, S)$. Рассмотрим грамматику $G' = (\Sigma, N', R', S')$, где
        \begin{itemize}
            \item $N' := N \cup \{S_A\}_{A \in N} \cup \{S'\}$, где $S_A$ --- копия $S$ для каждого $A \in N$,
            \item $R'$ состоит из правил
                \begin{itemize}
                    \item все правила из $R$,
                    \item $S_A \to q S_B p$ для всякого $(B \to pAq) \in R$ и всяких $A, B \in N$,
                    \item $S' \to q S_A p$ для всякого $(A \to pq) \in R$ и всякого $A \in N$,
                    \item $S_S \to \varepsilon$.
                \end{itemize}
        \end{itemize}
        Покажем по индукции по размеру дерева $w$, что если $w \in L_{G'}(S_A)$, то есть некоторое разбиение $w = vu$, что $uAv$ порождается $S$ в $G$.

        Рассмотрим первую подстановку. Если это подстановка $S_S \to \varepsilon$, то действительно, $S$ порождается символом $S$ в $G$. Тогда это $S_A \to q S_B p$, где $(B \to pAq) \in R$. Далее по дереву $p$ реализует некоторое слово $u$, а $q$ --- $v$. Причём поддерево $S_B$ имеет меньший размер, чем размер дерева $S_A$, значит по предположению индукции $S_B$ реализует какое-то слово $v'u'$, что $u'Bv'$ реализуется символом $S$ в $G$. Тогда $w =  vv'u'u$, а применяя правило $B \to pAq$, получаем что $S$ в $G$ также реализует $u'pAqv'$. Прикрепляя к $p$ и $q$ поддеревья, порождающие $u$ и $v$, получаем, что $S$ в $G$ порождает ещё и $u'uAvv'$. Следовательно, $w = (vv')(u'u)$ --- искомое разбиение, а $(u'u)A(vv')$ действительно порождается.

        Теперь покажем по индукции по размеру дерева $uAv$, что для всякого выражения $uAv$ ($u, v \in \Sigma^*$), порождаемого из $S$ в $G$, $vu \in L_{G'}(S_A)$.

        Если $uAv = S$, то, действительно, $\varepsilon \in L_{G'}(S_S)$, так как есть правило $S_S \to \varepsilon$. Иначе дерево $uAv$ нетривиально (состоит из $>1$ вершин, а значит имеет хотя бы второй уровень). Тогда рассмотрим подстановку, которой получается $A$ в $uAv$: $(B \to pAq) \in R$. Тогда $p$ порождает некоторое слово $u$, $q$ --- $v$, а строка $uAv$ разбивается как $u'uAvv'$. Тогда $S$ в $G$ порождает $u'Bv'$. Следовательно, $v'u' \in L_{G'}(S_B)$. Также в $G'$ из $S_A$ можно получить $q S_B p$. Следовательно, $q v'u' p$ получается из $S_A$. Вставляя поддеревья для $p$ и $q$, получаем, что $vv'u'u \in L_{G'}(S_A)$.

        Теперь заметим, что если $w \in L_{G'}(S')$, то $w$ --- циклический сдвиг слова из $L_G(S)$. Действительно, первая подстановка --- $S' \to q S_A p$, где $(A \to pq) \in R$. Значит $p$ порождает какое-то слово $u$, а $q$ --- $v$, т.е. $uv$ порождается $A$ в $G$. При этом $S_A$ порождает какое-то слово $v'u'$, где $u'Av'$ порождается $S$ в $G$. Значит $u'uvv' \in L(S)$. При этом $w = vv'u'u$.

        А если $uv \in L_G(S)$, то $vu \in L_{G'}(S')$. Действительно, рассммотрим первый (нижний) нетерминал в дереве $uv$, который разрывается границей между $u$ и $v$, --- $A$. Тогда $A$ порождает слева от границы $u'$, а справа $v'$ и тогда $u = u'' u'$, $v = v' v''$. Причём пусть на место данной $A$ была подставлена строка $pq$, где $p$ реализует $u'$, а $q$ --- $v'$. Тогда $S'$ реализует $q S_A p$, а значит и $v' S_A u'$, а значит и $v' v'' u'' u' = vu$, так как $S_A$ раелизует $v'' u''$ (так как $S$ в $G$ реализует $u'' A v''$).

        Следовательно, $L(G') = L_{G'}(S')$ --- циклический сдвиг $L_G(S) = L(G)$.
    \end{problem}
\end{document}