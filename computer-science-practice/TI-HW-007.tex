\documentclass[12pt,a4paper]{article}
\usepackage{solutions}
\usepackage{multicol}
\usepackage{float}
\usepackage{todonotes}

\title{Домашнее задание от 12.10.\\Теоретическая информатика. 2 курс.\\Решения.}
\author{Глеб Минаев @ 204 (20.Б04-мкн)}
% \date{}

\newcommand{\swap}{\mathrm{swap}}

\begin{document}
    \maketitle

    \begin{multicols}{2}
        \tableofcontents
    \end{multicols}

    \begin{problem}{4}
        Пусть дана грамматика $(\Sigma, N, R, S)$, задающая язык $L$. Рассмотрим грамматику $(\Sigma, N', R', S')$, где:
        \begin{itemize}
            \item $N'$ состоит из символов $A$, $A_{a \to b}$ и $A_{a \leftrightarrow b}$ (копии $A$) для всяких $A \in N$ и $a, b \in \Sigma$, а также ещё одного символа $S'$.
            \item $R'$ состоит из следующих правил. Для всякого правила $A \to u_0 B_1 \dots B_n u_n$ ($u_i \in \Sigma^*$, $B_i \in N$) из $R$ мы добавим в $R'$ правила
                \begin{enumerate}
                    \item то же правило $A \to u_0 B_1 \dots B_n u_n$,
                    \item правило $A_{a \to b} \to u_0 B_1 \dots u_{i-1} B_i v B_{i+1} u_{i+1} \dots B_n u_n$, где $v$ --- это строка $u_i$ (для некоторого $i$) с некоторым $a$, подменённым на $b$, ($u_i$ должна содержать $a$; иначе правило не добавляется),
                    \item правило $A_{a \to b} \to u_0 B_1 \dots B_{i-1} u_{i-1} (B_i)_{a \to b} u_i B_{i+1} \dots B_n u_n$, где ($n$ должно быть $>0$; иначе правило не добавляется),
                    \item правило $A_{a \leftrightarrow b} \to u_0 B_1 \dots B_i v B_{i+1} \dots B_j w B_{j+1} \dots B_n u_n$, где $v$ --- это строка $u_i$ (для некоторого $i$) с некоторым $a$, подменённым на $b$, ($u_i$ должна содержать $a$; иначе правило не добавляется), а $w$ --- это строка $u_j$ (для некоторого $j$) с некоторым $b$, подменённым на $a$, ($u_j$ должна содержать $b$; иначе правило не добавляется) ($i$ и $j$ никак не зависят друг от друга, даже порядком; если $i = j$ подмены считаются проведёнными одновременно, т.е. в $u_i$ некоторая пара символов $a$ и $b$ поменялась местами),
                    \item правило $A_{a \leftrightarrow b} \to u_0 B_1 \dots B_i v B_{i+1} \dots u_{j-1} (B_j)_{b \to a} u_j \dots B_n u_n$, где $v$ --- это строка $u_i$ (для некоторого $i$) с некоторым $a$, подменённым на $b$, ($u_i$ должна содержать $a$; иначе правило не добавляется) ($n$ должно быть $>0$; иначе правило не добавляется) ($i$ и $j$ никак не зависят друг от друга),
                    \item правило $A_{a \leftrightarrow b} \to u_0 B_1 \dots B_i v B_{i+1} \dots u_{j-1} (B_j)_{a \to b} u_j \dots B_n u_n$, где $v$ --- это строка $u_i$ (для некоторого $i$) с некоторым $b$, подменённым на $a$, ($u_i$ должна содержать $a$; иначе правило не добавляется) ($n$ должно быть $>0$; иначе правило не добавляется) ($i$ и $j$ никак не зависят друг от друга),
                    \item правило $A_{a \leftrightarrow b} \to u_0 B_1 \dots u_{i-1} (B_i)_{a \to b} u_i \dots u_{j-1} (B_j)_{b \to a} u_j \dots B_n u_n$ ($n$ должно быть $>1$; иначе правило не добавляется) ($i \neq j$; других зависимостей между $i$ и $j$ нет),
                    \item правило $A_{a \leftrightarrow b} \to u_0 B_1 \dots u_{i-1} (B_i)_{a \leftrightarrow b} u_i \dots B_n u_n$ ($n$ должно быть $>0$; иначе правило не добавляется).
                \end{enumerate}
                Также добавим правила вида $S' \to S_{a \to b}$ для любых $a, b \in \Sigma$.
        \end{itemize}

        Несложно видеть, что для каждого $A \in N$ правила для $A$ в новой грамматике те же, что и раньше, а значит язык каждого $A$ не изменился при переходе от старой грамматике к новой.

        Покажем, что всякое $w \in L(A_{a \to b})$ это слово из $L(A)$, в котором некоторый $a$ заменили на $b$, по индукции по размеру дерева $w$.

        Рассмотрим первую подстановку в дереве разбора $w$.
        \begin{itemize}
            \item Если это замена 2, то можно заменить $v$ заменить обратно на $u_i$, а $A_{a \to b}$ на $B$ и получить дерево разбора некоторого $w' \in L(A)$. Тогда при возвращении обратно мы меняем в $w'$ некоторый символ $a$ на $b$. Что и требовалось показать.
            \item Если это замена 3, то поддерево $(B_i)_{a \to b}$ образует некоторое слово $u$, получаемое из $u' \in L(B)$ заменой $a$ на $b$. Тогда заменяя $A_{a \to b}$ на $A$, $(B_i)_{a \to b}$ на $B_i$ и конструируя поддерево $B_i$ так, чтобы оно порождало $u'$, мы получаем дерево разбора $w' \in L(A)$. Возвращая всё назад мы заменяем $u'$ на $u$, что равносильно замене некоторого $a$ в $u'$ на $b$. Т.е. $w$ получается из $w' \in L(A)$ заменой некоторого $a$ на $b$.
        \end{itemize}

        По аналогии можно сконструировать всякое такое слово $w$, заменяя либо какую-то правильную строку $u_i$ на $u_i$ с проделанной заменой, либо делегируя эту работу $B_i$, заменяя его на $(B_i)_{a \to b}$.

        Таким образом $L(A_{a \to b})$ --- это язык $L(A)$, где во всяком слове подменили $a$ на $b$ (если в слове нет $a$, то оно просто игнорируется). Аналогично, $L(A_{a \leftrightarrow b})$ --- это язык $L(A)$, где во всяком слове поменяли мествами какие-то $a$ и $b$ (если хотя бы одного из $a$ и $b$ в слове нет, слово игнорируется). Это видно из правил, где разбираются случаи, где могут происходить замены и как их обрабатывать. Следовательно, первым действием $S'$ выбирает, какую замену хочет произвести, и запускает её, т.е.
        \[\swap(L) = \bigcup_{\substack{a, b \in \Sigma\\a \neq b}} L(S_{a \leftrightarrow b}) = L(S).\]
    \end{problem}
\end{document}