\documentclass[12pt,a4paper]{article}
\usepackage{../.tex/mcs-notes}
\usepackage{todonotes}
\usepackage{multicol}
\usepackage{float}

\settitle
{Дифференциальные уравнения и динамические системы.}
{С.Ю.Пилюгин}
{\%D0\%9C\%D0\%B0\%D1\%82\%D0\%90\%D0\%BD/MA.pdf}
\date{}

% \DeclareMathOperator{\Quot}{Quot}
% \DeclareMathOperator*{\osc}{osc}
% \DeclareMathOperator{\sign}{sign}
\DeclareMathOperator{\const}{const}
% \DeclareMathOperator{\grad}{grad}
% \newcommand{\eqdef}{\mathbin{\stackrel{\mathrm{def}}{=}}}
% \newcommand{\True}{\mathrm{True}}
% \newcommand{\False}{\mathrm{False}}
% \newcommand{\Id}{\mathrm{Id}}
% \renewcommand{\Re}{\mathrm{Re}}
% \renewcommand{\Im}{\mathrm{Im}}

\begin{document}
    \maketitle

    \listoftodos[TODOs]

    \tableofcontents

    \vspace{2em}
    Литература:
    \begin{itemize}
        \item В.И. Арнольд, ``Обыкновенные дифференциальные уравнения''.
        \item Ю.Н. Бибиков, ``Общий курс обыкновенных дифференциальных уравнений''.
        \item С.Ю. Пилюгин, ``Пространства динамических систем'', 2008.
    \end{itemize}

    \begin{definition}
        \emph{Дифференциальное уравнение} --- уравнение вида
        \[f(x, y, y', \dots, y^{(m)}) = 0,\]
        где $x$ --- независимая переменная, $f$ --- данная функция, а $y(x)$ --- искомая функция.

        \emph{Обыкновенное дифференциальное уравнение} --- дифференциальное уравнение над $\RR$
    \end{definition}

    \begin{remark*}
        Бывают ещё дифференциальные уравнения над комплексными числами и дифференциальные уравнения в частных производных. Но это уже совершенно другие области; а мы будем рассматривать только обыкновенные дифференциальные уравнения.
    \end{remark*}

    \subsection{Дифференциальные уравнения 1-го порядка, разрешённые относительно производных}

    Пусть $x$ --- независимая переменная, $y(x)$ --- искомая функция. Тогда будем рассматривать уравнения вида
    \[y' = f(x, y).\]
    $f$ будет всегда рассматриваться непрерывной.

    Зафиксируем область (открытое связное множество) $G$ в $\RR^2_{x, y}$. Будем также писать $f \in C(G)$.

    \begin{definition}
        $y: (a; b) \to \RR$ называется решением данного уравнения на $(a; b)$, если
        \begin{itemize}
            \item если $y$ дифференцируема на $(a; b)$,
            \item для всякого $x \in (a; b)$ $(x, y(x)) \in G$,
            \item $y'(x) = f(x, y(x))$ на $(a; b)$.
        \end{itemize}
    \end{definition}

    \begin{example}
        При $k > 0$, $f(x, y) := ky$, $G = \RR^2$ имеем уравнение
        \[y = ky'.\]
        Тогда всем известно, что $y(x) = c e^{kx}$ для некоторого $c \in \RR$.
    \end{example}

    \begin{definition}
        \emph{Интегральная кривая} --- график решения.
    \end{definition}

    \begin{definition}[задача Коши]
        Пусть фиксирована $(x_0, y_0) \in G$. $y(x)$ --- \emph{решение задачи Коши с начальными данными $(x_0, y_0)$}, если
        \begin{itemize}
            \item $y(x)$ --- решение дифференциального уравнения на некотором интервале $(a; b) \ni x$,
            \item $y(x_0) = y_0$.
        \end{itemize}
    \end{definition}

    \begin{example}
        В случае того же уравнения
        \[y' = ky\]
        решением будет $y(x) = y_0 e^{k(x - x_0)}$.
    \end{example}

    \begin{definition}
        $(x_0; y_0)$ называется \emph{точкой единственности}, если для всяких решений $y_1$ и $y_2$ задачи Коши с входными данными $(x_0; y_0)$ есть некоторая окрестность $x_0$, где $y_1$ и $y_2$ совпадают.
    \end{definition}

    \begin{example}
        Возьмём уравнение
        \[y' = 3 y^{2/3}\]
        с входными данными $(0; 0)$. Понятно, что сюда подойдёт всякое решение вида $y(x) = cx^3$ ($c \in \RR$), что уже говорит о неединственности данной точки. Но есть случаи ещё хуже: можно склеить кусок слева одного решения и кусок справа другого и получить новое решение!
    \end{example}

    \begin{definition}[поле направлений]
        Зададим в области $G$ поле направлений: в каждой точке $(x_0; y_0)$ поставим направление соответствующее производной $f(x_0, y_0)$. Это равносильно векторному полю, где вектор в точке $(x_0; y_0)$ --- $(1; f(x_0; y_0))$. Следовательно график всякого решения $y(x)$ будет касаться поля направлений в области определения, а векторное поле будет градиентом графиком решения с нативной параметризацией по $x$.
    \end{definition}

    \begin{theorem}[существования для дифференциального уравнения 1-го порядка]
        Пусть имеется дифференциальное уравнение
        \[y' = f(x, y)\]
        и $f \in C(G)$. Тогда для всякой точки $(x_0; y_0) \in G$ существует решение задачи Коши с начальными данными $(x_0; y_0)$.
    \end{theorem}

    \begin{theorem}[единственности для дифференциального уравнения 1-го порядка]
        Пусть имеется дифференциальное уравнение
        \[y' = f(x, y)\]
        и $f, \frac{\partial f}{\partial y} \in C(G)$. Тогда всякая точка $(x_0; y_0) \in G$ является точкой единственности.
    \end{theorem}

    \subsection{Интегрируемые дифференциальные уравнения 1-го порядка}

    Первый случай. Наше уравнение имеет вид
    \[y' = f(x).\]
    В таком случае
    \[y(x) = y_0 + \int_{x_0}^x f(t) dt.\]

    \begin{definition}
        Пусть имеется уравнение
        \[y' = f(x, y),\]
        где $f \in C(G)$, а $H$ --- подобласть $G$. Функция $U \in C^1(H, \RR)$ (т.е. $U: H \to \RR$ и $U$ дифференцируема на $H$) называется \emph{интегралом} этого уравнения в $H$, если
        \begin{itemize}
            \item $\frac{\partial U}{\partial y} \neq 0$ в $H$,
            \item если $y: (a; b) \to \RR$ --- решение в $H$, то $U(x, y(x)) = \const$ на $(a; b)$.
        \end{itemize}
    \end{definition}

    \begin{theorem}[о неявной функции]
        Пусть дана $F \in C^1(H, \RR)$ и есть некоторая точка $(x_0; y_0) \in H$, что $F(x_0, y_0) = 0$, а $\frac{\partial F}{\partial y}(x_0, y_0) \neq 0$. Тогда есть некоторые окрестности $I$ и $J$ точек $x_0$ и $y_0$ и функция $z \in C^1(I)$, что $z(x_0) = y_0$ и для всякой точки $(x; y) \in I \times J$, что $F(x, y) = 0$, будет верно $y = z(x)$.
    \end{theorem}

    \begin{theorem}[об интеграле для дифференциального уравнения 1-го порядка]
        Пусть имеется интеграл $U$ уравнения $y' = f(x, y)$ в $H \subseteq G$. Тогда для всякой точки $(x_0; y_0) \in H$ будут открытые $I$ и $J$, что $I \times J \subseteq H$, $x_0 \in I$, $y_0 \in J$, и некоторое $y(x) \in C^1(I)$, что
        \begin{itemize}
            \item $y(x)$ --- решение задачи Коши с начальными данными $(x_0; y_0)$,
            \item для всякой точки $(x_1; y_1) \in H$, что $U(x_1; y_1) = U(x_0; y_0)$, верно $y_1 = y(x_1)$.
        \end{itemize}
    \end{theorem}

    \begin{proof}
        Рассмотрим
        \[F(x, y) := U(x, y) - U(x_0, y_0).\]
        Заметим, что $F(x_0, y_0) = 0$, а $\frac{\partial F}{\partial y}(x_0, y_0) = \frac{\partial U}{\partial y}(x_0, y_0) \neq 0$, т.е. $F$ удовлетворяет условию теоремы о неявной функции. Тогда по данной теореме существуют некоторые окрестности $I_0$ и $J_0$ точек $x_0$ и $y_0$ и функция $y(x) \in C^1(I)$.
        
        По теореме о существовании существует решение $z(x)$ задачи Коши с начальными данными $(x_0, y_0)$ на $I \ni x_0$, что $(x, z(x)) \in I \times J$. По определению интеграла $U$ имеем, что $U(x, z(x)) = U(x_0, y_0)$, а значит $F(x, z(x)) = 0$. Тогда по теореме о неявной функции $z(x) = y(x)$ на всей области определения $y$ и $z$. 
    \end{proof}

    \begin{remark}
        Равенство $U(x, y) = c$ называют общим интегралом.
    \end{remark}

    \subsubsection{Дифференицальные уравнения с разделяющимися переменными}

    Будем рассматривать уравнение вида
    \[y' = m(x) n(y),\]
    $m \in C((a; b))$, $n \in C((\alpha; \beta))$, $G = (a; b) \times (\alpha; \beta)$.

    Первый случай. Пусть $n(y_0) = 0$. Тогда есть решение $y(x) \equiv y_0$.

    Второй случай. Рассмотрим некоторый интервал $I \subseteq (\alpha; \beta)$, что для всякого $y \in I$ верно $n(y) \neq 0$. Рассмотрим $y(x)$, что $(x, y(x)) \in (a; b) \times I$. Несложным преобразованием получаем, что
    \[\frac{y'(x)}{n(y(x))} = m(x).\]
    значит
    \[
        \int_{x_0}^x m(s) ds
        = \int_{x_0}^x \frac{y'(t) dt}{n(y(t))}
        = \int_{x_0}^x \frac{dy(t)}{n(y(t))}
        = \int_{y(x_0)}^{y(x)} \frac{dz}{n(z)}.
    \]
    Обозначим первообразные
    \[N(y) := \int \frac{dy}{n(y)} \qquad \text{ и } \qquad M(x) := \int m(x) dx.\]
    Тогда мы имеем, что
    \[N(y(x)) - N(y(x_0)) = M(x) - M(x_0).\]
    Определим
    \[U(x, y) := N(y) - M(x).\]
    Тогда
    \[U(x, y(x)) = N(y(x)) - M(x) = N(y(x_0)) - M(x_0) = \const.\]
    Также
    \[\frac{\partial U}{\partial y} = N' = \frac{1}{n(y)} \neq 0.\]
    Таким образом $U$ --- интеграл данного уравнения в $(a; b) \times I$.
\end{document}