\documentclass[12pt,a4paper]{article}
\usepackage{solutions}

\title{Листочек 1. Сходящийся.\\Математический анализ. 1 курс.\\Решения.}
\author{Глеб Минаев @ 102 (20.Б02-мкн)}
% \date{}

\begin{document}
    \maketitle
    \section*{Базовые задачи}

    \begin{enumproblem}[\textcolor{green}{сдано}]
        \begin{lemma}\label{1_good_set}
            $\phi([\varepsilon;\varepsilon + 11^{-n}))=[0;1]$, где $\varepsilon =_{11} 0{,}d_1\dots d_{n-1}A$. 
        \end{lemma}
        
        \begin{proof}
            Расмотрим любое число $\gamma =_{10} 0{,}g_1g_2\dots \in [0;1]$. Среди данных цифр нет $A$, поэтому если $\alpha = 0{,}d_1\dots d_{n-1}Ag_1g_2\dots$, то $\phi(\alpha) = \gamma$. Но $\alpha \in [\varepsilon;\varepsilon + 11^{-n})$, значит $\phi([\varepsilon;\varepsilon + 11^{-n})) = [0;1]$.
        \end{proof}

        \begin{remark}
            Единица включена в отрезок значений, так как равна $0{,}(9)$.
        \end{remark}

        \begin{lemma}\label{1_gset_anywhere}
            Для любого интервала $I$ на отрезке $[0;1]$ найдутся $n$ и $\varepsilon = 0{,}d_1\dots d_{n-1}A$, что $I \supseteq [\varepsilon; \varepsilon + 11^{-n})$.
        \end{lemma}

        \begin{proof}
            Рассмотрим $\delta = 11^{-m}$ для некоторого $m$, что $2\delta < |I|$. Тогда по принципу кузнечика Кронекера на $I$ найдутся две точки кратные $\delta$, тогда полуинтервал с концами в них будет подходить на роль $[\varepsilon;\varepsilon + 11^{-n})$.
        \end{proof}

        Используя лемму \ref{1_gset_anywhere} мы находим в данном интервале $I$ полуинтервал
        \[[0{,}d_1\dots d_{n-1}A, 0{,}d_1\dots d_{n-1}A + 11^{-n}],\]
        образ которого по лемме \ref{1_good_set} равен $[0;1]$. Тогда $\phi(I) \supseteq [0;1]$, но и $\phi(I) \subseteq [0;1]$, значит $\phi(I) = [0;1]$. Таким образом $\forall y \in [0;1]$ найдётся $x \in I$, что $\phi(x) = y$.
    \end{enumproblem}

    \begin{enumproblem}[\textcolor{green}{сдано}]
        Рассмотрим последовательность $\{y_n\}_{n=0}^\infty := \{x_n - 3\}_{n=0}^\infty$. Так мы получаем рекуренту
        \[y_{n+1} = \frac{3}{y_n + 3} + \frac{1}{n}-1 = \frac{1}{n}-\frac{y_n}{y_n+3}\]

        Заметим, что $y_1 = -2$, $y_2 = 3$, $y_3 = 0$. Тогда несложно видеть по индукции, что $\forall n > 2 \quad |y_n| < 1$:
        \[|y_{n+1}| \leqslant \left|\frac{1}{n}\right| + \left|\frac{y_n}{y_n + 3}\right| < \frac{1}{2} + \frac{|y_n|}{2} < \frac{1}{2} + \frac{1}{2} = 1\]

        Из этого заметим, что
        \[|y_{n+1}| < \frac{1}{n} + \frac{|y_n|}{2}\]
        
        Заметим по индукции, что $|y_n| < \frac{4}{n}$ для всех $n > 2$. База: для $n=3$ очевидно. Шаг:
        \[|y_{n+1}| < \frac{1}{n} + \frac{|y_n|}{2} < \frac{3}{n} \leqslant \frac{3}{n} + \frac{n-3}{n(n+1)} = \frac{3}{n} + \frac{4n - 3(n+1)}{n(n+1)}=\frac{3}{n} + \frac{4}{n+1}-\frac{3}{n}=\frac{4}{n+1}\]

        Тем самым мы получаем сразу несколько вещей:
        \begin{itemize}
            \item последовательность $\{y_n\}_{n=0}^\infty$ сходится, а с ней и $\{x_n\}_{n=0}^\infty$;
            \item $\lim \{x_n\}_{n=0}^\infty = \lim \{y_n\}_{n=0}^\infty + 3 = 3$;
            \item $N(\varepsilon) = \max(3,\frac{4}{\varepsilon})$ (эта функция одинакова для обеих последовательностей).
        \end{itemize}
    \end{enumproblem}

    \begin{enumproblem}[\textcolor{green}{сдано}]\ 
        \begin{itemize}
            \item Пусть $f$ полунепрерывна снизу; требуется показать, что для любого $a$ прообраз $(a; +\infty)$ открыт. Значит треубется показать для любой точки $t \in f^{-1}((a;+\infty))$, что она внутренняя, т.е. у неё есть окрестность, отображающаяся в $(a; +\infty)$. Пусть $b := f(t)$, тогда по определению полунепрерывности снизу
            \[\forall \varepsilon > 0\; \exists \delta > 0: f(U_{\delta}(t))\subseteq(b-\varepsilon; +\infty).\]
            Но поскольку $b \in (a; +\infty)$, то $b > a$, значит для $\varepsilon = b - a$ тоже найдётся такое $\delta$. А это и значит, что некоторая окрестность отображается в $(a; +\infty)$.
            \item Пусть известно, что для любого $a$ прообраз $(a;+\infty)$ открыт; требуется показать, что $f$ полунепрерывна снизу. Значит требуется показать для любой точки $t$, что
            \[\forall \varepsilon > 0\; \exists \delta > 0: f(U_\delta(t))\subseteq(f(t)-\varepsilon;+\infty).\]
            Рассмотрим $S := f^{-1}((f(t)-\varepsilon;+\infty))$. поскольку оно открыто и в нём лежит $t$, то в $S$ содержится некоторая $\delta$-окрестность $t$ как подмножество. Это и означает, что $f(U_\delta(t))\subseteq(f(t)-\varepsilon;+\infty)$.
        \end{itemize}
    \end{enumproblem}

    \begin{enumproblem}[\textcolor{green}{сдано}]
        Обозначим нашу функцию, которая отображает каждый элемент последовательности в следующий, как $f(x)$. Также подним её до функции $F$ над $\RR P^1$: $F((x: y)) = (2x^3: (3x^2-y^2)y)$. На всякий случай проверим, что она корректна: первая однородная координата зануляется только если $x=0$, но тогда вторая равна $(3\cdot 0^2 - 1^2)\cdot 1= -1$, т.е. всё корректно.

        \begin{lemma}
            Пусть известно, что последовательность сходится, тогда она сходится к неподвижной точке $F$.
        \end{lemma}

        \begin{proof}
            Заметим, что поскольку $F$ полиномиальна, то непрерывна. Поэтому
            \[\lim \{x_n\}_{n=1}^\infty = \lim \{F(x_n)\}_{n=0}^\infty = F(\lim \{x_n\}_{n=0}^\infty) = F(\lim \{x_n\}_{n=1}^\infty),\]
            что означает, что предел последовательности --- непожвижная точка $F$.
        \end{proof}

        \begin{lemma}
            Неподвижные точки $F$ --- $(-1:1)$, $(0:1)$, $(1:1)$ и $(1:0)$.
        \end{lemma}

        \begin{proof}
            Для этого решим уравнение:
            \begin{align*}
                (x: y) &= (2x^3: (3x^2 - y^2)y)&
                xy(3x^2-y^2) &= 2x^3y\\
                0 &= y(3x^3 - xy^2 - 2x^3)&
                0 &= yx(x^2-y^2)\\
                0 &= y \cdot x \cdot (x - y) \cdot (x + y)
            \end{align*}
        \end{proof}

        Также несложно заметить, что прообразы $\infty$ --- сама $\infty$, а также $\pm \sqrt{\frac{1}{3}}$.

        \begin{lemma}
            $f(-x)=-f(x)$.
        \end{lemma}

        \begin{proof}
            $f(x) = \frac{2x^3}{3x^2 - 1} = x\frac{2}{3-1/x^2}$. В последнем выражении при замене $x \mapsto -x$ дробь не меняется, а $x$ меняет знак, поэтому Q.E.D. 
        \end{proof}

        \begin{corollary}
            Можно рассматривать только $[0;+\infty)$, чтобы проанализировать данную динаммическую систему.
        \end{corollary}

        \begin{lemma}
            $\forall x \in (\sqrt{\frac{1}{3}}; +\infty)$ верно, что $f(x) \geqslant 1$.
        \end{lemma}

        \begin{proof}
            $3x^2 - 1 > 0$, поэтому $f(x) \geqslant 1 \Leftrightarrow 2x^3 \geqslant 3x^2 - 1$. Заметим, что $2x^3 - 3x^2 + 1 = (x-1)(2x^2 - x - 1) = (x-1)^2(2x+1)$, что очевидно не меньше $0$ на $(-\frac{1}{2}; +\infty) \supseteq (\sqrt{\frac{1}{3}}; +\infty)$.
        \end{proof}

        \begin{lemma}
            $\forall x \in (1; +\infty)$ верно, что $f(x) < x$.
        \end{lemma}

        \begin{proof}
            $f(x) = \frac{2x^3}{3x^2 - 1} = x\frac{2}{3-1/x^2}$. При $x > 1$ имеем, что $x^2 > 1$, поэтому $1/x^2 \in (0; 1)$, поэтому $3 - 1/x^2 \in (2; 3)$, поэтому $\frac{2}{3-1/x^2} \in (0;1)$. Поэтому $f(x) < x$.
        \end{proof}

        Из последних двух лемм мы получаем, что $f((\sqrt{\frac{1}{3}};1)) = (1;+\infty)$. А если какой-то член последовательности попал в $(1;+\infty)$, то после этого последовательность будет уменьшаться, не принижая $1$, значит будет сходиться; но сходиться не к чему кроме как к $1$.

        \begin{lemma}
            $\forall x \in (-\sqrt{\frac{1}{3}}; \sqrt{\frac{1}{3}})$ верно, что $|f(x)| > |x| \Leftrightarrow |x| > \sqrt{\frac{1}{5}}$.
        \end{lemma}

        \begin{proof}
            Заметим, что $|f(x)| = |-x\frac{2}{3-1/x^2}| = |x| \frac{2}{1/x^2-3}$, так как $1/x^2 > 3$. Но
            \begin{align*}
                \frac{2}{1/x^2 - 3} &> 1&
                2 &> 1/x^2 - 3&
                5 &> 1/x^2&
                x^2 &> \frac{1}{5}\\
                |x| &> \sqrt{\frac{1}{5}}&
            \end{align*}
        \end{proof}

        \begin{remark}
            Аналогичным рассуждением показывается, что $\sqrt{\frac{1}{5}}$ и $-\sqrt{\frac{1}{5}}$ переходят друг в друга, а всё, что меж ними, уменьшается по модулю.
        \end{remark}

        \begin{corollary}
            Если какой-то член последовательности находится на $(\frac{1}{5}; \frac{1}{3})$ (или же ему аналогично $(-\frac{1}{3}; -\frac{1}{5})$), то модуль последовательности будет увеличиваться. Остаться последовательность в $(-\sqrt{\frac{1}{3}};\sqrt{\frac{1}{3}})$ не может, так как иначе сойдётся к неправильному значению, поэтому рано или поздно вылетит вне, а значит либо перейдёт в $\infty$, либо сойдётся к $\pm 1$.
        \end{corollary}

        \begin{corollary}
            Если же какой-то член последовательности находится на $(-\sqrt{\frac{1}{5}}; \sqrt{\frac{1}{5}})$, то вся последовательность далее начинает уменьшаться по модулю, сходясь к $0$. 
        \end{corollary}

        Таким образом, чтобы попасть в $0$ нужно, чтобы $x_0 \in (-\sqrt{\frac{1}{5}}; \sqrt{\frac{1}{5}})$. А все досягаемые пределы --- это $-1$, $0$, $-1$ и $\infty$.
    \end{enumproblem}

    \begin{enumproblem}
        TBP
    \end{enumproblem}

    \begin{enumproblem}[\textcolor{green}{сдано}]
        \begin{lemma}
            Множество $L$ предельных точек $\{x_n\}_{n=0}^\infty$ \emph{связно}, т.е. $\forall a, b \in L \quad [a; b] \subseteq L$.
        \end{lemma}

        \begin{proof}
            Пусть $a$, $b$ --- предельные точки последовательности. Пусть также $c\in (a;b)$. Заметим, что так как разностная последовательность последовательности $\{x_n\}_{n=0}^\infty$ сходится, то $\forall \varepsilon > 0$ существует такой момент, когда последовательность начинает шагать с шагом меньшим $\varepsilon$. Также поскольку $a$ и $b$ предельные точки, то после любого момента эта последовательность заскочит в любую окрестность $a$ и любую окрестность $b$. Значит после любого момента будут существовать два других, где в первом она находилась в малой (посравнению с $|a-c|$ и $|b-c|$) окрестности $a$ (и была с одной сторона от $c$), а во втором --- в малой окрестности $b$ (а т.е. с другой стороны $c$). Значит между этими моментами последовательность ``перебегала от $a$ к $b$'', и был момент, когда она попала в $\varepsilon/2$-окрестность $c$ (т.к. $\varepsilon$ --- её шаг). Поскольку единственное ограничение на $\varepsilon$ --- быть $>0$, то это значит, что последовательность побывала во всех окрестностях $c$ бесконечно много раз. Значит $c$ --- предельная точка. Поскольку на $c$ не ставилось условий кроме ``$c\in (a;b)$'', то получаем, что $(a;b) \subseteq S$, или же $[a;b]\subseteq S$.
        \end{proof}

        \begin{lemma}
            Множество предельных точек любой последовательности замкнуто.
        \end{lemma}

        \begin{proof}
            Пусть $\{y_n\}_{n=0}^\infty$ --- данная последовательность, $L$ --- множество её предельных точек, а $m$ --- предельная точка $L$. Тогда имеем, что $\forall \varepsilon > 0\quad U_{\varepsilon/2}(m) \cap L \neq \varnothing$, т.е. $\exists l \in L \cap U_{\varepsilon/2}(m)$. При этом по определению $L$ последовательность посетит бесконечно много раз $\varepsilon/2$-окрестность $l$, а значит и $\varepsilon$-окрестность $m$. Таким образом последовательность посетит любую окрестность $m$ бесконечно много раз, значит $m$ --- тоже предельная.

            Так получаем, что $L$ содержит все свои предельные точки, значит $L$ замкнуто.
        \end{proof}

        Используя леммы выше получаем, что множество $L$ предельных точек связно и замкнуто; осталось доказать, что такими свойствами обладают только отрезок, луч или прямая.

        Заметим, что $L$ содержит свои инфимум и супремум, если они определены (каждый по отдельности). Тогда если они оба имеются, то $L$ --- отрезок; если $L$ имеет только один из них, то это закрытый луч, так как он содержит все точки между своим инфимумом (супремумом) и сколь угодно большими (маленькими) точками; если же он не имеет ни один из них, то это прямая, поскольку содержит все точки между сколь угодно большими и сколь угодно маленькими точками.
    \end{enumproblem}

    \begin{enumproblem}[\textcolor{green}{сдано}]
        \begin{lemma}\label{open_connected_set_classification_lemma}
            Связное открытое множество есть интервал, открытый луч или прямая.
        \end{lemma}

        \begin{proof}
            Пусть дано множество $S$. Заметим, что для $a$ и $b$ --- любых двух точек $S \cup S'$  --- интервал $(a;b)$ является подмножеством $S$, так как $\forall \varepsilon > 0$ найдутся $a_\varepsilon$ и $b_\varepsilon$ из $S$, лежащие в $\varepsilon$-окрестностях $a$ и $b$ соответственно. Поскольку тогда $(a_\varepsilon;b_\varepsilon) \subseteq S$, то $(a;b) \subseteq \bigcup_{\varepsilon > 0} (a_\varepsilon;b_\varepsilon) \subseteq \bigcup_{\varepsilon > 0} S = S$.

            Вспоминим, что инфимум и супремум $S$ являются его предельными точками, но не лежат в самом $S$. Тогда если у множества есть инфимум и супремум, то он является интервалом между ними, а если хотя бы один из них становится ``бесконечностью'', то соответствующий конец интервала становится бесконечностью (так получаются открытые лучи и вся прямая).
        \end{proof}

        \begin{lemma}\label{cover_countability_part}
            Пусть есть семейство $\Sigma$ попарно непересекающихся интервалов, длина которых хотя бы $L > 0$. Тогда $\Sigma$ не более, чем счётно.
        \end{lemma}

        \begin{proof}
            Рассмотрим множество $U$ верхних концов. Несложно понять, что расстояние между любыми двумя элементами $U$ хотя бы $L$. Тогда если разбить $\RR$ на полуинтервалы длины $L$, то на каждом из них будет не более одной точки из $U$, значит $|U| \leqslant |\NN|$. Но поскольку $|\Sigma| = |U|$, то $\Sigma$ не более, чем счётно.
        \end{proof}

        \begin{lemma}\label{cover_countability_lemma}
            Пусть есть семейство $\Sigma$ попарно непересекающихся интервалов. Тогда $\Sigma$ не более, чем счётно.
        \end{lemma}

        \begin{proof}
            Предстваим $\Sigma$ как дизъюнктное объединение $\{\Sigma_n\}_{n \in \ZZ}$, где $\Sigma_n$ --- множество интервалов $\Sigma$, длина которых лежит на $[2^n;2^{n+1})$ (несложно проверить, что это и вправду дизъюнктное объединение). Тогда по лемме \ref{cover_countability_part} каждое $\Sigma_n$ не более чем счётно, значит $\Sigma$ является объединением счётного числа не более чем счётных множеств, а значит само не более чем счётно.
        \end{proof}

        Рассмотрим отношение $\sim$ на $X$, где $a\sim b \Leftrightarrow [a;b] \subseteq X$. Несложно заметить, что $\sim$ --- отношение эквивалентности. Тогда можно рассмотреть $\Sigma := X/\sim$.
        
        Заметим, что любой элемент $\Sigma$ --- открытое (т.к. у каждой точки есть окрестность, с которой она лежит в $X$ и которая связна, поэтому лежит в одном класссе эквивалентности) связное (по определению класса эквивалентности) множество, а значит по лемме \ref{open_connected_set_classification_lemma} это интервал, открытый луч или прямая.

        Также по лемме \ref{cover_countability_lemma} $\Sigma$ не более чем счётно. Таким образом $X$ получилось представимым в виде дизъюнктного объединения не более чем счётного числа интервалов, открытых лучей и прямых.

        P.S. Формально задача для $X \neq \RR$, а прямых быть не должно, но это две равносильные задачи.
    \end{enumproblem}

    \section*{Рейтинговые задачи}

    \begin{enumproblem}[\textcolor{green}{сдано}]\ 
        \ItemedProblem
        \begin{enumerate}
            \item Заметим, что $f$ неопределена на открытом множестве, которое представимо в виде дизъюнктного объединения интервалов. Тогда на каждом таком интервале определим функцию очень просто: как линейную функцию, интерполированную по значениям в концах интервала.
            
            Пусть точка $x \in M$. Посмотрим на пределы с каждой из сторон (с правой и с левой); WLOG с правой. Если она является границей интервала с правой стороны, то существует некоторая константа $k$, что $\forall \varepsilon > 0: f((x; x + k\varepsilon)) \subseteq U_\varepsilon(f(x))$ (эта константа есть модуль обратного значения к коэффициенту наклона отрезка на данном интервале). В ином же случае есть ``сколь угодно близкие интервалы'' (всё ещё справа). Тогда $\delta(\varepsilon)$ можно строить следующим образом. Поскольку есть некоторая (правая) окрестность, в которой значения во всех точках $M$ лежат в $\varepsilon$-окрестности $f(x)$, то можно взять за границу новой (правой) окрестности любой элемент $M$ из этой окрестности; в таком случае все линейные промежутки будут из-за простого неравенства лежать в $U_\varepsilon(f(x))$, так как их концы будут. Так мы получаем с обеих сторон по ограничению, и беря минимум, мы получаем искомую функцию $\delta(\varepsilon)$.

            Теперь осталось доказать непрерывность для $x \notin M$. Но это очевидно, поскольку тогда $x$ вместе со своей некоторой окрестностью лежит в $\overline{M}$, а значит на некотором (одном!) интервале, образующем $\overline{M}$, а там функция просто линейна. Ну а линейные функции, очевидно непрерывны.
            
            \item Для начала поймём, что липшицевые функции непрерывны. Поэтому доопределим функцию $f$ на $M'$ просто предельными значениями в этих точках. Получим $f_1$.
            
            \begin{lemma}
                Пусть дана $n \in M' \setminus M$. Тогда $f_1$ липшицева с константой $L$ на $M\cup\{n\}$.
            \end{lemma}

            \begin{proof}
                Есть $\{m_i\}_{i=0}^\infty \in \NN^M$, предел которой есть $n$. Предположим противное: есть $b \in M$, что $|f_1(b) - f_1(n)| > L |b-n|$. Но поскольку $\{|f_1(b)-f_1(m_i)|\}_{i=0}^\infty \leqslant L\{|b-m_i|\}_{i=0}^\infty$, то $|f_1(b) - f_1(n)| = \lim \{|f_1(b)-f_1(m_i)|\}_{i=0}^\infty \leqslant L\lim \{|b-m_i|\}_{i=0}^\infty = L |b-n|$ --- противоречие.
            \end{proof}

            \begin{lemma}
                $f_1$ липшицева с константой $L$ на $M \cup M'$.
            \end{lemma}

            \begin{proof}
                Предположим противное. Тогда противное могло вызваться между точками $n_1, n_2 \in M \setminus M'$. Но заметим, что есть $\{m_i\}_{i=0}^\infty \in \NN^M$, сходящаяся к $n_2$, тогда $|f_1(n_1) - f_1(n_2)| = \lim \{|f_1(n_1)-f_1(m_i)|\}_{i=0}^\infty \leqslant L\lim \{|n_1-m_i|\}_{i=0}^\infty = L |n_1-n_2|$ --- противоречие.
            \end{proof}

            Теперь дополним $f_1$ как в пункте (а): на каждом интервале дополняющего множества просто определим линейную функцию. Получим $f_2$.

            \begin{lemma}
                $f_2$ липшицева всё с той же константой на $\RR$.
            \end{lemma}

            \begin{proof}
                Предположим противное, т.е. всё-таки возникает конфликт между какими-то точками $p_1$ и $p_2$. Пусть $S := M \cup M'$.

                Поймём, что оба $p_1$ и $p_2$ не могут лежать в $S$, так как это противоречит предыдущей аксиоме. WLOG $p_1$ лежит на некотором интервале $(a;b)$ из дизъюнктного разложения $\overline{S}$. Пусть $p_2$ лежит не на том же интервале или лежит в $S$. Тогда
                \begin{align*}
                    |f_2(p_1) - f_2(p_2)|
                    &= \left|\frac{f_2(a)(p_1-b) + f_2(b)(a-p_1)}{a-b} - f_2(p_2)\right|\\
                    &= \left|\frac{(f_2(a) - f_2(p_2))(p_1-b) + (f_2(b) - f_2(p_2))(a-p_1)}{a-b}\right|\\
                    &\leqslant \frac{|f_2(a) - f_2(p_2)|(p_1-b) + |f_2(b) - f_2(p_2)|(a-p_1)}{a-b}\\
                    &\leqslant \frac{L|a-p_2|(p_1-b) + L|b-p_2|(a-p_1)}{a-b}\\
                    &= L\frac{|a-p_2|(p_1-b) + |b-p_2|(a-p_1)}{a-b}\\
                \end{align*}
                Поскольку $p_2$ не лежит на $(a;b)$, то несложно видеть, что
                \[\frac{|a-p_2|(p_1-b) + |b-p_2|(a-p_1)}{a-b} = |p_1-p_2|\]
                А тогда $|f_2(p_1) - f_2(p_2)| \leqslant L |p_1-p_2|$ --- противоречие.

                Пусть же $p_1$ и $p_2$ лежат на одном интервале $(a;b)$, тогда $|f_2(p_1) - f_2(p_2)| \leqslant L |p_1-p_2|$ очевидно следует из того, что $\frac{|f_2(p_1) - f_2(p_2)|}{|p_1-p_2|} = \frac{|f_2(a) - f_2(b)|}{|a - b|} \leqslant L$.
            \end{proof}

            Тем самым задача решена.

            \item 
        \end{enumerate}
    \end{enumproblem}

    \begin{enumproblem}
        TBP
    \end{enumproblem}

    \begin{enumproblem}
        TBP
    \end{enumproblem}

    \begin{enumproblem}
        TBP
    \end{enumproblem}

    \begin{problem}{12}
        Предположим противное. Тогда существует последовательность $\{(x_i, y_i)\}_{i=0}^\infty$, сходящаяся к $(0,0)$, что значение
        \[\frac{\ln(P(x_i, y_i))}{\ln(x_i^2+y_i^2)}\]
        не органичено сверху. Тогда рассмотрим подпоследовательность, для которой это значение монотонно сходится к $+\infty$.

        Затем выделим подпоследовательность, что направление (по модулю $2\pi$, а не $\pi$) вектора $(x_i, y_i)$ сходится; затем уберём лишние члены, чтобы сходимость была монотонной и с одной стороны от этого направления. Сделаем поворот координат, чтобы направление сходилось к направлению $(1, 0)$, а дальше сделаем при необходимости замену $(x_i, y_i) \mapsto (x_i, -y_i)$, чтобы $y_i \geqslant 0$. Таким же образом затребуем монотонную сходимость по $x$. Таким образом мы получили последовательность $\{(x_i, y_i)\}_{i=0}^\infty$ и многочлен $P$, что сама последовательность сходится к нулю, $x_i > 0$ и монотонно сходится к $0$, $y_i \geqslant 0$, значение $y_i/x_i$ монотонно сходится к $0$, а значение
        \[\frac{\ln(P(x_i, y_i))}{\ln(x_i^2+y_i^2)}\]
        монотонно сходится к $+\infty$.

        В таком случае сразу заметим, что
        \[\ln(x_i^2+y_i^2) = \ln(x_i^2) + \ln\left(1 + \frac{y_i^2}{x_i^2}\right) = 2\ln(x_i) + \ln\left(1 + \frac{y_i^2}{x_i^2}\right)\]
        При этом $y_i^2/x_i^2 \to 0$, и тем самым полуачается, что $2\ln(x) \to -\infty$, а $\ln\left(1+\frac{y_i^2}{x_i^2}\right) \to 0$, поэтому можно рассматривать не $\ln(P)/\ln(x^2+y^2)$, а $\ln(P)/\ln(x)$.

        Если $\{y_i\}_{i=0}^\infty$ в какой-то момент занулилась, то после этого она тождественно равна нулю. А тогда мы имеем, что $P$ вырождается в полином от одной переменной, а тогда этот полином от одной переменной в нуле уменьшается быстрее любого монома, что очевидно неверно (хватит просто взять самый младший ненулевой член: все остальные мономы будут $o$-малыми от него, а значит в достаточно малой окрестности нуля, многочлен будет больше чем половина своего младшего ненулевого члена). Таким образом $y_i > 0$.

        Расмотрим $c_i := \frac{\ln(y_i)}{\ln(x_i)}$. Мы и так знаем, что $\ln(x_i) \to -\infty$ и $\ln(y_i) - \ln(x_i) \to -\infty$. Поэтому с некоторого момента $0 > \ln(x_i) > \ln(y_i)$, а значит $c_i \geqslant 1$. Если $c_i$ неограничено сверху, то можно выделить подпоследовательность, что $c_i$ будет монотонно сходиться к $+\infty$, а тогда мы имеем, что для любого многочлена $Q$, не равного тождественно нулю, верно, что $|Q(x)|/y \to +\infty$. Это значит, что если мы распишем $P(x, y)$ как $P_k(x)y^k + \dots + P_n(x)y^n$, то $P_0(x)y^0$ (а он ненулевой, так как иначе $P$ занулялся бы на всей прямой $y=0$) на фоне остальных членов суммы будет бесконечно великим, поэтому $P(x, y) \approx P_0(x)y^0 \succ x^{\deg(P_0)+1}$ --- противоречие.

        Тогда $c_i$ ограничено, а значит есть подпоследовательность, что $c_i$ сходится к некоторому $c$. В таком случае $y = x^c \tau$, а значит $\frac{\ln(\tau)}{\ln(x)}\to 0$. Отсюда следует, что любой полином от $\tau$, разделённый на $x$, стремится к бесконечности. В таком случае $P(x, y) = P_1(\tau)x^{d_1} + \dots + P_m(\tau)x^{d_m}$, где $0 < d_1 < \dots < d_m$ ($d_i$ не обязательно целое, так как равно $p + cq$ для $p, q \in \NN \cup \{0\}$), а тогда $P(x, y) \approx P_1(\tau)x^{d_1} \succ x^{d_1+\varepsilon}$ для любого $\varepsilon > 0$ --- противоречие.

        Если $\tau$ не имеет предельной точкой корень $P_1$ (который ненулевой), то $P_1(\tau)$ имеет оценку снизу в виде константы. Иначе выделим подпоследовательность, что $\tau$ сходится к константе $t$. Тогда сделаем замену $\tau = t + \sigma$. Заметим, что тогда $P_1(\tau) \approx \sigma^k$ для некоторого $k \in \NN\setminus\{0\}$. А значит $P \approx x^{d_1} \sigma^k$. Но мы также получаем, что $x^{d_1} \sigma^k$ асимптотически меньше любого мономы от $x$, значит то же верно для $\sigma^k$, а значит и для $\sigma$.

        Тогда представим $P$ как многочлен от $x$ и $\sigma$, но только степени $x$ могут быть не только целыми. Заметим, что если выделить многочлен при $\sigma^0$, то он тождественно равен $0$, так как иначе мы получаем мономиальную от $x$ оценку снизу на $P$. Тогда можем рассмотреть последовательность точек отличающейся от данной лишь тем, что значения $x$ те же, а $\sigma$ --- $0$. Тогда значение $P$ равно $0$ в этой последовательности, чего быть не может.
    \end{problem}

    \begin{enumproblem}
        TBP
    \end{enumproblem}
\end{document}