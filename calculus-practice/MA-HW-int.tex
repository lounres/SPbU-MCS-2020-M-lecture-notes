\documentclass[12pt,a4paper]{article}
\usepackage{solutions-en}
\usepackage{float}
\usepackage{multicol}

\title{Home integrals.\\Calculus.\\Solutions.}
\author{Gleb Minaev @ 102 (20.Б02-мкн)}
\date{}

\begin{document}
    \maketitle

    \begin{problem}{4039}
        The given domain $\Omega := \{(x; y) \mid x \geqslant 0 \wedge y \geqslant 0 \wedge x + y = 1\}$. The surface is defined as $z^2 = 2xy$, but it means that upper part ($z = \sqrt{2xy}$) and lower part ($z = -\sqrt{2xy}$) of it are symmetrical, hence the area of upper part $z := \sqrt{2xy}$ is a half of the total area. Then the area of the surface is
        \[
            A = 2\int_{\Omega} \sqrt{1 + \left(\frac{\partial z}{\partial x}\right)^2 + \left(\frac{\partial z}{\partial y}\right)^2} dy dx.
        \]
        But $\frac{\partial z}{\partial x} = \frac{\partial \sqrt{2xy}}{\partial x} = \frac{\sqrt{2y}}{2\sqrt{x}} = \sqrt{\frac{y}{2x}}$; similarly $\frac{\partial z}{\partial y} = \sqrt{\frac{x}{2y}}$. Then
        \[
            A = 2\int_\Omega \sqrt{1 + \frac{x}{2y} + \frac{y}{2x}} dx dy.
        \]
        Let $u := \sqrt{x/y}$ and $v := x + y$. Then $v = \frac{x}{y} y + y = y (u^2 + 1)$, so $y = \frac{v}{u^2 + 1}$ and $x = \frac{v u^2}{u^2 + 1}$. Also $v$ goes from $0$ to $1$ and $u$ goes from $0$ to $+\infty$, Jacobian of the substitution is
        \[
            \begin{vmatrix}
                \frac{\partial x}{\partial u}& \frac{\partial x}{\partial v}\\
                \frac{\partial y}{\partial u}& \frac{\partial y}{\partial v}\\
            \end{vmatrix}
            =
            \begin{vmatrix}
                \frac{2vu}{(u^2 + 1)^2}& \frac{u^2}{(u^2 + 1)}\\
                \frac{-2vu}{(u^2 + 1)^2}& \frac{1}{(u^2 + 1)}\\
            \end{vmatrix}
            = \frac{2vu + 2vu^3}{(u^2+1)^3}
            = \frac{2vu}{(u^2+1)^2},
        \]
        and $(x; y) \in \Omega$ iff $u \in [0; +\infty] \wedge v \in [0; 1]$. Hence
        \begin{align*}
            A
            &= 2\int_0^1 \int_0^{+\infty} \sqrt{1 + \frac{u^2}{2} + \frac{1}{2u^2}} \frac{2vu}{(u^2+1)^2} du dv\\
            &= \int_0^1 2v dv \int_0^{+\infty} \sqrt{1 + \frac{u^2}{2} + \frac{1}{2u^2}} \frac{2u}{(u^2+1)^2} du\\
            &= \left.v^2\right|_0^1 \int_0^{+\infty} \sqrt{\frac{u^4 + 2u^2 + 1}{2u^2}} \frac{2u}{(u^2+1)^2} du\\
            &= \int_0^{+\infty} \frac{u^2 + 1}{\sqrt{2}u} \frac{2u}{(u^2+1)^2} du\\
            &= \int_0^{+\infty} \frac{\sqrt{2}du}{u^2+1} du\\
            &= \left.\sqrt{2}\tan^{-1}(u)\right|_0^{+\infty}\\
            &= \frac{\pi}{\sqrt{2}}.
        \end{align*}
    \end{problem}

    \begin{problem}{4044}
        Let $p = x/a$, $q=y/a$, $r=z/a$. Then we are looking for area of surface $r = (p^2 + q^2)/2$ inside cylinder $(p^2 + q^2)^2 = 2 pq$.
        
        Let $p = l \cos(\varphi)$, $q = l \sin(\varphi)$. Then $p^2 + q^2 = l^2$, $2pq = l^2 \sin(2\varphi)$. Hence the curve $(p^2 + q^2)^2 = 2 pq$ is equivalent to curve $l^4 = l^2 \sin(2\varphi) \Leftrightarrow l^2 = \sin(2\varphi)$. Hence we are looking for area of the surface in region $l^2 \leqslant \sin(2\varphi)$ (region $l^2 \geqslant \sin(2\varphi)$ is not bounded, because contatin region $l\geqslant 1$).

        Jacobian of substitution $(p; q) \mapsto (x; y)$ is obviously $a^2$, and Jacobian of $(r; \varphi) \mapsto (p; q)$ is
        \[
            \begin{vmatrix}
                \frac{\partial p}{\partial l}& \frac{\partial p}{\partial \varphi}\\
                \frac{\partial q}{\partial l}& \frac{\partial q}{\partial \varphi}\\
            \end{vmatrix}
            =
            \begin{vmatrix}
                \cos(\varphi)& -l\sin(\varphi)\\
                \sin(\varphi)& l\cos(\varphi)\\
            \end{vmatrix}
            = l.
        \]
        So the area is
        \begin{align*}
            A
            &= \int_\Omega \sqrt{1 + \left(\frac{\partial z}{\partial x}\right)^2 + \left(\frac{\partial z}{\partial y}\right)^2} dx dy\\
            &= \int_\Omega \sqrt{1 + \left(\frac{\partial (ar)}{\partial (ap)}\right)^2 + \left(\frac{\partial (ar)}{\partial (aq)}\right)^2} a^2 dp dq\\
            &= a^2 \int_\Omega \sqrt{1 + \left(\frac{\partial r}{\partial p}\right)^2 + \left(\frac{\partial r}{\partial q}\right)^2} dp dq\\
            &= a^2 \int_\Omega \sqrt{1 + p^2 + q^2} dp dq\\
            &= a^2 \int_\Omega \sqrt{1 + l^2} l dl d\varphi\\
            &= 2a^2 \int_0^{\pi/2} \int_0^{\sqrt{\sin(2\varphi)}} \sqrt{1 + l^2} l dl d\varphi\\
            &= 2a^2 \int_0^{\pi/2} \left. \frac{1}{3} (1 + l^2)^{3/2} \right|_0^{\sqrt{\sin(2\varphi)}} d\varphi\\
            &= 2a^2 \int_0^{\pi/2} \frac{1}{3} ((1 + \sin(2\varphi))^{3/2}-1) d\varphi\\
            &= 2a^2 \int_0^{\pi/2} \frac{1}{3} ((2\sin(\varphi + \pi/4)^2)^{3/2}-1) d\varphi\\
            &= 2a^2 \int_0^{\pi/2} \frac{1}{3} (2\sqrt{2}\sin(\varphi + \pi/4)^3-1) d\varphi\\
            &= 2a^2 \int_0^{\pi/2} \frac{1}{3} (\sqrt{2}(2\sin(\varphi + \pi/4)^2)\sin(\varphi + \pi/4)-1) d\varphi\\
            &= 2a^2 \int_0^{\pi/2} \frac{1}{3} (\sqrt{2}(\cos(0) - \cos(2\varphi + \pi/2))\sin(\varphi + \pi/4)-1) d\varphi\\
            &= 2a^2 \int_0^{\pi/2} \frac{1}{3} (\sqrt{2}\sin(\varphi + \pi/4) - \sqrt{2}\cos(2\varphi + \pi/2)\sin(\varphi + \pi/4)-1) d\varphi\\
            &= 2a^2 \int_0^{\pi/2} \frac{1}{3} (\sqrt{2}\sin(\varphi + \pi/4) - \frac{1}{\sqrt{2}}(\sin(3\varphi + 3\pi/4) + \sin(-\varphi-\pi/4))-1) d\varphi\\
            &= 2a^2 \int_0^{\pi/2} \frac{1}{3} ((\sqrt{2} + \frac{1}{\sqrt{2}})\sin(\varphi + \pi/4) - \frac{1}{\sqrt{2}}\sin(3\varphi + 3\pi/4)-1) d\varphi\\
            &= 2a^2 \int_0^{\pi/2} \frac{1}{3\sqrt{2}} (3\sin(\varphi + \pi/4) - \sin(3\varphi + 3\pi/4)-\sqrt{2}) d\varphi
        \end{align*}
        \begin{align*}
            &= 2a^2 \left.\frac{1}{3\sqrt{2}} (-3\cos(\varphi + \pi/4) + \frac{1}{3}\cos(3\varphi + 3\pi/4)-\sqrt{2}\varphi) \right|_0^{\pi/2} \\
            &= 2a^2 \frac{1}{3\sqrt{2}} \Bigl((-3\cos(3\pi/4) + \frac{1}{3}\cos(9\pi/4)-\pi/\sqrt{2})\Bigr.\\
            &\qquad -\Bigl. (-3\cos(\pi/4) + \frac{1}{3}\cos(3\pi/4))\Bigr)\\
            &= 2a^2 \frac{1}{3\sqrt{2}} \Bigl(-3\frac{-1}{\sqrt{2}} + \frac{1}{3}\frac{1}{\sqrt{2}} - \pi/\sqrt{2} + 3\frac{1}{\sqrt{2}} - \frac{1}{3} \frac{-1}{\sqrt{2}}\Bigr)\\
            &= 2a^2 \frac{1}{3\sqrt{2}} \Bigl(\frac{3}{\sqrt{2}} + \frac{1}{3\sqrt{2}} + \frac{3}{\sqrt{2}} + \frac{1}{3\sqrt{2}} - \pi/\sqrt{2}\Bigr)\\
            &= 2a^2 \frac{1}{3\sqrt{2}} \Bigl(\sqrt{2} \frac{10}{3} - \pi/\sqrt{2}\Bigr)\\
            &= 2a^2 \frac{1}{3} \Bigl(\frac{10}{3} - \pi/2\Bigr)\\
            &= 2a^2 \Bigl(\frac{10}{9} - \pi/6\Bigr)\\
            &= a^2 \Bigl(\frac{20}{9} - \pi/3\Bigr)
        \end{align*}
    \end{problem}

    \begin{problem}{4017}
        Similarly to previous problem we are looking for volume of solid figure
        \[\Omega := \{(x; y; z) \mid x^2 + y^2 \geqslant az \wedge z \geqslant 0 \wedge (x^2 + y^2)^2 \leqslant a^2 (x^2 - y^2)\}.\]
        Let $x = ar \cos(\varphi)$, $y = ar \sin(\varphi)$, $z = aq$. Then $(x; y; z) \in \Omega$ iff
        \[
            (r; \varphi; q) \in \Gamma := \{(r; \varphi; q) \mid r^2 \geqslant q \geqslant 0 \wedge r^2 \leqslant \cos(2\varphi) \wedge 0 \leqslant r \wedge \varphi \in [0; 2\pi]\}.
        \]
        Also Jacobian of the substitution $(r; \varphi; q) \to (x; y; z)$ is
        \[
            \begin{vmatrix}
                \frac{\partial x}{\partial r}& \frac{\partial x}{\partial \varphi}& \frac{\partial x}{\partial q}\\
                \frac{\partial y}{\partial r}& \frac{\partial y}{\partial \varphi}& \frac{\partial y}{\partial q}\\
                \frac{\partial z}{\partial r}& \frac{\partial z}{\partial \varphi}& \frac{\partial z}{\partial q}\\
            \end{vmatrix}
            =
            \begin{vmatrix}
                a\cos(\varphi)& -ar\sin(\varphi)& \\
                a\sin(\varphi)& ar\cos(\varphi)& \\
                & & a\\
            \end{vmatrix}
            = a^3 r.
        \]
        Hence volume of solid figure $\Omega$ is
        \begin{align*}
            V
            &= \int_\Omega 1 dx dy dz\\
            &= \int_\Gamma a^3 r dq dr d\varphi\\
            &= 4\int_0^{\pi/4} \int_0^{\sqrt{\cos(2\varphi)}} \int_0^{r^2} a^3 r dq dr d\varphi\\
            &= 4a^3 \int_0^{\pi/4} \int_0^{\sqrt{\cos(2\varphi)}} r \int_0^{r^2} dq dr d\varphi\\
            &= 4a^3 \int_0^{\pi/4} \int_0^{\sqrt{\cos(2\varphi)}} r^3 dr d\varphi\\
            &= 4a^3 \int_0^{\pi/4} \left. \frac{r^4}{4} \right|_0^{\sqrt{\cos(2\varphi)}} d\varphi
        \end{align*}
        \begin{align*}
            &= a^3 \int_0^{\pi/4} \cos(2\varphi)^2 d\varphi\\
            &= \frac{1}{2} a^3 \int_0^{\pi/4} 1 + \cos(4\varphi) d\varphi\\
            &= \frac{1}{2} a^3 \left. \varphi + \frac{\sin(4\varphi)}{4} \right|_0^{\pi/4}\\
            &= \frac{1}{2} a^3 \left.\left( \varphi + \frac{\sin(4\varphi)}{4} \right)\right|_0^{\pi/4}\\
            &= \frac{1}{2} a^3 \frac{\pi}{4}\\
            &= \frac{\pi a^3}{8}
        \end{align*}
    \end{problem}

    \begin{problem}{4104}
        Similarly we are looking for volume of solide figure
        \[\Omega := \{(x; y; z) \mid \sqrt{x^2 + y^2} \geqslant z \wedge az \geqslant x^2 + y^2\}.\]
        Let $x = ar \cos(\varphi)$, $y = ar \sin(\varphi)$, $z = aq$. Then $(x; y; z) \in \Omega$ iff
        \[(r; \varphi; q) \in \Gamma := \{(r; \varphi; q) \mid r \geqslant q \geqslant r^2 \wedge r \geqslant 0 \wedge \varphi \in [0; 2\pi]\}.\]
        Also Jacobian of the substitution $(r; \varphi; q) \to (x; y; z)$ is
        \[
            \begin{vmatrix}
                \frac{\partial x}{\partial r}& \frac{\partial x}{\partial \varphi}& \frac{\partial x}{\partial q}\\
                \frac{\partial y}{\partial r}& \frac{\partial y}{\partial \varphi}& \frac{\partial y}{\partial q}\\
                \frac{\partial z}{\partial r}& \frac{\partial z}{\partial \varphi}& \frac{\partial z}{\partial q}\\
            \end{vmatrix}
            =
            \begin{vmatrix}
                a\cos(\varphi)& -ar\sin(\varphi)& \\
                a\sin(\varphi)& ar\cos(\varphi)& \\
                & & a\\
            \end{vmatrix}
            = a^3 r.
        \]
        Hence volume of solid figure $\Omega$ is
        \begin{align*}
            V
            &= \int_\Omega 1 dx dy dz\\
            &= \int_\Gamma a^3 r dq dr d\varphi\\
            &= \int_0^{2\pi} \int_0^1 \int_{r^2}^r a^3 r dq dr d\varphi\\
            &= 2\pi a^3 \int_0^1 r \int_{r^2}^r dq dr\\
            &= 2\pi a^3 \int_0^1 r(r - r^2) dr\\
            &= 2\pi a^3 \int_0^1 r^2 - r^3 dr\\
            &= 2\pi a^3 \left.\left(\frac{r^3}{3} - \frac{r^4}{4}\right)\right|_0^1\\
            &= \frac{\pi a^3}{6}
        \end{align*}
    \end{problem}
\end{document}