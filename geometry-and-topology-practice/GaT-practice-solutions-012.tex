\documentclass[12pt,a4paper]{article}
\usepackage{solutions}
% \usepackage{float}
\usepackage{inkscape}

\title{Занятие от 11.03.\\Геометрия и топология. 1 курс.\\Решения.}
\author{Глеб Минаев @ 102 (20.Б02-мкн)}
% \date{}

\DeclareMathOperator{\Img}{Im}
\DeclareMathOperator{\Rea}{Rea}
\DeclareMathOperator{\Ker}{Ker}
\DeclareMathOperator{\Cl}{Cl}
\DeclareMathOperator{\Int}{Int}
\DeclareMathOperator{\Fr}{Fr}
\DeclareMathOperator{\Id}{Id}
\DeclareMathOperator{\sign}{sign}
\DeclareMathOperator{\tr}{tr}
\newcommand{\FAC}{\ensuremath{\mathrm{FAC}}\xspace}
\newcommand{\SAC}{\ensuremath{\mathrm{SAC}}\xspace}
\newcommand{\T}{\ensuremath{\mathrm{T}}\xspace}
\newcommand{\SL}{\ensuremath{\mathrm{SL}}\xspace}
\newcommand{\PGL}{\ensuremath{\mathrm{PGL}}\xspace}

\begin{document}
    \maketitle

    \begin{problem}{114}
        Давайте определим функцию
        \[g: V \times V \to \RR, (u, v) \mapsto \frac{p(u+v)^2 - p(u)^2 - p(v)^2}{2}\]
        Покажем, что $g$ является скалярным произведением на $V$.
        \begin{enumerate}
            \item Симметричность очевидна:
                \[
                    g(u, v)
                    = \frac{p(u+v)^2 - p(u)^2 - p(v)^2}{2}
                    = \frac{p(v+u)^2 - p(v)^2 - p(u)^2}{2}
                    = g(v, u)
                \]

            \item Теперь покажем аддитивность по аргументу (покажем для правого, а для левого будет следовать из симметричности). Вспомним тождество, заданное на $p$
                \[p(u+v)^2 + p(u-v)^2 = 2(p(u)^2 + p(v)^2)\]
                Применим его трижды для пар $u$ и $v$, $u$ и $v-w$, $u$ и $v+w$:
                \begin{align}
                    p(u+v)^2 + p(u-v)^2 &= 2(p(u)^2 + p(v)^2) \label{eq1}\\
                    p(u+v-w)^2 + p(u-v+w)^2 &= 2(p(u)^2 + p(v-w)^2) \label{eq2}\\
                    p(u+v+w)^2 + p(-u+v+w)^2 &= 2(p(u)^2 + p(v+w)^2) \label{eq3}
                \end{align}
                Рассмотрим таблицу \ref{result-table}: в ней циклично просуммированы выражения выше и показано, какие квадраты модулей с какими коэффициентами входят. Суммируя всё вместе с правильными (указанными в таблице) коэффициентами мы получаем
                \[p(u+v+w)^2 - p(u+v)^2 - p(v+w)^2 - p(w+u)^2 + p(u)^2 + p(v)^2 + p(w)^2 = 0\]
                так как сложили несколько выражений равных нулю. При этом несложно видеть, что полученное равенство равносильно равенству
                \[2g(u, v+w) - 2g(u, v) - 2g(v, w) = 0\]
                Отсюда мы и получаем, что
                \[g(u, v + w) = g(u, v) + g(u, w)\]

                \begin{table}
                    \centering
                    \begin{tabular}{c||c|c|c||c}
                        $p(\ldots)^2$&
                        $\frac{1}{3} \sum\limits_{\mathrm{cyc}} \hyperref[eq1]{(\ref*{eq1})}$&
                        $\frac{-1}{6} \sum\limits_{\mathrm{cyc}} \hyperref[eq2]{(\ref*{eq2})}$&
                        $\frac{-1}{3} \sum\limits_{\mathrm{cyc}} \hyperref[eq3]{(\ref*{eq3})}$&
                        Сумма\\
                        \hline
                        \raisebox{-10pt}{\rule{0pt}{28pt}} $u + v + w$& $3 \cdot \frac{1}{3}$& & & $1$\\
                        \raisebox{-10pt}{\rule{0pt}{28pt}} $u + v - w$& $1 \cdot \frac{1}{3}$& $2 \cdot \frac{-1}{6}$& & $0$\\
                        \raisebox{-10pt}{\rule{0pt}{28pt}} $v + w - u$& $1 \cdot \frac{1}{3}$& $2 \cdot \frac{-1}{6}$& & $0$\\
                        \raisebox{-10pt}{\rule{0pt}{28pt}} $w + u - v$& $1 \cdot \frac{1}{3}$& $2 \cdot \frac{-1}{6}$& & $0$\\
                        \raisebox{-10pt}{\rule{0pt}{28pt}} $u + v$& $-2 \cdot \frac{1}{3}$& & $1 \cdot \frac{-1}{3}$& $-1$\\
                        \raisebox{-10pt}{\rule{0pt}{28pt}} $v + w$& $-2 \cdot \frac{1}{3}$& & $1 \cdot \frac{-1}{3}$& $-1$\\
                        \raisebox{-10pt}{\rule{0pt}{28pt}} $w + u$& $-2 \cdot \frac{1}{3}$& & $1 \cdot \frac{-1}{3}$& $-1$\\
                        \raisebox{-10pt}{\rule{0pt}{28pt}} $u - v$& & $-2 \cdot \frac{-1}{6}$& $1 \cdot \frac{-1}{3}$& $0$\\
                        \raisebox{-10pt}{\rule{0pt}{28pt}} $v - w$& & $-2 \cdot \frac{-1}{6}$& $1 \cdot \frac{-1}{3}$& $0$\\
                        \raisebox{-10pt}{\rule{0pt}{28pt}} $w - u$& & $-2 \cdot \frac{-1}{6}$& $1 \cdot \frac{-1}{3}$& $0$\\
                        \raisebox{-10pt}{\rule{0pt}{28pt}} $u$& $-2 \cdot \frac{1}{3}$& $-2 \cdot \frac{-1}{6}$& $-4 \cdot \frac{-1}{3}$& $1$\\
                        \raisebox{-10pt}{\rule{0pt}{28pt}} $v$& $-2 \cdot \frac{1}{3}$& $-2 \cdot \frac{-1}{6}$& $-4 \cdot \frac{-1}{3}$& $1$\\
                        \raisebox{-10pt}{\rule{0pt}{28pt}} $w$& $-2 \cdot \frac{1}{3}$& $-2 \cdot \frac{-1}{6}$& $-4 \cdot \frac{-1}{3}$& $1$\\
                    \end{tabular}
                    \caption{}\label{result-table}
                \end{table}

            \item Теперь покажем, пропорциональность по правому аргументу, т.е. что $g(u, \lambda v) = \lambda g(u, v)$. Для этого заметим, что
                \begin{enumerate}
                    \item \[g(u, \overrightarrow{0}) = \frac{p(u)^2 - p(u)^2 - p(\overrightarrow{0})^2}{2} = 0\]
                    \item $\forall n \in \NN$ \[g(u, nv) = \underbrace{g(u, v) + \dots + g(u, v)}_n = n g(u, v)\]
                    \item $\forall n \in \NN$ \[g\left(u, \frac{1}{n}v\right) = \frac{1}{n} \cdot ng\left(u, \frac{1}{n}v\right) = \frac{1}{n} g\left(u, n\frac{1}{n}v\right) = \frac{1}{n} g(u, v)\]
                    \item 
                        \begin{multline*}
                            g(u, -v) = g(u, -v) + g(u, v) - g(u, v) =\\
                            g(u, -v + v) - g(u, v) = g(u, \overrightarrow{0}) - g(u, v) = -g(u, v)
                        \end{multline*}
                    \item $\forall n \in \ZZ, m \in \NN$ \[g\left(u, \frac{n}{m} v\right) = \sign(n)g\left(u, \frac{|n|}{m} v\right) = n g\left(u, \frac{1}{m} v\right) = \frac{n}{m} g\left(u, v\right)\]
                \end{enumerate}
                т.е. для всех рациональных $\lambda$ утверждение доказано. При этом заметим, что сама по себе функция $p$ непрерывна, поэтому непрерывна и $g$ (хотя потребуется непрерывность только по одному аргументу). Следовательно
                \[g(u, \lambda v) = \lim_{n \to \infty} g(u, q_n v) = \lim_{n \to \infty} q_n g(u, v) = \lambda g(u, v)\]
                где $(q_n)_{n=0}^\infty$ --- случайная последовательность рациональных, сходящаяся к $\lambda$.

            \item И последнее.
                \[g(u, u) = \frac{p(u + u)^2 - p(u)^2 - p(u)^2}{2} = p(u)^2 \geqslant 0\]
                При этом равенство достигается тогда и только тогда, когда $u \neq \overrightarrow{0}$.
        \end{enumerate}
    \end{problem}

    \begin{problem}{115}
        Для начала заметим, что биекция
        \[
            \varphi: \mathrm{Mat}_n(\RR) \to \RR^{n^2},
            \begin{pmatrix}
                a_{1,1}& \cdots& a_{1, n}\\
                \vdots& \ddots& \vdots\\
                a_{n,1}& \cdots& a_{n, n}
            \end{pmatrix}
            \mapsto (a_{1,1}, \ldots, a_{1,n}, a_{2,1}, \ldots, a_{n,n})
        \]
        сопоставляет скалярному произведению $\tr(A^T B)$ в $\mathrm{Mat}_n(\RR)$ обычное скалярное произведение в $\RR^{n^2}$. Действительно,
        \begin{align*}
            \tr(A^T B)
            &= \sum_{i=1}^n (A^T B)_{i, i}&
            &= \sum_{i=1}^n \sum_{j=1}^n (A^T)_{i,j} \cdot B_{j, i}&
            &= \sum_{i=1}^n \sum_{j=1}^n A_{j,i} \cdot B_{j, i}&
            &= \varphi(A) \cdot \varphi(B)
        \end{align*}
        В таком случае будем рассматривать вместо $\mathrm{Mat}_2(\RR)$ $\RR^4$.

        Также заметим, что $U$ описывается как пространство, перпендикулярное векторам $K = \varphi(E^T) = \varphi(E) = (1; 0; 0; 1)$ и $L = \varphi(J^T) = (0; -1; 1; 0)$. Поскольку $X = X^\parallel + X^\perp$, где $X^\parallel \in U$, а $X^\perp \perp U$, то $X^\perp \in \langle K, L \rangle$, а $X^\parallel \perp \langle K, L \rangle$. Таким образом
        \[X = \alpha K + \beta L + X^\parallel\]
        
        Ещё раз напомним конкретные значения:
        \begin{align*}
            K &= (1; 0; 0; 1)&
            L &= (0; -1; 1; 0)&
            X &= (3; 4; 2; -5)
        \end{align*}

        Запишем банальные уравнения:
        \begin{align*}
            \alpha (K \cdot K) + \beta (L \cdot K) &= X \cdot K&
            \alpha (K \cdot L) + \beta (L \cdot L) &= X \cdot L\\
        \end{align*}
        т.е.
        \[
            \begin{pmatrix}
                K \cdot K& K \cdot L\\
                L \cdot K& L \cdot L\\
            \end{pmatrix}
            \begin{pmatrix}
                \alpha\\ \beta
            \end{pmatrix}
            = \begin{pmatrix}
                X \cdot K\\
                X \cdot L
            \end{pmatrix}
        \]
        Следовательно
        \begin{align*}
            \begin{pmatrix}
                \alpha\\ \beta
            \end{pmatrix}
            &=
            \begin{pmatrix}
                K \cdot K& K \cdot L\\
                L \cdot K& L \cdot L\\
            \end{pmatrix}^{-1}
            \begin{pmatrix}
                X \cdot K\\
                X \cdot L
            \end{pmatrix}\\
            &=\frac{1}{(K \cdot K)(L \cdot L) - (K \cdot L)^2}
            \begin{pmatrix}
                L \cdot L& -K \cdot L\\
                -L \cdot K& K \cdot K\\
            \end{pmatrix}
            \begin{pmatrix}
                X \cdot K\\
                X \cdot L
            \end{pmatrix}\\
            &=\frac{1}{2 \cdot 2 - 0^2}
            \begin{pmatrix}
                2& -0\\
                -0& 2\\
            \end{pmatrix}
            \begin{pmatrix}
                -2\\
                -2
            \end{pmatrix}\\
            &=\frac{4}{4}
            \begin{pmatrix}
                1& 0\\
                0& 1\\
            \end{pmatrix}
            \begin{pmatrix}
                -1\\
                -1
            \end{pmatrix}
            =
            \begin{pmatrix}
                -1\\ -1
            \end{pmatrix}\\
        \end{align*}
        Тогда
        \begin{align*}
            &X^\perp = \alpha K + \beta L = (-1; 1; -1; -1)&
            &X^\parallel = X - X^\perp = (4; 3; 3; -4)
        \end{align*}
        Прообразы этих векторов легко восстанавливаются:
        \begin{align*}
            &X^\perp =
            \begin{pmatrix}
                -1& 1\\
                -1& -1
            \end{pmatrix}&
            &X^\parallel =
            \begin{pmatrix}
                4& 3\\
                3& -4
            \end{pmatrix}
        \end{align*}
    \end{problem}
\end{document}