\documentclass[12pt,a4paper]{article}
\usepackage{solutions}
% \usepackage{float}

\title{Занятие от 3.12.\\Геометрия и топология. 1 курс.\\Решения.}
\author{Глеб Минаев @ 102 (20.Б02-мкн)}
% \date{}

\DeclareMathOperator{\Cl}{Cl}
\DeclareMathOperator{\Int}{Int}
\DeclareMathOperator{\Fr}{Fr}
\DeclareMathOperator{\Id}{Id}
\newcommand{\FAC}{\ensuremath{\mathrm{FAC}}\xspace}
\newcommand{\SAC}{\ensuremath{\mathrm{SAC}}\xspace}
\newcommand{\T}{\ensuremath{\mathrm{T}}\xspace}

\begin{document}
    \maketitle

    \begin{problem}{48}\ 
        \begin{itemize}
            \item Покажем, что это действительно топологическое пространство.
                \begin{enumerate}
                    \item $\RR$ и $\varnothing$ безусловно открыты.
                    \item Пусть даны открытые в обычном смысле множества $U_1$, \dots, $U_n$ и счётные множества $C_1$, \dots, $C_n$. Тогда
                        \[
                            \bigcap_{i=1}^n U_i \setminus C_i
                            = \left(\bigcap_{i=1}^n U_i\right) \setminus \left(\bigcup_{i=1}^n C_i\right)
                        \]
                        При этом первая скобка --- открытое множество, а вторая --- счётное. Следовательно пересечение конечного числа открытых в новом смысле множеств открыто в новом смысле.
                    \item Пусть дано семейство $\{U_i\}_{i \in I}$ открытых в обычном смысле множеств и семейство $\{C_i\}_{i \in I}$ счётных множеств. Заметим, что
                        \[\bigcup_{i \in I} U_i\]
                        --- открытое в обычном смысле множество, значит раскладывается в счётное объединение попарно непересекающихся интервалов. Пусть $I$ --- такой интервал разложения. Тогда $I$ является объединением счётного числа отрезков, каждый из которых покрывается конечным набором открытых множеств из $\{U_i\}_{i \in I}$, следовательно и $I$ покрывается счётным набором множеств из $\{U_i\}_{i \in I}$. Значит
                        \[\bigcup_{i \in I} U_i \setminus C_i\]
                        содержит всё тот же интервал без счётного числа точек, поэтому
                        \[\bigcup_{i \in I} U_i \setminus C_i = \left(\bigcup_{i \in I} U_i\right) \setminus C\]
                        где $C$ счётно. Таким образом объединение любого семейства открытых в новом смысле множеств открыто в новом смысле.
                \end{enumerate}
            \item Покажем, что пространство хаусдорфово. Поскольку обычные метрические окрестности открыты в новом смысле, а любые две точки можно разделить метрическими окрестностями правильных размеров, то $\T_2$ верна для нового пространства.
            \item По теореме с лекции достаточно предъявить точку $a$ и её окрестность $U$, что замыкание всякой подокрестности $U$ точки $a$ не будет подмножеством $U$.

            Для этого рассмотрим $a := \sqrt{2}$ и $U := \RR \setminus \QQ$. Пусть $V$ --- некоторая окрестность $a$, являющаяся подмножеством $U$. Тогда $V$ содержит как подмножество некоторую метрическую окрестность $I$ точки $a$ без счётного множества $C$. При этом понятно, что $C \supseteq \QQ$.
            
            Покажем, что $\Cl(V) \supseteq I$. Пусть $t$ --- некоторая точка $I \cap C$. Если $t$ --- непредельная точка $V$, то в некоторой окрестности (в новом смысле) точки $t$ нет точек $V$. Это значит, что в некоторой метрической окрестности $t$ не более чем счётное число точек $V$. Но на деле их там континуум (поскольку из $V$ в этой окрестности лежит он весь без счётного количество точек (например, без $C$)) --- противоречие. Следовательно всякая точка $t \in I$ является предельной точкой $V$. Т.е. $I \subseteq \Cl(V)$.

            Заметим, что $I \nsubseteq U$, значит $\Cl(V) \nsubseteq U$. Поскольку это утверждение верно независимо от $V$, то имеем, что новое пространство не регулярно.
        \end{itemize}
    \end{problem}
\end{document}