\documentclass[12pt,a4paper]{article}
\usepackage{solutions}
% \usepackage{float}
\usepackage{inkscape}

\title{Занятие от 11.02.\\Геометрия и топология. 1 курс.\\Решения.}
\author{Глеб Минаев @ 102 (20.Б02-мкн)}
% \date{}

\DeclareMathOperator{\Cl}{Cl}
\DeclareMathOperator{\Int}{Int}
\DeclareMathOperator{\Fr}{Fr}
\DeclareMathOperator{\Id}{Id}
\newcommand{\FAC}{\ensuremath{\mathrm{FAC}}\xspace}
\newcommand{\SAC}{\ensuremath{\mathrm{SAC}}\xspace}
\newcommand{\T}{\ensuremath{\mathrm{T}}\xspace}

\begin{document}
    \maketitle

    \begin{problem}{73} 
        Пусть поверхность $S$ собрана из граней $\Gamma_1$, \dots, $\Gamma_n$ (которые в свою очередь являются замкнутыми поверхностями). Для каждой грани $\Gamma_i$ найдём развёртку. Хотим соединить изначальное развёртку $S$ с развёртками каждой её грани, а потом посмотреть, как изменялось значение выражения $V - E + F$.

        Будем по очереди заменять каждую $\Gamma_i$ на её развёртку. Но есть единственная проблема: на границе $\Gamma_i$, по которой она склеивается с остальными гранями $S$, вершины и рёбра у развёрток $S$ и $\Gamma_i$ могут не соответствовать.
        \begin{figure}[h]
            \centering
            \inkscapepicture{GaT-practice-solutions-008-1}
        \end{figure}
        Для этого возьмём в качестве вершин объединение вершин обеих развёрток, а рёбра раздробим на множества рёбер строго по этим вершинам. Тогда мы можем объединить эти развёртку в одну развёртку. При этом во время разбиения рёбер выражение $V - E + F$ не изменило своего значения, а после замены $\Gamma_i$ на её развертку изменилось на $- \chi(D_1) + \chi(\Gamma_i) = \chi(\Gamma_i) - 1$.

        Таким образом мы получаем, что
        \[\chi(S) = V - E + F + \sum_{i=1}^n (\chi(\Gamma_i) - 1) = V - E + \sum_{k=1}^n \chi(\Gamma_i)\]

        Поскольку у каждой грани $\Gamma_i$ есть грань, то $\chi(\Gamma_i) - 1 \leqslant 0$. Следовательно
        \[\chi(S) \leqslant V - E + F\]
    \end{problem}
\end{document}