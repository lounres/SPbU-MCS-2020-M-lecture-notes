\documentclass[12pt,a4paper]{article}
\usepackage{solutions}
% \usepackage{float}

\title{Занятие от 26.11.\\Геометрия и топология. 1 курс.\\Решения.}
\author{Глеб Минаев @ 102 (20.Б02-мкн)}
% \date{}

\DeclareMathOperator{\Cl}{Cl}
\DeclareMathOperator{\Int}{Int}
\DeclareMathOperator{\Fr}{Fr}
\DeclareMathOperator{\Id}{Id}

\begin{document}
    \maketitle

    \begin{problem}{39}\ 
        \begin{enumerate}
            \renewcommand{\theenumi}{\asbuk{enumi}}
            \renewcommand{\labelenumi}{\theenumi)}
            \setcounter{enumi}{2}
            \item Пусть есть некоторые $r_{A, B} > 0$ и $r_{B, C} > 0$, что
                \begin{align*}
                    A \subseteq U_{r_{A, B}}(B) &\wedge B \subseteq U_{r_{A, B}}(A)&
                    &\text{ и }&
                    B \subseteq U_{r_{B, C}}(C) &\wedge C \subseteq U_{r_{B, C}}(B)
                \end{align*}
                Заметим, что для всяких множеств $X$ и $Y$ и положительных значений $r$, $r_1$ и $r_2$ верно, что $X \subseteq Y \rightarrow U_r(X) \subseteq U_r(Y)$, а $U_{r_1}(U_{r_2}(X)) = U_{r_1+r_2}(X)$. Следовательно
                \begin{align*}
                    &A \subseteq U_{r_{A, B}}(B) \subseteq U_{r_{A, B} + r_{B, C}}(C)&
                    &C \subseteq U_{r_{B, C}}(B) \subseteq U_{r_{A, B} + r_{B, C}}(A)
                \end{align*}
                Таким образом
                \begin{multline*}
                    \{r > 0: A \subseteq U_r(B) \wedge B \subseteq U_r(A)\}
                    + \{r > 0: B \subseteq U_r(C) \wedge C \subseteq U_r(B)\}\\
                    \subseteq \{r > 0: A \subseteq U_r(C) \wedge C \subseteq U_r(A)\}
                \end{multline*}
                Следовательно
                \begin{multline*}
                    \inf \{r > 0: A \subseteq U_r(B) \wedge B \subseteq U_r(A)\}
                    + \inf \{r > 0: B \subseteq U_r(C) \wedge C \subseteq U_r(B)\}\\
                    \geqslant \inf \{r > 0: A \subseteq U_r(C) \wedge C \subseteq U_r(A)\}
                \end{multline*}
                т.е.
                \[d_H(A, B) + d_H(B, C) \geqslant d_H(A, C)\]
            
            \item Заметим, что $B \subseteq \Cl(A)$ равносильно тому, что во всякой окрестности всякой точки из $B$ есть точка из $A$. Это равносильно тому, что для всякого $r > 0$ для всякой точки $b \in B$ есть точка $a \in A$, что $a \in U_r(b)$, следовательно $b \in U_r(a)$, что равносильно тому, что всякая точка $B$ находится в $r$-окрестности $A$, т.е. $B \subseteq U_r(A)$. Таким образом $B \in \Cl(A)$ равносильно тому, что для всякого $r > 0$ верно, что $B \subseteq U_r(A)$.
            
            При этом $\Cl(A) = \Cl(B)$ равносильно тому, что $A \subseteq \Cl(B)$ и $B \subseteq \Cl(A)$, а значит равносильно тому, что $d_H(A, B) = 0$.
        \end{enumerate}
    \end{problem}
\end{document}