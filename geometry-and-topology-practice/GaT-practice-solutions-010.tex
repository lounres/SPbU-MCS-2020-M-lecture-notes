\documentclass[12pt,a4paper]{article}
\usepackage{solutions}
% \usepackage{float}
\usepackage{inkscape}

\title{Занятие от 04.03.\\Геометрия и топология. 1 курс.\\Решения.}
\author{Глеб Минаев @ 102 (20.Б02-мкн)}
% \date{}

\DeclareMathOperator{\Img}{Im}
\DeclareMathOperator{\Ker}{Ker}
\DeclareMathOperator{\Cl}{Cl}
\DeclareMathOperator{\Int}{Int}
\DeclareMathOperator{\Fr}{Fr}
\DeclareMathOperator{\Id}{Id}
\newcommand{\FAC}{\ensuremath{\mathrm{FAC}}\xspace}
\newcommand{\SAC}{\ensuremath{\mathrm{SAC}}\xspace}
\newcommand{\T}{\ensuremath{\mathrm{T}}\xspace}

\begin{document}
    \maketitle

    \begin{problem}{97}
        \begin{lemma}
            Пусть даны аффинные подпространства $A$, $B$, $A'$ и $B'$ пространства $\RR^n$ ($V_A$, $V_B$, $V_{A'}$ и $V_{B'}$ --- их векторные пространства). Следующие утверждения равносильны.
            \begin{enumerate}
                \item Существует аффинная биекция $\RR^n$ на себя, переводящая $A$ в $A'$ и $B$ в $B'$.
                \item $\dim(A) = \dim(A')$, $\dim(B) = \dim(B')$, $\dim(V_A \cap V_{A'}) = \dim(V_B \cap V_{B'})$ и $A \cap B = \varnothing \leftrightarrow A' \cap B' = \varnothing$.
            \end{enumerate}
        \end{lemma}

        \begin{proof}
            \begin{itemize}
                \item[$1 \Rightarrow 2$)] Пусть такая биекция существует. Тогда все утверждения из пункта (2) очевидны для $V_A$ и $V_B$, а значит и для самих аффинных пространств.
                \item[$2 \Rightarrow 1$)] Возьмём любой базис $\{e_i\}_{i=1}^{\deg(V_A \cap V_B)}$ в $V_A \cap V_B$ и дополним его в $V_A$ и в $V_B$ до базисов множествами $\{g_i\}_{i=1}^{\deg(V_A) - \deg(V_A \cap V_B)}$ и $\{h_i\}_{i=1}^{\deg(V_B) - \deg(V_A \cap V_B)}$ соответственно. Заметим, что объединение трёх множеств есть базис $V_A + V_B$. Действительно, пусть это не так и есть такие нетривиальные коэффициенты, что
                \[\sum e_i a_i + \sum g_i b_i + \sum h_i c_i = \overline{0}\]
                Тогда имеем, что
                \[\sum e_i a_i + \sum g_i b_i = \sum h_i (-c_i)\]
                При этом левая часть выражения лежит в $V_A$, а правая в $V_B$, следовательно в $V_A \cap V_B$. Но так как $\{e_i\} \cup \{h_i\}$ и $\{e_i\}$ --- базисы $V_B$ и $V_A \cap V_B$ соответственно, то $c_i = 0$ для всех $i$; аналогично $b_i = 0$. Следовательно
                \[\sum e_i a_i = 0\]
                т.е. $a_i = 0$ --- противоречие.
                Следовательно в $V_A + V_B$ можно построить ``правильный'' базис состоящий из базисов $V_A \cap V_B$, $V_A$ и $V_B$, и по аналогии его можно получить для $V_{A'}$ и $V_{B'}$.
                
                    Рассмотрим два случая:
                    \begin{itemize}
                        \item Пусть $A \cap B$ и $A' \cap B'$ непусты. Тогда в них можно выбрать по точке $p$ и $p'$ соответственно. Тогда $A = p + V_A$, $B = p + V_B$, $A' = p' + V_{A'}$, $B' = p' + V_{B'}$. Тогда можно рассмотреть аффинное преобразование, которое переводит $p$ в $p'$, а в пространстве векторов переводит $e_i$ в $e'_i$, $g_i$ в $g'_i$ и $h_i$ в $h'_i$. Оно то и переведёт $A$ в $A'$ и $B$ в $B'$.

                        \item Пусть $A \cap B = \varnothing = A' \cap B'$. Выберем во всех четырёх пространствах по точке $a$, $b$, $a'$ и $b'$ соответственно. Несложно видеть, что $\overrightarrow{ab} \notin V_A + V_B$, так как иначе $V_A$ и $V_B$ пересекаются. Тогда можно рассмотреть аффинное преобразование, которое переводит $a$ в $a'$, а в пространстве векторов переводит $\overrightarrow{ab}$ в $\overrightarrow{a'b'}$, $e_i$ в $e'_i$, $g_i$ в $g'_i$ и $h_i$ в $h'_i$. Оно то и переведёт $b$ в $b'$, а следовательно $A$ в $A'$ и $B$ в $B'$.
                    \end{itemize}
            \end{itemize}
        \end{proof}

        Заметим, что искомые классы эквивалентности эквивалентны парам $(\deg(V_A \cap V_B), [A \cap B = \varnothing])$, т.е. паре из числа равному размерности $V_A \cap V_B$ и булевому значению, определяющему пусто ли $A \cap B$.
        
        Заметим, что $A \cap B$ может быть пусто тогда и только тогда, когда $\dim(V_A + V_B) < n$. Следовательно искомое количество компонент связности равно количеству достижимых выше описанных пар. Несложно видеть, что $\deg(V_A \cap V_B)$ может принимать все значения от $\max(0, k + m - n)$ до $\min(k, m)$ и только их. И во всех случаях кроме $\deg(V_A \cap V_B)$ второе значение в паре может принимать два значение; в исключённом случае только одно. Таким образом искомый ответ равен
        \begin{align*}
            &2(\min(k, m) - \max(0, k + m - n) + 1) - [\max(0, k + m - n) \leqslant n \leqslant \min(k, m)]\\
            =&2(\min(k, m) - \max(0, k + m - n) + 1) - [\max(0, k + m - n) \leqslant n \wedge n \leqslant \min(k, m)]\\
            =&2(\min(k, m) - \max(0, k + m - n) + 1) - [0 \leqslant n \wedge k + m - n \leqslant n \wedge n \leqslant k \wedge n \leqslant m]\\
            =&2(\min(k, m) - \max(0, k + m - n) + 1) - [k + m \leqslant 2n \wedge n \leqslant k \wedge n \leqslant m]\\
            =&2(\min(k, m) - \max(0, k + m - n) + 1) - [k = n \wedge m = n]\\
        \end{align*}
        (где $[*]$ --- \href{https://ru.wikipedia.org/wiki/\%D0\%A1\%D0\%BA\%D0\%BE\%D0\%B1\%D0\%BA\%D0\%B0_\%D0\%90\%D0\%B9\%D0\%B2\%D0\%B5\%D1\%80\%D1\%81\%D0\%BE\%D0\%BD\%D0\%B0}{``скобка Айверсона''}).

        Отсюда рассматривая конкретные $k$ и $m$ несложно получить ответ.
    \end{problem}

    \begin{problem}{100}
        Заметим, что $f$ имеет вид $Ax + b$ для некоторых линейного оператора $A$ и вектора $b$. Тогда несложно видеть, что
        \[\underbrace{f \circ \dots \circ f}_\text{$n$ раз} = A^n x + (A^{n-1} + A^{n-2} + \dots + A^0) b\]

        Заметим также, что для всяких операторов $A$ и $B$ и вектора $b$ верно следующее.
        \begin{align*}
            &\text{Уравнение $BAx + Bb = 0$ имеет корень.}\\
            \Longleftrightarrow&\text{Уравнение $Ax + b \in \Ker B$ имеет корень.}\\
            \Longleftrightarrow&b \in \{s - t \mid s \in \Ker B \wedge t \in \Img A\}\\
            \Longleftrightarrow&b \in \Ker B + \Img A\\
        \end{align*}

        Теперь поймём, что существование неподвижной точки $f$ равносильно разрешимости уравнения $(A-1)x + b = 0$, а $f^n$ --- $(A^n - \Id)x + (A^{n-1} + \dots + A^0) = 0$, что равносильно $(A^{n-1} + \dots + A^0)((A-1)x + b) = 0$. Таким образом нужно показать, что если $b \in \Ker(A^{n-1} + \dots + A^0) + \Img(A-1)$, то $b \in \Img(A-1)$. Т.е. нужно показать, что $\Ker(A^{n-1} + \dots + A^0) \subseteq \Img(A-1)$.
        
        Перейдём в поле комплексных чисел и докажем данное утверждение там; после сужения поля факт несомненно останется верным. Обозначим за $\zeta$ первообразный корень из $1$ степени $n$. Тогда несложно видеть, что
        \[A^{n-1} + \dots + A^0 = \prod_{i=1}^{n-1} A - \zeta^i\]

        Покажем по индукции по $k$, что для любых различных констант $\lambda_1$, \dots, $\lambda_k$ и для всякого оператора $A$ верно, что
        \[\Ker \prod_{i=1}^k A - \lambda_i = \sum_{i=1}^k \Ker(A - \lambda_i)\]

        {\bfseries База.} При $k = 0, 1$ очевидно.

        {\bfseries Шаг.} Заметим, что по предположению индукции всякий элемент $\Ker\prod_{i=1}^{k-1} A - \lambda_i$ имеет вид $v_1 + \dots + v_{k-1}$, где $v_i \in \Ker(A - \lambda_i)$. Соответственно если $(\prod_{i=1}^k A - \lambda_i) x = 0$, то $(A - \lambda_k)x$ имеет вид $v_1 + \dots + v_{k-1}$. Следовательно
        \[\left(\prod_{i=1}^k A - \lambda_i \right)\left(x - \frac{v_1}{\lambda_1 - \lambda_k} - \dots - \frac{v_1}{\lambda_{k-1} - \lambda_k}\right) = 0\]
        Таким образом $x$ имеет вид
        \[\frac{v_1}{\lambda_1 - \lambda_k} + \dots + \frac{v_1}{\lambda_{k-1} - \lambda_k} + v_k\]
        где $v_i \in \Ker(A - \lambda_i)$, т.е.
        По той же причине несложно видеть, что всякий элемент $\sum_{i=1}^k \Ker(A - \lambda_i)$ лежит в ядре $\prod_{i=1}^k A - \lambda_i$.

        Теперь покажем (ещё раз), что $\Ker(A - \zeta^i) \subseteq \Img(A - 1)$ при $i \in \{1; \dots; n-1\}$. Отсюда будет следовать требуемое утверждение. Действительно, если $v \in \Ker(A - \zeta^i)$, то
        \[(A-1)\frac{v}{\zeta^i - 1} = \frac{1}{\zeta^i - 1}(A-1)v = \frac{1}{\zeta^i - 1}(\zeta^i v - v) = v\]
        т.е. $v \in \Img(A - 1)$.
    \end{problem}
\end{document}