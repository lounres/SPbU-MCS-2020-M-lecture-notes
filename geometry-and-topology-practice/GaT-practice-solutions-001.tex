\documentclass[12pt,a4paper]{article}
\usepackage{solutions}
\usepackage{float}

\title{Занятие от 5.11.\\Геометрия и топология. 1 курс.\\Решения.}
\author{Глеб Минаев @ 102 (20.Б02-мкн)}
% \date{}

\begin{document}
    \maketitle

    \begin{problem}{9}
        \begin{lemma}
            $f$ монотонно не убывает на $\RR_+$.
        \end{lemma}

        \begin{proof}
            Предположим противное. Тогда есть $s$ и $t$, что $s > t$, а $f(s) < f(t)$. В таком случае для всех $x > s$ верно, что
            \[
                \alpha := \frac{s-t}{x-t} \in [0; 1]\qquad
                \text{и}\qquad
                \alpha x + (1-\alpha) t = \frac{(s-t)x}{x-t} + \frac{(x-s)t}{(x-t)} = \frac{s(x-t)}{(x-t)} = s
            \]
            а тогда по выпуклости $f$
            \begin{align*}
                f(x)
                &= \frac{\alpha f(x) + (1-\alpha) f(t) - (1-\alpha) f(t)}{\alpha}&
                &\leqslant \frac{f(\alpha x + (1-\alpha) t) - (1-\alpha) f(t)}{\alpha}\\
                &= \frac{f(s) - f(t)}{\alpha} + f(t)&
                &= \frac{f(s) - f(t)}{s - t} (x - t) + f(t)
            \end{align*}
            При этом последнее --- линейный многочлен с отрицательным старшим членом, поэтому при достаточно больших $x$ значение $f$ станет отрицательным, что не может быть, так как область значений $f$ --- $\RR_+$. Поэтому $f$ не убывает.
        \end{proof}

        \begin{lemma}
            Для любых $a, b \in \RR_+$ верно, что $f(a) + f(b) \geqslant f(a + b)$.
        \end{lemma}

        \begin{proof}
            \[
                \left.
                \begin{aligned}
                    f(a) &\geqslant \frac{a}{a+b} f(a+b) + \frac{b}{a+b} f(0)\\
                    f(b) &\geqslant \frac{b}{a+b} f(a+b) + \frac{a}{a+b} f(0)\\
                \end{aligned}
                \right\} \Longrightarrow
                f(a) + f(b) \geqslant f(a + b) + f(0) = f(a + b)
            \]
        \end{proof}

        Докажем, что $f \circ d$ --- метрика.
        \begin{enumerate}
            \item $\forall x, y \in X\quad d(x, y) \geqslant 0 \Rightarrow f(d(x, y)) \geqslant 0$.
            \item $\forall x, y \in X\quad f(d(x, y)) = 0 \leftrightarrow d(x, y) = 0 \leftrightarrow x = y$.
            \item $\forall x, y \in X\quad d(x, y) = d(y, x) \Rightarrow f(d(x, y)) = f(d(y, x))$.
            \item $\forall x, y, z \in X$
                \[d(x, y) + d(y, z) \geqslant d(x, z)\quad \Longrightarrow\quad f(d(x, y)) + f(d(y, z)) \geqslant f(d(x, y) + d(y, z)) \geqslant f(d(x, z)).\]
        \end{enumerate}
        Поэтому $f \circ d$ --- метрика.
    \end{problem}
\end{document}