\documentclass[12pt,a4paper]{article}
\usepackage{solutions}
% \usepackage{float}
\usepackage{inkscape}

\title{Занятие от 18.02.\\Геометрия и топология. 1 курс.\\Решения.}
\author{Глеб Минаев @ 102 (20.Б02-мкн)}
% \date{}

\DeclareMathOperator{\Cl}{Cl}
\DeclareMathOperator{\Int}{Int}
\DeclareMathOperator{\Fr}{Fr}
\DeclareMathOperator{\Id}{Id}
\newcommand{\FAC}{\ensuremath{\mathrm{FAC}}\xspace}
\newcommand{\SAC}{\ensuremath{\mathrm{SAC}}\xspace}
\newcommand{\T}{\ensuremath{\mathrm{T}}\xspace}

\begin{document}
    \maketitle

    \begin{problem}{79}
        Обозначим нашу поверхность за $S$, а количества вершин, рёбер и граней графа за $V$, $E$ и $F$ соответственно.

        Вспомним, что по задаче 73
        \[V - E + F \geqslant \chi(S).\]

        Если мы просуммируем количества рёбер у каждой грани, то с одной стороны получим $2E$, а с другой --- значение, неменьшее $kF$; т.е.
        \[2E \geqslant kF\]
        Тогда мы получаем, что
        \[V - \chi(S) \geqslant E - F \geqslant E - \frac{2}{k} E = \frac{k-2}{k} E.\]
        Следовательно
        \[(k-2) E \leqslant k(V - \chi(S)).\]
    \end{problem}

    \begin{problem}{83}
        Пусть $K_{4,4}$ вложен в некоторую поверхность $S$. Несложно видеть, что тогда любая грань будет иметь хотя бы 4 ребра границы. Действительно, поскольку $K_{4,4}$ двудолен, то всякий цикл имеет чётную длину; при этом поскольку нет кратных рёбер, то всякий рёберно простой цикл имеет длину хотя бы $4$. Таким образом по предыдущей задаче
        \[\frac{k-2}{k} E \leqslant V - \chi(S)\]
        где $k = 4$. Следовательно
        \[\chi(S) \leqslant V - \frac{k-2}{k} E = 8 - \frac{2}{4} (4 \cdot 4) = 0\]
        т.е. $S$ не может быть проективной плоскостью ($\chi(\RR P^2) = 1 > 0$).

        Приведём пример вложения для тора. Его можно видеть на рис. \ref{K_4,4-in-torus}
        \begin{figure}[p]
            \Huge
            \centering
            \inkscapepicture{GaT-practice-solutions-009-1}
            \caption{Вложение $K_{4,4}$ в тор.}
            \label{K_4,4-in-torus}
        \end{figure}
    \end{problem}
\end{document}