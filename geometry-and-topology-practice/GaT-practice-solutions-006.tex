\documentclass[12pt,a4paper]{article}
\usepackage{solutions}
% \usepackage{float}

\title{Занятие от 10.12.\\Геометрия и топология. 1 курс.\\Решения.}
\author{Глеб Минаев @ 102 (20.Б02-мкн)}
% \date{}

\DeclareMathOperator{\Cl}{Cl}
\DeclareMathOperator{\Int}{Int}
\DeclareMathOperator{\Fr}{Fr}
\DeclareMathOperator{\Id}{Id}
\newcommand{\FAC}{\ensuremath{\mathrm{FAC}}\xspace}
\newcommand{\SAC}{\ensuremath{\mathrm{SAC}}\xspace}
\newcommand{\T}{\ensuremath{\mathrm{T}}\xspace}

\begin{document}
    \maketitle

    \begin{problem}{56}
        \begin{lemma}
            Найдётся линейно независимый набор векторов $\{v_n\}_{n=0}^\infty$, что:
            \begin{itemize}
                \item $\forall n \in \NN \cup \{0\}\qquad \|v_n\| = 1$.
                \item $\forall n \in \NN \cup \{0\}\; \exists \varepsilon > 0:\qquad U_{\varepsilon}(v_n) \cap \{v_k\}_{k=0}^\infty = \{v_n\}$. 
            \end{itemize}
        \end{lemma}

        \begin{proof}
            Поскольку пространство бесконечномерное, то у него есть бесконечный базис $\Sigma$. Также сразу будем подразумевать под $\Sigma$ множество $\{v/\|v\| \mid v \in \Sigma\}$.
            
            Если в $\Sigma$ есть такой вектор $v$, что для всякого $\varepsilon > 0$ верно, что
            \[|\Sigma \setminus U_{\varepsilon}(v_n)| \in \NN,\]
            то построим искомую последовательность следующим образом. На место $v_0$ возьмём любой элемент $\Sigma \setminus \{v\}$. Далее каждый следующий элемент $v_{n+1}$ определим как случайный элемент
            \[\Sigma \cap U_{d(v_n, v)/2}(v) \setminus \{v\};\]
            это множество непусто, так как иначе $\Sigma$ конечно. Следовательно для всяких $m > n$ верно, что $d(v_m, v) > 2d(v_n, v)$, а значит
            \begin{align*}
                &d(v_m, v_n) \geqslant d(v_m, v) - d(v_n, v) > d(v_n, v)&
                &d(v_m, v_n) \geqslant d(v_m, v) - d(v_n, v) > d(v_m, v)/2&
            \end{align*}
            Таким образом $U_{d(v_n, v)/2}(v_n) \cap \{v_k\}_{k=0}^\infty = \{v_n\}$ для всякого $n \in \NN \cup \{0\}$.

            
        \end{proof}
    \end{problem}
\end{document}