\documentclass[12pt,a4paper]{article}
\usepackage{solutions}
% \usepackage{float}

\title{Занятие от 19.11.\\Геометрия и топология. 1 курс.\\Решения.}
\author{Глеб Минаев @ 102 (20.Б02-мкн)}
% \date{}

\DeclareMathOperator{\Cl}{Cl}
\DeclareMathOperator{\Int}{Int}
\DeclareMathOperator{\Fr}{Fr}
\DeclareMathOperator{\Id}{Id}

\begin{document}
    \maketitle

    \begin{problem}{30}
        Пусть $\{A_n\}_{n=0}^\infty$ --- набор нигде не плотных множеств. Это значит, что $\Cl(\Int(\RR \setminus A_n)) = \RR$, а значит любое открытое множество будет пересекаться с $\Int(\RR \setminus A_n)$. Тогда если мы рассмотрим любое непустое открытое $U$, то множество $U \cap \Int(\RR \setminus A_n)$ будет открытым и непустым. Это можно перефразировать так: в любом интервале прямой найдётся подинтервал, лежащий в $\Int(\RR \setminus A_n)$ как подмножество. Тогда это будет значить, что в любом отрезке прямой будет подотрезок, лежащий в $\Int(\RR \setminus A_n)$ как подмножество.

        Тогда рассмотрим любой отрезок $S_0 \subseteq \Int(\RR \setminus A_0)$, затем любой его подотрезок $S_1 \subseteq \Int(\RR \setminus A_1)$ и т.д. Таким образом получим последовательность отрезков $(S_n)_{n=0}^\infty$. Заметим, что множество $\bigcap_{n = 0}^\infty S_n$ непусто. Следовательно есть точка $y$, лежащая в каждом из отрезков $S_n$, а значит и в каждом множестве $A_n$. Таким образом для всякого $n \geqslant 0$ имеем, что $y \in \RR \setminus A_n$, а значит $y \notin A_n$. Таким образом $\bigcup_{n = 0}^\infty A_n$ не содержит $y$, а значит не совпадает с $\RR$.

        Это буквально значит, что $\RR$ не представляется в виде счётного набора нигде не плотных множеств.
    \end{problem}

    \begin{problem}{31}
        Заметим, что если $U$ открыто, то оно есть дизъюнктное объединение интервалов некоторого не более чем счётного семейства $\Sigma$. Пусть $x$ --- граничная точка $U$. Тогда $x \notin U$, но в каждой окрестности $x$ находится некоторый элемент $U$. Тогда либо $x$ является концом отрезка, либо во всякой окрестности $x$ есть некоторый интервал из $\Sigma$. А тогда понятно, что $\Fr(U)$ есть замыкание множества концов интервалов из $\Sigma$.

        Также очевидно, что $x \in \Fr(U)$ является концом интервала из $\Sigma$ тогда и только тогда, когда в какой-то правой или левой окрестности $x$ нет точек $\Fr(U)$.

        Заметим ещё раз, что $\Fr(U)$ замкнуто, а тогда рассмотрим $F := \RR \setminus \Fr(U)$. Очевидно, что $U \subseteq F$. При этом в каждой окрестности $x \in \Fr(U)$ есть точка из $F$, а значит и из $F$, поэтому $\Cl(F) = \RR$ (поэтому в том числе $\Fr(F) = \Fr(U)$).
        
        Рассмотрим $\Lambda$ --- семейство интервалов, что их дизъюнктное объединение равно $F$. Тогда заметим, что для всякого $(a; b) \in \Sigma$ верно, $(a; b) \subseteq F$, а $a, b \in \Fr(F)$, следовательно $(a; b) \in \Lambda$. Поэтому $\Sigma \subseteq \Lambda$. Значит если у некоторого семейства открытых множеств границы равны границе $U$, то каждое множество из этого семейства есть объединение некоторого набора интервалов из $\Lambda$. т.е. их не более континуума.

        Покажем, что континуум достигается. Рассмотрим семейство $\Lambda$ интервалов, которые выкидываются из отрезка $[0; 1]$ при построении канторового множества (обозначим его за $C$): т.е. это $(1/3; 2/3)$, $(1/9; 2/9)$, $(7/9; 8/9)$, и т.д. Рассмотрим также последовательность $(I_n)_{n=0}^\infty := \bigl((1/3^{n+1}; 2/3^{n+1})\bigr)_{n=0}^\infty$ интервалов из $\Lambda$. Также определим $\Lambda' := \Lambda \setminus \{I_n\}_{n=0}^\infty$.

        Несложно видеть, что граница $F := \bigcup_{I \in \Lambda} I$ есть $C$, так как это множество, которое содержит все концы интервалов из $\Lambda'$, а каждая точка $C$ является пределом этих концов: каждая точка $C$ есть пересечение счётного числа отрезков, каждый из которых в $3$ раза меньше предыдущего, но один из концов каждого отрезка совпадает с концом некоторого интервала, следовательно каждая точка $C$ является пределом некоторых концов интервалов $\Lambda$.

        Теперь поймём, что граница $F' := \bigcup_{I \in \Lambda'} I$ есть тоже $C$. Если $x \in C$ является концом интервала из $\Lambda'$, то оно так же лежит на границе $F'$. Если $x \in C$ было такой точкой, что в любой её правой (левой) проколотой окрестности были границы интервалов $\Lambda$, а $x \neq 0$, то в некоторой правой (левой) проколотой окрестности $x$ множества $U$ и $U'$ (т.е. их пересечения с этой окрестностью совпадают), поэтому $x$ является пределом концов интервалов из $\Lambda'$, а значит лежит на границе $U'$. Заметим также, что концы каждого интервала $I_n$ также являются пределами других концов интервалов из $\Lambda$, поэтому они также лежат на границе $U'$. Осталось показать, что $0$ лежит на границе $U'$. Можно легко заметить, что для всякого $n \geqslant 0$ интервал $(7/3^{n + 2}; 8/3^{n+2})$ лежит в $\Lambda'$, следовательно точка $7/3^{n+2}$ является концом интервала из $\Lambda'$, а значит их предел и даст $0$. Таким образом $\Fr(U') = \Fr(U) = C$.

        Значит, если мы возьмём любое $\Sigma$, что $\Lambda' \subseteq \Sigma \subseteq \Lambda$, то граница $V := \bigcup_{I \in \Sigma} I$ так же совпадёт с $C$. При этом $\Lambda \setminus \Lambda'$ счётно, поэтому таких $\Sigma$ континуум, а значит и $V$ тоже континуум.
    \end{problem}
\end{document}