\documentclass[12pt,a4paper]{article}
\usepackage{solutions}
% \usepackage{float}
\usepackage{inkscape}

\title{Занятие от 11.03.\\Геометрия и топология. 1 курс.\\Решения.}
\author{Глеб Минаев @ 102 (20.Б02-мкн)}
% \date{}

\DeclareMathOperator{\Img}{Im}
\DeclareMathOperator{\Rea}{Rea}
\DeclareMathOperator{\Ker}{Ker}
\DeclareMathOperator{\Cl}{Cl}
\DeclareMathOperator{\Int}{Int}
\DeclareMathOperator{\Fr}{Fr}
\DeclareMathOperator{\Id}{Id}
\newcommand{\FAC}{\ensuremath{\mathrm{FAC}}\xspace}
\newcommand{\SAC}{\ensuremath{\mathrm{SAC}}\xspace}
\newcommand{\T}{\ensuremath{\mathrm{T}}\xspace}
\newcommand{\SL}{\ensuremath{\mathrm{SL}}\xspace}
\newcommand{\PGL}{\ensuremath{\mathrm{PGL}}\xspace}

\begin{document}
    \maketitle

    \begin{problem}{105}
        Рассмотрим картину происходящего со стороны векторных пространств, проективизациями которых и являются $X$, $Y$, $Z_1$ и $Z_2$.
        
        Нам даны векторное пространство $V_X$ и его векторные подпространства $V_Y$, $V_{Z_1}$ и $V_{Z_2}$, что $\dim(V_X) = n+1$, $\dim(V_Y) = k + 1$, $\dim(V_{Z_1}) = \dim(V_{Z_2}) = n - k = \dim(V_X) - \dim(V_Y)$ и $V_Y \cap V_{Z_1} = V_Y \cap V_{Z_2} = \{\overrightarrow{0}\}$. Тогда понятно, что
        \[V_X = V_Y \oplus V_{Z_1} = V_Y \oplus V_{Z_2}\]
        и следовательно $V_{Z_1} \simeq V_X / V_Y \simeq V_{Z_2}$, и при этом отображения
        \begin{align*}
            &g_1: V_{Z_1} \to V_X / V_Y, v \mapsto v + V_Y&
            &g_2: V_{Z_2} \to V_X / V_Y, v \mapsto v + V_Y
        \end{align*}
        являются изоморфизмами.

        Обозначим также канонические проекции $Z_1$ и $Z_2$ за $p_1$ и $p_2$ соответственно.

        Теперь перепишем определение $f$ на языке рассматриваемых векторных пространств. Для всякого $x \in Z_1$ рассматривается соответствующее одномерное подпространство $V_x := p_1(x)$, затем оболочка
        \[U_x := \langle V_x, V_Y \rangle\]
        и наконец $f(x)$ определяется как точка, соответствующая одномерному подпространству в пересечении $U_x$ и $V_{Z_2}$. На деле несложно видеть, что
        \[(g_2^{-1} \circ g_1)(v): V_{Z_1} \to v_{Z_2}, v \mapsto (v + V_Y) \cap V_{Z_2}\]
        и является изоморфизмом. Следовательно
        \[U_x \cap V_{Z^2} = g_2^{-1}(U_x) = g_2^{-1}(g_1(V_x))\]
        и является одномерным подпространством $V_{Z_2}$, так как является образом одномерного подпространства при изоморфизме. Следовательно мы просто имеем, что
        \[f = p_2^{-1} \circ g_2^{-1} \circ g_1 \circ p_1\]
        т.е. $f$ есть проективизация $g_2^{-1} \circ g_1$. Поэтому $f$ корректно определено и является проективным отображением.
    \end{problem}

    \begin{problem}{107}
        Заметим, что $\PGL(2, \CC)$ (она же группа автоморфизмов $\CC P^1$, и она же изоморфна $\SL(2, \CC)$) сохраняет ориентацию $\CC P^1$ как сферы. Действительно, всякие проективный автоморфизм можно представить в виде
        \[z \mapsto a + \frac{1}{bz + c}\]
        для некоторых $a, b, c \in \CC$, а соответственно в виде композиции преобразований видов $z \mapsto z + a$, $z \mapsto bz$ и $z \mapsto 1/z$. И действительно:
        \begin{itemize}
            \item $z \mapsto z + a$ --- просто параллельный перенос, поэтому ориентация очевидно сохраняется;
            \item $z \mapsto bz$ --- поворотная гомотетия относительно нуля, которая тоже очевидно сохраняет ориентацию;
            \item $z \mapsto 1/z$ --- инверсия в нуле вкупе с симметрией относительно вещественной оси; поскольку обе меняют ориентацию сферы, то их композиция ориентацию не меняет.
        \end{itemize}

        Следовательно всякий автоморфизм $\CC P^1$ сохраняет $H$ на месте тогда и только тогда, когда оставляет на месте $\RR P^1$ и не меняет её ориентации. Действительно:
        \begin{itemize}
            \item[$\Rightarrow$)] Если $H$ остаётся на месте, то $\RR P^1$ как граница $H$ на сфере остаётся на месте (понятно, что всякий автоморфизм $\CC P^1$ является непрерывным в смысле топологии сферы), а значит $\RR P^1$ перейдёт в себя. Да и то, что $H$ перешло в себя, а не в другую полусферу будет означать сохранение ориентации $\RR P^1$. Действительно, из того, что всякий автоморфизм сохраняет ориентацию $\CC P^1$, то понятно, что при сохранении ориентации $\RR P^1$ $H$ перейдёт в себя, а при смене --- в другую полусферу. Следовательно то, что $H$ осталась на месте значит, что $\RR P^1$ не поменяла ориентации.

            \item[$\Leftarrow$)] Если $\RR P^1$ переходит в себя, то $H$ и другая полусфера переходят либо в себя, либо в друг друга. И как мы выяснили, сохранение ориентации $\RR P^1$ отбрасывает второй случай, что значит, что $H$ переходит в себя. 
        \end{itemize}

        Несложно видеть, что если оператор переводит $\RR P^1$ в себя, то он лежит в $\PGL(2, \RR)$, т.е. можно считать, что имеет вещественные коэффициенты. Также рассматривая вместо $\RR P^1$ его образующее векторное пространство, получим, что сохранение ориентации $\RR P^1$ равносильно неотрицательности определителя. Но поскольку нам матрица оператора нужна с точностью до гомотетии, а определители гомотетий достигают всех положительных значений, то можно считать, что определитель равен $1$.

        Таким образом можно считать, что наша матрица лежит в $\SL(2, \RR)$.

        Это действительно сочетается с пунктом (1), а вот с пунктом (2) есть вопросы:
        \begin{itemize}
            \item Если в (2) просится рассмотреть матрицы из $\SL(2, \RR)$, то там получаются матрицы вида $(\begin{smallmatrix}a&b\\-b&a\end{smallmatrix})$, что не покрывает всё $\SL(2, \RR)$. Например, остаётся матрица $(\begin{smallmatrix}1&1\\0&1\end{smallmatrix})$.
            \item Если в (2) просится рассмотреть матрицы из $\SL(2, \CC)$, то там получаются матрицы, которые не переводят $H$ в себя. Например, матрица
                \[\begin{pmatrix}1&i\\\frac{1}{2i}&\frac{3}{2}\end{pmatrix})\]
                лежит в $\SL(2, \CC)$, оставляет $i$ на месте, но $0$ отправляет в $\frac{2i}{3}$, т.е. не переводит $\RR P^1$ в себя, а значит не переводит и $H$ в себя.
        \end{itemize}
        Поэтому непонятно, как требуется сравнить группы в (2) и (3).
    \end{problem}
\end{document}