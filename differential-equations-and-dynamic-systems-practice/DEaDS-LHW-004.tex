\documentclass[12pt,a4paper]{article}
\usepackage{solutions}
\usepackage{float}

\title{Теоретическое домашнее задание от 24.09\\Дифференциальные уравнения и динамические системы}
\author{Глеб Минаев @ 204 (20.Б04-мкн)}
% \date{}

\newcommand{\dom}{\mathrm{dom}}

\begin{document}
    \maketitle
    
    \begin{problem}{4}
        Поскольку $(x_0, y_0)$ --- точка неединственности, то есть два решения $\gamma_1$ и $\gamma_2$ задачи Коши с этими начальными данными, определённые в некоторой окрестности $x_0$ и отличные в любой окрестности $x_0$. WLOG $\gamma_1$ и $\gamma_2$ отличны в любой правой окрестности $x_0$. Тогда есть некоторая точка $(x_1, y_1)$, что $x_1 > x_0$
        \[(\gamma_1(x_1) - y_1)(\gamma_2(x_1) - y_1) < 0,\]
        т.е. $y_1$ находится строго между $\gamma_1(x_1)$ и $\gamma_2(x_1)$.

        Докажим, что есть решение той же задачи Коши на отрезке $[x_0; x_1]$. Для этого рассмотрим множество $S$ решений $\alpha$ уравнения $y' = f(x, y)$ на некотором интервале $(t; x_1]$ ($x_0 \leqslant t < x_1$; $t$ у каждого $\alpha$ своё), проходящих через $(x_1, y_1)$ и лежащих нестрого между $\gamma_1$ и $\gamma_2$. Введём на $S$ отношение $\preccurlyeq$ по правилу
        \[\alpha \preccurlyeq \beta \qquad \Longleftrightarrow \qquad \dom(\alpha) \subseteq \dom(\beta) \wedge \alpha = \beta|_{\dom(\alpha)},\]
        где $\dom(f)$ --- область определения функции $f$. Пусть в данном ЧУМ имеется некоторая цепь $\Sigma$. Тогда для всякой точки $t \in [x_0; x_1]$ верно, что $\alpha(t)$ для любой $\alpha \in \Sigma$, что $t \in \dom(\alpha)$, не зависит от $\alpha$. Действительно, если $\alpha_1, \alpha_2 \in \Sigma$ и $t \in \alpha_1$, $t \in \alpha_2$, то $\alpha_1$ и $\alpha_2$ сравнимы (WLOG $\alpha_1 \preccurlyeq \alpha_2$), а тогда
        \[\alpha_2(t) = (\alpha_2|_{\dom(\alpha_1)}) (t) = \alpha_1(t).\]
        Таким образом можно взять ``объединение'' функций $\Sigma$: рассмотреть функцию
        \[\widetilde{\alpha}: \bigcup_{\alpha \in \Sigma} \dom(\alpha) \to \RR, t \mapsto \alpha(t) \text{ для любого $\alpha \in \Sigma$, определённой на $t$}\]
        (с точки зрения теории множеств это будет буквальным объединением функций). Очевидно $\widetilde{\alpha}$ --- верхняя граница $\Sigma$. Это значит, что построенный ЧУМ удовлетворяет условию леммы Цорна. Следовательно есть максимальный элемент.

        Также заметим, что если имеется некоторое решение $\alpha: (t; x_1] \to \RR$, лежащее между $\gamma_1$ и $\gamma_2$, то $|\alpha'|$ ограничен, так как область
        \[G := \{(x; y) \mid x \in [x_0; x_1] \wedge y \in [\gamma_1(x); \gamma_2(x)]\}\]
        ограничена, а значит $f$ на ней ограничена. Следовательно
        \[\lim_{p \to t^+} \alpha(p)\]
        определён, так как иначе по теореме Лагранжа о среднем значении в окрестности $t$ будут достигаться сколь угодно большие по модулю значения $\alpha'$. Следовательно
        \[\lim_{p \to t^+} \alpha'(p) = \lim_{p \to t^+} f(p, \alpha(p)) = f(t, \alpha(t)),\]
        а тогда по известной теореме $\alpha$ и $\alpha'$ доопределяются на $[t; x_1]$, оставляя $\alpha$ решением рассматриваемого уравнения.

        Ещё заметим, что если $\alpha: [t; x_1]$ ($x_0 < t < x_1$) есть решение, лежащее между $\gamma_1$ и $\gamma_2$, что $\alpha(t) = \gamma_1(t)$, то склеив $\gamma_1|_{[x_0; t]}$ и $\alpha|_{[t; x_1]}$ получаем решение того же уравнения; при этом если $\alpha$ проходило через $(x_1; y_1)$, то мы получили корень уравнения на $[x_0; x_1]$, проходящий чрез $(x_0; y_0)$ и $(x_1; y_1)$. Если $\alpha(t)$ не совпадает с $\gamma_1(t)$ и $\gamma_2(t)$, то есть решение задачи Коши в точке $(t; \alpha(t))$, а тогда склеивая его часть до $t$ и $\alpha$, получаем небольшое продолжение $\alpha$ влево по прямой, лежащее в $S$.

        Таким образом максимальный элемент в $S$ не может быть определён на $(t; x_1]$, где $t > x_0$, так как тогда его можно замкнуть и продолжить немного левее. Значит максимальный элемент определён на $(x_0; x_1]$. Замыкая его, мы получаем решение уравнения на $[x_0; x_1]$, проходящее через $(x_0; y_0)$ и $(x_1; y_1)$.
        
        Таким образом мы получили для всякого $y_1$ решение $\alpha$ на $[x_0; x_1]$, что $\alpha(x_1) = y_1$. Поскольку $y_1 \in (\gamma_1(x_1); \gamma_2(x_1))$, то имеется континуальное количество таких решений. Чтобы каждое из этих решений превратить в решение задачи Коши, их все нужно доопределить немного левее $x_0$; это можно несложно сделать, если склеить каждую из функций с частью $\gamma_1$ до $x_0$.
    \end{problem}
\end{document}