\documentclass[12pt,a4paper]{article}
\usepackage{solutions}
\usepackage{float}

\title{Теоретическое ДЗ\\Дифференциальные уравнения и динамические системы}
\author{Глеб Минаев @ 204 (20.Б04-мкн)}
% \date{}

\begin{document}
    \maketitle
    
    \begin{problem*}[№1]
        Пусть $y: (a; b) \to \RR$ --- некоторое решение дифференциального уравнения $y' = f(y)$.

        \begin{lemma}
            Пусть имеются точки $x_1, x_2 \in (a; b)$, где $y(x_1) = y(x_2) = \alpha$. Если $f(\alpha) \neq 0$, то есть точка $x_3 \in (x_1; x_2)$, что $y(x_3) = \alpha$. 
        \end{lemma}

        \begin{proof}
            WLOG $f(\alpha) > 0$, $x_1 < x_2$. Тогда
            \[y'(x_0) = y'(x_1) = f(\alpha) > 0,\]
            а значит $x_0$ и $x_1$ имеют некоторые окрестности $I_1$ и $I_2$, где
            \[
                \forall x \in I_1 \quad \frac{y(x) - y(x_1)}{x - x_1} \in (\frac{1}{2} \alpha; \frac{3}{2} \alpha)
                \qquad \text{ и } \qquad
                \forall x \in I_2 \quad \frac{y(x) - y(x_2)}{x - x_2} \in (\frac{1}{2} \alpha; \frac{3}{2} \alpha).
            \]
            Тогда выберем любые точки $t_1 \in I_1$ и $t_2 \in I_2$, что $x_1 < t_1 < t_2 < x_2$. В таком случае
            \[y(t_1) > y(x_1) = \alpha \qquad \text{ и } \qquad y(t_2) < y(x_2) = \alpha.\]
            Тогда по теореме о промежуточном значении есть точка $x_3 \in (t_1; t_2)$, где $y(x_3) = \alpha$.
        \end{proof}

        \begin{lemma}
            Пусть имеются точки $x_1, x_2 \in (a; b)$, где $y(x_1) = y(x_2) = \alpha$. Тогда $f(\alpha) = 0$.
        \end{lemma}
        

        \begin{proof}
            Предположим противное. Тогда $f(\alpha) \neq 0$. Значит между $x_1$ и $x_2$ есть точка $x_3$, где $y$ имеет то же значение $\alpha$. Повторяя операцию для отрезков $[x_1; x_3]$ и $[x_3; x_2]$, получаем ещё 2 такие же точки и т.д.: повторяя такую операцию счётное число раз, получаем, что на отрезке $[x_1; x_2]$ есть счётное число точек, где $y$ принимает значение $\alpha$. Это значит, что есть какая-то последовательность точек $(t_i)_{i=0}^\infty$, сходящаяся к некоторой точке $t$, где $y(t_i) = \alpha$. Значит $y(t) = \alpha$ по непрерывности $y$. При этом $y'(t) = 0$, так как
            \[\lim_{i \to \infty} \frac{y(t_i) - y(t)}{t_i - t} = \lim_{i \to \infty} 0 = 0.\]
            Значит
            \[f(\alpha) = f(y(t)) = y'(t) = 0\]
            --- противоречие. Следовательно $f(\alpha) = 0$ с самого начала.
        \end{proof}

        \begin{corollary}
            Если какое-то значение принимается дважды, то во всех точках, где оно принимается, производная $y'$ зануляется.
        \end{corollary}

        \begin{lemma}
            Пусть имеются точки $x_1, x_2 \in (a; b)$, где $y(x_1) = y(x_2) = \alpha$. Тогда $y \equiv \alpha$ на $(x_1; x_2)$.
        \end{lemma}

        \begin{proof}
            Пусть $S = \sup_{[x_1; x_2]} y$ и $I = \inf_{[x_1; x_2]} y$. Поскольку $y$ непрерывна, а $[x_1; x_2]$ компактен, то $S$ является не просто супремумом, а максимумом и принимается в некоторой точке $t_S \in [x_1; x_2]$; аналогично для $I$. WLOG $x_1 \leqslant t_S \leqslant t_I \leqslant x_2$. Тогда множество значений $y$ на $[x_1; x_2]$ есть $[I; S]$. При этом каждое значение из $(\alpha; S)$ принимается на $(x_1; t_S)$ и $(t_S; t_I)$, каждое значение из $(I; \alpha)$ --- на $(t_S; t_I)$ и $(t_I; x_2)$.
            
            Значит всякое значение из $(S; I) \setminus \{\alpha\}$ принимается дважды на $(x_1; x_2)$, а значит во всех точках, где оно принимается $y' = 0$. Аналогично можно сказать про $\alpha$, так как оно принимается в $x_1$ и $x_2$, и $S$ и $I$, так как они являются супремумом и инфимумом, а производная в экстремальных точках равна $0$.
            
            Таким образом $y' \equiv 0$ на $(x_1; x_2)$, откуда следует, что $y \equiv \alpha$ на $(x_1; x_2)$.
        \end{proof}

        Теперь мы можем показать монотонность $y$ на $(a; b)$. Немонотонность $y$ означает, что есть какие-то точки $x_1, x_2, x_3 \in (a; b)$, что WLOG
        \[y(x_1) < y(x_2) > y(x_3).\]
        Но это значит, что есть некоторые точки $t_1 \in (x_1; x_2)$ и $t_2 \in (x_2; x_3)$, где $y$ принимает одно и то же значение, меньшее $y(x_2)$. Тогда мы получаем противоречие с последней леммой. Значит $y$ (нестрого) монотонна.
    \end{problem*}
\end{document}