\documentclass[12pt,a4paper]{article}
\usepackage{solutions}
\usepackage{float}

\title{Рейтинговое домашнее задание от 24.09\\Дифференциальные уравнения и динамические системы}
\author{Глеб Минаев @ 204 (20.Б04-мкн)}
\date{}

\newcommand{\const}{\mathrm{const}}

\begin{document}
    \maketitle

    \begin{problem}{3}
        Сначала давайте просто решим решим уравнение.
        \[y' - y = - \frac{1}{x}\]
        Введём функцию
        \[\varphi(x) := e^{\int_0^x -1 dt} = e^{-x}.\]
        Следовательно,
        \begin{gather*}
            \varphi y' - \varphi y = - \frac{\varphi}{x}\\
            \varphi y' + \varphi' y = - \frac{\varphi}{x}\\
            (\varphi y)' = - \frac{\varphi}{x}\\
            \varphi y = \int -\frac{\varphi(t)}{t} dt = \int_1^{\infty} \frac{\varphi(t)}{t} dt - \int_1^x \frac{\varphi(t)}{t} dt + C = \int_x^{\infty} \frac{\varphi(t)}{t} dt + C\\
            y = e^x \left(\int_x^{\infty} \frac{\varphi(t)}{t} dt + C\right) = \int_x^{\infty} \frac{e^{x - t}}{t} dt + C e^x
        \end{gather*}
        Тогда
        \[\lim_{x \to \infty} \int_x^{\infty} \frac{e^{x - t}}{t} dt = 0,\]
        а
        \[
            \lim_{x \to \infty} C e^x =
            \begin{cases}
                \pm \infty& \text{ если } C \neq 0,\\
                0& \text{ если } C = 0.
            \end{cases}
        \]
        Таким образом понятно, что $\lim_{x \to \infty} y(x) = 0$ тогда и только тогда, когда $C = 0$. Т.е. единственное решение, обладающее оговоренным в условии свойством, есть функция
        \[y(x) = \int_x^{\infty} \frac{e^{x - t}}{t} dt.\]
    \end{problem}
\end{document}