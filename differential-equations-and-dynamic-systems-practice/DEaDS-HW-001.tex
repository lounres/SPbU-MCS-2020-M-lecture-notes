\documentclass[12pt,a4paper]{article}
\usepackage{solutions}
\usepackage{float}

\title{Домашнее задание от 07.09\\Дифференциальные уравнения и динамические системы}
\author{Глеб Минаев @ 204 (20.Б04-мкн)}
% \date{}

\begin{document}
    \maketitle
    
    \begin{problem*}[17]
        Заметим, что
        \[
            y = e^{Cx}
            \qquad \Longleftrightarrow \qquad
            \ln(y) = Cx
            \qquad \Longrightarrow \qquad
            \Bigl(\ln(y)\Bigr)' = C.
        \]
        Следовательно все линии данного семейства удовлетворяют уравнению
        \[\ln(y) = \Bigl(\ln(y)\Bigr)' x.\]
        Другая форма этого уравнения: $x y' = y \ln(y)$.
    \end{problem*}

    \begin{problem*}[27]
        Из нашего равенства следует, что
        \[\frac{y'}{y} = a + by' \qquad \text{ и } \qquad \frac{y''y - {y'}^2}{y^2} = by''.\]
        Из этих равенств мы получаем, что
        \[b = \frac{y''y - {y'}^2}{y^2 y''} \qquad \text{ и } \qquad a = \frac{y'}{y} - by' = \frac{yy'y'' - yy'y'' + {y'}^2}{y^2y''} = \frac{{y'}^2}{y^2 y''}.\]
        Подставляя в изначальное равенство, получаем
        \[\ln(y) = \frac{x {y'}^2 + y^2 y'' - y {y'}^2}{y^2 y''}.\]
    \end{problem*}

    \begin{problem*}[32]
        Скажем последнее условие на окружности немного по-другому: их центры лежат на прямой $y = x$. Таким образом мы получаем семейство кривых, заданных уравнениями
        \[(y - a)^2 + (x - a)^2 = a^2.\]
        Дифференцируя, получаем
        \[y'(y-a) + (x-a) = 0.\]
        Отсюда выражаем $a$:
        \[a = \frac{y'y + x}{y' + 1}.\]
        Подставляя в начальное равенство, имеем
        \begin{align*}
            \left(y - \frac{yy' + x}{y' + 1}\right)^2 + \left(x - \frac{yy' + x}{y' + 1}\right)^2 = \left(\frac{yy' + x}{y' + 1}\right)^2
        \end{align*}
    \end{problem*}

    \begin{problem*}[40]
        Две кривые пересекаются под углом $\varphi$ значит, что угол между касательными к ним в их общей точке равен $\varphi$. Пусть $u$ --- вектор, лежащий на одной касательной, а $v$ --- на другой. В таком случае условие на угол выглядит как
        \[u \cdot v = |u| |v| |\cos(\varphi)| \qquad \text{ или по-другому } \qquad (u \cdot v)^2 = |u|^2 |v|^2 \cos(\varphi)^2.\]
        Несложно видеть, что вектор касательной нашей кривой есть ничто иное как $(1; y')$. При этом данное семейство кривых является линиями уровней функции $x^2 + y^2$, градиент которой --- $(x; y)$; следовательно касательная --- перпендикулярный к градиенту вектор $(-y; x)$ Тогда имеем, что
        \[(-y + xy')^2 = (x^2 + y^2) (1 + {y'}^2) \frac{1}{2}.\]
    \end{problem*}
\end{document}