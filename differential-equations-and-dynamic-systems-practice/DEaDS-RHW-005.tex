\documentclass[12pt,a4paper]{article}
\usepackage{solutions}
\usepackage{float}

\title{Рейтинговое домашнее задание от 08.10\\Дифференциальные уравнения и динамические системы}
\author{Глеб Минаев @ 204 (20.Б04-мкн)}
\date{}

\begin{document}
    \maketitle

    \begin{problem}{5}
        Покажем, что производная данного двойного отношения равна нулю.
        \begin{gather*}
            \left(\frac{y_4 - y_2}{y_4 - y_1} : \frac{y_3 - y_2}{y_3 - y_1}\right)'
            = \left(\frac{(y_4 - y_2)(y_3 - y_1)}{(y_4 - y_1)(y_3 - y_2)}\right)'\\
            = \frac{((y_4 - y_2)(y_3 - y_1))' (y_4 - y_1)(y_3 - y_2) - (y_4 - y_2)(y_3 - y_1) ((y_4 - y_1)(y_3 - y_2))'}{(y_4 - y_1)^2(y_3 - y_2)^2}.
        \end{gather*}
        Заметим, что
        \[
            (y_i - y_j)'
            = p (y_i^2 - y_j^2) + q (y_i - y_j)
            = (y_i - y_j) (p(y_i + y_j) + q).
        \]
        Тогда
        \begin{gather*}
            E_{i, j, k, l} := ((y_i - y_k)(y_j - y_l))' (y_i - y_l)(y_j - y_k)\\
            = (y_i - y_k)' (y_j - y_l)(y_i - y_l)(y_j - y_k) + (y_i - y_k)(y_j - y_l)' (y_i - y_l)(y_j - y_k)\\
            = (y_i - y_k)(y_j - y_l)(y_i - y_l)(y_j - y_k)(p(y_i + y_k) + q + p(y_j + y_l) + q)\\
            = (y_i - y_k)(y_j - y_l)(y_i - y_l)(y_j - y_k)(p(y_i + y_j + y_k + y_l) + 2q),
        \end{gather*}
        и значит $E_{i, j, k, l} = E_{j, i, k, l}$. Тем самым
        \begin{gather*}
            \left(\frac{y_4 - y_2}{y_4 - y_1} : \frac{y_3 - y_2}{y_3 - y_1}\right)'\\
            = \frac{E_{4, 3, 2, 1} - E_{3, 4, 2, 1}}{(y_4 - y_1)^2(y_3 - y_2)^2}\\
            = 0.
        \end{gather*}

        Равенство производной данного двойного отношения нулю означает константность данного двойного отношения.
    \end{problem}
\end{document}