\documentclass[12pt,a4paper]{article}
\usepackage{math-text}

\title{Алгебра.\footnote{\textcolor{red}{\bfseries Этот текст содержит критические недочёты (см. с. \pageref{TODO-list}). Вы можете помочь исправить их, написав мне \href{mailto:minaevgleb@yandex.ru}{по почте} (minaevgleb@yandex.ru) или \href{https://vk.com/lounres}{в ВКонтакте} (@lounres).}}
}
\author{В. А. Петров\\lektorium.tv}
\date{}

\usepackage{todonotes}
\setlength{\marginparwidth}{2.5cm}
\setlength{\textwidth}{470 pt}

\begin{document}
    \maketitle

    Зарождение --- Аль Хорезин, ``Китхаб Альджебр валь мукабалт''. ``Альджебр'' значит ``перенос из одной части уравнения в другую'', а ``мукабалт'' --- ``приведение подобных''.

    Литература:
    \begin{itemize}
        \item Ван дер Варден ``Алгебра''
        \item Лэнг ``Алгебра''
        \item Винберг ``Курс Алгебры''
    \end{itemize}

    \begin{definition}
        Алгебраическая структура --- это множество $M$ + заданные на нём операции + аксиомы на операциях.
    \end{definition}

    \begin{definition}
        Абелева группа --- набор $(M, +: M^2 \to M, 0 \in M)$ с аксиомами:
        \begin{description}
            \item[$A_1$)] $\forall a, b, c \in M: (a + b) + c = a + (b + c)$ --- ассоциативность сложения
            \item[$A_2$)] $\forall a \in M: a + 0 = a = 0 + a$ --- нейтральный по сложению элемент
            \item[$A_3$)] $\forall a, b \in M: a + b = b + a$ --- коммутативность сложения
            \item[$A_4$)] $\forall a \in M: \exists -a: a + (-a) = 0 = (-a) + a$ --- существование противоположного
        \end{description}
    \end{definition}

    \begin{definition} Опишем следующие аксиомы на наборе $(M, +: M^2 \to M, \cdot: M^2 \to M, 0 \in M, 1 \in M)$:
        \begin{description}
            \item[$D$)] $\forall a, b, k \in M: k(a + b) = ka + kb$, $(a + b)k = ak + bk$ --- дистрибутивность
            \item[$M_1$)] $\forall a, b, c \in M: (a \cdot b) \cdot c = a \cdot (b \cdot c)$ --- ассоциативность умножения
            \item[$M_2$)] $\forall a \in M: a \cdot 1 = a = 1 \cdot a$ --- нейтральный по умножению элемент
            \item[$M_3$)] $\forall a, b \in M: a \cdot b = b \cdot a$ --- коммутативность умножения
            \item[$M_4$)] $\forall a \in M \setminus \{0\}: \exists\, a^{-1}: a \cdot a^{-1} = 1 = a^{-1} \cdot a$ --- существование обратного
        \end{description}
        По этим аксиомам определим следующие понятия:
        \begin{description}
            \item[\emph{Кольцо}] --- набор $(M, +, \cdot, 0)$, что верны $A_1$, $A_2$, $A_3$, $A_4$ и $D$.
            \item[\emph{Ассоциативное кольцо}] --- кольцо с $M_1$.
            \item[\emph{Кольцо с единицей}] --- кольцо с $M_2$.
            \item[\emph{Тело}] --- кольцо с $M_1$, $M_2$.
            \item[\emph{Поле}] --- кольцо с $M_1$, $M_2$, $M_3$, $M_4$.
            \item[\emph{Полукольцо}] --- кольцо без $A_4$.
        \end{description}
    \end{definition}

    \begin{example}
        Если взять $\RR^3$, то векторное произведение в нём неассоциативно и антикоммутативно. Но есть
        \begin{lemma*}[Тождество Якоби]
            $u\times (v\times w) + v\times (w\times u) + w\times (u\times v) = 0$
        \end{lemma*}
    \end{example}

    \begin{example}
        Если взять $R^4 = R \times R^3$ и рассмотреть $\cdot: ((a; u); (b; v)) \mapsto (ab-u\cdot v; av + bu + u\times v)$ и $+: ((a; u); (b; v)) \mapsto (a + b, u + v)$, тогда получим $\HH$ --- ассоциативное некоммутативное тело кватернионов. Ассоциативность доказал Гамильтон.
    \end{example}

    \begin{lemma*}
        $0 \cdot a = 0$
    \end{lemma*}

    \begin{definition}
        Коммутативное кольцо без делителей нуля называетсся \emph{областью (целостности)}.
    \end{definition}

    \begin{definition}
        Пусть $m\in\NN$. Тогда \emph{множество остатков при делении на $m$} или $\ZZ/m\ZZ$ --- это фактор-множество по отношению эквивалентности $a\sim b \Leftrightarrow (a-b) \mid m$.
    \end{definition}

    \begin{definition}
        \emph{Подкольцо} --- это подмножество кольца, согласованное с его операциями.

        Как следствие ноль и обратимость соглассуются автоматически.
    \end{definition}

    \begin{statement}
        Если $R$ --- подкольцо области целостности $S$, то $R$ --- область целостности.
    \end{statement}

    \begin{definition}
        \emph{Целые Гауссовы числа} или $\ZZ{}[i]$ --- это $\{a + bi \mid a, b \in \ZZ\}$.
    \end{definition}

    \begin{definition}
        Некоторое подмножество $R$ кольца $S$ \emph{замкнуто относительно сложения (умножения)}, если $\forall a, b \in R: a + b \in R$ ($ab \in R$ соответственно).
    \end{definition}

    \begin{remark}
        Замкнутое относительно сложения \textbf{И} умножения подмножество --- подкольцо.
    \end{remark}

    \begin{example}
        Пусть $d$ --- целое, не квадрат. Тогда $\ZZ{}[\sqrt{d}]$ --- область целостности.
    \end{example}

    \section{Теория делимости}

    Пусть $R$ --- область целостности.

    \begin{definition}
        ``$a$ \emph{делит} $b$'' или же $a \mid b$ значит, что $\exists c \in R: b = ac$.
    \end{definition}

    \begin{statement}
        Отношение ``$\mid$'' рефлексивно и транзитивно.
    \end{statement}

    \begin{definition}
        $a$ и $b$ \emph{ассоциативны}, если $a \mid b$ и $b \mid a$. Обозначение: $a \sim b$.
    \end{definition}

    \begin{statement}
        ``$\sim$'' --- отношение эквивалентности.
    \end{statement}

    \begin{statement}
        $a \sim b \Leftrightarrow \exists \text{ обратимый } \varepsilon: a = \varepsilon b$.
    \end{statement}

    \begin{proof}
        Пусть $a \sim b$. Тогда $\exists c, d: ac = b, bd = a$. Тогда $a(1-cd) = a - acd = a - bd = a - a = 0$, значит либо $a = 0$, либо $cd = 1$. В первом случае $b = ac = 0c = 0$, значит можно просто взять $\varepsilon = 1$. Во втором случае, $cd = 1$, значит $c$ и $d$ обратимы, тогда можно взять $\varepsilon = d$. следствие в одну сторону доказано.

        Пусть $a = \varepsilon b$, где $\varepsilon$ обратим. Значит:
        \begin{enumerate}
            \item $b\mid a$;
            \item $\exists \delta: \delta\varepsilon = 1$, значит $\delta a = \delta \varepsilon b = b$, значит $a \mid b$.
        \end{enumerate}
        Таким образом $a \sim b$.
    \end{proof}

    \begin{example}
        В $\ZZ{}[i]$ есть только следующие обратимые элементы: $1$, $-1$, $i$ и $-i$. Поэтому все ассоциативные элементы получаются друг из друга домножением на один из $1$, $-1$, $i$, $-i$ и вместе образуют квадрат (на \href{https://ru.wikipedia.org/wiki/%D0%9A%D0%BE%D0%BC%D0%BF%D0%BB%D0%B5%D0%BA%D1%81%D0%BD%D0%B0%D1%8F_%D0%BF%D0%BB%D0%BE%D1%81%D0%BA%D0%BE%D1%81%D1%82%D1%8C}{комплексной плоскоти}) с центром в нуле.
    \end{example}

    \begin{definition}
        \emph{Главным идеалом} элемента $a$ называется множество $M := \{ak \mid k \in R\} = \{b \mid a \text{ делит } b\}$. Обозначение: $(a)$ или $aR$.
    \end{definition}

    \begin{statement}
        $a \mid b \Leftrightarrow b \in aR \Leftrightarrow bR \subseteq aR$. 
    \end{statement}

    \begin{statement}
        $a \sim b \Leftrightarrow aR = bR$.
    \end{statement}

    \begin{statement}$\forall a \in R$
        \begin{enumerate}
            \item $0 \in aR$
            \item $x \in aR \Rightarrow -x \in aR$
            \item $x, y \in aR \Rightarrow x + y \in aR$
            \item $x \in aR, r \in R \Rightarrow xr \in aR$
        \end{enumerate}
    \end{statement}

    \begin{remark}
        То же верно и в некоммутативном $R$.
    \end{remark}

    \begin{example}
        В поле есть только $0R$ и $1R$.
    \end{example}

    \begin{example}
        В $\ZZ$ есть только $m\ZZ$ для каждого $m \in \NN \cup \{0\}$.
    \end{example}

    \begin{definition}
        Пусть $P$ --- кольцо. $I\subseteq P$ называется \emph{правым идеалом}, если
        \begin{enumerate}
            \item $0 \in I$;
            \item $a, b \in I \Rightarrow a + b \in I$;
            \item $a \in I \Rightarrow -a \in I$;
            \item $a \in I, r \in R \Rightarrow ar \in I$.
        \end{enumerate}
        $I$ называется \emph{левым идеалом}, ессли аксиому 4 заменить на ``$a \in I, r \in R \Rightarrow ra \in I$''. Также $I$ называется \emph{двухсторонним идеалом}, если является левым и правым идеалом, и обозначается как $I \triangleleft P$.
    \end{definition}

    \begin{remark}
        В коммутативном кольце (и в частности в области целостности) все идеалы двухсторонние.
    \end{remark}

    \begin{example}
        Пусть дано кольцо $P$ и фиксированы $a_1, \dots, a_n \in P$. Тогда $a_1 P + \dots + a_n P = \{a_1 x_1 + \dots + a_n x_n \mid x_1, \dots, x_n \in P\}$ есть правый (конечнопорождённый) идеал, попрождённый элементами $a_1, \dots, a_n$. Аналогично $P a_1 + \dots + P a_n = \{x_1 a_1 + \dots + x_n a_n \mid x_1, \dots, x_n \in P\}$ --- левый (конечнопорождённый) идеал, попрождённый элементами $a_1, \dots, a_n$.
    \end{example}

    \begin{definition}
        \emph{Область главных идеалов} --- область целостности, где все идеалы главные.
    \end{definition}

    \begin{definition}
        Область целостности $R$ называется \emph{Евклидовой}, если существует функция (``Евклидова норма'') $N: R\setminus\{0\} \to \NN$, что
        \[\forall a, b \neq 0\; \exists q, r: a = bq + r \wedge (r = 0 \vee N(r) < N(b))\]
    \end{definition}

    \begin{theorem}
        Евклидово кольцо --- область главных идеалов.
    \end{theorem}

    \begin{proof}
        Пусть наше кольцо --- $R$. Если $I = \{0\}$, то $I = 0R$. Иначе возьмём $d \in I \setminus \{0\}$ с минимальной Евклидовой нормой. Тогда $\forall a \in I$ либо $d \mid a$, либо $\exists q, r: a = dq - r$. Во втором случае $dq \in I$, $r = a - dq \in I$, но $N(r) < N(d)$ --- противоречие. Значит $I = dR$.
    \end{proof}

    \begin{definition}
        \emph{Общим делителем} $a$ и $b$ называется $c$, что $c \mid a$ и $c \mid b$. \emph{Наибольшим общим делителем (НОД)} $a$ и $b$ называется общий делитель $a$ и $b$, делящийся на все другие общие делители $a$ и $b$.
    \end{definition}

    \begin{theorem}[алгоритм Евклида]
        В Евклидовом кольце у любых двух чисел есть НОД.
    \end{theorem}

    \begin{proof}
        Заметим, что $(a, b) = (a + bk, b)$.
        
        Пусть даны $a$ и $b$. Предположим, что $\phi(a) \geqslant \phi(b)$, иначе поменяем их местами. Тем самым по аксиоме Евклида найдутся $q$ и $r$, что $a = bq + r$, а $\phi(r) < \phi(b) \leqslant \phi(a)$, значит $\phi(a) + \phi(b) > \phi(r) + \phi(b)$. При этом $(a, b) = (r, b)$. Значит бесконечно $\phi(a) + \phi(b)$ не может бесконечнго уменьшаться, так как натурально, значит за конечное кол-во переходов мы получим, что одно из чисел делит другое, а значит НОД стал определён.
    \end{proof}

    \begin{theorem}[линейное представление НОД]
        $\forall a, b \in R\; \exists p, q \in R: ap + bq = (a, b)$.
    \end{theorem}

    \begin{proof}
        Докажем по индукции по $N(a) + N(b)$.

        \textbf{База.} $N(a) + N(b)=0$. Значит $N(a)=N(b)=0$, а тогда $a$ и $b$ не могут не делиться друг на друга, значит НОД --- любой из них. А в этом случае разложение очевидно.
        
        \textbf{Шаг.} WLOG $N(a) \geqslant N(b)$. Если $b \mid a$, то $b$ --- НОД, а тогда разложение очевидно. Иначе по аксиоме Евклида $\exists q, r: a = bq + r$. Заметим, что $(a, b) = (b, r) = d$, но $N(a) + N(b) \geqslant N(b) + N(b) > N(b) + N(r)$. Таким образом по предположению индукции для $b$ и $r$ получаем, что $d = bk + rl$ для некоторых $k$ и $l$, значит $d = bk + (a - bq)l = al + b(k - ql)$.
    \end{proof}

    \begin{definition}
        Элемент $p$ области целостности $R$ назвыается \emph{неприводимым}, если $\forall d \mid p$ либо $d \sim 1$, либо $d \sim p$.
    \end{definition}
    
    \begin{definition}
        Элемент $p$ области целостности $R$ назвыается \emph{простым}, если из условия $p \mid ab$ следует, что $p \mid a$ или $p \mid b$.
    \end{definition}

    \begin{statement}
        Любое простое неприводимо.
    \end{statement}

    \begin{proof}
        Предположим противное, т.е. некоторое простое $p$ представляется в виде произведения неделителей единицы $a$ и $b$. Тогда WLOG $p \mid a$. Значит $p\sim a$, а $b \sim 1$ --- противоречие.
    \end{proof}

    \begin{statement}
        В области главных идеалов неприводимые просты.
    \end{statement}

    \begin{proof}
        Пусть неприводимое $p$ делит $ab$. Пусть тогда $pR + aR = dR$. В таком случае $d \sim p$, значит либо $d \sim p$, либо $d \sim 1$. Если $d \sim p$, то $p \mid a$. Иначе $px + ay = 1$, значит $pxb + aby = b$. Но $p \mid pxb$ и $p \mid aby$, значит $p \mid b$. Поскольку рассуждение не зависит от $a$ и $b$, то $p$ просто.  
    \end{proof}

    \todo[inline]{Проверить начиная с этого момента до конца раздела формальные моменты и довести доказательства.}
    \begin{definition}
        Назовём разложением числа $a$ на неприводимые множители предстваление его в виде произведения $\alpha \prod_{i=1}^n p_i$ с точностью до перестановки членов произведения и домножения их на делители $1$ (т.е. их ассоциативности), где все $p_i$ просты.

        Область целостности называется \emph{факториальным кольцом}, если в нём все ненулевые элементы разложимы на неприводимые \todo{Точно ли \emph{неприводимые}? Или всё-таки \emph{простые}?} множители единственным образом (также говорят, что в области целостности верна \emph{основная теорема арифметики}).
    \end{definition}

    \begin{theorem}
        Область главныых идеалов удовлетворяет условию обрыва цепи главныых идеалов.
    \end{theorem}

    \begin{proof}
        Пусть наша область --- $R$. Сначала докажем существование разложения.

        Предположим противное, т.е. существует последовательность $\{a_n\}_{n=0}^\infty$, что $a_{n+1}$ --- собственный делитель $a_n$ (т.е. $a_{n+1} \mid a_n \wedge a_n \nsim a_{n+1}$). Тогда $a_0 R \subsetneq a_1 R \subsetneq a_2 R \subsetneq \dots$. Тогда $\exists x: xR = \bigcup_{n=0}^\infty a_n R$, так как  это объединение --- идеал. Но тогда $x \in a_j R$ для некоторого $j$, а значит $x R \subseteq a_j R$, а тогда $a_{j+1} R \subseteq a_j R$ --- противоречие.
    \end{proof}

    \begin{theorem}
        Пусть $R$ --- область целостности. Тогда следующие утверждения равносильны.
        \begin{enumerate}
            \item $R$ факториально.
            \item $R$ удовлетворяет условию обрыва возрастающей цепи главных идеалов.
        \end{enumerate}
    \end{theorem}

    \begin{proof}
        Сначала докажем в одну сторону. Пусть $R$ факториально.

        \begin{lemma}\todo{Тонкий момент. Проверить.}
            Любой неприводимый в $R$ прост.
        \end{lemma}

        \begin{proof}
            Пусть $p \mid ab$ для любого неприводимого $p$ и каких-то $a$ и $b$. Тогда имеем $px=ab$ для некоторых $x$, а значит по факториальности, что $a = p_1 \dots p_k$, $b = q_1 \dots q_l$, $x = r_1 \dots r_m$. Тогда имеем два разложения $px=ab$ на простые: $p \cdot r_1 \cdot \dots \cdot r_m$ и $p_1 \cdot \dots \cdot p_k \cdot q_1 \cdot \dots \cdot q_l$. Значит по факториальности оба разложения совпадают, а значит $p$ равно какому-то неприводимому из разложений $a$ и $b$. А значит $p \mid a$ или $p \mid b$, что и означает простоту $p$.
        \end{proof}

        \begin{lemma}\todo{Тонкий момент. Проверить.}
            Если $a \mid b$, то разложение на неприводимые в $a$ есть подмножество разложения на неприводимые в $b$.
        \end{lemma}

        \begin{proof}
            Пусть $p$ --- неприводимый из разложения $a$. Тогда $p \mid b$, а значит $p$ делит какой-то неприводимый $q$ из разложения $b$, значит $p \sim q$, значит $a' \mid b'$, где $a' p = a$, $b' q = b$. Тогда спуском по кол-ву неприводимых в разложении $a$ получаем требуемое утверждение.
        \end{proof}

        Пусть нашлась возрастающая последовательность главных иделаов $\{a_n R\}_{n=0}^\infty$, тогда $a_{n+1}$ --- собственный делитель $a_n$ для всех $n$. Тогда разложение на неприводимые $a_0$ является собственным подмножеством разложением $a_n$, а поскольку разложение по определению конечно, то такой последовательности быть не может.

        Теперь докажем в обратную сторону. Пусть $R$ удовлетворяет условию обрыва возрастающей цепи главных идеалов.

        \begin{lemma}
            $ab \sim cd$, $b\sim d$, значит $a \sim c$.
        \end{lemma}

        \begin{proof}
            $b = \varepsilon d$, $ab = \delta cd$, где $\varepsilon \sim \delta \sim 1$. тогда $(a\varepsilon - \delta c)d = a \varepsilon d - \delta cd = ab - \delta cd = 0$, значит $a = \varepsilon^{-1}\delta c$, значит $a \sim c$.
        \end{proof}

        Теперь предположим имеется $R$, удовлетворяющее условию обрыва возрастающей цепи главных идеалов. Тогда у каждого числа есть неприводимый делитель, так как иначе есть подъём идеалов: $a_0 = a_1 b_1$, $a_1 = a_2 b_2$ и т.д., значит $a_0 R \subsetneq a_1 R \subsetneq a_2 R \subsetneq \dots$ --- противоречие. Тогда и каждое число \todo{Завершить доказательство.}
    \end{proof}

    \section{Идеалы}

    \begin{theorem}
        Пусть даны $I \triangleleft R$ и $a \sim b \Leftrightarrow a-b \in I$. Тогда $\sim$ --- отношение эквивалентности, а $R/I:=R/\sim$ --- кольцо. 
    \end{theorem}

    \begin{proof}
        Проверим, что $\sim$ --- отношение эквивалентности:
        \begin{itemize}
            \item $a - a = 0 \in I$, значит $a \sim a$;
            \item $a\sim b$, значит $a - b \in I$, значит $b-a = -(a - b) \in I$, значит $a \sim a$;
            \item $a\sim b$, $b\sim c$, значит $a-b \in I$, $b - c \in I$, значит $a-c = (a-b) + (b-c) \in I$, значит $a \sim c$.
        \end{itemize}

        Определим на $R/I$ операции сложения и умножения, нуля, противоположного, единицы и обратного:
        \begin{itemize}
            \item $[a] + [b] := [a + b]$;
            \item $[a] \cdot [b] := [a \cdot b]$;
            \item $0 := [0] = I$;
            \item $-[a] := [-a]$;
            \item $1 := [1]$;
            \item $[a]^{-1} := [a^{-1}]$.
        \end{itemize}
        Покажем, что $R/I$ --- кольцо:
        \begin{description}
            \item[$A_1$)] $\forall a, b, c \in R: ([a] + [b]) + [c] = [a + b] + [c] = [(a + b) + c] = [a + (b + c)] = [a] + [b + c] = [a] + ([b] + [c])$
            \item[$A_2$)] $\forall a \in R: [a] + [0] = [a + 0] = a = [0 + a] = [0] + [a]$
            \item[$A_3$)] $\forall a, b \in R: [a] + [b] = [a + b] = [b + a] = [b] + [a]$
            \item[$A_4$)] $\forall a \in R: [a] + -[a] = [a] + [-a] = [a + (-a)] = [0] = [(-a) + a] = [-a] + [a] = -[a] + [a]$
            \item[$D$)] $\forall a, b, k \in R: [k]([a] + [b]) = [k][a + b] = [k(a + b)] = [ka + kb] = [ka] + [kb] = [k][a] + [k][b]$, $([a] + [b])[k] = [a + b][k] = [(a + b)k] = [ak + bk] = [ak] + [bk] = [a][k] + [b][k]$
            \item[$M_1$)] $\forall a, b, c \in R: ([a] \cdot [b]) \cdot [c] = [a \cdot b] \cdot [c] = [(a \cdot b) \cdot c] = [a \cdot (b \cdot c)] = [a] \cdot [b \cdot c] = [a] \cdot ([b] \cdot [c])$
            \item[$M_2$)] $\forall a \in R: [a] \cdot [1] = [a \cdot 1] = [a] = [1 \cdot a] = [1] \cdot [a]$
            \item[$M_3$)] $\forall a, b \in R: [a] \cdot [b] = [a \cdot b] = [b \cdot a] = [b] \cdot [a]$
            \item[$M_4$)] $\forall a \in R \setminus \{0\}: [a] \cdot [a]^{-1} = [a] \cdot [a^{-1}] = [a \cdot a^{-1}] = [1] = [a^{-1} \cdot a] = [a^{-1}] \cdot [a] = [a]^{-1} \cdot [a]$
        \end{description}
    \end{proof}

    \begin{remark}
        Доказательство для классов эквивалентности каждой аксиомы основывалось только на соответсвующей аксиоме и определениях ранее.
    \end{remark}

    \pagebreak

    \listoftodos\label{TODO-list}
\end{document}