\documentclass[12pt,a4paper]{article}
\usepackage{../.tex/mcs-notes}
\usepackage{todonotes}
\usepackage{multicol}
\usepackage[all]{xy}
\CompileMatrices

\settitle
{Алгебра.}
{В. А. Петров}
{\%D0\%90\%D0\%BB\%D0\%B3\%D0\%B5\%D0\%B1\%D1\%80\%D0\%B0/Algebra.pdf}
\date{}

\DeclareMathOperator{\Img}{Im}
\DeclareMathOperator{\Ker}{Ker}
\DeclareMathOperator{\Frac}{Frac}
\newcommand{\A}{\ensuremath{\mathrm{A}}\xspace}
\newcommand{\D}{\ensuremath{\mathrm{D}}\xspace}
\newcommand{\M}{\ensuremath{\mathrm{M}}\xspace}
\newcommand{\Hom}{\mathrm{Hom}}
\newcommand{\Mor}{\mathrm{Mor}}
\newcommand{\dom}{\mathrm{dom}}
\newcommand{\cod}{\mathrm{cod}}
\newcommand{\Ob}{\mathrm{Ob}}
\newcommand{\id}{\mathrm{id}}
\newcommand{\Sets}{\mathrm{Sets}}
\newcommand{\Ens}{\mathrm{Ens}}
\newcommand{\Groups}{\mathrm{Groups}}
\newcommand{\Rings}{\mathrm{Rings}}
\newcommand{\CommRings}{\mathrm{CommRings}}
\newcommand{\Vect}{\mathrm{Vect}}
\newcommand{\Mod}{\mathrm{Mod}}
\newcommand{\Top}{\mathrm{Top}}
\newcommand{\HTop}{\mathrm{HTop}}
\newcommand{\Rels}{\mathrm{Rels}}

% \setlength{\marginparwidth}{2.5cm}
% \setlength{\textwidth}{470 pt}

\begin{document}
    \maketitle

    \listoftodos[TODOs]

    \tableofcontents

    \vspace{2em}

    Литература:
    \begin{itemize}
        \item Ван дер Варден ``Алгебра''
        \item Лэнг ``Алгебра''
        \item Винберг ``Курс Алгебры''
    \end{itemize}

    \subsection*{Немного истории}

    Зарождение --- Аль Хорезин, ``Китхаб Альджебр валь мукабалт''. ``Альджебр'' значит ``перенос из одной части уравнения в другую'', а ``мукабалт'' --- ``приведение подобных''.

    \section{Основные понятия.}

    \begin{definition}
        Алгебраическая структура --- это множество $M$ + заданные на нём операции + аксиомы на операциях.
    \end{definition}

    \begin{definition}
        Абелева группа --- набор $(M, +: M^2 \to M)$ с аксиомами:
        \begin{description}
            \item[$\A_1$)] $\forall a, b, c \in M: (a + b) + c = a + (b + c)$ --- ассоциативность сложения
            \item[$\A_2$)] $\exists 0 \in M: \forall a \in M: a + 0 = a = 0 + a$ --- нейтральный по сложению элемент
            \item[$\A_3$)] $\forall a, b \in M: a + b = b + a$ --- коммутативность сложения
            \item[$\A_4$)] $\forall a \in M: \exists -a: a + (-a) = 0 = (-a) + a$ --- существование противоположного
        \end{description}
    \end{definition}

    \begin{definition} Опишем следующие аксиомы на наборе $(M, +: M^2 \to M, \cdot: M^2 \to M)$ в добавок к $\A_1$, \dots, $\A_4$:
        \begin{description}
            \item[$\D$)] $\forall a, b, k \in M: k(a + b) = ka + kb$, $(a + b)k = ak + bk$ --- дистрибутивность
            \item[$\M_1$)] $\forall a, b, c \in M: (a \cdot b) \cdot c = a \cdot (b \cdot c)$ --- ассоциативность умножения
            \item[$\M_2$)] $\exists 1 \in M: \forall a \in M: a \cdot 1 = a = 1 \cdot a$ --- нейтральный по умножению элемент
            \item[$\M_3$)] $\forall a, b \in M: a \cdot b = b \cdot a$ --- коммутативность умножения
            \item[$\M_4$)] $\forall a \in M \setminus \{0\}: \exists\, a^{-1}: a \cdot a^{-1} = 1 = a^{-1} \cdot a$ --- существование обратного
        \end{description}
        По этим аксиомам определим следующие понятия:
        \begin{description}
            \item[\emph{Кольцо}] --- набор $(M, +, \cdot, 0)$, что верны $\A_1$, $\A_2$, $\A_3$, $\A_4$ и $\D$.
            \item[\emph{Ассоциативное кольцо}] --- кольцо с $\M_1$.
            \item[\emph{Кольцо с единицей}] --- кольцо с $\M_2$.
            \item[\emph{Тело}] --- кольцо с $\M_1$, $\M_2$, $\M_4$.
            \item[\emph{Поле}] --- кольцо с $\M_1$, $\M_2$, $\M_3$, $\M_4$.
            \item[\emph{Полукольцо}] --- кольцо без $\A_4$.
        \end{description}
    \end{definition}

    \begin{example}
        Если взять $\RR^3$, то векторное произведение в нём неассоциативно и антикоммутативно. Но есть
        \begin{lemma*}[Тождество Якоби]
            $u\times (v\times w) + v\times (w\times u) + w\times (u\times v) = 0$
        \end{lemma*}
    \end{example}

    \begin{example}
        Если взять $R^4 = R \times R^3$ и рассмотреть $\cdot: ((a; u); (b; v)) \mapsto (ab-u\cdot v; av + bu + u\times v)$ и $+: ((a; u); (b; v)) \mapsto (a + b, u + v)$, тогда получим $\HH$ --- ассоциативное некоммутативное тело кватернионов. Ассоциативность доказал Гамильтон.
    \end{example}

    \begin{lemma}
        $0 \cdot a = 0$
    \end{lemma}

    \begin{definition}
        Коммутативное кольцо без делителей нуля называется \emph{областью (целостности)}.
    \end{definition}

    \begin{definition}
        Пусть $m\in\NN$. Тогда \emph{множество остатков при делении на $m$} или $\ZZ/m\ZZ$ --- это фактор-множество по отношению эквивалентности $a\sim b \Leftrightarrow (a-b) \mid m$.
    \end{definition}

    \begin{definition}
        \emph{Подкольцо} --- это подмножество кольца, согласованное с его операциями.

        Как следствие ноль и обратимость согласуются автоматически.
    \end{definition}

    \begin{statement}
        Если $R$ --- подкольцо области целостности $S$, то $R$ --- область целостности.
    \end{statement}

    \begin{definition}
        \emph{Целые Гауссовы числа} или $\ZZ{}[i]$ --- это $\{a + bi \mid a, b \in \ZZ\}$.
    \end{definition}

    \begin{definition}
        Некоторое подмножество $R$ кольца $S$ \emph{замкнуто относительно сложения (умножения)}, если $\forall a, b \in R: a + b \in R$ ($ab \in R$ соответственно).
    \end{definition}

    \begin{remark}
        Замкнутое относительно сложения \textbf{И} умножения подмножество --- подкольцо.
    \end{remark}

    \begin{example}
        Пусть $d$ --- целое, не квадрат. Тогда $\ZZ{}[\sqrt{d}]$ --- область целостности.
    \end{example}

    \section{Теория делимости}

    Пусть $R$ --- область целостности.

    \begin{definition}
        ``$a$ \emph{делит} $b$'' или же $a \mid b$ значит, что $\exists c \in R: b = ac$.
    \end{definition}

    \begin{statement}
        Отношение ``$\mid$'' рефлексивно и транзитивно.
    \end{statement}

    \begin{definition}
        $a$ и $b$ \emph{ассоциированы}, если $a \mid b$ и $b \mid a$. Обозначение: $a \sim b$.
    \end{definition}

    \begin{statement}
        ``$\sim$'' --- отношение эквивалентности.
    \end{statement}

    \begin{statement}
        $a \sim b \Leftrightarrow \exists \text{ обратимый } \varepsilon: a = \varepsilon b$.
    \end{statement}

    \begin{proof}
        Пусть $a \sim b$. Тогда $\exists c, d: ac = b, bd = a$. Тогда $a(1-cd) = a - acd = a - bd = a - a = 0$, значит либо $a = 0$, либо $cd = 1$. В первом случае $b = ac = 0c = 0$, значит можно просто взять $\varepsilon = 1$. Во втором случае, $cd = 1$, значит $c$ и $d$ обратимы, тогда можно взять $\varepsilon = d$. следствие в одну сторону доказано.

        Пусть $a = \varepsilon b$, где $\varepsilon$ обратим. Значит:
        \begin{enumerate}
            \item $b\mid a$;
            \item $\exists \delta: \delta\varepsilon = 1$, значит $\delta a = \delta \varepsilon b = b$, значит $a \mid b$.
        \end{enumerate}
        Таким образом $a \sim b$.
    \end{proof}

    \begin{example}
        В $\ZZ{}[i]$ есть только следующие обратимые элементы: $1$, $-1$, $i$ и $-i$. Поэтому все ассоциативные элементы получаются друг из друга домножением на один из $1$, $-1$, $i$, $-i$ и вместе образуют квадрат (на \href{https://ru.wikipedia.org/wiki/%D0%9A%D0%BE%D0%BC%D0%BF%D0%BB%D0%B5%D0%BA%D1%81%D0%BD%D0%B0%D1%8F_%D0%BF%D0%BB%D0%BE%D1%81%D0%BA%D0%BE%D1%81%D1%82%D1%8C}{комплексной плоскоти}) с центром в нуле.
    \end{example}

    \begin{definition}
        \emph{Главным идеалом} элемента $a$ называется множество $M := \{ak \mid k \in R\} = \{b \mid a \text{ делит } b\}$. Обозначение: $(a)$ или $aR$.
    \end{definition}

    \begin{statement}
        $a \mid b \Leftrightarrow b \in aR \Leftrightarrow bR \subseteq aR$. 
    \end{statement}

    \begin{statement}
        $a \sim b \Leftrightarrow aR = bR$.
    \end{statement}

    \begin{statement}$\forall a \in R$
        \begin{enumerate}
            \item $0 \in aR$
            \item $x \in aR \Rightarrow -x \in aR$
            \item $x, y \in aR \Rightarrow x + y \in aR$
            \item $x \in aR, r \in R \Rightarrow xr \in aR$
        \end{enumerate}
    \end{statement}

    \begin{remark}
        То же верно и в некоммутативном $R$.
    \end{remark}

    \begin{example}
        В поле есть только $0R$ и $1R$.
    \end{example}

    \begin{example}
        В $\ZZ$ есть только $m\ZZ$ для каждого $m \in \NN \cup \{0\}$.
    \end{example}

    \begin{definition}
        Пусть $P$ --- кольцо. $I\subseteq P$ называется \emph{правым идеалом}, если
        \begin{enumerate}
            \item $0 \in I$;
            \item $a, b \in I \Rightarrow a + b \in I$;
            \item $a \in I \Rightarrow -a \in I$;
            \item $a \in I, r \in R \Rightarrow ar \in I$.
        \end{enumerate}
        $I$ называется \emph{левым идеалом}, если аксиому 4 заменить на ``$a \in I, r \in R \Rightarrow ra \in I$''. Также $I$ называется \emph{двухсторонним идеалом}, если является левым и правым идеалом, и обозначается как $I \triangleleft P$.
    \end{definition}

    \begin{remark}
        В коммутативном кольце (и в частности в области целостности) все идеалы двухсторонние.
    \end{remark}

    \begin{example}
        Пусть дано кольцо $P$ и фиксированы $a_1, \dots, a_n \in P$. Тогда $a_1 P + \dots + a_n P = \{a_1 x_1 + \dots + a_n x_n \mid x_1, \dots, x_n \in P\}$ есть правый (конечнопорождённый) идеал, порождённый элементами $a_1, \dots, a_n$. Аналогично $P a_1 + \dots + P a_n = \{x_1 a_1 + \dots + x_n a_n \mid x_1, \dots, x_n \in P\}$ --- левый (конечнопорождённый) идеал, порождённый элементами $a_1, \dots, a_n$.
    \end{example}

    \begin{definition}
        \emph{Область главных идеалов (ОГИ)} --- область целостности, где все идеалы главные.
    \end{definition}

    \begin{definition}
        Область целостности $R$ называется \emph{Евклидовой}, если существует функция (``Евклидова норма'') $N: R\setminus\{0\} \to \NN$, что
        \[\forall a, b \neq 0\; \exists q, r: a = bq + r \wedge (r = 0 \vee N(r) < N(b))\]
    \end{definition}

    \begin{theorem}
        Евклидово кольцо --- область главных идеалов.
    \end{theorem}

    \begin{proof}
        Пусть наше кольцо --- $R$. Если $I = \{0\}$, то $I = 0R$. Иначе возьмём $d \in I \setminus \{0\}$ с минимальной Евклидовой нормой. Тогда $\forall a \in I$ либо $d \mid a$, либо $\exists q, r: a = dq - r$. Во втором случае $dq \in I$, $r = a - dq \in I$, но $N(r) < N(d)$ --- противоречие. Значит $I = dR$.
    \end{proof}

    \begin{definition}
        \emph{Общим делителем} $a$ и $b$ называется $c$, что $c \mid a$ и $c \mid b$. \emph{Наибольшим общим делителем (НОД)} $a$ и $b$ называется общий делитель $a$ и $b$, делящийся на все другие общие делители $a$ и $b$.
    \end{definition}

    \begin{theorem}[алгоритм Евклида]
        В Евклидовом кольце у любых двух чисел есть НОД.
    \end{theorem}

    \begin{proof}
        Заметим, что $(a, b) = (a + bk, b)$.
        
        Пусть даны $a$ и $b$. Предположим, что $\varphi(a) \geqslant \varphi(b)$, иначе поменяем их местами. Тем самым по аксиоме Евклида найдутся $q$ и $r$, что $a = bq + r$, а $\varphi(r) < \varphi(b) \leqslant \varphi(a)$, значит $\varphi(a) + \varphi(b) > \varphi(r) + \varphi(b)$. При этом $(a, b) = (r, b)$. Значит бесконечно $\varphi(a) + \varphi(b)$ не может бесконечного уменьшаться, так как натурально, значит за конечное кол-во переходов мы получим, что одно из чисел делит другое, а значит НОД стал определён.
    \end{proof}

    \begin{theorem}[линейное представление НОД]
        $\forall a, b \in R\; \exists p, q \in R: ap + bq = (a, b)$.
    \end{theorem}

    \begin{proof}
        Докажем по индукции по $N(a) + N(b)$.

        \textbf{База.} $N(a) + N(b)=0$. Значит $N(a)=N(b)=0$, а тогда $a$ и $b$ не могут не делиться друг на друга, значит НОД --- любой из них. А в этом случае разложение очевидно.
        
        \textbf{Шаг.} WLOG $N(a) \geqslant N(b)$. Если $b \mid a$, то $b$ --- НОД, а тогда разложение очевидно. Иначе по аксиоме Евклида $\exists q, r: a = bq + r$. Заметим, что $(a, b) = (b, r) = d$, но $N(a) + N(b) \geqslant N(b) + N(b) > N(b) + N(r)$. Таким образом по предположению индукции для $b$ и $r$ получаем, что $d = bk + rl$ для некоторых $k$ и $l$, значит $d = bk + (a - bq)l = al + b(k - ql)$.
    \end{proof}

    \begin{definition}
        Элемент $p$ области целостности $R$ называется \emph{неприводимым}, если $\forall d \mid p$ либо $d \sim 1$, либо $d \sim p$.
    \end{definition}
    
    \begin{definition}
        Элемент $p$ области целостности $R$ называется \emph{простым}, если из условия $p \mid ab$ следует, что $p \mid a$ или $p \mid b$.
    \end{definition}

    \begin{statement}
        Любое простое неприводимо.
    \end{statement}

    \begin{proof}
        Предположим противное, т.е. некоторое простое $p$ представляется в виде произведения неделителей единицы $a$ и $b$. Тогда WLOG $p \mid a$. Значит $p\sim a$, а $b \sim 1$ --- противоречие.
    \end{proof}

    \begin{statement}
        В области главных идеалов неприводимые просты.
    \end{statement}

    \begin{proof}
        Пусть неприводимое $p$ делит $ab$. Пусть тогда $pR + aR = dR$. В таком случае $d \sim p$, значит либо $d \sim p$, либо $d \sim 1$. Если $d \sim p$, то $p \mid a$. Иначе $px + ay = 1$, значит $pxb + aby = b$. Но $p \mid pxb$ и $p \mid aby$, значит $p \mid b$. Поскольку рассуждение не зависит от $a$ и $b$, то $p$ просто.  
    \end{proof}

    \begin{definition}
        Область целостности $R$ \emph{удовлетворяет условию обрыва возрастающих цепей главных идеалов (APCC)}, если не существует последовательности $d_0 R \subsetneq d_1 R \subsetneq \dots$. Такое кольцо область целостности называют нётеровой.
    \end{definition}

    \begin{theorem}
        ОГИ нётерова.
    \end{theorem}

    \begin{proof}
        Пусть наша область --- $R$. Предположим противное, т.е. существует последовательность $\{a_n\}_{n=0}^\infty$, что $a_{n+1}$ --- собственный делитель $a_n$ (т.е. $a_{n+1} \mid a_n \wedge a_n \nsim a_{n+1}$). Тогда $a_0 R \subsetneq a_1 R \subsetneq a_2 R \subsetneq \dots$. Тогда $\exists x: xR = \bigcup_{n=0}^\infty a_n R$, так как  это объединение --- идеал. Но тогда $x \in a_j R$ для некоторого $j$, а значит $x R \subseteq a_j R$, а тогда $a_{j+1} R \subseteq a_j R$ --- противоречие.
    \end{proof}

    \begin{definition}
        Область целостности называется \emph{факториальной областью}, если в нём все неприводимые просты и оно нётерово.
    \end{definition}

    \begin{example}
        ОГИ факториальна.
    \end{example}

    \begin{theorem}[основная теорема арифметики]
        Пусть $R$ факториально. Тогда любое число представимо единственным образом в виде произведения простых с точностью до перестановки множителей и ассоциированности.
    \end{theorem}

    \begin{proof}
        \begin{thlemma}
            У каждого числа есть неприводимый делитель.
        \end{thlemma}

        \begin{proof}
            Пусть это не так. Тогда есть подъём идеалов: $a_0 = a_1 b_1$, $a_1 = a_2 b_2$ и т.д., значит $a_0 R \subsetneq a_1 R \subsetneq a_2 R \subsetneq \dots$ --- противоречие.
        \end{proof}

        \begin{thlemma}
            Каждое число представимо в виде произведения простых.
        \end{thlemma}

        \begin{proof}
            Пусть это не так. Тогда есть подъём идеалов: $a_0 = p_1 a_1$, где $p_1$ прост, $a_1 = p_2 a_2$, где $p_2$ прост, и т.д., значит $a_0 R \subsetneq a_1 R \subsetneq a_2 R \subsetneq \dots$ --- противоречие.
        \end{proof}

        Это доказывает существование разложения.

        \begin{thlemma}
            Если $p_1 \cdot \dots p_n = q_1 \cdot \dots \cdot q_m$ для простых $p_1$, \dots, $p_n$, $q_1$, \dots, $q_m$, то эти два набора совпадают с точностью до перестановки и ассоциированности.
        \end{thlemma}

        \begin{proof}
            Докажем индукцией по $n$.

            \textbf{База:} Для $n=0$ утверждение очевидно, так как тогда $1 = q_1 \cdot \dots \cdot q_m$, значит $m=0$.

            \textbf{Шаг:} Несложно видеть, что $p_n \mid q_1 \cdot \dots \cdot q_m$, значит $p_n \mid q_i$ для некоторого $i$, значит $p_n \sim q_i$. Переставим $q_k$, что $q'_m = q_i$. Значит $p_1 \cdot \dots \cdot p_{n-1} = q'_1 \cdot \dots \cdot q'_{m-1}$. По предположению индукции эти два набора совпадают с точностью до перестановки и ассоциированности, значит таковы и начальные наборы.
        \end{proof}

        Это доказывает единственность разложения.
    \end{proof}

    \section{Идеалы и морфизмы}

    \begin{theorem}
        Пусть даны $I \triangleleft R$ и $a \sim b \Leftrightarrow a-b \in I$. Тогда $\sim$ --- отношение эквивалентности, а $R/I:=R/\sim$ --- кольцо. 
    \end{theorem}

    \begin{proof}
        Проверим, что $\sim$ --- отношение эквивалентности:
        \begin{itemize}
            \item $a - a = 0 \in I$, значит $a \sim a$;
            \item $a\sim b$, значит $a - b \in I$, значит $b-a = -(a - b) \in I$, значит $a \sim a$;
            \item $a\sim b$, $b\sim c$, значит $a-b \in I$, $b - c \in I$, значит $a-c = (a-b) + (b-c) \in I$, значит $a \sim c$.
        \end{itemize}

        Определим на $R/I$ операции сложения и умножения, нуля, противоположного, единицы и обратного:
        \begin{itemize}
            \item $[a] + [b] := [a + b]$;
            \item $[a] \cdot [b] := [a \cdot b]$;
            \item $0 := [0] = I$;
            \item $-[a] := [-a]$;
            \item $1 := [1]$;
            \item $[a]^{-1} := [a^{-1}]$.
        \end{itemize}
        Покажем, что $R/I$ --- кольцо:
        \begin{description}
            \item[$\A_1$)] $\forall a, b, c \in R: ([a] + [b]) + [c] = [a + b] + [c] = [(a + b) + c] = [a + (b + c)] = [a] + [b + c] = [a] + ([b] + [c])$
            \item[$\A_2$)] $\forall a \in R: [a] + [0] = [a + 0] = a = [0 + a] = [0] + [a]$
            \item[$\A_3$)] $\forall a, b \in R: [a] + [b] = [a + b] = [b + a] = [b] + [a]$
            \item[$\A_4$)] $\forall a \in R: [a] + -[a] = [a] + [-a] = [a + (-a)] = [0] = [(-a) + a] = [-a] + [a] = -[a] + [a]$
            \item[$\D$)] $\forall a, b, k \in R: [k]([a] + [b]) = [k][a + b] = [k(a + b)] = [ka + kb] = [ka] + [kb] = [k][a] + [k][b]$, $([a] + [b])[k] = [a + b][k] = [(a + b)k] = [ak + bk] = [ak] + [bk] = [a][k] + [b][k]$
            \item[$\M_1$)] $\forall a, b, c \in R: ([a] \cdot [b]) \cdot [c] = [a \cdot b] \cdot [c] = [(a \cdot b) \cdot c] = [a \cdot (b \cdot c)] = [a] \cdot [b \cdot c] = [a] \cdot ([b] \cdot [c])$
            \item[$\M_2$)] $\forall a \in R: [a] \cdot [1] = [a \cdot 1] = [a] = [1 \cdot a] = [1] \cdot [a]$
            \item[$\M_3$)] $\forall a, b \in R: [a] \cdot [b] = [a \cdot b] = [b \cdot a] = [b] \cdot [a]$
            \item[$\M_4$)] $\forall a \in R \setminus \{0\}: [a] \cdot [a]^{-1} = [a] \cdot [a^{-1}] = [a \cdot a^{-1}] = [1] = [a^{-1} \cdot a] = [a^{-1}] \cdot [a] = [a]^{-1} \cdot [a]$
        \end{description}
    \end{proof}

    \begin{remark}
        Доказательство для классов эквивалентности каждой аксиомы основывалось только на соответствующей аксиоме и определениях ранее.
    \end{remark}

    \begin{definition}
        \emph{Гомоморфизм} --- такое отображение $\varphi: R \to S$ --- это отображение, сохраняющее операции:
        \begin{itemize}
            \item $\varphi(a + b) = \varphi(a) + \varphi(b)$;
            \item $\varphi(a \cdot b) = \varphi(a) \cdot \varphi(b)$;
            \item $\varphi(0) = 0$;
            \item $\varphi(-a) = -\varphi(a)$.
        \end{itemize}

        \emph{Гомоморфизм кольца с 1} --- гомоморфизм, что $\varphi(1) = 1$.
    \end{definition}

    \begin{statement}
        Композиция гомоморфизмов --- гомоморфизм.
    \end{statement}

    \begin{definition}
            Пусть $f: X \to Y$. Несложно видеть, что $f$ раскладывается в композицию сюръекции $f: X \to f(X)$ и инъекции $id: f(X) \to Y$. Тогда $\Img(f) = \{f(x) \mid x \in X\}$ --- \emph{множество значений} $f$, а классы значений $X$, переходящих в один $y\in Y$ суть \emph{слои} --- $f^{-1}(y) = \{x \mid f(x) = y\}$ для некоторого $y$.
        \[
            \xymatrix{
                X \ar@{->>}[rd]^f \ar[rr]^f && Y\\
                & f(X) = \Img(f) \ar@{^{(}->}[ur]^{id}
            }
        \]
    \end{definition}

    \begin{definition}
        Пусть $\varphi: R \to S$ --- гомоморфизм. Тогда \emph{ядром} $\varphi$ называется $\Ker(\varphi) := \{r \in R \mid \varphi(r) = 0\}$.
    \end{definition}

    \begin{statement}
        Ядро гомоморфизма --- двусторонний идеал.
    \end{statement}

    \begin{definition}
        $\varphi: S \to R$ --- \emph{изоморфизм}, если это биективный гомоморфизм.
    \end{definition}

    \begin{definition}
        Два кольца называются изоморфными, если между ними есть изоморфизм. Обозначение: $R \cong S$.
    \end{definition}

    \begin{statement}
        Пусть $R \cong S$. Тогда
        \begin{itemize}
            \item Если $R$ коммутативно, то и $S$ коммутативно.
            \item Если $R$ --- область целостности, то и $S$ --- область целостности.
            \item Если $R$ --- ОГИ, то и $S$ --- ОГИ.
        \end{itemize}
    \end{statement}

    \begin{statement}\ 
        \begin{enumerate}
            \item $R \cong R$.
            \item $R \cong S \Leftrightarrow S \cong R$.
            \item $R \cong S \cong T \Rightarrow R \cong T$.
        \end{enumerate}
    \end{statement}

    \begin{theorem}[о гомоморфизме]
        Пусть $\varphi: R \to S$ --- гомоморфизм. (Вспомним, что $\Ker(\varphi) \triangleleft R$, а $\Img(\varphi) = \varphi(R)$.) Тогда $R/\Ker(\varphi) \cong \Img(\varphi)$, где изоморфизм переводит $[a] \mapsto \varphi(a)$.
        \[
            \xymatrix{
                R \ar@{->>}[d]_{r \mapsto [r]} \ar[rr]^\varphi && S\\
                R/\Ker(\varphi) \ar[rr]^-\sim_-{[r] \mapsto \varphi(r)} && \Img(\varphi) \ar@{^{(}->}[u]_{id}
            }
        \]
    \end{theorem}

    \begin{proof}\ 
        \begin{enumerate}
            \item Корректность. $[a] = [a'] \Leftrightarrow a - a' \in \Ker(\varphi) \Leftrightarrow \varphi(a - a') = 0 \Leftrightarrow \varphi(a) = \varphi(a')$.
                \begin{remark}
                    Классы эквивалентности по $\Ker(\varphi)$ как раз слои $\varphi$.
                \end{remark}
            \item Заметим, что работают следующие операции:
                \begin{itemize}
                    \item $[a] + [b] = [a + b] \mapsto \varphi(a) + \varphi(b) = \varphi(a + b)$;
                    \item $[a] \cdot [b] = [a \cdot b] \mapsto \varphi(a) \cdot \varphi(b) = \varphi(a \cdot b)$. 
                \end{itemize}
            \item Сюръективность следует из того, что $\varphi(a) = \varphi(b) \Leftrightarrow [a] = [b]$.
            \item Инъективность следует из того, что каждый элемент в $\Img(\varphi)$ имеет прообраз.
        \end{enumerate}
    \end{proof}

    \begin{theorem}[китайская теорема об остатках (КТО) для двух чисел]
        Пусть $m$ и $n$ взаимно просты. Тогда $\ZZ/mn\ZZ \cong \ZZ/m\ZZ \times \ZZ/n\ZZ$.
    \end{theorem}

    \begin{proof}
        Рассмотрим $\varphi: \ZZ/mn\ZZ \to \ZZ/m\ZZ \times \ZZ/n\ZZ, [a]_{mn} \mapsto ([a]_m; [a]_n)$. Несложно заметить, что ядро $\varphi$ тривиально, поэтому $\ZZ/mn\ZZ \cong \ZZ/mn\ZZ/\Ker(\varphi) \cong \Img(\varphi)$. Но в последнем элементов не менее $mn$, так как $\Img(\varphi) \cong \ZZ/mn\ZZ$, но и не более, так как $|\ZZ/m\ZZ \times \ZZ/n\ZZ| = mn$, поэтому $\Img(\varphi) = \ZZ/m\ZZ \times \ZZ/n\ZZ$, поэтому $\ZZ/mn\ZZ \cong \ZZ/m\ZZ \times \ZZ/n\ZZ$.
    \end{proof}

    \begin{theorem}[КТО]
        Пусть $m_1, \dots, m_k$ --- попарно взаимно простые числа. Тогда
        \[\ZZ/m_1\dots m_k \cong \ZZ/m_1\ZZ \times \dots \times \ZZ/m_k\ZZ\]
    \end{theorem}

    \begin{proof}
        По индукции по $k$ с помощью КТО для двух чисел.
    \end{proof}

    \begin{theorem}[Универсальное свойтсво фактор-кольца]
        Пусть есть $I \triangleleft R$ и гомоморфизмы $\pi: R \to R/I$ --- нативный гомоморфизм, и $\varphi: R \to S$, что $\pi(I) = \{0\}$. Тогда существует и единственен гомоморфизм $\varphi': R/I \to S$, что $\varphi' \circ \pi = \varphi$.
        \[
            \xymatrix{
                R \ar@{->>}[rd]_{\pi} \ar[rr]^\varphi && S\\
                & R/I \ar[ur]_{\varphi'}
            }
        \]
    \end{theorem}

    \begin{proof}
        $\varphi'([a]) = (\varphi' \circ \pi)(a) = \varphi(a)$ --- это означает единственность; так функцию и определим. Осталось показать корректность.

        Несложно заметить, что если $[a] = [b]$, то $a - b \in I$, значит $\varphi(a - b) = 0$, значит $\varphi(a) = \varphi(b)$. Теперь проверим операции:
        \begin{itemize}
            \item $\varphi'([a] + [b]) = \varphi'([a + b]) = \varphi(a + b) = \varphi(a) + \varphi(b) = \varphi'([a]) + \varphi'([b])$.
            \item $\varphi'([a] \cdot [b]) = \varphi'([a \cdot b]) = \varphi(a \cdot b) = \varphi(a) \cdot \varphi(b) = \varphi'([a]) \cdot \varphi'([b])$
        \end{itemize}
    \end{proof}

    \begin{definition}
        Пусть $R$ --- область целостности. Тогда рассмотрим $Q = R \times (R \setminus \{0\})$ и отношение $\sim$ на $Q$, что $(a; b) \sim (c; d) \Leftrightarrow ad = bc$. Несложно видеть, что $\sim$ --- отношение эквивалентности. Тогда \emph{полем частных} области целостности $R$ называется $\Frac(R) = Q/\sim$, где операции:
        \begin{itemize}
            \item $[(a; b)] + [(c; d)] := [(ad + bc; bd)]$;
            \item $[(a; b)] \cdot [(c; d)] := [(ac; bd)]$;
            \item $0 := [(0; 1)]$;
            \item $- [(a; b)] := [(-a; b)]$;
            \item $1 := [(1; 1)]$;
            \item $[(a; b)]^{-1} = [(b; a)]$.
        \end{itemize}
        Несложно видеть, что все операции корректны, а поле частных --- поле.
    \end{definition}

    \begin{remark}
        Есть нативный инъективный гомоморфизм из $R$ в $\Frac(R)$:
        \[\varphi: R \to \Frac(R), r \mapsto [(r; 1)]\]
    \end{remark}

    \begin{theorem}[Уникальное свойтсво поля частных]
        Пусть $R$ --- область целостности, $F$ --- поле, $\varphi: R \to F$ --- инъективный гомоморфизм, сохраняющий $1$, $\pi: R \to \Frac(R)$ --- нативный гомоморфизм. Тогда существует единственный гомоморфизм $\varphi': \Frac(R) \to F$, что $\varphi' \circ \pi = \varphi$.
        \[
            \xymatrix{
                R \ar@{^{(}->}[rd]_{\pi} \ar@{^{(}->}[rr]^\varphi && F\\
                & \Frac(R) \ar[ur]_{\varphi'}
            }
        \]
    \end{theorem}

    \begin{remark}\label{field_homomorphism_remark}
        Если $\varphi: E \to F$ --- гомоморфизм полей, сохраняющий $1$, то он инъективен. Действительно, $\Ker(\varphi)$ --- идеал, значит ${0}$ или $E$, так как $E$ поле, но случай $E$ не подходит, так как не сохраняется $0$, значит $\Ker(\varphi) = {0}$, значит $\varphi$ инъективно.
    \end{remark}

    \begin{proof}
        \begin{thlemma}
            $\varphi'(1/b) = 1/\varphi'(b)$
        \end{thlemma}

        \begin{proof}
            По замечанию \ref{field_homomorphism_remark} $\varphi'$ --- инъективен, но $\varphi'(0) = 0$, а тогда для всякого $a \neq 0$ верно, что $\varphi'(a) \neq 0$, значит $\varphi'(a) \cdot \varphi'(a^{-1}) = \varphi'(1) = 1$, значит $\varphi'(a)^{-1} = \varphi'(a^{-1})$.
        \end{proof}

        \begin{thlemma}
            $\varphi'(a/b) = \varphi'(a)/\varphi'(b)$.
        \end{thlemma}

        \begin{proof}
            $\varphi'(a/b) = \varphi'(a) \cdot \varphi'(b^{-1}) = \varphi'(a) \cdot \varphi'(b)^{-1} = \varphi'(a)/\varphi'(b)$.
        \end{proof}

        Заметим, что $\varphi'(a) = \varphi'(\pi(a)) = \varphi(a)$, поэтому $\varphi'(a/b) = \varphi(a)/\varphi(b)$ --- это означает единственность $\varphi'$.

        Теперь рассмотрим соответствующую $\varphi': a/b \mapsto \varphi(a)/\varphi(b)$. Проверим корректность:
        \begin{align*}
            \frac{a}{b} &= \frac{c}{d}&
            &\Rightarrow&
            ad &= bc&
            &\Rightarrow&
            \varphi(ad) &= \varphi(bc)&
            &\Rightarrow\\
            \varphi(a)\varphi(d) &= \varphi(b)\varphi(c)&
            &\Rightarrow&
            \frac{\varphi(a)}{\varphi(b)} &= \frac{\varphi(c)}{\varphi(d)}&
            &\Rightarrow&
            \varphi'\left(\frac{a}{b}\right) &= \varphi'\left(\frac{c}{d}\right)
        \end{align*}
        Теперь проверим согласованность с операциями:
        \begin{itemize}
            \item \[\varphi'\left(\frac{a}{b}\cdot \frac{c}{d}\right) = \frac{\varphi(ac)}{\varphi(bd)}=\frac{\varphi(a)}{\varphi(b)}\cdot\frac{\varphi(c)}{\varphi(d)}=\varphi'\left(\frac{a}{b}\right)\cdot\varphi'\left(\frac{c}{d}\right);\]
            \item
                \begin{multline*}
                    \varphi'\left(\frac{a}{b} + \frac{c}{d}\right) = \varphi'\left(\frac{ad + bc}{bd}\right) = \frac{\varphi(ad + bc)}{\varphi(bd)} = \\
                    \frac{\varphi(a)\varphi(d) + \varphi(b)\varphi(c)}{\varphi(b)\varphi(d)} = \frac{\varphi(a)}{\varphi(b)} + \frac{\varphi(c)}{\varphi(d)} = \varphi'\left(\frac{a}{b}\right) + \varphi'\left(\frac{c}{d}\right)
                \end{multline*}
        \end{itemize}
        
    \end{proof}

    \section{Многочлены}

    \begin{theorem}
        Пусть дано кольцо $R$. Рассмотрим множество $S$ финитных бесконечных последовательностей элементов из $R$; т.е. все такие последовательности $(a_n)_{n=0}^\infty$, что всякое $a_n \in R$ и есть такое $N$, что для всякого $n > N$ верно, что $a_n = 0_R$. Также рассмотрим операции сложения и умножения на $S$:
        \begin{align*}
            &{+}: S^2 \to S, ((a_n)_{n=0}^\infty, (b_n)_{n=0}^\infty) \mapsto (a_n + b_n)_{n=0}^\infty&
            &{\cdot}: S^2 \to S, ((a_n)_{n=0}^\infty, (b_n)_{n=0}^\infty) \mapsto \left(\sum_{k=0}^n a_k \cdot b_{n-k}\right)_{n=0}^\infty
        \end{align*}
        Тогда
        \begin{enumerate}
            \item $S$ является кольцом, где $+$ --- операция сложения, $\cdot$ --- операция умножения, $(0_R)_{n=0}^\infty$ --- нейтральный по сложению элемент.
            \item $S$ наследует от $R$ аксиомы $\M_1$, $\M_2$ и $\M_3$.
            \item $R$ изоморфно подкольцу $S$, состоящему из элементов вида $(a, 0, 0 , \dots)$, где $a \in R$.
        \end{enumerate}
    \end{theorem}

    \begin{definition}
        Множество $S$ из прошлой теоремы называется \emph{кольцом многочленов над $R$} и обозначается $R[x]$. При этом всякий его элемент $(a_n)_{n=0}^\infty$ обозначается как $a_0 + \cdots + a_n x^n + \cdots = \sum_{n=0}^\infty a_n x^n$.
    \end{definition}

    \begin{proof}
        \begin{enumerate}
            \item Важно сказать, что из $\A_1$ следует корректность определения умножения. Проверим аксиомы:
                \begin{description}
                    \item[$\A_1$)] $\forall (a_n)_{n=0}^\infty, (b_n)_{n=0}^\infty, (c_n)_{n=0}^\infty \in S:$
                        \begin{align*}
                            ((a_n)_{n=0}^\infty + (b_n)_{n=0}^\infty) + (c_n)_{n=0}^\infty
                            &= (a_n + b_n)_{n=0}^\infty + (c_n)_{n=0}^\infty\\
                            &= ((a_n + b_n) + c_n)_{n=0}^\infty\\
                            &= (a_n + (b_n + c_n))_{n=0}^\infty\\
                            &= (a_n)_{n=0}^\infty + (b_n + c_n)_{n=0}^\infty\\
                            &= (a_n)_{n=0}^\infty + ((b_n)_{n=0}^\infty + (c_n)_{n=0}^\infty)
                        \end{align*}
                    \item[$\A_2$)] $\forall (a_n)_{n=0}^\infty \in R:$
                        \begin{align*}
                            (a_n)_{n=0}^\infty + (0)_{n=0}^\infty
                            &= (a_n + 0)_{n=0}^\infty\\
                            &= (a_n)_{n=0}^\infty\\
                            &= (0 + a_n)_{n=0}^\infty\\
                            &= (0)_{n=0}^\infty + (a_n)_{n=0}^\infty
                        \end{align*}
                    \item[$\A_3$)] $\forall (a_n)_{n=0}^\infty, (b_n)_{n=0}^\infty \in R:$
                        \begin{align*}
                            (a_n)_{n=0}^\infty + (b_n)_{n=0}^\infty
                            &= (a_n + b_n)_{n=0}^\infty\\
                            &= (b_n + a_n)_{n=0}^\infty\\
                            &= (b_n)_{n=0}^\infty + (a_n)_{n=0}^\infty
                        \end{align*}
                    \item[$\A_4$)] $\forall (a_n)_{n=0}^\infty \in R:$
                        \begin{align*}
                            (a_n)_{n=0}^\infty + (-a_n)_{n=0}^\infty
                            &= (a_n + -a_n)_{n=0}^\infty\\
                            &= (0)_{n=0}^\infty\\
                            &= (-a_n + a_n)_{n=0}^\infty\\
                            &= (-a_n)_{n=0}^\infty + (a_n)_{n=0}^\infty
                        \end{align*}
                    \item[$\D$)] $\forall (a_n)_{n=0}^\infty, (b_n)_{n=0}^\infty, (k_n)_{n=0}^\infty \in R:$
                        \begin{align*}
                            (k_n)_{n=0}^\infty((a_n)_{n=0}^\infty + (b_n)_{n=0}^\infty)
                            &= (k_n)_{n=0}^\infty \cdot (a_n + b_n)_{n=0}^\infty\\
                            &= \left(\sum_{t=0}^n k_t(a_{n-t} + b_{n-t})\right)_{n=0}^\infty\\
                            &= \left(\sum_{t=0}^n k_t \cdot a_{n-t} + \sum_{t=0}^n k_t \cdot b_{n-t}\right)_{n=0}^\infty\\
                            &= \left(\sum_{t=0}^n k_t \cdot a_{n-t}\right)_{n=0}^\infty + \left(\sum_{t=0}^n k_t \cdot b_{n-t}\right)_{n=0}^\infty\\
                            &= (k_n)_{n=0}^\infty (a_n)_{n=0}^\infty + (k_n)_{n=0}^\infty (b_n)_{n=0}^\infty
                        \end{align*}
                        и
                        \begin{align*}
                            ((a_n)_{n=0}^\infty + (b_n)_{n=0}^\infty)(k_n)_{n=0}^\infty
                            &= (a_n + b_n)_{n=0}^\infty \cdot (k_n)_{n=0}^\infty\\
                            &= \left(\sum_{t=0}^n (a_{n-t} + b_{n-t})k_t\right)_{n=0}^\infty\\
                            &= \left(\sum_{t=0}^n a_{n-t} \cdot k_t + \sum_{t=0}^n b_{n-t} \cdot k_t\right)_{n=0}^\infty\\
                            &= \left(\sum_{t=0}^n a_{n-t} \cdot k_t\right)_{n=0}^\infty + \left(\sum_{t=0}^n b_{n-t} \cdot k_t\right)_{n=0}^\infty\\
                            &= (a_n)_{n=0}^\infty (k_n)_{n=0}^\infty + (b_n)_{n=0}^\infty (k_n)_{n=0}^\infty
                        \end{align*}
                \end{description}
            \item Проверим наследственность для каждой аксиомы:
                \begin{description}
                    \item[$\M_1$)] $\forall (a_n)_{n=0}^\infty, (b_n)_{n=0}^\infty, (c_n)_{n=0}^\infty \in R:$
                        \begin{align*}
                            ((a_n)_{n=0}^\infty \cdot (b_n)_{n=0}^\infty) \cdot (c_n)_{n=0}^\infty
                            &= \left(\sum_{k=0}^n a_k \cdot b_{n-k}\right)_{n=0}^\infty \cdot (c_n)_{n=0}^\infty\\
                            &= \left(\sum_{k=0}^n \left(\sum_{l=0}^k a_l \cdot b_{k-l}\right) \cdot c_{n-k}\right)_{n=0}^\infty\\
                            &= \left(\sum_{\substack{0 \leqslant k\\ l \leqslant 0\\ k+l \leqslant n}} (a_k \cdot b_l) \cdot c_{n-k-l}\right)_{n=0}^\infty\\
                            &= \left(\sum_{\substack{0 \leqslant k\\ l \leqslant 0\\ k+l \leqslant n}} a_k \cdot (b_l \cdot c_{n-k-l})\right)_{n=0}^\infty\\
                            &= \left(\sum_{k=0}^n a_{n-k} \cdot \left(\sum_{l=0}^k b_l \cdot c_{k-l}\right)\right)_{n=0}^\infty\\
                            &= (a_n)_{n=0}^\infty \cdot \left(\sum_{k=0}^n b_k \cdot c_{n-k}\right)_{n=0}^\infty\\
                            &= (a_n)_{n=0}^\infty \cdot ((b_n)_{n=0}^\infty \cdot (c_n)_{n=0}^\infty)
                        \end{align*}
                    \item[$\M_2$)] Обозначим за $1$ в $S$ последовательность $(t_n)_{n=0}^\infty$, где $t_0 = 1$, а все остальные члены равны $0$. Тогда $\forall (a_n)_{n=0}^\infty \in R:$
                        \begin{align*}
                            (a_n)_{n=0}^\infty \cdot 1
                            &= \left(\sum_{k=0}^n a_k \cdot t_{n-k}\right)_{n=0}^\infty\\
                            &= (a_n)_{n=0}^\infty\\
                            &= \left(\sum_{k=0}^n t_{n-k} \cdot a_k\right)_{n=0}^\infty\\
                            &= 1 \cdot (a_n)_{n=0}^\infty
                        \end{align*}
                    \item[$\M_3$)] $\forall (a_n)_{n=0}^\infty, (b_n)_{n=0}^\infty \in R:$
                        \begin{align*}
                            (a_n)_{n=0}^\infty \cdot (b_n)_{n=0}^\infty
                            &= \left(\sum_{k=0}^n a_k \cdot b_{n-k}\right)_{n=0}^\infty\\
                            &= \left(\sum_{k=0}^n b_k \cdot a_{n-k}\right)_{n=0}^\infty\\
                            &= (b_n)_{n=0}^\infty \cdot (a_n)_{n=0}^\infty
                        \end{align*}
                \end{description}
            \item Рассмотрим отображение $\varphi: R \to S, a \mapsto (a, 0, 0, \dots)$. Тогда
                \begin{itemize}
                    \item $\varphi(a) + \varphi(b) = (a + b, 0, \dots) = \varphi(a + b)$
                    \item $\varphi(a) \cdot \varphi(b) = (ab, 0, \dots) = \varphi(a \cdot b)$
                    \item $\varphi(0) = (0, 0, \dots) = 0$
                    \item (в случае $\M_2$) $\varphi(1) = (1, 0, \dots) = 1$
                \end{itemize}
                Значит $\Ker(\varphi) = \{0\}$, $R \cong \Img(\phi)$. При этом несложно видеть, что $\Img(\phi)$ и есть множество всех последовательностей вида $(a, 0, 0, \dots)$.
        \end{enumerate}
    \end{proof}

    \newpage\null\thispagestyle{empty}\newpage
    \section{Теория категорий}

    \begin{definition}
        \emph{Категория} $C$ есть совокупность семейства (не обязательно множества) объектов $\Ob(C)$ и семейства \emph{морфизмов} (также ``arrows''), что выполнены следующие условия.
        \begin{enumerate}
            \item У всякого морфизма $f$ есть прообраз (также ``начало'', ``source'', ``domain''; обозначение: $s(f)$ или $\dom(f)$) и образ (также ``конец'', ``target'', ``codomain''; обозначение: $t(f)$ или $\cod(f)$), являющиеся объектами из рассмотренного семейства. Семейства всех морфизмов из $X$ в $Y$ (т.е. с прообразом $X$ и образом $Y$) обозначается $\Hom(X, Y)$ или $\Mor(X, Y)$.
            \item На семействе морфизмов введён не полностью определённый бинарный оператор $\circ$ (можно считать, функциональное отношение из $M \times M$ в $M$, где $M$ --- семейство морфизмов), что для всяких $X, Y, Z \in \Ob(C)$ и $f \in \Hom(X, Y)$, $g \in \Mor(Y, Z)$ значение $g \circ f$ определено и лежит в $\Hom(X, Z)$. Данный оператор называется \emph{композицией}, а $g \circ f$ --- композицией $g$ и $f$.
            \item Операция композиции морфизмов ассоциативна: для всяких $X, Y, Z, T \in \Ob(C)$ и $f \in \Hom(X, Y)$, $g \in \Hom(Y, Z)$, $h \in \Hom(Z, T)$
                \[(f \circ g) \circ h = f \circ (g \circ h).\]
            \item Для всякого $X \in \Ob(C)$ есть выделенный морфизм $\id_X \in \Hom(X, X)$ (также $1_X$). Он называется тождественным морфизмом $X$.
            \item Для всяких $X, Y \in \Ob(C)$ для всякого $f \in \Hom(X, Y)$ верно, что
                \[f \circ \id_X = f = \id_Y \circ f.\]
        \end{enumerate}
    \end{definition}

    \begin{example}\ 
        \begin{enumerate}
            \item $\Sets$ ($\Ens$):
                \begin{itemize}
                    \item $\Ob(\Sets)$ --- все множества,
                    \item $\Hom(X, Y)$ --- все отображения из $X$ в $Y$,
                    \item $\circ$ --- обычная композиция отображений,
                    \item $\id_X$ --- тождественное отображение $X \to X$.
                \end{itemize}
            \item $\Groups$:
                \begin{itemize}
                    \item $\Ob(\Groups)$ --- все группы,
                    \item $\Hom(G, H)$ --- все гомоморфизмы $G \to H$,
                    \item $\circ$ --- обычная композиция гомоморфизмов,
                    \item $\id_G$ --- тождественный гомоморфизм $G \to G$.
                \end{itemize}
            \item Аналогично описываются категории $\Rings$ колец, $\CommRings$ коммутативных колец (если в случаях $\Rings$ и $\CommRings$ рассматриваются кольца с единицей, то надо требовать, чтобы гомоморфизмы переводили единицу в единицу), $\Vect_F$ векторных пространств над полем $F$, $R-\Mod$ $R$-модулей, и т.д. для всякой алгебраической структуры.
            \item $\Top$:
                \begin{itemize}
                    \item $\Ob(\Top)$ --- все топологические пространства,
                    \item $\Hom(G, H)$ --- все непрерывные отображения,
                    \item $\circ$ --- обычная композиция отображений,
                    \item $\id_G$ --- тождественное отображение $G \to G$.
                \end{itemize}
            \item $\HTop$:
                \begin{itemize}
                    \item $\Ob(\HTop)$ --- все ``хорошие'' (компактно порождённые) топологические пространства,
                    \item $\Hom(G, H)$ --- все непрерывные отображения по модулю гомотопии,
                    \item $\circ$ --- обычная композиция отображений,
                    \item $\id_G$ --- тождественное отображение $G \to G$.
                \end{itemize}
            \item $\Ob(C) = \{X\}$. В таком случае мы получаем \emph{моноид} некоторых отображений $X$ на себя: у нас есть множество морфизмов $X$ на себя с операцией композиции (произведение в моноиде), которая ассоциативна и имеет нейтральный элемент (но не обязательно обратима).
            \item Частичный предпорядок задаёт категорию:
                \begin{itemize}
                    \item $\Ob(C) = M$,
                    \item
                        $\Hom(x, y) = \begin{cases}
                            \{\star_{x\to y}\} \text{ если } x \leqslant y,\\
                            \varnothing \text{ иначе},
                        \end{cases}$
                    \item $\star_{y\to z} \circ \star_{x\to y} := \star_{x\to z}$,
                    \item $\id_x := \star_{x\to x}$.
                \end{itemize}
            \item $\Rels$ --- категория отношений:
                \begin{itemize}
                    \item $\Ob(\Rels)$ --- все множества;
                    \item $\Hom(X, Y)$ --- все подмножества $X \times Y$;
                    \item для всяких $S \in \Hom(X, Y)$ и $R \in \Hom(Y, Z)$
                        \[R \circ S := \{(x, z) \in X \times Z \mid \exists y \in Y: \qquad (x, y) \in S \wedge (y, z) \in R\};\]
                    \item $\id_X := \{(x, x)\}_{x \in X}$. 
                \end{itemize}
            \item Пустая категория: нет объектов, нет морфизмов.
            \item Категория с единственным объектом и единственным тождественным морфизмом на нём.
            \item Дискретная категория: нет нетождественных морфизмов.
        \end{enumerate}
    \end{example}

    \begin{definition}
        $X, Y \in \Ob(C)$ называются \emph{изоморфными}, если есть $f \in \Hom(X, Y)$ и $g \in \Hom(Y, X)$, что
        \[f \circ g = \id_Y \qquad \text{ и } \qquad g \circ f = \id_X.\]
    \end{definition}

    \begin{definition}
        \emph{Подкатегория} $S$ категории $C$ --- категория, семейства объектов и морфизмов которой суть подсемейства объектов и морфизмов категории $C$ соответственно.
    \end{definition}
\end{document}