\documentclass[12pt,a4paper]{article}
\usepackage{my_math}

\title{Алгебра.}
\author{В. А. Петров\\lektorium.tv}
\date{}

\begin{document}
    \maketitle

    Зарождение --- Аль Хорезин, ``Китхаб Альджебр валь мукабалт''. ``Альджебр'' значит ``перенос из одной части уравнения в другую'', а ``мукабалт'' --- ``приведение подобных''.

    Литература:
    \begin{itemize}
        \item Ван дер Варден ``Алгебра''
        \item Лэнг ``Алгебра''
        \item Винберг ``Курс Алгебры''
    \end{itemize}

    \begin{definition}
        Алгебраическая структура --- это множество $M$ + заданные на нём операции + аксиомы на операциях.
    \end{definition}

    \begin{definition}
        Абелева группа --- набор $(M, +: M^2 \to M, 0 \in M)$ с аксиомами:
        \begin{itemize}
            \item[$A_1$)] $\forall a, b, c \in M: (a + b) + c = a + (b + c)$ --- ассоциативность сложения
            \item[$A_2$)] $\forall a \in M: a + 0 = a = 0 + a$ --- нейтральный по сложению элемент
            \item[$A_3$)] $\forall a, b \in M: a + b = b + a$ --- коммутативность сложения
            \item[$A_4$)] $\forall a \in M: \exists -a: a + (-a) = 0 = (-a) + a$ --- существование противоположного
        \end{itemize}
    \end{definition}

    \begin{definition} Опишем следующие аксиомы на наборе $(M, +: M^2 \to M, \cdot: M^2 \to M, 0 \in M, 1 \in M)$:
        \begin{itemize}
            \item[$D$)] $\forall a, b, k \in M: k(a + b) = ka + kb$, $(a + b)k = ak + bk$ --- дистрибутивность
            \item[$M_1$)] $\forall a, b, c \in M: (a \cdot b) \cdot c = a \cdot (b \cdot c)$ --- ассоциативность умножения
            \item[$M_2$)] $\forall a \in M: a \cdot 1 = a = 1 \cdot a$ --- нейтральный по умножению элемент
            \item[$M_3$)] $\forall a, b \in M: a \cdot b = b \cdot a$ --- коммутативность умножения
            \item[$M_4$)] $\forall a \in M: \exists a^{-1}: a \cdot a^{-1} = 1 = a^{-1} \cdot a$ --- существование обратного
        \end{itemize}
        По этим аксиомам определим следующие понятия:
        \begin{itemize}
            \item \emph{Кольцо} --- набор $(M, +, \cdot, 0)$, что верны $A_1$, $A_2$, $A_3$, $A_4$ и $D$.
            \item \emph{Ассоциативное кольцо} --- кольцо с $M_1$.
            \item \emph{Кольцо с единицей} --- кольцо с $M_2$.
            \item \emph{Тело} --- кольцо с $M_1$, $M_2$.
            \item \emph{Поле} --- кольцо с $M_1$, $M_2$, $M_3$, $M_4$.
            \item \emph{Полукольцо} --- кольцо без $A_4$.
        \end{itemize}
    \end{definition}

    \begin{example}
        Если взять $\RR^3$, то векторное произведение в нём неассоциативно и антикоммутативно. Но есть
        \begin{lemma*}[Тождество Якоби]
            $u\times (v\times w) + v\times (w\times u) + w\times (u\times v) = 0$
        \end{lemma*}
    \end{example}

    \begin{example}
        Если взять $R^4 = R \times R^3$ и рассмотреть $\cdot: ((a; u); (b; v)) \mapsto (ab-u\cdot v; av + bu + u\times v)$ и $+: ((a; u); (b; v)) \mapsto (a + b, u + v)$, тогда получим $\HH$ --- ассоциативное некоммутативное тело кватернионов. Ассоциативность доказал Гамильтон.
    \end{example}

    \begin{lemma*}
        $0 \cdot a = 0$
    \end{lemma*}

    \begin{definition}
        Кольцо без делителей нуля называетсся \emph{областью (целостности)}.
    \end{definition}

    \begin{definition}
        Пусть $m\in\NN$. Тогда \emph{множество остатков при делении на $m$} или $\ZZ/m\ZZ$ --- это фактор-множество по отношению эквивалентности $a\sim b \Leftrightarrow (a-b) \mid m$.
    \end{definition}

    \begin{definition}
        \emph{Подкольцо} --- это подмножество кольца, согласованное с его операциями.

        Как следствие ноль и обратимость соглассуются автоматически.
    \end{definition}

    \begin{statement}
        Если $R$ --- подкольцо области целостности $S$, то $R$ --- область целостности.
    \end{statement}

    \begin{definition}
        \emph{Целые Гауссовы числа} или $\ZZ{}[i]$ --- это $\{a + bi \mid a, b \in \ZZ\}$.
    \end{definition}

    \begin{definition}
        Некоторое подмножество $R$ кольца $S$ \emph{замкнуто относительно сложения (умножения)}, если $\forall a, b \in R: a + b \in R$ ($ab \in R$ соответственно).
    \end{definition}

    \begin{remark}
        Замкнутое относительно сложения \textbf{И} умножения подмножество --- подкольцо.
    \end{remark}

    \begin{example}
        Пусть $d$ --- целое, не квадрат. Тогда $\ZZ{}[\sqrt{d}]$ --- область целостности.
    \end{example}

    \section{Теория делимости}

    Пусть $R$ --- область целостности.

    \begin{definition}
        ``$a$ \emph{делит} $b$'' или же $a \mid b$ значит, что $\exists c \in R: b = ac$.
    \end{definition}

    \begin{statement}
        Отношение ``$\mid$'' рефлексивно и транзитивно.
    \end{statement}

    \begin{definition}
        $a$ и $b$ \emph{ассоциативны}, если $a \mid b$ и $b \mid a$. Обозначение: $a \sim b$.
    \end{definition}

    \begin{statement}
        ``$\sim$'' --- отношение эквивалентности.
    \end{statement}

    \begin{statement}
        $a \sim b \Leftrightarrow \exists \text{обратимый} \varepsilon: a = \varepsilon b$.
    \end{statement}

    \begin{proof}
        Пусть $a \sim b$. Тогда $\exists c, d: ac = b, bd = a$. Тогда $a(1-cd) = a - acd = a - bd = a - a = 0$, значит либо $a = 0$, либо $cd = 1$. В первом случае $b = ac = 0c = 0$, значит можно просто взять $\varepsilon = 1$. Во втором случае, $cd = 1$, значит $c$ и $d$ обратимы, тогда можно взять $\varepsilon = d$. следствие в одну сторону доказано.

        Пусть $a = \varepsilon b$, где $\varepsilon$ обратим. Значит:
        \begin{enumerate}
            \item $b\sim a$;
            \item $\exists \delta: \delta\varepsilon = 1$, значит $\delta a = \delta \varepsilon b = b$, значит $a \sim b$.
        \end{enumerate}
        Таким образом $a \sim b$.
    \end{proof}

    \begin{example}
        В $\ZZ{}[i]$ есть только следующие обратимые элементы: $1$, $-1$, $i$ и $-i$. Поэтому все ассоциативные элементы получаются друг из друга домножением на один из $1$, $-1$, $i$, $-i$ и вместе образуют квадрат (на \href{https://ru.wikipedia.org/wiki/%D0%9A%D0%BE%D0%BC%D0%BF%D0%BB%D0%B5%D0%BA%D1%81%D0%BD%D0%B0%D1%8F_%D0%BF%D0%BB%D0%BE%D1%81%D0%BA%D0%BE%D1%81%D1%82%D1%8C}{комплексной плоскоти}) с центром в нуле.
    \end{example}

    \begin{definition}
        \emph{Главным идеалом} элемента $a$ называется множество $M := \{ak \mid k \in R\} = \{b \mid a \text{ делит } b\}$. Обозначение: $(a)$ или $aR$.
    \end{definition}

    \begin{statement}
        $a \mid b \Leftrightarrow b \in aR \Leftrightarrow bR \subseteq aR$. 
    \end{statement}

    \begin{statement}
        $a \sim b \Leftrightarrow aR = bR$.
    \end{statement}

    \begin{statement}$\forall a \in R$
        \begin{enumerate}
            \item $0 \in aR$
            \item $x \in aR \Rightarrow -x \in aR$
            \item $x, y \in aR \Rightarrow x + y \in aR$
            \item $x \in aR, r \in R \Rightarrow xr \in aR$
        \end{enumerate}
    \end{statement}

    \begin{remark}
        То же верно и в некоммутативном $R$.
    \end{remark}

    \begin{example}
        В поле есть только $0R$ и $1R$.
    \end{example}

    \begin{example}
        В $\ZZ$ есть только $m\ZZ$ для каждого $m \in \NN \cup \{0\}$.
    \end{example}
\end{document}