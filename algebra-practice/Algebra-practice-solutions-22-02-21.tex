\documentclass[12pt,a4paper]{article}
\usepackage{solutions}
\usepackage{multicol}

\title{Основы математической логики.\\ Практика. 1 курс.\\Решения.}
\author{Глеб Минаев @ 102 (20.Б02-мкн)}
% \date{}

\begin{document}
    \maketitle

    \begin{multicols}{2}
        \tableofcontents
    \end{multicols}

    \begin{enumproblem}
        \begin{align*}
            &\begin{pmatrix}
                3& x\\
                2& y
            \end{pmatrix}
            \begin{pmatrix}
                1\\
                2
            \end{pmatrix}
            =
            \begin{pmatrix}
                3 + 2x\\
                2 + 2y
            \end{pmatrix}
            =
            \begin{pmatrix}
                \lambda\\
                2\lambda
            \end{pmatrix}&
            &\begin{pmatrix}
                3& x\\
                2& y
            \end{pmatrix}
            \begin{pmatrix}
                2\\
                1
            \end{pmatrix}
            =
            \begin{pmatrix}
                6 + x\\
                4 + y
            \end{pmatrix}
            =
            \begin{pmatrix}
                2\mu\\
                \mu
            \end{pmatrix}&
        \end{align*}
        Следовательно
        \begin{align*}
            &\left\{
                \begin{aligned}
                    2(3 + 2x) &= 2 + 2y\\
                    6 + x &= 2(4 + y)\\
                \end{aligned}
            \right.&
            &\left\{
                \begin{aligned}
                    4 + 4x - 2y &= 0\\
                    -2 + x - 2y &= 0\\
                \end{aligned}
            \right.&
            &\left\{
                \begin{aligned}
                    6 + 3x &= 0\\
                    -2 + x - 2y &= 0\\
                \end{aligned}
            \right.&
            &\left\{
                \begin{aligned}
                    x &= -2\\
                    y &= -1 + \frac{1}{2}x\\
                \end{aligned}
            \right.&
            &\left\{
                \begin{aligned}
                    x &= -2\\
                    y &= -2\\
                \end{aligned}
            \right.
        \end{align*}
        B следовательно $\lambda = 3 + 2x = -1$, $\mu = 4 + y = 2$.
    \end{enumproblem}

    \begin{enumproblem}
        Заметим, что
        \begin{align*}
            &\begin{pmatrix}
                1& 2& 0& 0\\
                -1& -1& 0& 0\\
                1& 1& 1& -2\\
                2& 1& 2& -3\\
            \end{pmatrix}
            \begin{pmatrix}
                a\\
                b\\
                c\\
                d
            \end{pmatrix}
            =
            \lambda
            \begin{pmatrix}
                a\\
                b\\
                c\\
                d
            \end{pmatrix}&
            &\Longrightarrow&
            &\begin{pmatrix}
                1& 2\\
                -1& -1\\
            \end{pmatrix}
            \begin{pmatrix}
                a\\
                b
            \end{pmatrix}
            =
            \lambda
            \begin{pmatrix}
                a\\
                b
            \end{pmatrix}
        \end{align*}
        Несложно видеть, что собственные числа последней матрицы --- $\pm i$, а собственные вектора --- $(\begin{smallmatrix}
            2\\
            \pm i - 1
        \end{smallmatrix})$ соответственно.

        Если первые две координаты не равны обе нулю, то можно считать, что рассматриваемое собственное число $\lambda = \pm i$, а $a = 2$, $b = \lambda - 1$. WLOG $\lambda = i$; второй вариант получается сопряжением. Тогда получим, что
        \begin{align*}
            &\left\{
                \begin{aligned}
                    2 + i - 1 + c - 2d &= i c\\
                    4 + i - 1 + 2c - 3d &= i d\\
                \end{aligned}
            \right.\\
            &\left\{
                \begin{aligned}
                    (1 - i) c - 2d &= -1 - i\\
                    2c - (3 + i)d &= -3 - i\\
                \end{aligned}
            \right.\\
            &\left\{
                \begin{aligned}
                    \left(\frac{(3 + i)(1 - i)}{2} - 2\right)d &= \frac{(3 + i)(1 - i)}{2} -1 - i\\
                    2c - (3 + i)d &= -3 - i\\
                \end{aligned}
            \right.\\
            &\left\{
                \begin{aligned}
                    -i d &= 1 - 2i\\
                    2c - (3 + i)d &= -3 - i\\
                \end{aligned}
            \right.\\
            &\left\{
                \begin{aligned}
                    d &= 2 + i\\
                    2c &= -3 - i + (3 + i)d\\
                \end{aligned}
            \right.\\
            &\left\{
                \begin{aligned}
                    d &= 2 + i\\
                    c &= 1 + 2i\\
                \end{aligned}
            \right.\\
        \end{align*}
        Т.е. два собственных значения --- $\pm i$, а вектора --- $(4; -2 \pm 2i; 1 \pm 2i; 2 \pm i)$.

        Если же первые две координаты нулевые, то мы имеем, что
        \[
            \begin{pmatrix}
                1& -2\\
                2& -3\\
            \end{pmatrix}
            \begin{pmatrix}
                c\\
                d
            \end{pmatrix}
            =
            \lambda
            \begin{pmatrix}
                c\\
                d
            \end{pmatrix}
        \]
        Несложно видеть, что собственное значение --- $-1$, а собственный вектор --- $(0; 0; 1; 1)$.
    \end{enumproblem}

    \begin{enumproblem}
        TODO
    \end{enumproblem}

    \begin{enumproblem}
        TODO
    \end{enumproblem}

    \begin{enumproblem}
        TODO
    \end{enumproblem}

    \begin{enumproblem}
        TODO
    \end{enumproblem}

    \begin{enumproblem}
        TODO
    \end{enumproblem}

    \begin{enumproblem}
        TODO
    \end{enumproblem}

    \begin{enumproblem}
        TODO
    \end{enumproblem}

    \begin{enumproblem}
        TODO
    \end{enumproblem}

    \begin{enumproblem}
        TODO
    \end{enumproblem}

    \begin{enumproblem}
        TODO
    \end{enumproblem}

    \begin{enumproblem}
        TODO
    \end{enumproblem}

    \begin{enumproblem}
        TODO
    \end{enumproblem}

    \begin{enumproblem}
        TODO
    \end{enumproblem}
\end{document}