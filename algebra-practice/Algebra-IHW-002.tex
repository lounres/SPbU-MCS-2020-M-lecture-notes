\documentclass[12pt,a4paper]{article}
\usepackage{solutions}
\usepackage{float}

\title{Индивидуальное ДЗ\\Алгебра}
\author{Глеб Минаев @ 102 (20.Б02-мкн)}
% \date{}

\begin{document}
    \maketitle

    \begin{problem*}
        Нам дана матрица
        \[
            M =
            \begin{pmatrix}
                -2& 0& 12& -20& 0& 0\\
                -3& 3& 15& -26& -1& -1\\
                0& 0& 2& 0& 0& 0\\
                0& 0& 0& 2& 0& 0\\
                -7& 1& 26& -43& 1& -1\\
                0& 0& 1& -3& 0& 2
            \end{pmatrix}
        \]
        Для начала найдём базис, в котором наша матрица будет верхнетреугольной.
        \begin{enumerate}
            \item Найдём собственный вектор с собственным значением $-2$. Для этого нам нужно найти ядро
                \[
                    M + 2E =
                    \begin{pmatrix}
                        0& 0& 12& -20& 0& 0\\
                        -3& 5& 15& -26& -1& -1\\
                        0& 0& 4& 0& 0& 0\\
                        0& 0& 0& 4& 0& 0\\
                        -7& 1& 26& -43& 3& -1\\
                        0& 0& 1& -3& 0& 4
                    \end{pmatrix}
                \]
                Пусть искомый вектор --- $(x_i)_{i=1}^6$, тогда (меняя строки на их линейные комбинации, равносильно изменяем СЛУ)
                \begin{align*}
                    \begin{pmatrix}
                        & & 12& -20& & \\
                        -3& 5& 15& -26& -1& -1\\
                        & & 4& & & \\
                        & & & 4& & \\
                        -7& 1& 26& -43& 3& -1\\
                        & & 1& -3& & 4
                    \end{pmatrix}
                    \begin{pmatrix}
                        x_1\\x_2\\x_3\\x_4\\x_5\\x_6
                    \end{pmatrix}
                    &= 0&
                    \begin{pmatrix}
                        -3& 5& 15& -26& -1& -1\\
                        -7& 1& 26& -43& 3& -1\\
                        & & 1& & & \\
                        & & & 1& & \\
                        & & 12& -20& & \\
                        & & 1& -3& & 4
                    \end{pmatrix}
                    \begin{pmatrix}
                        x_1\\x_2\\x_3\\x_4\\x_5\\x_6
                    \end{pmatrix}
                    &= 0\\
                    \begin{pmatrix}
                        -3& 5& 15& -26& -1& -1\\
                        -7& 1& 26& -43& 3& -1\\
                        & & 1& & & \\
                        & & & 1& & \\
                        & & & & & \\
                        & & & & & 4
                    \end{pmatrix}
                    \begin{pmatrix}
                        x_1\\x_2\\x_3\\x_4\\x_5\\x_6
                    \end{pmatrix}
                    &= 0&
                    \begin{pmatrix}
                        -3& 5& 15& -26& -1& -1\\
                        -7& 1& 26& -43& 3& -1\\
                        & & 1& & & \\
                        & & & 1& & \\
                        & & & & & 1\\
                    \end{pmatrix}
                    \begin{pmatrix}
                        x_1\\x_2\\x_3\\x_4\\x_5\\x_6
                    \end{pmatrix}
                    &= 0
                \end{align*}
                \begin{align*}
                    \begin{pmatrix}
                        -3& 5& & & -1& \\
                        -7& 1& & & 3& \\
                        & & 1& & & \\
                        & & & 1& & \\
                        & & & & & 1\\
                    \end{pmatrix}
                    \begin{pmatrix}
                        x_1\\x_2\\x_3\\x_4\\x_5\\x_6
                    \end{pmatrix}
                    &= 0&
                    \begin{pmatrix}
                        -3& 5& & & -1& \\
                        -1& -9& & & 5& \\
                        & & 1& & & \\
                        & & & 1& & \\
                        & & & & & 1\\
                    \end{pmatrix}
                    \begin{pmatrix}
                        x_1\\x_2\\x_3\\x_4\\x_5\\x_6
                    \end{pmatrix}
                    &= 0\\
                    \begin{pmatrix}
                        & 32& & & -16& \\
                        -1& -9& & & 5& \\
                        & & 1& & & \\
                        & & & 1& & \\
                        & & & & & 1\\
                    \end{pmatrix}
                    \begin{pmatrix}
                        x_1\\x_2\\x_3\\x_4\\x_5\\x_6
                    \end{pmatrix}
                    &= 0&
                    \begin{pmatrix}
                        & 2& & & -1& \\
                        1& 9& & & -5& \\
                        & & 1& & & \\
                        & & & 1& & \\
                        & & & & & 1\\
                    \end{pmatrix}
                    \begin{pmatrix}
                        x_1\\x_2\\x_3\\x_4\\x_5\\x_6
                    \end{pmatrix}
                    &= 0\\
                    \begin{pmatrix}
                        & 2& & & -1& \\
                        1& -1& & & & \\
                        & & 1& & & \\
                        & & & 1& & \\
                        & & & & & 1\\
                    \end{pmatrix}
                    \begin{pmatrix}
                        x_1\\x_2\\x_3\\x_4\\x_5\\x_6
                    \end{pmatrix}
                    &= 0
                \end{align*}
                Отсюда мы получаем, что $x_3 = x_4 = x_6 = 0$, а $(x_1, x_2, x_5) \sim (1, 1, 2)$. Следовательно искомый вектор ---
                \[
                    e_1 :=
                    \begin{pmatrix}
                        1\\1\\0\\0\\2\\0
                    \end{pmatrix}
                \]
                Рассматривая матрицу
                \[
                    S =
                    \begin{pmatrix}
                        1&&&&&\\
                        1& 1&&&&\\
                        && 1&&&\\
                        &&& 1&&\\
                        2&&&& 1&\\
                        &&&&& 1
                    \end{pmatrix}
                \]
                получаем, что
                \[
                    S^{-1} \cdot M \cdot S =
                    \begin{pmatrix}
                        -2&& 12& -20&&\\
                        & 3& 3& -6& -1& -1\\
                        && 2&&&\\
                        &&& 2&&\\
                        & 1& 2& -3& 1& -1\\
                        && 1& -3&&2 
                    \end{pmatrix}
                \]
            
            \item Рассматривая наше пространство по модулю $e_1$, получаем, что оператор $S^{-1} \cdot M \cdot S$ получает вид
                \[
                    M_1 =
                    \begin{pmatrix}
                        3& 3& -6& -1& -1\\
                        & 2&&&\\
                        && 2&&\\
                        1& 2& -3& 1& -1\\
                        & 1& -3&&2 
                    \end{pmatrix}
                \]
                Мы знаем, что $\chi(M_1)(\lambda + 2) \sim \chi(M)$, значит
                \[\chi(M_1) \sim (\lambda - 2)^5\]

                Найдём собственные вектора $M_1$. Для этого нужно найти ядро
                \[
                    M_1 - 2E =
                    \begin{pmatrix}
                        1& 3& -6& -1& -1\\
                        & 0&&&\\
                        && 0&&\\
                        1& 2& -3& -1& -1\\
                        & 1& -3&&0
                    \end{pmatrix}
                \]
                Значит нам нужно решить СЛУ; тогда (меняя строки на их линейные комбинации, равносильно изменяем СЛУ)
                \begin{align*}
                    \begin{pmatrix}
                        1& 3& -6& -1& -1\\
                        & 0&&&\\
                        && 0&&\\
                        1& 2& -3& -1& -1\\
                        & 1& -3&&0
                    \end{pmatrix}
                    \begin{pmatrix}
                        x_1\\x_2\\x_3\\x_4\\x_5
                    \end{pmatrix}
                    &= 0&
                    \begin{pmatrix}
                        1& 3& -6& -1& -1\\
                        1& 2& -3& -1& -1\\
                        & 1& -3&&
                    \end{pmatrix}
                    \begin{pmatrix}
                        x_1\\x_2\\x_3\\x_4\\x_5
                    \end{pmatrix}
                    &= 0\\
                    \begin{pmatrix}
                        1& 3& -6& -1& -1\\
                        & 1& -3&&
                    \end{pmatrix}
                    \begin{pmatrix}
                        x_1\\x_2\\x_3\\x_4\\x_5
                    \end{pmatrix}
                    &= 0&
                    \begin{pmatrix}
                        1&& 3& -1& -1\\
                        & 1& -3&&
                    \end{pmatrix}
                    \begin{pmatrix}
                        x_1\\x_2\\x_3\\x_4\\x_5
                    \end{pmatrix}
                    &= 0
                \end{align*}
                Таким образом несложно видеть, что ядро порождается векторами
                \[
                    e_2 :=
                    \begin{pmatrix}
                        1\\0\\0\\0\\1
                    \end{pmatrix},
                    \qquad
                    e_3 :=
                    \begin{pmatrix}
                        1\\0\\0\\1\\0
                    \end{pmatrix},
                    \qquad \text{ и } \qquad
                    e_4 :=
                    \begin{pmatrix}
                        -3\\3\\1\\0\\0
                    \end{pmatrix}
                \]
                Приписывая в начало к ним по нулю, получаем собственные с точностью до $e_1$ вектора $S^{-1} \cdot M \cdot S$, а значит и $M$. Дополняя $\{e_1; \dots; e_4\}$ до базиса, получаем матрицу
                \[
                    S =
                    \begin{pmatrix}
                        1&&&&&\\
                        1& 1& 1& -3&1&\\
                        &&& 3&&1\\
                        &&& 1&&\\
                        2&&1&&&\\
                        &1&&&&
                    \end{pmatrix}
                \]
                Следовательно получаем, что
                \[
                    S^{-1} \cdot M \cdot S =
                    \begin{pmatrix}
                        -2&&&16&&12\\
                        &2&&&&1\\
                        &&2&&1&2\\
                        &&&2&&\\
                        &&&&2&\\
                        &&&&&2 
                    \end{pmatrix}
                \]
        \end{enumerate}

        Фактически, теперь мы почти получили жорданов базис: вектора полученного базиса по применению оператора $M$ домножаются на соответствующие им значения и увеличиваются на другие, меньшие по индексу базисные вектора.
        
        Сначала отделим блоки векторов с разными собственными значениями. Для этого вычтем из векторов $e_2$, \dots, $e_6$ (с собственным значением $2$) вектор $e_1$ (с собственным значением $-2$) с правильными коэффициентами, чтобы по применению оператора $M$ вектора $e_2$, \dots, $e_6$ переходили в линейные комбинации только $e_2$, \dots, $e_6$ (без $e_1$). Для этого определим новое значение матрицы
        \[
            S :=
            \begin{pmatrix}
                1&&&&&\\
                1& 1& 1& -3&1&\\
                &&& 3&&1\\
                &&& 1&&\\
                2&&1&&&\\
                &1&&&&
            \end{pmatrix}
            \begin{pmatrix}
                1&&&4&&3\\
                &1&&&&\\
                &&1&&&\\
                &&&1&&\\
                &&&&1&\\
                &&&&&1\\
            \end{pmatrix}
            =
            \begin{pmatrix}
                1&&& 4&& 3\\
                1& 1& 1& 1& 1& 3\\
                &&& 3&& 1\\
                &&& 1&&\\
                2&& 1& 8&& 6\\
                & 1&&&& 
            \end{pmatrix}
        \]
        и получим, что
        \[
            S^{-1} \cdot M \cdot S =
            \begin{pmatrix}
                -2&&&&&\\
                &2&&&&1\\
                &&2&&1&2\\
                &&&2&&\\
                &&&&2&\\
                &&&&&2 
            \end{pmatrix}
        \]

        Теперь будем исправлять блоки векторов одинаковых собственных значений.
        \begin{enumerate}
            \item Видим, что $e_6$ по применению $M$ порождает кроме себя $e_2 + 2e_3$; поэтому заменим $e_3$ на $2e_3 + e_2$: определим новое значение матрицы
                \[
                    S :=
                    \begin{pmatrix}
                        1&&& 4&& 3\\
                        1& 1& 1& 1& 1& 3\\
                        &&& 3&& 1\\
                        &&& 1&&\\
                        2&& 1& 8&& 6\\
                        & 1&&&&
                    \end{pmatrix}
                    \begin{pmatrix}
                        1&&&&&\\
                        &1&1&&&\\
                        &&2&&&\\
                        &&&1&&\\
                        &&&&1&\\
                        &&&&&1
                    \end{pmatrix}
                    =
                    \begin{pmatrix}
                        1&&& 4&& 3\\
                        1& 1& 3& 1& 1& 3\\
                        &&& 3&& 1\\
                        &&& 1&&\\
                        2&&2& 8&& 6\\
                        &1&1&&&
                    \end{pmatrix}
                \]
                Получаем
                \[
                    S^{-1} \cdot M \cdot S =
                    \begin{pmatrix}
                        -2&&&&&\\
                        &2&&&-\frac{1}{2}&\\
                        &&2&&\frac{1}{2}&1\\
                        &&&2&&\\
                        &&&&2&\\
                        &&&&&2 
                    \end{pmatrix}
                \]

            \item Видим, что $e_6$ порождает только $e_3$; хотим чтобы никакой вектор больше не порождал $e_3$. Для этого вычтем $e_6$ из всех остальных векторов, порождающих $e_3$, с правильными коэффициентами: определим новое значение матрицы
                \[
                    S :=
                    \begin{pmatrix}
                        1&&& 4&& 3\\
                        1& 1& 3& 1& 1& 3\\
                        &&& 3&& 1\\
                        &&& 1&&\\
                        2&&2& 8&& 6\\
                        &1&1&&&
                    \end{pmatrix}
                    \begin{pmatrix}
                        1&&&&&\\
                        &1&&&&\\
                        &&1&&&\\
                        &&&1&&\\
                        &&&&1&\\
                        &&&&-\frac{1}{2}&1
                    \end{pmatrix}
                    =
                    \begin{pmatrix}
                        1&&& 4& -\frac{3}{2}& 3\\
                        1& 1& 3& 1& -\frac{1}{2}& 3\\
                        &&& 3& -\frac{1}{2}& 1\\
                        &&& 1&&\\
                        2&& 2& 8& -3& 6\\
                        & 1& 1&&&
                    \end{pmatrix}
                \]
                Получаем
                \[
                    S^{-1} \cdot M \cdot S =
                    \begin{pmatrix}
                        -2&&&&&\\
                        &2&&&-\frac{1}{2}&\\
                        &&2&&&1\\
                        &&&2&&\\
                        &&&&2&\\
                        &&&&&2 
                    \end{pmatrix}
                \]

            \item Видим, что $e_6$ порождает только $e_3$, а $e_3$ ничто больше не порождает. Поэтому можем переставить $e_3$ и $e_5$: определим новое значение матрицы
                \[
                    S :=
                    \begin{pmatrix}
                        1&&& 4& -\frac{3}{2}& 3\\
                        1& 1& 3& 1& -\frac{1}{2}& 3\\
                        &&& 3& -\frac{1}{2}& 1\\
                        &&& 1&&\\
                        2&& 2& 8& -3& 6\\
                        & 1& 1&&&
                    \end{pmatrix}
                    \begin{pmatrix}
                        1&&&&&\\
                        &1&&&&\\
                        &&&&1&\\
                        &&&1&&\\
                        &&1&&&\\
                        &&&&&1
                    \end{pmatrix}
                    =
                    \begin{pmatrix}
                        1&& -\frac{3}{2}& 4&& 3\\
                        1& 1& -\frac{1}{2}& 1& 3& 3\\
                        && -\frac{1}{2}& 3&& 1\\
                        &&& 1&&\\
                        2&& -3& 8& 2& 6\\
                        & 1&&& 1&
                    \end{pmatrix}
                \]
                Получаем
                \[
                    S^{-1} \cdot M \cdot S =
                    \begin{pmatrix}
                        -2&&&&&\\
                        &2& -\frac{1}{2}&&&\\
                        &&2&&&\\
                        &&&2&&\\
                        &&&&2&1\\
                        &&&&&2 
                    \end{pmatrix}
                \]

            \item $e_5$ ничто не порождает --- идём дальше.

            \item $e_4$ ничто не порождает --- идём дальше.

            \item Видим, что $e_3$ порождает только $e_2$, а $e_2$ ничто больше не порождает. $e_2$ стоит на правильном месте, но не с тем коэффициентом, поэтому разделим его на $-2$: определим новое значение матрицы
                \[
                    S :=
                    \begin{pmatrix}
                        1&& -\frac{3}{2}& 4&& 3\\
                        1& 1& -\frac{1}{2}& 1& 3& 3\\
                        && -\frac{1}{2}& 3&& 1\\
                        &&& 1&&\\
                        2&& -3& 8& 2& 6\\
                        & 1&&& 1&
                    \end{pmatrix}
                    \begin{pmatrix}
                        1&&&&&\\
                        & -\frac{1}{2}&&&&\\
                        &&1&&&\\
                        &&&1&&\\
                        &&&&1&\\
                        &&&&&1
                    \end{pmatrix}
                    =
                    \begin{pmatrix}
                        1&& -\frac{3}{2}& 4&& 3\\
                        1& -\frac{1}{2}& -\frac{1}{2}& 1& 3& 3\\
                        && -\frac{1}{2}& 3&& 1\\
                        &&& 1&&\\
                        2&& -3& 8& 2& 6\\
                        & -\frac{1}{2}&&& 1&
                    \end{pmatrix}
                \]
                Получаем
                \[
                    S^{-1} \cdot M \cdot S =
                    \begin{pmatrix}
                        -2&&&&&\\
                        &2& 1&&&\\
                        &&2&&&\\
                        &&&2&&\\
                        &&&&2&1\\
                        &&&&&2 
                    \end{pmatrix}
                \]
        \end{enumerate}

        Таким образом в жордановом базисе
        \[\left\{
            \begin{pmatrix}
                1\\1\\0\\0\\2\\0
            \end{pmatrix},
            \begin{pmatrix}
                0\\-\frac{1}{2}\\0\\0\\0\\-\frac{1}{2}
            \end{pmatrix},
            \begin{pmatrix}
                -\frac{3}{2}\\-\frac{1}{2}\\-\frac{1}{2}\\0\\-3\\0
            \end{pmatrix},
            \begin{pmatrix}
                4\\1\\3\\1\\8\\0
            \end{pmatrix},
            \begin{pmatrix}
                0\\3\\0\\0\\2\\1
            \end{pmatrix},
            \begin{pmatrix}
                3\\3\\1\\0\\6\\0
            \end{pmatrix}
        \right\}\]
        матрица имеет вид жордановой матрицы
        \[
            \begin{pmatrix}
                -2&&&&&\\
                &2& 1&&&\\
                &&2&&&\\
                &&&2&&\\
                &&&&2&1\\
                &&&&&2 
            \end{pmatrix}
        \]
    \end{problem*}
\end{document}