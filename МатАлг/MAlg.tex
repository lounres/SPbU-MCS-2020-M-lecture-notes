\documentclass[12pt,a4paper]{article}
\usepackage{math-text}

\title{Математические основы алгоритмов}
\author{А. С. Охотин}
\date{}

\begin{document}
    \maketitle

    \begin{definition}
        \emph{Машина с произвольным доступом в память (RAM)}. У нас есть память в виде ячеек на $\ZZ$, где хранятся целые числа. Будем называть
        \begin{itemize}
            \item \emph{константой} (и писать ``$n$'') всякую целочисленную константу,
            \item \emph{прямой адресацией} (и писать ``$x_n$'') получение значения по адресу, заданным константой $n$,
            \item \emph{косвенной адресацией} (и писать ``$x_{x_n}$'') получение значения по адресу, заданным значением по адресу, заданным значением константой $n$.
        \end{itemize}
        Программы --- конечные последовательности, состоящие из команд (строчек команд) следующего типа:
        \begin{itemize}
            \item присваивание: \verb|A = B|, где \verb|B| может быть константой или адресацией, а $A$ может быть только адресацией;
            \item арифметические операции: \verb|A = B + C|, \verb|A = B - C|, \verb|A = B × C|, \verb|A = B / C|, \verb|A = B ÷ C|, где \verb|B| и \verb|C| --- константа или адресация, а \verb|A| --- адресация;
            \item перевод чтения программы на строку с номером $n$: \verb|GOTO n|;
            \item перевод чтения программы на строку с номером $x_n$ (значения ячейки по адресу $n$): \verb|GOTO x_n|;
            \item \verb|IF A == B THEN GOTO n| (вместо \verb|A == B| может быть \verb|A >= B|; вместо \verb|n| может быть \verb|x_n|), где \verb|A| и \verb|B| --- константы или адресации;
            \item команда остановки: \verb|HALT|.
        \end{itemize}
    \end{definition}

    \begin{remark*}
        Вся суть вопросов заключена в том, чтобы вычислить какую-то функцию. В таком случае задачу можно воспринимать так, что в ячейке $0$ записано количество входных значений $n$, а в ячейках от $1$ до $n$ записаны значения входных параметров, а требуется вычислить функцию от этих входящих значений и оставить их в таком же виде.

        Также иногда можно писать менее низкоуровневые команды, если понятна их низкоуровневая реализация. Например, ``провести ребро из $v$ в $u$'', ``просумировать $n$ конкретных значений'', а не две как это реализуется командой \verb|A = B + C| и т.д.
    \end{remark*}

    \begin{example}
        Функция вычисления факвториала $n$ выглядит следующим образом.
        \begin{verbatim}
функция f(n):
    если n = 0
        ответ 1
    иначе
        ответ n * f(n - 1)
        \end{verbatim}
        что низкоуровнево может быть реализуемо как
        \begin{verbatim}
1. A = n
2. RES = 1
3. IF A == 0 GOTO 7
4. RES = RES * A
5. A = A - 1
6. GOTO 3
7. HALT
        \end{verbatim}
        Также нерекурсивно можно реализовать так.
        \begin{verbatim}
функция f(n):
    пусть x = 1
    для i = 1, ..., n
        x = x * i
    ответ x
        \end{verbatim}
        что низкоуровнево может быть реализуемо как
        \begin{verbatim}
1. RES = 1
2. ITER = 1
3. IF ITER > n GOTO 7
4. RES = RES * ITER
5. ITER = ITER + 1
6. GOTO 3
7. HALT
        \end{verbatim}
    \end{example}

    \begin{definition}
        \emph{Сложность работы программы} --- это функция $t(n)$ равная максимуму затрачиваемого времени по всем входным данным длины $n$. При этом время вычисляется как сумма стоимостей всех выполненных операций.
        
        В обычной модели стоимость равна $1$ для каждой операции. Но бывает проблемма, что, например, если хочется найти $a + b$ и $c + d$, то можно записать $a$ и $c$ в одну ячейку (например, как $a + 10^k * c$), а $b$ и $d$ в другую и сложить их за одну операцию вместо двух. Да и вообще бывает проблема, что складывание двух больших чисел и двух маленьких в реальности являются задачами разной сложности. Поэтому есть также модель ``log-cost'', где каждая операция стоит логарифм от входящих в неё значений.
    \end{definition}

    \begin{definition}
        \emph{Сложность памяти программы} --- это такая функция $s(n)$ равная максимуму затрачиваемого места по всем входным данным длины $n$. В качестве затрачиваемого места подразумевается количество изменённых (хоть раз) ячеек.
    \end{definition}

    \begin{definition}
        Для всяких функций $f(n)$ и $g(n)$ скажем, что
        \begin{itemize}
            \item $g = o(f)$, если $\lim_{n \to \infty} \frac{|g(n)|}{|f(n)|} = 0$,
            \item $g = O(f)$, если $|g| \leqslant C|f|$ для некоторой константы $C > 0$ с некоторого момента,
            \item $g = \omega(f)$, если $\lim_{n \to \infty} \frac{|g(n)|}{|f(n)|} = +\infty$,
            \item $g = \Omega(f)$, если $|g| \geqslant C|f|$ для некоторой константы $C > 0$ с некоторого момента,
            \item $g = \Theta(f)$, если $c|f| \leqslant |g| \geqslant C|f|$ для некоторых констант $c, C > 0$ с некоторого момента.
        \end{itemize}
    \end{definition}

    \begin{theorem}
        Умножение двух чисел длины не более $n$ можно посчитать за время $O(n^2)$.
    \end{theorem}

    \begin{proof}
        Действительно, можно посчитать произведение в столбик. В таком случае будет произведено $n^2$ умножений и $\approx n^2$ сложений. В таком случае произведение можно посчитать за $O(n^2)$ шагов. При этом памяти можно использовать не более $2n$ при том же времени работы, если сразу прибавлять полученные произведения к нулю.
    \end{proof}

    \begin{theorem}[Карацуба]
        Умножение двух чисел длины не более $n$ можно посчитать за время $O(n^{\log_2(3)})$ ($\log_2(3) \approx 1.58$).
    \end{theorem}

    \begin{proof}
        Пусть даны два числа $a = \overline{A_1A_2}$ и $b = \overline{B_1B_2}$, где $A_1$, $A_2$, $B_1$, $B_2$ --- последовательности цифр длины $k$. Тогда $c = ab$ имеет вид $\overline{C_1C_2C_3}$, где
        \[
            C_1 = A_1 B_1,
            \qquad
            C_2 = A_1 B_2 + A_2 B_1 = (A_1 + A_2)(B_1 + B_2) - A_1 B_1 - A_2 B_2,
            \qquad
            C_3 = A_2 B_2.
        \]
        Следовательно посчитать произведение можно с помощью всего трёх произведений (и операции переноса через разряды занимает линейное время от длины): $A_1 B_1$, $A_2 B_2$ и $(A_1 + A_2) (B_1 + B_2)$.

        Давйте разобьём наши числа $\overline{a_{n-1}\dots a_0}$ и $\overline{b_{n-1}\dots b_0}$ на две приемерно равные половины: $\overline{a_{n-1}\dots a_0} = \overline{A_1A_2}$, $\overline{b_{n-1}\dots b_0} = \overline{B_1B_2}$. (Важно чтобы длины последовательностей были одинаковой длины, но можно у $A_1$ и $B_1$ добавить фиктивные нули в начале.) Тогда асимптотика перемножения для длины $n$ будет описываться формулой
        \[T(n) = 3 T(n/2) + O(n).\]
        Следовательно несложно понять по индукции, что $T(n) = O(n^{\log_2(3)})$.
    \end{proof}

    \begin{theorem}\ 
        \begin{enumerate}
            \item Сортировка пузырьком работает за $O(n^2)$.
            \item Сортировка вставками (добавлением элемента) работает за $\approx n^2/2$.
        \end{enumerate}
    \end{theorem}

    \begin{theorem}
        Сортировка слиянием (merge sort) работает за $O(n\log(n))$.
    \end{theorem}

    \begin{proof}
        Заметим, что слияние двух отсортированных массивов длины не более $n$ можно реализовать за $2n$ операций. Следовательно асимптотика сортировки для длины $n$ будет описываться формулой
        \[T(n) = 2T(n/2) + O(n),\]
        откуда $T(n) = O(n \log(n))$.
    \end{proof}

    \begin{theorem}[``быстрая сортировка'', Хоар (Hoar), 1961]
        Быстрая сортировка работает за $O(n \log(n))$.
    \end{theorem}

    \begin{proof}
        Рассмотрим следующий алгоритм.
        \begin{enumerate}
            \item Выберем случайный элемент $y$.
            \item Разделим весь массив без $y$ на две части: элементы $\leqslant y$ и элементы $> y$, и поставим их в порядке ``элементы $\leqslant y$, $y$, элементы $> y$''.
            \item Применим быструю сортировку к полученным частям.
        \end{enumerate}
        Понятно, что после сортировки каждой из оставшихся частей, массив станет останется отсортированным. Давайте более конкретно опишем алгоритм:
        \begin{verbatim}
function quicksort(l, m)
    if m - l >= 2 then
        i = partition(l, m)
        quicksort(l, i)
        quicksort(i+1, m)

function partition(l, m)
    choose random p among l, ..., m-1
    y = x_p
    x_p <-> x_{m-1}
    i = l
    j = m - 1
    while i < j do
        if x_i < y then
            i = i + 1
        else if x_j >= y then
            j = j - 1
        else
            x_i <-> x_j
    x_{i-1} <-> x_{m-1}
    return i-1
        \end{verbatim}

        Пусть $C(n)$ --- количество сравнений во время работы алгоритма. Далее несложно убедиться по индукции, что
        \begin{itemize}
            \item $C(n) = \Omega(n \log(n))$,
            \item минимальное количество сравнений на одном и том же массиве будет достигаться только при выборе медиан,
            \item $C(n) = O(n^2)$,
            \item максимальное количество сравнений на одном и том же массиве будет достигаться только при выборе крайних элементов.
        \end{itemize}

        Также заметим, что если выбор случайного элемента $y$ имеет равновероятное распределение, то вероятность того, что $y_i$ и $y_j$ в отсортированном массиве будут сравнены, равна $2/(j-i+1)$. Действительно, в самом начале обрабатывается массив содержащий все элементы $y_i$, $y_{i+1}$, \dots, $y_j$, и пока никакой из этих элементов не будет выбран как опорный, то все они будут находится в одном обрабатываемом массиве и $y_i$ и $y_j$ не будут сравнены. Если же будет выбран элемент $y_k$ для $i < k < j$, то $y_i$ и $y_j$ будут сравнены с $y_k$, попадут в разные обрабатываемые массивы и больше никогда не будут сравнены. Если же будет выбран один из $y_i$ и $y_j$, то они будут сравнены единожды. Значит вероятность того, что $y_i$ и $y_j$ будут сравнены равна $2/(j-i+1)$. Значит мат. ожидание количества сравнений равно
        \[
            \sum_{i=1}^{n-1} \sum_{j=i+1}^n \frac{2}{j-i+1}
            = \sum_{i=1}^{n-1} \sum^{k=1}^{n-i} \frac{2}{k+1}
            \approx \sum_{i=1}^{n-1} 2\ln(n-i)
            \approx 2n \ln(n).
        \]

        
    \end{proof}


\end{document}